\providecommand{\main}{../main}
\documentclass[\main/main.tex]{subfiles}
\graphicspath{{../images/}}
\newcommand{\NP}[1]{{}:#1:{}}
\begin{document}
\subsection{
    自由場の理論
}
抽象的な議論ばかりで具体的な計算に欠けるので、自由場と摂動論の計算を扱う。
相関関数の母関数である生成汎函数を以下のように導入する。
\begin{align}
    Z[J]
    = ∫\𝒟ϕ\e^{-S[ϕ,J]}
    = ∫\𝒟ϕ\exp(-S[ϕ]+∫\d^d{x}J(x)ϕ(x))
\end{align}
$J(x)$を源(source)と呼ぶ。
すると$n$点関数は、
\begin{align}
    ⟨ϕ(x₁)⋯ϕ(xₙ)⟩
    = \f{1}{Z[0]}\δ{J(x₁)}⋯\δ{J(xₙ)}Z[J]\eval_{J=0}
\end{align}
と表すことができる。
回りくどく感じるかもしれないが、作用が場の2次形式で表される場合にはこれが具体的に計算できてしまう。
このとき作用は以下のような2次形式で表される。
\begin{align}
    S[ϕ] = \f{1}{2}∫\d[d]x\d[d]y ϕ(x)K(x,y)ϕ(y)
\end{align}
$K(x,y)$は$x,y$の交換について対称な超関数であり、例えば
\begin{align}
    K(x,y)
    =\∂{x^μ}\∂{y_μ}δ^d(x-y) + m^2δ(x-y)
\end{align}
の場合は
\begin{align}
    S[ϕ]
    &
    = \f{1}{2}∫\d[d]x\d[d]y ϕ(x)\l(
        \∂{x^μ}\∂{y_μ}δ^d(x-y) + m^2δ(x-y)
    \r)ϕ(y)
    \∅ &
    = \f{1}{2}∫\d[d]x(∂_μϕ(x)∂^μϕ(x)+m^2ϕ(x)^2)
\end{align}
となる。
生成汎函数は源を加えて、
\begin{align}
    Z[J]
    = ∫\𝒟ϕ\exp(
        -∫\d[d]x\d[d]yϕ(x)K(x,y)ϕ(y)
        +∫\d[d]x J(x)ϕ(x)
    )
\end{align}
となる。
これを計算するにあたって、いきなり連続的な場合を考えるのではなく離散的な場合を考える。
すなわち、$N×N$対称行列$K_{ij}$によって
\begin{align}
    Z[J] = ∫ ∏_i\dd{ϕ_i}\exp(
        -\f{1}{2}ϕ_iK_{ij}ϕ_j + J_iϕ_i
    )
\end{align}
と表してやる。
これは単なるGauss積分であり、簡単に解ける。
まず平方完成によって
\begin{align}
    -\f{1}{2}ϕ_iK_{ij}ϕ_j + J_iϕ_i
    = -\f{1}{2}\Q(ϕ_i + J_k K^{-1}_{ki})K_{ij}(ϕ_j + K^{-1}_{jl} J_l)
    +\f{1}{2} J_i K^{-1}_{ij} J_j
\end{align}
と変形する。
$K_{ij}$が対称であることから$J_kK^{-1}_{ki} = K^{-1}_{ik}J_k$が成り立つので、$\~ϕ_i = ϕ_i + J_kK^{-1}_{ki}$とおくことで、
\begin{align}
    Z[J]
    = \exp(\f{1}{2} J_i K^{-1}_{ij} J_j)
    ∫ ∏_i\dd{ϕ_i}\exp(-\f{1}{2}\~ϕ_iK_{ij}\~ϕ_j)
    ≕ 𝒩\exp(\f{1}{2} J_i K^{-1}_{ij} J_j)
\end{align}
と書ける。ここで、$J$に依存しない積分は全て定数$𝒩$としてまとめてしまった。
さらに、分配関数は定数倍して定義し直せるので、
\begin{align}
    Z[J] = \exp(\f{1}{2} J_i K^{-1}_{ij} J_j)
\end{align}
と書ける。
この式を、連続極限でも信用することにする。
すなわち、
\begin{align}
    Z[J] = \exp(
        \f{1}{2}∫\d[d]x\d[d]y J(x)K^{-1}(x,y)J(y)
    )
\end{align}
とする。$K^{-1}(x,y)$は$K(x,y)$に対するGreen関数であり、
\begin{align}
    ∫\d[d]y K^{-1}(x,y)K(y,z) = δ(x-z)
\end{align}
となるような関数である。
あとで零質量の自由場を扱うが、その場合は
\begin{align}
    K(x,y) = ∂_μ∂^μδ(x-y)
\end{align}
であり、Green関数は
\begin{align}
    K^{-1}(x,y) = \f{C}{|x-y|^{d-2}},\␣
    C ≔ \f{1}{(d-2)S_d}
\end{align}
となる。
ここで$S_d$は$d-1$次元超球面の面積であり、具体的には
\begin{align}
    S_d = \f{2π^{d/2}}{Γ(d/2)}
\end{align}
である。

生成汎函数が求まったので、あとはこれをひたすら汎函数微分していけば、相関関数が求まる。
まず、$Z[J]$を$J(x)$で汎函数微分すると、
\begin{align}
    \δ{J(x)}Z[J] = ∫\d[d]y K^{-1}(x,y)J(y) Z[J]
\end{align}
となる。したがって$J(x) = 0$とおくと、1点関数は
\begin{align}
    \⟨ϕ(x)\⟩ = \f{1}{Z[0]}\δ{J(x)}Z[J]\eval_{J=0} = 0
\end{align}
となる。さらに$J(y)$で汎函数微分すると、
\begin{align}
    \δ{J(x)}\δ{J(y)}Z[J]
    = K^{-1}(x,y)Z[J] + 𝒪(J^2)
\end{align}
となるので、2点関数は
\begin{align}
    \⟨ϕ(x)ϕ(y)\⟩
    = \f{1}{Z[0]}\δ{J(x)}\δ{J(y)}Z[J]\eval_{J=0} = K^{-1}(x,y)
\end{align}
となる。
\begin{align}
    &
    \⟨ϕ(x₁)⋯ϕ(x_{2n+1})\⟩ = 0
    \\ &
    \⟨ϕ(x₁)⋯ϕ(x_{2n})\⟩
    =\f{1}{2^nn!}∑_{σ ∈ 𝔖_{2n}}
        \⟨ϕ(x_{σ(1)})ϕ(x_{σ(2)})\⟩
            ⋯
        \⟨ϕ(x_{σ(2n-1)})ϕ(x_{σ(2n)})\⟩
\end{align}
\subsection{
    摂動論
}
\begin{align}
    S[ϕ]
    = ∫\d[d]x\l(
        \f{1}{2}∂_μϕ(x)∂^μϕ(x)
        +\f{1}{2}m^2ϕ(x)^2
        +\f{1}{4!}λϕ(x)^4
    \r)
\end{align}
\begin{align}
    ∫\𝒟ϕF[ϕ]\e^{-S[ϕ]}
    &
    = ∫\𝒟ϕF[ϕ]\e^{-S_0[ϕ]-S_\t{int}[ϕ]}
    \∅ &
    = ∫\𝒟ϕF[ϕ]\l(
        1+\f{1}{4!}λ∫\d[d]xϕ(x)^4
        +⋯
    \r)\e^{-S_0[ϕ]}
\end{align}
\subsection{
    正規積
}
場の量子論において、同一点にある2つの場の積は発散する。
そこで自由場の正規積を以下のように定義する。
\begin{align}
    \NP{A(x)B(y)} = A(x)B(y) - \⟨A(x)B(y)\⟩
\end{align}
これは発散を差し引く形になっており、$x→y$で収束する。
また$n+1$個の演算子の正規積は再帰的に
\begin{align}
    \NP{A_1⋯A_{n+1}}
    = \NP{A_1⋯A_n}A_{n+1}
    - ∑_{i=1}^n\⟨A_{n+1}A_i\⟩\NP{A_1⋯\cancel{A_i}⋯A_n}
\end{align}
と定義する。ここで$\NP{A(x)} = A(x)$とする。
たとえば
\begin{align}
    \NP{A₁A₂A₃}
    &
    = \NP{A_1A_2}A_3 - \⟨A_1A_3\⟩\NP{A_2} - \⟨A_2A_3\⟩\NP{A_1}
    \∅ &
    = A₁A₂A₃ - \⟨A₁A₂\⟩A₃ - \⟨A₂A₃\⟩A₁ - \⟨A₃A₁\⟩A₂
\end{align}
である。
通常の場の積を正規積によって表すこともできる。
\begin{align}
    A_1A_2A_3A_4
    = \NP{ A_1A_2A_3A_4}
    &
    +\f{1}{4}∑_{σ∈𝔖_4}\⟨A_{σ(1)}A_{σ(2)}\⟩\NP{A_{σ(3)}A_{σ(4)}}
    \∅ &
    +\f{1}{8}∑_{σ∈𝔖_4}\⟨A_{σ(1)}A_{σ(2)}\⟩\⟨A_{σ(3)}A_{σ(4)}\⟩
\end{align}
係数の$1/4,1/8$は重複して数えている寄与を除くための因子である。
両辺の期待値をとると、(\ref{Wick theorem})から$\⟨\NP{A_1⋯A_4}\⟩ = 0$が分かる。
同様の議論によって一般に、
\begin{align}
    A_1⋯A_n =
    &
    \NP{A_1⋯A_n}
    \∅ &
    + \f{1}{2(n-2)!}∑_{σ∈𝔖_n}\⟨A_{σ(1)}A_{σ(2)}\⟩\NP{A_{σ(3)}⋯A_{σ(n)}}
    \∅ &
    + \f{1}{4(n-4)!}∑_{σ∈𝔖_n}\⟨A_{σ(1)}A_{σ(2)}\⟩\⟨A_{σ(3)}A_{σ(4)}\⟩\NP{A_{σ(5)}⋯A_{σ(n)}}
    \∅ &
    + ⋯
\end{align}
と書ける。ここから(\ref{Wick theorem})により$\⟨\NP{A_1⋯A_n}\⟩=0$となる。
この式もWickの定理と呼ばれる。

()において、$A_{m+1},…,A_n$を$B_1,…,B_l$と書くことにする。
右辺にある$A$どうしの縮約と$B$どうしの縮約を全て左辺に移項すると、
左辺は$\NP{A_1⋯A_m}\NP{B_1⋯B_l}$と表せる。
また右辺には$A$と$B$の間の縮約だけが残り、
\begin{align}
    &
    \NP{A_1⋯A_m}\NP{B_1⋯B_l}
    \∅ &
    = \NP{A_1⋯A_mB_1⋯B_l}
    \∅ & \␣
    + (\t{係数})× ∑_{σ∈𝔖_m}∑_{σ'∈𝔖_l}\⟨A_{σ(1)}B_{σ'(1)}\⟩
    \NP{\t{残り}}
    \∅ & \␣
    + (\t{係数})× ∑_{σ∈𝔖_m}∑_{σ'∈𝔖_l}
    \⟨A_{σ(1)}B_{σ'(1)}\⟩\⟨A_{σ(2)}B_{σ'(2)}\⟩
    \NP{\t{残り}}
    \∅ & \␣
    +⋯
    \label{general Wick theorem}
\end{align}
と書ける。ここで$(\t{係数})$は重複して数えている寄与を除くための因子であり、具体的に書いてもあまり意味がないので省略した。

正規積について複雑な式を並べてきたが、これらは単純な母関数表示にまとめることができる。
まず、以下の等式が成り立つ。
\begin{align}
    \NP{\e^{A}}\NP{\e^{B}}
    = \e^{\⟨AB\⟩}\NP{\e^{A+B}}
\end{align}
これは両辺を$A,B$についてTaylor展開した結果が
(\ref{general Wick theorem})で添字の違いを無視した場合の式に一致することから分かる。
この式を2回使うと、
\begin{align}
    \NP{\e^{A}}\NP{\e^{B}}\NP{\e^{C}}
    &
    = \e^{\⟨AB\⟩}\NP{\e^{A+B}}\NP{\e^{C}}
    \∅ &
    = \e^{\⟨AB\⟩+\⟨BC\⟩+\⟨CA\⟩}\NP{\e^{A+B+C}}
\end{align}
が得られる。同様の議論を繰り返して
\begin{align}
    ∏_i\NP{\e^{A_i}}
    = \e^{∑_{i<j}\⟨A_i A_j\⟩}\NP{\e^{∑_i A_i}}
\end{align}
を得る。これはWickの定理の母関数表示となっている。

% \begin{align}
%    \^ϕ(t,x)
%     = ∫\f{\dd{k}}{\√{2πω}}
%         (\^aₖ\e^{\i kx} + \^aₖ^†\e^{-\i kx})
% \end{align}
% \begin{align}
%     \^ϕ(t,x) = \^ϕ^{(+)}(t,x)+\^ϕ^{(-)}(t,x)
% \end{align}
% \begin{align}
%     \⟨ϕ(t,x)ϕ(t',x')\⟩
%     &
%     = ⟨0|\^ϕ(t,x)\^ϕ(t',x')|0⟩
%     \∅ &
%     = ⟨0|[\^ϕ^{(-)}(t,x),\^ϕ^{(+)}(t',x')]|0⟩
%     \∅ &
%     = [\^ϕ^{(-)}(t,x),\^ϕ^{(+)}(t',x')]
% \end{align}
% \begin{align}
%     \NP{\^ϕ(t,x)\^ϕ(t',x')}
%     = \^ϕ(t,x)\^ϕ(t',x')-[\^ϕ^{(-)}(t,x),\^ϕ^{(+)}(t',x')]
% \end{align}

\subsection{
    演算子積展開
}
\begin{align}
    ϕ(x)ϕ(0)
    &
    =\f{C}{|x|^{d-2}}+\NP{ϕ(x)ϕ(0)}
    \∅ &
    =\f{C}{|x|^{d-2}}1
        + ϕ^2(0) + x⋅∂ϕ(0)ϕ(0)
        + \f{1}{2}x^μx^ν∂_μ∂_νϕ(0)ϕ(0) + ⋯
\end{align}
\begin{align}
    ϕ^2(x)ϕ^2(0)
    &
    = \f{2C^2}{|x|^{2(d-2)}}
        +\f{4C}{|x|^{d-2}}\NP{ϕ(x)ϕ(0)}
        +\NP{ϕ(x)^2ϕ(0)^2}
    \∅ &
    = \f{2C^2}{|x|^{2(d-2)}}1
        +\f{4C}{|x|^{d-2}}ϕ^2(0)
        + ϕ^4(0) + ⋯
\end{align}
\begin{align}
    ϕ^4(x)ϕ^2(0)
    = \f{12C^2}{|x|^{2(d-2)}}ϕ^2(0)
        +\f{8C}{|x|^{(d-2)}}ϕ^4(0)
        + ⋯
\end{align}
\begin{align}
    ϕ^4(x)ϕ^4(0)
    = \f{24C^4}{|x|^{4(d-2)}}1
        +\f{96C^3}{|x|^{3(d-2)}}ϕ^2(0)
        +\f{72C^2}{|x|^{2(d-2)}}ϕ^4(0)
        + ⋯
\end{align}

% まずは時空が1点のみの場合の自由場の理論を考える。場はひとつの実数$ϕ$で表され、作用は
% \begin{align}
%     S(ϕ) = -\f{1}{2}gϕ^2
% \end{align}
% とする。分配関数はGauss積分によって
% \begin{align}
%     Z = ∫_{-∞}^∞\e^{-gϕ^2/2} = \Q(\f{2π}{g})^{1/2}
%     \label{eq: gauss integral}
% \end{align}
% と計算される。相関関数は
% \begin{align}
%     \⟨ϕ^n\⟩ = \f{1}{Z}∫_{-∞}^∞ϕ^n\e^{-gϕ^2/2}
% \end{align}
% と定義される。$n$が奇数の場合、これは明らかに$0$である。
% また$n$が偶数の場合は(\ref{eq: gauss integral})を$g$で$n/2$回微分することで、
% \begin{align}
%     \⟨ϕ^n\⟩ = \Q(\f{2π}{g})^{-1/2}\Q(-2\dv{g})^{n/2}\Q(\f{2π}{g})^{1/2} = \f{(n-1)!!}{g^{n/2}}
% \end{align}
% を得る。ここで、$(n-1)!!$が$n$個の点を2個ずつのペアに分けるような場合の数に注目すると、この式を以下のように図形的に表現することができる。

% 次に、時空が2点からなる場合を考える。場は2つの実数$ϕ_1,ϕ_2$であり、作用は
% \begin{align}
%     S(ϕ_1,ϕ_2) = \f{1}{2}g_1ϕ_1^2 + \f{1}{2}g_2ϕ_2^2
% \end{align}
% とする。分配関数は
% \begin{align}
%     Z = ∫_{-∞}^∞\dd{ϕ_1}∫_{-∞}^∞\dd{ϕ_2}\e^{-S(ϕ_1,ϕ_2)}
%     = \Q(\f{2π}{g_1})^{1/2}\Q(\f{2π}{g_2})^{1/2}
% \end{align}
% である。相関関数$\⟨ϕ_1^nϕ_2^m\⟩$を計算すると
% \begin{align}
%     \⟨ϕ_1^nϕ_2^m\⟩
%     = \f{1}{Z}∫_{-∞}^∞\dd{ϕ_1}ϕ_1^n\e^{-g_1ϕ_1^2/2}∫_{-∞}^∞\dd{ϕ_2^m}\e^{-g_2ϕ_2^2/2}
%     = \⟨ϕ_1^n\⟩\⟨ϕ_2^m\⟩
% \end{align}
% となる。$\⟨ϕ_1^n\⟩,\⟨ϕ_2^m\⟩$は先ほど求めたものと同じである。
% ここで、またもや図形的な表現を考えてみよう。
% グラフの頂点として、$n$個の$ϕ_1$と$m$個の$ϕ_2$を考え、それらを2個ずつのペアにして線で結んでいく。
% $11$のペアには$1/g_1$を、$22$のペアには$1/g_2$を、$12$のペアと余った頂点には$0$を対応づける。
% これらの積をとって、あらゆる線の結び方について足しあげればよい。

% ここで、$ϕ_1,ϕ_2$の線型結合によって新たな場$ϕ'_1,ϕ'_2$を定義すると、作用は対称行列$G_{ij}$によって
% \begin{align}
%     S(ϕ'_1,ϕ'_2) = \f{1}{2}ϕ'_i G_{ij} ϕ'_j,\␣
%     (G_{ij} = G_{ji})
% \end{align}
% と書ける。ここで、$ϕ_1,ϕ_2$に対する図形の線型結合によって$ϕ'_1,ϕ'_2$に対する図形を定義することで、$\⟨{ϕ'_1}^n{ϕ'_2}^m\⟩$の図形による計算方法が直ちに得られる。
% すなわち、$n$個の$ϕ'_1$と$m$個の$ϕ'_2$の間を線で結び、$ϕ'_i$と$ϕ'_j$の間の線に$\⟨ϕ'_iϕ'_j\⟩$を対応づければ良い。
% このような図形による計算はGauss積分において一般的であり、Wickの定理と呼ばれる。

% 次に、一気に飛躍して$d$次元Euclid空間上の自由場を考えよう。
% 作用は
% \begin{align}
%     S[ϕ] = \f{1}{2}∫\d^d{x}ϕ(x)(-Δ+m^2)ϕ(x)
% \end{align}
% とする。この場合にもWickの定理によるダイアグラム計算は有効である。
% 例えば、$\⟨ϕ(x)ϕ(y)ϕ(z)^4\⟩$という相関関数を考える。
% \footnote{
% この相関関数は$ϕ^4$理論
% \begin{align}
%     S'[ϕ] = ∫\d^d{x}\Q[\f{1}{2}ϕ(x)(-Δ+m^2)ϕ(x)+\f{1}{4!}λϕ(x)^4]
% \end{align}
% の2点関数
% \begin{align}
%     \⟨ϕ(x)ϕ(y)\⟩' = \⟨ϕ(x)ϕ(y)\exp(-\f{1}{4!}λ∫\d^d{z}ϕ(z)^4)\⟩
% \end{align}
% を$λ$に関する摂動展開したときに、1次の項として現れる。
% }
\end{document}