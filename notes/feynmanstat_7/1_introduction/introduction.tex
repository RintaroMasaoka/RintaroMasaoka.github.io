\documentclass[../main.tex]{subfiles}
\begin{document}
\title{ファインマン統計力学 7章}
\subtitle{スピン波}
\author{理学部物理学科3年 政岡凜太郎}
\date{2022年 9月 26日}
\frame{\titlepage}
\begin{frame}{目次}
    \tableofcontents
\end{frame}
\section{導入}
\begin{frame}{本日の目標}
    \begin{itemize}
        \item 磁性体の低温での振る舞いを,スピン波の概念を使って定性的に
        理解する.
        \item 1次元Heisenberg模型の厳密解の構成方法について知る.
    \end{itemize}
\end{frame}

\begin{frame}{Pauli行列}
    Pauli行列を
    \begin{align}
        \sigma_i^x = \mqty(0&1\\1&0), \quad
        \sigma_i^y = \mqty(0&-i\\i&0),\quad
        \sigma_i^z = \mqty(1&0\\0&-1)
    \end{align}
    と定義する.ただし,この行列は$\ket{\uparrow}_i,\ket{\downarrow}_i$を基底とする空間に作用するものである.
    また,
    \begin{align}
        \sigma_i^{\pm} = \sigma_i^x \pm i\sigma_i^y,
        \quad \sigma_i^+ = \mqty(0&2\\0&0),
        \quad \sigma_i^- = \mqty(0&0\\2&0)
    \end{align}
    とする.
    % スピンの直積を取る場合,$\ket{\uparrow}_1 \otimes \ket{\uparrow}_2$のように書くか,省略して$\ket{\uparrow\uparrow}$のように書く.
\end{frame}

\begin{frame}{Heisenberg模型}
    今回主に扱うハミルトニアンは,一般的には
    \begin{align}
        H =  \sum_{\expval{i,j}} J_{ij}\bm{\sigma}_i \cdot \bm{\sigma}_j
    \end{align}
    と書かれる.これをHeisenberg模型という.
    最も簡単な例としては,1次元で最近接相互作用のみを取り入れた場合,
    \begin{align}
        H_{XXX} = -J \sum_{n=1}^N \bm{\sigma}_n \cdot \bm{\sigma}_{n+1}
    \end{align}
    と書かれる.この場合も単にHeisenberg模型と呼んだり,あるいはXXX模型と呼んだりもする.
    $z$軸方向の相互作用の強さを変えたものは,XXZ模型と呼ばれ,
    \begin{align}
        H_{XXZ} = -J \sum_{n=1}^N (\sigma_n^x \sigma_{n+1}^z + \sigma_n^y \sigma_{n+1}^y + \Delta \sigma_n^z \sigma_{n+1}^z)
    \end{align}
    と書かれる.
\end{frame}

\begin{frame}{スピン置換演算子}
    まず,2スピン系$H = -J\bm{\sigma}_1 \cdot \bm{\sigma}_2$を考える.
    \begin{align}
        \Pi^{1,2} = \frac{\bm{\sigma}_1\cdot \bm{\sigma}_2 + 1}{2}
        = \frac{\sigma_1^z\sigma_2^z + 1}{2}
        + \frac{\sigma_1^+\sigma_2^- + \sigma_1^-\sigma_2^+}{4}
    \end{align}
    と定義すると,$H = -J(2\Pi^{1,2}-1)$と書ける.ここで,
    \begin{align*}
        &
        \Pi^{1,2} \ket{\uparrow\uparrow} = \ket{\uparrow\uparrow},
        \quad
        \Pi^{1,2} \ket{\downarrow\downarrow} = \ket{\downarrow\downarrow},
        \\ &
        \Pi^{1,2} \ket{\uparrow\downarrow} = \ket{\downarrow\uparrow},
        \quad
        \Pi^{1,2} \ket{\downarrow\uparrow} = \ket{\uparrow \downarrow}
    \end{align*}
    となる.$\Pi^{1,2}$はスピン1とスピン2を入れ替える変換であり,スピン置換演算子という.
\end{frame}

\begin{frame}{古典近似}
    ハミルトニアンは
    \begin{align}
        H = -J \sum_{n=1}^N (2\Pi^{n,n+1} - 1) = -J \sum_{n=1}^N \bm{\sigma}_n\cdot \bm{\sigma}_{n+1}
    \end{align}
    である.Heisenbergの運動方程式
    \begin{align}
        \dot{\bm{\sigma}}_n = \frac{i}{\hbar}[H, \bm{\sigma}_n]
    \end{align}
    は,交換関係
    \begin{align}
        [\sigma_1^i\sigma_2^i, \sigma_1^j] = 2i\epsilon_{ijk}\sigma_1^k \sigma_2^i = 2i(\bm{\sigma}_1 \times \bm{\sigma}_2)^j
    \end{align}
    より,
    \begin{align}
        \hbar \dot{\bm{\sigma}}_n = 2J \bm{\sigma}_n \times (\bm{\sigma}_{n+1} + \bm{\sigma}_{n-1})
    \end{align}
    と書ける.
\end{frame}

\begin{frame}{}
    $\bm{\sigma}_n$を大きさ$1$の古典的なベクトルとみなす.$\bm{\sigma}_z \approx 1$のとき,方程式を線形化でき,
    \begin{align}
        \hbar \dot{\bm{\sigma}}_n = 4J \bm{\sigma}_n \times \bm{e}_z + \bm{e}_z \times (\bm{\sigma}_{n+1} + \bm{\sigma}_{n-1})
        \label{classical Heisenberg eq}
    \end{align}
    と近似できる.$\bm{\sigma}_{n+1}+\bm{\sigma}_{n-1}= 2\bm{\sigma}_n$と近似すると,$\bm{\sigma}_n$は単に$z$軸まわりのラーモア歳差運動をする.
    次に,
    \begin{align}
        \sigma_n^x \approx c\sin\omega t e^{ink}
        \\
        \sigma_n^y \approx c\cos\omega t e^{ink}
    \end{align}
    という解を仮定する.これを(\ref{classical Heisenberg eq})に代入すると,
    \begin{align}
        \hbar\omega = 4J(1- \cos k)
    \end{align}
    を得る.
    ここで,波数の次元について注意しておく.今回扱うのは全て格子間隔を1とした格子であり,波数は無次元量になる.通常の$1/(\text{距離})$の次元を持つ波数にしたければ,格子間隔を$a$として,$k \to ka$と置き換えれば良い.
\end{frame}

% \begin{frame}{}
%     連続体近似を行って,$\bm{\sigma}_n$を$\bm{\sigma}(x)$で置き換えると,(\ref{classical Heisenberg eq})は
%     \begin{align}
%         \hbar \pdv{\bm{\sigma}}{t} = 2J \bm{\sigma} \times \pdv[2]{\bm{\sigma}}{x}
%     \end{align}
%     と書ける.時間・空間のスケールを取り直すことで,Heisenberg強磁性体方程式
%     \begin{align}
%         \pdv{\bm{S}}{t} = \bm{S} \times \pdv[2]{\bm{S}}{x},
%         \quad
%         |\bm{S}|^2 = 1
%     \end{align}
%     を得る.この方程式は$N$-ソリトン解をもつことが知られている.1-ソリトン解を具体的に書くと,
%     \begin{align}
%         &
%         \bm{S} = (\sin\theta\cos\phi,\sin\theta\sin\phi,\cos\theta),
%         \\ &
%         \cos\theta = 1 -2b^2 \sech^2(b\sqrt{\Omega}(x-vt-x_0)),
%         \\ &\nonumber
%         \phi = \Omega t + \phi_0 + \frac{1}{2}v(x-x_0-vt)
%         \\ & \qquad
%         +\arctan\qty[\qty(\frac{b^2}{1-b^2})^{1/2}\tanh(b\sqrt{\Omega}(x-vt-x_0))].
%     \end{align}
% \end{frame}
\end{document}