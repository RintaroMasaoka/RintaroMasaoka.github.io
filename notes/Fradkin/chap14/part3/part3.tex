\documentclass[8pt,unicode,xcolor=svgnames]{beamer}

% beamer settings %
\usepackage{luatexja}
\usepackage[ipaex]{luatexja-preset}% IPAex font
\renewcommand{\kanjifamilydefault}{\gtdefault}
\usetheme[
    outer/progressbar=foot,
    % outer/numbering=none
]{metropolis}
\usefonttheme{professionalfonts}
\setbeamercolor{background canvas}{bg=white}
\setbeamercolor{frametitle}{fg=white,bg=Teal}
\setbeamercolor{progress bar}{fg=Teal,bg=white!90!black}
\setbeamercolor{colorbox}{bg=Teal!10}
\setbeamercolor{block body}{bg=white!90!black}
\setbeamercolor{block title}{fg=white,bg=DarkGreen}
\setbeamercolor{block title example}{fg=white,bg=DarkSeaGreen}
\setbeamercolor{block title alerted}{fg=white,bg=DarkKhaki}
\setbeamercolor{itemize item}{fg=ForestGreen}
\makeatletter
\newcommand*{\currentname}{\@currentlabelname}
\setlength{\metropolis@titleseparator@linewidth}{2pt}
\setlength{\metropolis@progressonsectionpage@linewidth}{2pt}
\setlength{\metropolis@progressinheadfoot@linewidth}{2pt}
\makeatother

% packages %
\usepackage{amsmath,amssymb,mathtools,diffcoeff}
\usepackage[oldfont,exchangeupit]{unicommand}

% graphics %
\usepackage{graphicx,float,tikz}
\newcommand\Includegraphics[2][]{\vcenter{\hbox{\includegraphics[#1]{#2}}}}
\graphicspath{{../images/}}

% biblatex %
\usepackage[backend=biber]{biblatex}
\addbibresource{../ref/ref.bib}

% tcolorbox %
\usepackage{tcolorbox}
\tcbuselibrary{skins,theorems}
\tcbset{
    enhanced,
    frame hidden,
    sharp corners,
    colback=Teal!10,
    boxsep=0pt,
    fonttitle=\bfseries
}

% hyperref %
\usepackage{hyperref}
\hypersetup{
    unicode,
    setpagesize=false,
    bookmarksnumbered=true,
    bookmarksopen=true,
    colorlinks=true,
    linkcolor=DarkCyan!80!black,
    citecolor=Chocolate,
    urlcolor=Chocolate
}

\usepackage{subfiles}

% options %
\numberwithin{equation}{section}

\begin{document}

\section{14.5 Superconductors as topological fluids}
\begin{frame}{\currentname}
    BCS理論によると、spin-singlet superconductorの基底状態は以下のpair-field
    \begin{align}
        Δ(𝒙,𝒚) = ϵ_{σσ'}ψ_{σ}^†(𝒙)ψ_{σ'}^†(𝒚),␣
        Δ(𝒌) = ϵ_{σσ'}ψ_{σ}^†(𝒌)ψ_{σ'}^†(-𝒌)
    \end{align}
    がノンゼロの期待値を取る状態として特徴づけられる。
    pair-fieldは大域的な$\U(1)$変換$ψ_σ(𝒙) ↦ ℯ^{¡θ}ψ_{σ'}(𝒙)$に対して$Δ ↦ ℯ^{-2¡θ} Δ$となる。

    超伝導の基底状態は$\U(1)$対称性を自発的に破る状態であり、その考えのもとにGinzburg--Landau理論が展開される。
    しかし、厳密には\alert{超伝導における秩序変数の概念には問題がある。}
\end{frame}
\begin{frame}{\currentname}
    次にgauge invariantな秩序変数を考える。
    \begin{align}
        𝒪(𝒙,𝒚) = Δ(𝒙, 𝒚)\exp(¡∫\𝑑^2z 𝑨(𝒛)⋅𝑬_c(𝒛))
    \end{align}
    ここで
    \begin{align}
        ∇⋅𝑬_c(𝒛) = δ(𝒛-𝒙) + δ(𝒛-𝒚)
    \end{align}
    とする。
\end{frame}
\begin{frame}{\currentname}
    超伝導のスペクトラムには中性のfermionicな励起が存在することが知られている。
\end{frame}
\begin{frame}{\currentname}
    Coulombゲージでは
    \begin{align}
        \exp(-¡∫\𝑑^2z 𝑨(z) ⋅∇U_c(z)) = \exp(¡e∫\𝑑^2z ∇⋅𝑨(z) U_c(z)) = 1
    \end{align}
\end{frame}
\begin{frame}{\currentname}
    \begin{align}
        ℒ_{topo}[a_μ, b_μ] = ÷1{𝜋}ϵ^{μνλ}a_μ∂_νb_λ - a^μj_{qp}^μ - j^𝑣_μb^μ
    \end{align}
    \begin{align}
        K = \( 0&2\\2&0\)
    \end{align}
\end{frame}
\begin{frame}{\currentname}
    \begin{align}
        ℒ = (D_μϕ)^*D^μϕ + ÷1{4}F_{μν}² + V(ϕ^*ϕ) - A_μj^μ_{up}
    \end{align}
    \begin{align}
        D_μ = ∂_μ + 2¡A_μ
    \end{align}
    $V(ϕ^*ϕ)$はメキシカンハット型のポテンシャルとし、$ϕ = √{ρ_𝑠}ℯ^{¡φ}$に最小をもつとする。ここで$ρ_s$は$|Δ|$に比例する。
    渦の寄与を入れるために$𝑑φ$を$𝑑η + 𝛿χ$で置き換える。
    ただし$𝑑(𝑑η+𝛿χ) = Δχ$が渦密度を表すとする。
    すると
    \begin{align}
        ∮_{∂Σ}𝛿χ  = ∫_Σ Δχ = 2𝜋N_𝑣[Σ]
    \end{align}
    となる。
\end{frame}
\begin{frame}{\currentname}
    さらに$a ≔ ÷1{2} 𝛿χ$とする。
    vortex currentは
    \begin{align}
        j^𝑣 = ÷1{𝜋}𝑑a
    \end{align}
    によって定義する。
    スムースな場$η$の方はgauge場に``食べられて''、massiveな場に変わる (Anderson--Higgs機構):
    \begin{align}
        ℒ_{eff} = ÷1{4}F²_{μν} + ÷{m_s²}{2}(A_μ-a_μ)² - A_μj^μ
    \end{align}
    ここで$m_s$とLondon侵入長$λ_𝐿$の間の関係は$m_s = λ_𝐿^{-1}$で与えられる。
    \begin{align}
        Z[j_μ^{qp}, j_μ^𝑣]
        &
        =∫\𝒟{A_μ}\𝒟{a_μ}δ(÷1{𝜋}ϵ_{μνλ}∂^νa^λ - j_μ^𝑣)
        \exp(-∫\𝑑^3xℒ_{eff}[A_μ,a_μ,b_μ]) \∅
        &
        = ∫\𝒟{a_μ}\𝒟{b_μ}\exp(-∫\𝑑^3xℒ_{eff}[a_μ,b_μ])
    \end{align}
\end{frame}
\section{14.9 Topological superconductors}
\end{document}