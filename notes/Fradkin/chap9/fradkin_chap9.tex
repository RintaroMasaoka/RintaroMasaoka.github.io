\documentclass[8pt,unicode,xcolor=svgnames]{beamer}

% beamer settings %
\usepackage{luatexja}
\usepackage[ipaex]{luatexja-preset}% IPAex font
\renewcommand{\kanjifamilydefault}{\gtdefault}
\usetheme[
    outer/progressbar=foot
]{metropolis}
\usefonttheme{professionalfonts}
\usecolortheme{spruce}
\setbeamercolor{colorbox}{bg=Teal!10}
\setbeamercolor{block body}{bg=white!90!black}
\setbeamercolor{block title}{fg=white,bg=DarkGreen}
\setbeamercolor{block title example}{fg=white,bg=DarkSeaGreen}
\setbeamercolor{block title alerted}{fg=white,bg=DarkKhaki}
\setbeamercolor{itemize item}{fg=ForestGreen}
\usepackage{nameref}
\makeatletter
\newcommand*{\currentname}{\@currentlabelname}
\setlength{\metropolis@titleseparator@linewidth}{2pt}
\setlength{\metropolis@progressonsectionpage@linewidth}{2pt}
\setlength{\metropolis@progressinheadfoot@linewidth}{2pt}
\makeatother

% packages %
\usepackage{amsmath,amssymb,mathtools,physics}
\usepackage[math-style=ISO,bold-style=ISO]{unicode-math}

% graphics %
\usepackage{graphicx,float,tikz}
\newcommand\Includegraphics[2][]{\vcenter{\hbox{\includegraphics[#1]{#2}}}}
\graphicspath{{../images/}}

% biblatex %
\usepackage[backend=biber]{biblatex}
\addbibresource{../ref/ref.bib}

% tcolorbox %
\usepackage{tcolorbox}
\tcbuselibrary{skins,theorems}
\tcbset{
    enhanced,
    frame hidden,
    sharp corners,
    colback=Teal!10,
    boxsep=0pt,
    fonttitle=\bfseries
}

% hyperref %
\usepackage{hyperref}
\hypersetup{
    unicode,
    setpagesize=false,
    bookmarksnumbered=true,
    bookmarksopen=true,
    colorlinks=true,
    linkcolor=DarkCyan!80!black,
    citecolor=Chocolate,
    urlcolor=Chocolate
}

% macro %
\usepackage{shortcommands}
\usepackage{subfiles}
\DeclareMathOperator{\Int}{Int}

% options %
\numberwithin{equation}{section}

\begin{document}
\title{Quantum dimers and gauge theories}
\author{政岡凜太郎}
\frame{\titlepage}
\begin{frame}{目次}\tableofcontents\end{frame}

% \section{
%     ホモロジー・コホモロジー
% }
% \begin{frame}{チェイン}
%     グラフ\footnote{
%         正確にはCW複体というべき。
%     }$G$の頂点の集合を$𝒱(G)$、
%     辺の集合を$ℰ(G)$、
%     面の集合を$ℱ(G)$と書く。
%     より高次のセルは考えない。

%     $𝒱(G),ℰ(G),ℱ(G)$の要素を基底とする形式的なベクトル空間を考え、
%     それぞれ$C₀(G),C₁(G),C₂(G)$と書く。
%     $C_p(G)$の元を$p$-チェインと呼ぶ。
    
%     バウンダリー作用素$∂$は$p$-チェインに対して
%     その境界の$(p-1)$-チェインを返す線形写像で、
%     \begin{align}
%         ∂² = 0
%     \end{align}
%     を満たしている。
%     $C_p(G)$および$∂$の作用はグラフに関する全ての情報をもつ。
% \end{frame}
% \begin{frame}{コチェイン}
%     双対グラフ$G^*$を
%     \begin{align}
%         𝒱(G^*) = ℱ(G),␣ ℰ(G^*) = ℰ(G),␣ℱ(G^*) = 𝒱(G)
%     \end{align}
%     によって構成する。また
%     \begin{align}
%         C₀(G^*) = C₂(G),␣
%         C₁(G^*) = C₁(G),␣
%         C₂(G^*) = C₀(G),␣
%     \end{align}
%     と対応付ける。$C_p(G^*)$の元を$p$-コチェインと呼ぶ。
%     \footnote{
%         今の構成ではベクトル空間とその双対空間を同一視しているので、
%         チェインとコチェインは同じものである。
%         ただし少数の部分を除いて大体サイクルなものはチェインと呼び、
%         大体コサイクルなものをコチェインと呼ぶことにする。
%     }
    
%     コバウンダリー作用素$\~∂$を
%     \begin{align}
%         \~∂ = ∂^𝑇
%     \end{align}
%     によって定義する。(Stokesの定理)
    
%     $\~∂$は双対格子におけるバウンダリー作用素とみなせる。
%     明らかに$\~∂² = 0$が成り立つ。
% \end{frame}

% \begin{frame}{(コ)サイクル・(コ)バウンダリー・(コ)ホモロジー}
%     サイクルは
%     \begin{align}
%         ∂C = 0
%     \end{align}
%     を満たすチェインのことである。バウンダリーは
%     \begin{align}
%         C = ∂S
%     \end{align}
%     と書けるチェインのことである。
%     $∂² = 0$からバウンダリーはサイクルである。
%     サイクルにおいてバウンダリーの差異を無視した類をホモロジー類という。

%     コサイクルは
%     \begin{align}
%         \~∂a = 0
%     \end{align}
%     を満たすコチェインのことである。コバウンダリーは
%     \begin{align}
%         a = \~∂λ
%     \end{align}
%     と書けるコチェインのことである。
%     $\~∂² = 0$からコバウンダリーはコサイクルである。
%     コサイクルにおいてコバウンダリーの差異を無視した類をコホモロジー類という。
% \end{frame}

% \begin{frame}{正方格子の場合}
%     一般的に語ったが、正方格子しか使わないので、正方格子の場合を考える。
%     ベクトル空間の基底は
%     \begin{align}
%         {v_i},{e_{ij}},{f_{ijkl}}
%     \end{align}
%     である。
%     添字どうしの位置関係は文脈で読み取ってほしい。
%     また
%     \begin{align}
%         e_{ji} = -e_{ij},␣
%         f_{σ(i)σ(j)σ(k)σ(l)} = \sign(σ)f_{ijkl}
%     \end{align}
%     を課す。
%     $∂$の作用は
%     \begin{align}
%         ∂v_i = 0,␣ ∂e_{ij} = v_i - v_j,␣
%         ∂f_{ijkl} = e_{ij} + e_{jk} + e_{kl} + e_{li}
%     \end{align}
%     である。これを転置すると、
%     \begin{align}
%         \~∂v_i = e_{ij} + e_{ik} + e_{il} + e_{im},␣
%         \~∂e_{ij} = f_{ijkl} - f_{mnji},␣
%         \~∂f_{ijkl} = 0
%     \end{align}
%     のようになる。

%     正方格子における$1$-サイクルは、ご想像の通り。
%     $1$-コサイクルは双対格子での$1$-サイクルを元の格子に戻してきたものである。
%     もちろん$\~∂$のカーネルと考えてもいい。
% \end{frame}

% \begin{frame}{正方格子の場合}
%     次に成分表示の見方をする。
%     グラフ上の$0$-形式、$1$-形式、$2$-形式
%     \begin{align}
%         ϕ₀ = ∑_{v ∈ 𝒱} ϕ₀(v)v,␣
%         ϕ₁ = ∑_{e ∈ ℰ} ϕ₁(e)e,␣
%         ϕ₂ = ∑_{f ∈ ℱ} ϕ₂(f)f
%     \end{align}
%     を考える。これらに$\~∂$を作用させると
%     \begin{align}&
%         \~∂ϕ₀ = ∑_{e_{ij} ∈ ℰ} (ϕ₀(v_i) - ϕ₀(v_j)) e_{ij},\∅
%         &
%         \~∂ϕ₁ = ∑_{f_{ijkl} ∈ ℱ} 
%         (ϕ₁(e_{ij})+ϕ₁(e_{jk})+ϕ₁(e_{kl})+ϕ₁(e_{li})) f_{ijkl},\∅
%         &
%         \~∂ϕ₂ = 0
%     \end{align}
%     となって、ほとんど外微分である。
%     よって格子間隔を小さくする極限で
%     \begin{align}
%         \~∂ → 𝑑,␣ ∂ → δ = (-1)^p ★^{-1}𝑑★
%     \end{align}
%     となる。
% \end{frame}

% \begin{frame}{コホモロジーの構成}
%     コホモロジー群はホモロジー群の双対空間としても実現できる。

%     ホモロジー類の代表元としてサイクル$C$をとってくる。
%     内積を
%     \begin{align}
%         (v,v') = δ_{v,v'},␣
%         (e,e') = δ_{e,e'},␣
%         (f,f') = δ_{f,f'}
%     \end{align}
%     によって定める。
%     $C$とコチェイン$a$との内積$(a,C)$が$C$の取り方に依らないためには、
%     \begin{align}
%         (a,∂S) = (\~∂a,S) = 0
%     \end{align}
%     が必要である。よって$a$はコサイクルである。
%     一方$a$をコバウンダリー$\~∂λ$だけ変形すると、
%     \begin{align}
%         (a+\~∂λ,C) = (a,C) + (a,∂C) = (a,C)
%     \end{align}
%     となるから、$a$にコバウンダリーを足しても線形写像として同じものが得られる。

%     ホモロジー群の双対空間はコホモロジー群となる。
% \end{frame}
% \section{
%     量子ダイマー模型
% }

% \begin{frame}{\currentname}
%     ダイマー配位$D ∈ 𝒟(G)$とは、グラフ$G$の辺集合の部分集合で、
%     互いに共通の頂点を持たず、
%     かつ全ての頂点を覆うようなもののことである。
%     \footnote{
%         ダイマー配位が存在するためには頂点の総数$|𝒱(G)|$が偶数である必要がある。
%     }

%     ダイマー配位はグラフ$G$上の$ℤ₂$係数$1$-チェインであって
%     \begin{align}
%         ∂D = ∑_{v_i ∈ 𝒱(G)} v_i
%     \end{align}
%     となるものとしても定義できる。
%     ここで$𝒱(G)$はグラフ$G$の頂点集合である。
%     $v_i$との内積をとると、
%     \begin{align}
%         (v_i,∂D) = (\~∂v_i,D) = 1
%     \end{align}
%     となる。これは教科書の(9.2)式に対応する。

%     2つのダイマー配位$D₀,D₁ ∈ 𝒟(G)$に対して
%     \begin{align}
%         ∂(D₁-D₀) =  0
%     \end{align}
%     が成り立つ。
%     したがって、1つの基準$D₀$を決めてしまえば任意の$D ∈ 𝒟(G)$は$1$-サイクルとしても表せる。
% \end{frame}
% \begin{frame}{\currentname}
%     量子ダイマー模型のハミルトニアンは
%     \begin{align}
%         H_\𝚞{QDM} = H_\𝚞{res} + H_\𝚞{diag}
%     \end{align}
%     である。$H_\𝚞{res}$は共鳴項
%     \begin{align}
%         H_\𝚞{res} = \=J ∑_{\𝚞{plaquettes}}(
%             |{=}⟩⟨‖| + |‖⟩⟨{=}|
%         )
%     \end{align}
%     であり、$H_\𝚞{diag}$は対角項
%     \begin{align}
%         H_\𝚞{diag} = V ∑_{\𝚞{plaquettes}}(
%             |{=}⟩⟨{=}| + |‖⟩⟨‖|
%         )
%     \end{align}
%     である。
%     $|V| ≫ |J|$の場合、基底状態を求めることは容易である。
%     $V < 0$の場合、figure 8.12 (a) のように全てのダイマーが平行になる。
%     $V > 0$の場合、figure 8.12 (b) のようにダイマーが互い違いに並ぶ。
% \end{frame}
% \begin{frame}{\currentname}
%     $J/V=-1$をRokhsar - Kivelson (RK) pointといい、
%     この点で厳密な基底状態が構成できる。
%     具体的には
%     \begin{align}
%         |Ψ_{\𝚞{sRVB}}⟩ = ∑_{D ∈ 𝒟(G)} |D⟩
%     \end{align}
%     とする。つまり全てのダイマー配位を等しい重みで足し合わせた状態である。
%     これは$J/V=-1$の場合に基底状態となっている。
%     \begin{align}
%         H_\𝚞{QDM}|Ψ_\𝚞{sRVB}⟩ = 0
%     \end{align}
%     $|Ψ_\𝚞{sRVB}⟩$は規格化されておらず、
%     \begin{align}
%         ‖Ψ_\𝚞{sRVB}‖² = Z_\𝚞{dimer}
%     \end{align}
%     となる。$Z_\𝚞{dimer}$は古典ダイマー模型の分配関数である。
%     ダイマー基底について対角な演算子$𝒪$をもってくると
%     \begin{align}
%         ÷{⟨Ψ_\𝚞{sRVB}|𝒪|Ψ_\𝚞{sRVB}⟩}{‖Ψ_\𝚞{sRVB}‖²}
%         = ÷{1}{Z_\𝚞{dimer}}∑_{D ∈ 𝒟(G)}⟨D|𝒪|D⟩
%         = ⟨𝒪⟩_\𝚞{dimer}
%     \end{align}
%     となり、期待値が古典ダイマー模型の相関関数にマップできる。
% \end{frame}
% \begin{frame}{\currentname}
%     正方格子の古典ダイマー模型の相関関数では、2つの平行な辺の間の相関関数が
%     \begin{align}
%         G(R) ∝ ÷{1}{R²}
%     \end{align}
%     となるらしい。
%     これはcriticalな振る舞いになっている。

%     三角格子の場合、相関関数は指数関数的に減衰する。
% \end{frame}
% \section*{
%     格子$U(1)$ゲージ理論
% }
% \begin{frame}{\currentname}
%     グラフの各辺に$\U(1) ≅ S¹$上の自由粒子のHilbert空間を考え、
%     全ての積について直積をとる。
%     \begin{align}
%         ℋ = ⨂_{e ∈ ℰ} ℋ_e,␣
%         ℋ_e ≅ ℋ_{\U(1)}
%     \end{align}
%     このとき各辺の自由度は角度変数$0 ≤ A ≤ 2𝜋$か
%     、角運動量$l ∈ ℤ$で特徴づけられる。
%     $e$上の角度演算子を$A(e)$、角運動量演算子を$E(e)$と書く。
%     また$E(e)$の固有状態を$|l(e)⟩ ∈ ℋ_e$と書く。
    
%     $A(e), E(e)$は正準交換関係
%     \begin{align}
%         [A(e),E(e')] = ¡δ_{ee'}
%     \end{align}
%     を満たす。交換関係から明らかに
%     \begin{align}
%         ℯ^{-¡m\ad a}E = E + m
%         \label{eq: adjoint action of a}
%     \end{align}
%     が成り立つ。
%         $|0⟩$から一般の$|l⟩$は
%     \begin{align}
%         |l⟩ = ℯ^{¡ma}|0⟩
%     \end{align}
%     によって構成される。
%     これは$|l⟩$の波動関数表示だが
%     (\ref{eq: adjoint action of a})からもわかる。
% \end{frame}
% \begin{frame}{\currentname}
%     $A = ∑_{e ∈ ℰ} A(e)e$に対してゲージ変換は
%     \begin{align}
%         A → A + \~∂α
%     \end{align}
%     と定義される。
%     これを演算子として表現すると、
%     \begin{align}
%         ℯ^{¡(\~∂α,E)}
%         = ℯ^{¡(α,Q)},␣
%         Q(v) ≔ ∂E(v) = ∑_{e ∈ \~∂v} E(e)
%     \end{align}
%     となる。
%     ただし$E = ∑_{e ∈ ℰ}E(e)e,~ Q = ∑_{v ∈ 𝒱}Q(v)v$である。
%     $Q(v)$と交換する演算子はゲージ不変な物理量となる。
%     まず$[Q(v),E(e)] = 0$なので、$E(e)$はゲージ不変な物理量である。
%     電磁場$\~∂A$やWilsonループ
%     \begin{align}
%         W(Γ) = ℯ^{(A,Γ)},␣ (∂Γ = 0)
%     \end{align}
%     もゲージ不変である。
%     $Γ = ∂f$と取ったものを足し合わせてみると、
%     \begin{align}
%         ∑_f W(∂f)
%         &
%         = 1 + ∑_{f ∈ ℱ} (A,∂f) + ∑_{f ∈ ℱ}(A,∂f)² + ⋯ \∅
%         &
%         = 1 + (\~∂A,\~∂A) + ⋯ ≈ ℯ^{(\~∂A,\~∂A)}
%     \end{align}
%     となってMaxwell作用が出てくる。
% \end{frame}
% \begin{frame}{\currentname}
%     量子ダイマー模型が$U(1)$ゲージ理論として書けることを示そう。
%     まず、各辺の自由度を$l = 0,1$に限るために、
%     \begin{align}
%         H_\𝚞{dimer} = ÷{1}{2k}∑_e[(E(e)-÷{1}{2})²-÷{1}{4}]
%     \end{align}
%     を加えて$k → 0$とする。
%     ここで2部グラフを仮定して、
%     辺の向きは常に副格子$A$から副格子$B$への向きにとる。
%     次に、共鳴項は
%     \begin{align}
%         H_\𝚞{res} = 2\=J∑_v \cos(\~∂A(v))
%     \end{align}
%     と書ける。また対角項は
%     \begin{align}
%         H_\𝚞{diag} = -V∑_{f_{ijkl}}(
%             E(e_{ij})E(e_{kl})
%             + E(e_{jk})E(e_{li})
%         )
%     \end{align}
%     と書ける。ダイマー配位の拘束条件は
%     \begin{align}
%         ∂E = Q = ∑v_B - ∑v_A
%     \end{align}
%     となる。
%     $E$を電場とみなすならば、
%     この条件はstaggerdな背景電荷を表しているとみなせる。
% \end{frame}
% \section{
%     古典ダイマー模型
% }
% \begin{frame}{\currentname}
%     せっかく量子をやるのだから、古典ダイマー模型についても少し触れておく。

%     グラフ$G$上のダイマー配位$D$に対し、作用を
%     \begin{align}
%         ℯ^{-S[D,w]} = ∏_{e_{ij} ∈ D} w_{ij}
%     \end{align}
%     で定める。ここで$w$は辺上に定められた正値の重みである。
%     ダイマー模型の分配関数は
%     \begin{align}
%         Z[w] = ∑_{D ∈ 𝒟(G)}ℯ^{-S[D,w]}
%     \end{align}
%     で与えられる。
%     $w_{ij}$を行列と考えれば、
%     $Z[w]$はハフニアンと呼ばれる量に一致する。
%     \begin{align}
%         Z[w] = \haf w
%         = ÷{1}{𝒩} ∑_{σ ∈ S_{2n}}w_{σ(1)σ(2)}w_{σ(3)σ(4)}⋯w_{σ(2n-1)σ(2n)}
%     \end{align}
%     ここで$|𝒱(G)| = 2n$とおいた。
%     $𝒩$は数え上げの重複を除く因子であり、
%     分配関数において定数倍には興味がないので省略した。

%     ハフニアンはとても計算しにくい量なので、パフィアンに変換したくなってくる。
% \end{frame}
% \begin{frame}{\currentname}
%     辺$e_{ij} ∈ ℰ(G)$に対し、向き付け
%     \begin{align}
%        ε_{ij} = ±1,␣ ε_{ji} = -ε_{ij}
%     \end{align}
%     を考える。
%     これを用いて、
%     \begin{align}
%         a_{ij} ≔ ε_{ij}w_{ij}
%     \end{align}
%     と定義する。
%     $a_{ij}$は構成から反対称行列である。
%     向きのついたダイマー模型の分配関数は
%     \begin{align}
%         Z'[w,ε] = \Pf a
%         = ÷{1}{𝒩}∑_{σ ∈ S_{2n}}
%             \sign(σ)a_{σ(1)σ(2)}a_{σ(3)σ(4)}⋯a_{σ(2n-1)σ(2n)}
%     \end{align}
%     と表される。
%     \alert{$G$が平面グラフの場合、}$Z'[w,ε] = Z[w]$となるような向き付け$ε$が常に存在することが知られている。
%     すると、$(\Pf a)² = \det a$から分配関数の計算は$a$の固有値問題に帰着する。
%     あるいは、
%     \begin{align}
%         \Pf a = ∫\𝒟{χ} ℯ^{χ_ia_{ij}χ_j}
%     \end{align}
%     と書いて自由なマヨラナフェルミオンに帰着する。
% \end{frame}
% \begin{frame}{\currentname}
%     良い向き付け$ε$について、少し解像度を高めておこう。
%     ただし、具体的に構成する方法については省略する。
%     \footnote{
%         気になる人はFisher-Kasteleyn-Temperleyのアルゴリズムで検索してみて。
%     }
%     満たすべき条件は
%     \begin{align}
%         \sign(σ)ε_{σ(1)σ(2)}⋯ε_{σ(2n-1)σ(2n)} = \const
%     \end{align}
%     となることである。ここで、
%     \begin{align}&
%         σ'(1) = σ(2), σ'(2) = σ(3), …, σ'(2l) = σ(1)\\
%         &
%         σ'(i) = σ(i) ␣(i > 2l)
%     \end{align}
%     となるような新たな置換$σ'$を考えてみよう。
%     $\sign(σσ'^{-1}) = -1$から
%     \begin{align}
%         ε_{σ(1)σ(2)}ε_{σ(2)σ(3)}⋯ε_{σ(2l-1)σ(2l)}ε_{σ(2l)σ(1)} = -1
%     \end{align}
%     が成り立たなければならない。
%     そこで、グラフの任意の面$f ∈ ℱ$について
%     \begin{align}
%         ∏_{e ∈ ∂f} ε(e) = -1
%         \label{Z2 gauge invariance for classical dimer}
%     \end{align}
%     を課す。実は条件としてこれで十分である。
%     \footnote{
%         $g ≠ 0$の閉曲面上ではちょっと事情は複雑になる。
%     }
% \end{frame}
% \begin{frame}{\currentname}
%     次に同値な向き付けの概念を導入しよう。

%     行列$a$の$i$行成分を全て反転すると、
%     各々のダイマー配位の寄与は全て$-1$倍される。
%     したがって同じ統計力学系が得られる。

%     言い換えると、ある頂点に対して接する全ての辺の向きを反転させても系は不変である。
%     この操作は$ℤ₂$ゲージ変換になる。まず
%     \begin{align}
%         ε_{ij} = ℯ^{¡𝜋K_{ij}}
%     \end{align}
%     と書く。2つの向き付け$ε'_{ij},ε_{ij}$の比が
%     \begin{align}
%         ε'_{ij}/ε_{ij} = ℯ^{¡𝜋(K'_{ij}-K_{ij})}
%         = ℯ^{¡𝜋(δ_{ki} - δ_{kj})}
%     \end{align}
%     となるとき、分配関数が不変だと言っている。

%     ここで$δ_{ki}-δ_{kj}$はコバウンダリー$\~∂v_k$の成分表示である。
%     さらに、このようなコバウンダリーをいくつ足しても変わらないのだから、
%     $K-K₀$を$ℤ₂$コホモロジー類の元、あるいは$ℤ₂$ゲージ場とみなすことができる。
% \end{frame}
% \begin{frame}{\currentname}
%     同値でないゲージ場の数を数えてみよう。
%     \begin{itemize}
%         \item ゲージ場の自由度は$|ℰ(G)|$個
%         \item 満たすべき条件は$|ℱ(G)|$個
%         \item 同値なゲージ場への変換が$|𝒱(G)|-1$個
%     \end{itemize}
%     である。
%     ここですべての頂点について$ℤ₂$ゲージ変換を行うと元のゲージ場に戻ってくることに注意。
%     よって平面グラフでは
%     \begin{align}
%         2^{|ℰ(G)| - |ℱ(G)| - |𝒱(G)| + 1} = 2^0 = 1
%     \end{align}
%     個だけ同値でないゲージ場が存在する。
%     種数が$g ≠ 0$の閉曲面では非自明なゲージ場が可能なことも、予想できるだろう。
%     \footnote{
%         その場合も古典ダイマー模型との間に色々と面白い対応があるのだが、
%         スピン構造やArf不変量といった概念が出てきて、いまいち分かっていない。
%     }

%     ここらで撤退して、本筋に戻る。
% \end{frame}
\section{
    $ℤ₂$ゲージ理論
}
\begin{frame}{\currentname}
    ここからFradkinの内容。

    グラフ$G$を2次元正方格子とする。
    古典論では、各辺に2値を取る向きの自由度を考えた。
    これに対応する量子論の状態空間として、各辺$e$に対しスピン$1/2$のHilbert空間$ℋ_e$を考え、
    全ての辺についてテンソル積したもの考える。
    \begin{align}
        ℋ = ⨂_{e ∈ ℰ} ℋ_e,␣ ℋ_e ≅ \Span\{|0⟩,|1⟩\}
    \end{align}
    また$σ^z$の固有値$1$を右・上向きに、固有値$-1$を左・下向きに対応させる。
    ハミルトニアンを
    \begin{align}
        H = -g ∑_{e ∈ ℰ} σ^x(e)
            - ÷{1}{g}∑_{f ∈ ℱ} W(∂f)
            \label{eq: Hamiltonian of Z2 gauge theory}
    \end{align}
    と定義する。ここで$W(∂f)$はプラケット演算子
    \begin{align}
        W(∂f) ≔ ∏_{e ∈ ∂f} σ^z(e)
    \end{align}
    である。
\end{frame}
\begin{frame}{\currentname}
    頂点$α$に対し、バーテックス演算子を
    \begin{align}
        Q(v) ≔ \~W(\~∂v) ≔ ∏_{e ∈ \~∂v} σ^x(e)
    \end{align}
    と定義する。
    この演算子は$Q(v)² = 1$を満たし、
    コバウンダリー$\~∂v$に対する$ℤ₂$ゲージ変換を生成する。
    ゲージ不変な状態は
    \begin{align}
        Q(v)|\Phys⟩ = |\Phys⟩
    \end{align}
    を満たす。
    容易に
    \begin{align}
        [Q(v),Q(v')] = 0,␣ [Q(v),H] = 0
    \end{align}
    が示せるので、$\{Q(v)\},H$を同時対角化できる。
    よって任意のゲージ変換について不変な状態をとってこれる。
\end{frame}
\begin{frame}{\currentname}
    ゲージ不変な物理量は以下の通り。 3.はゲージ不変なのか?
    \begin{enumerate}
        \item サイクル$Γ$上のWilsonループ
        \begin{align}
            W(Γ) = ∏_{e ∈ Γ}σ^z(e)
        \end{align}
        \item 辺$e$上の電場 $σ^x(e)$
        \item 頂点$v$上の電荷。
        \begin{align}
            ∏_{e ∈ γ(v)} σ^z(e)
        \end{align}
        ただし$γ(v)$は$v$を端点にもつチェイン。
        もう一つの端点は無限遠点。
        \item 非自明なコサイクル$\~Γ$上の 't Hooft ループ
        \begin{align}
            \~W(\~Γ) = ∏_{e ∈ \~Γ} σ^x(e)
        \end{align}
        \item 面$f$上の磁荷
        \begin{align}
            τ^z(f) = ∏_{e ∈ \~γ(f)} σ^x(e)
        \end{align}
        ただし$\~γ(f)$は$f$を端点にもつコチェイン。もう一つの端点は無限遠点。
    \end{enumerate}
\end{frame}
\begin{frame}{\currentname}
    ハミルトニアン(\ref{eq: Hamiltonian of Z2 gauge theory})
    を双対格子の言葉に直そう。
    磁荷演算子$τ^z(f)$はWilsonループ$W(∂f)$と反交換する。
    \begin{align}
        {W(∂f),τ^z(f)} = 0
    \end{align}
    $W(∂f)² = 1$から、
    Wilsonループは双対格子上のPauli行列$τ^x(f) ≔ W(∂f)$とみなすことができる。
    また
    \begin{align}
        σ^x(e) = ∏_{f ∈ \~∂e}τ^z(f)
    \end{align}
    が成り立つ。よって、
    \begin{align}
        H = -g∑_{⟨f,f'⟩}τ^z(f)τ^z(f') - ÷{1}{g}∑_f τ^x(f)
    \end{align}
    となる。なんと、横磁場Ising模型になった。
    ということで、横磁場Ising模型は$(2+1)$次元$ℤ₂$ゲージ理論と双対の関係にある。
    $τ^z$は横磁場Ising模型の秩序変数だが、ゲージ理論側では磁荷に対応する。
\end{frame}
\section{
    $ℤ₂$閉じ込め相
}
\begin{frame}{\currentname}
    $g$が大きい場合には強結合展開(低温展開)を行える。
    まず基底状態は
    \begin{align}
        |𝐺_∞⟩ = ∏_{e ∈ ℰ}|σ^x(e) = +1⟩
    \end{align}
    となる。
    励起状態を作るためには有限のエネルギーが必要で、
    拘束条件から最小のエネルギーは$σ^x = -1$の辺がプラケットを構成する場合。
    このときのエネルギーは$8g$である。

    $g$が有限の場合、基底状態はループの重ね合わせで表される。
    $g → g_𝑐$とすると、ギャップは0に近づき、
    典型的なループの大きさが発散するため、
    強結合展開は破綻する。

    双対な横磁場Ising模型では、
    $g → ∞$はIsing相互作用が横磁場に比べて大きい極限なので、秩序相に対応する。
    これはモノポール$τ^z(f)$が凝縮した相とみなすことができる。
\end{frame}
\begin{frame}{\currentname}
    単一電荷が存在する状態は、
    \begin{align}
        Q(v₀)|ψ⟩ = -|ψ⟩,␣ Q(v)|ψ⟩ = |ψ⟩␣(v ≠ v₀)
    \end{align}
    で定義される。基底状態からこの状態を作ろうとすると、
    \begin{align}
        |ψ⟩ = ∏_{e ∈ γ(v₀)}σ^z(e) |𝐺⟩
    \end{align}
    とする必要がある。ここで$γ(v₀)$は$v₀$と無限遠点を端点にもつチェインである。
    このためには無限の$σ^x$をフリップしないといけないので、
    gappedな相では単一の電荷を作ることはできない。

    ただし、2つの電荷を用意してその間をチェインで繋げば有限のエネルギーで電荷のペアを作ることができる。$R$だけ離れた電荷の間のエネルギーは
    \begin{align}
        𝛥E(R)  = σR,␣ σ = 2g + O(1/g)
    \end{align}
    となる。$σ$を "string tension" と呼ぶ。$σ$は$g → g_𝑐$で0に近づく。

    これは$\SU(3)$ゲージ理論の強結合領域で電荷(色荷)が単独で存在できない事情と同じなので、
    これをもって$ℤ₂$閉じ込め相と呼ぶ。
\end{frame}
\begin{frame}{\currentname}
    次にWilsonループの期待値を考える。
    まず$g → ∞$では$⟨𝐺_∞|W(∂S)|𝐺_∞⟩ = 0$である。
    有限の$g$では$|𝐺⟩$はBrillouin-Wigner摂動論によって$1/g²$で展開できる。
    有限の寄与が出てくるのは$|S|$は面$S$の面積として、$(1/g²)^{|S|}$次以降であるから
    \begin{align}
        ⟨𝐺(g)|W(∂S)|𝐺(g)⟩ ∼ (÷{1}{g²})^{|S|}
        = ℯ^{-μ(g)|S|},␣
        μ(g) = \log(g²) + 𝒪(1)
    \end{align}
    となるだろう。
    
    かなり雑に議論してしまったので、ちゃんと考えてみる。
    まず
    \begin{align}
        H = g(E_∞ + H_∞ - ÷{1}{g²} ∑_f W(∂f)),␣
        H_∞ = ∑_{e ∈ ℰ}(1-σ^x(e))
    \end{align}
    とおく。ここで$gE_∞$は$g → ∞$での基底状態のエネルギーを表す。
    基底状態$|𝐺⟩$は以下の自己無撞着方程式を満たす。
    \begin{align}
        |𝐺⟩ = |𝐺_∞⟩ + ÷{1}{g²}
         ÷{1-|𝐺_∞⟩⟨𝐺_∞|}{H_∞ + ÷{1}{g²} ∑_f ⟨𝐺_∞|W(∂f)|𝐺⟩}
         ∑_f W(∂f)|𝐺⟩
     \end{align}
     である。
\end{frame}
\begin{frame}{\currentname}
    $g → ∞$で主要な項を取り出すと、
    \begin{align}
       ⟨𝐺|W(∂S)|𝐺⟩
       ≈ (÷{1}{g²})^{|S|}∑⟨𝐺|W(∂S)∏_{f_i ∈ S}÷{W(∂f_i)}{H_∞}|𝐺_∞⟩
    \end{align}
    と書ける。
    $∑$はWilsonループを掛けるあらゆる順序についての足し合わせである。
    これを平均場近似を使って評価しよう。
    $S$の中でランダムに面を塗りつぶして行って、$S' ⊂ S$を構成する。
    このとき$∂S'$の長さは
    \begin{align}
        |∂S'| ≈ 2|S|⋅2p(1-p),␣ p = ÷{|S'|}{|S|}
    \end{align}
    と評価できる。
    ここで$S$の内部に含まれる辺の総数がおおよそ$2|S|$であることを用いた。
    ここから
    \begin{align}
        ÷{1}{H_∞}∏_{f_i∈S'⊂S}W(∂f_i)|𝐺_∞⟩
        &
        ≈ ÷{1}{2|∂S'|}∏_{f_i∈S'⊂S}W(∂f_i)|𝐺_∞⟩\∅
        &
        = ÷{|S|}{8|S'|(|S|-|S'|)}
            ∏_{f_i∈S'⊂S}W(∂f_i)|𝐺_∞⟩
    \end{align}
    となる。
\end{frame}
\begin{frame}{\currentname}
    よって
    \begin{align}
        ⟨𝐺|W(∂S)|𝐺⟩
        &
        ≈ (÷{1}{g²})^{|S|}(|S|+1)! ∏_{n = 1}^{|S|}
            ÷{|S|}{8n(|S|-n)}\∅
            &
        = (÷{|S|}{8g²})^{|S|}÷{(|S|+1)!}{(|S|!)²}\∅
        &
        ∼  \exp(\log(|S|/8g²)|S| - |S|\log |S| + |S|)\∅
        &
        = \exp(-μ(g)|S|)
    \end{align}
    となる。
    ここで、$μ(g) = \log(8g²) - 1$である。
    \footnote{
        相転移点を$μ(g)=0$となる点とすれば、$1/g_𝑐² ∼ 8/ℯ = 2.94$と推定できる。
        これを数値計算の結果$1/g_𝑐² = 3.044$と比較すると、割といい線いっている。
    }

    粗っぽい評価だが、Wilsonループの期待値が閉じ込め相で面積則に従うことがわかった!

    疑問: 面積則は閉じ込め相に対して普遍的な性質か?
\end{frame}
\section{
    $ℤ₂$非閉じ込め相
}
\begin{frame}{\currentname}
    次に弱結合相を考える。
    $g → 0$の基底状態は全てのプラケット演算子$W(∂f)$に対して
    固有値$1$をもつ状態である。
    これは$ℤ₂$ゲージ場が平坦であると言い換えることができる。
    $W(∂f)$に対して対角化された基底は
    \begin{align}
        |ε⟩ = ⨂_{e ∈ ℰ}|σ^z(e) = ε(e)⟩
    \end{align}
    のように書ける。
    ここで$ε(e)$は$±1$を取る辺上の場である。

    ゲージ変換で移り合う状態を同一視するとき、ゲージ固定をすると便利である。
    \footnote{
        古典ダイマー模型におけるFisher-Kasteleyn-Temperleyのアルゴリズムとは、まさにゲージ固定を系統的に構成する方法である。
    }
    ここでは正方格子において$x$軸方向の辺で全て$σ^z = 1$とするゲージをとる。
    このとき基底状態は
    \begin{align}
        |𝐺₀⟩ = ⨂_{e ∈ ℰ}|σ^z(e) = 1⟩
    \end{align}
    となる。
\end{frame}

\begin{frame}{\currentname}
    ゲージ変換で移り合う基底を手で同一視するのではなく、
    そもそもゲージ不変な基底状態を構成することも可能である。
    この場合、
    $W(∂f)$だけでなく$\~W(\~∂v)$についても対角化するので、
    Toric code
    \begin{align}
        H = - ∑_{f ∈ ℱ}W(∂f) - ∑_{v ∈ 𝒱}\~W(\~∂v)
    \end{align}
    の基底状態を求めれば良い。(別に今はToricとは限定していないが。)
    これはKitaev状態
    \begin{align}
        |𝐺⟩_\𝚞{Kitaev} = ∑_{A ∼ A₀}|[A]⟩
    \end{align}
    として与えられる。
    ここで$A ∼ A₀$は$A = A₀ + \~∂Λ$を意味する。
    $|[A]⟩$はコサイクル$A = ∑_e A(e)e$に対して
    \begin{align}
        |[A]⟩ = ⨂_{e ∈ ℰ}|σ^z(e) = (-1)^{A(e)}⟩
    \end{align}
    として定義される。
    ゲージ変換はコホモロジー類の元に対する置換として作用するので、
    同じ重みで重ね合わせた$|𝐺⟩_\𝚞{Kitaev}$は任意のゲージ変換について不変になっている。
\end{frame}

\begin{frame}{\currentname}
    励起状態について考えよう。

    $|𝐺⟩_\𝚞{Kitaev}$にコチェイン上の演算子
    $∏_{e ∈ \~γ}σ^x(e)$を掛ければ、
    モノポール$τ^z(f)$の対を生成できる。
    $g=0$では励起エネルギーは常にモノポールと重なる$W(∂f)$に対して発生し、
    モノポール間の距離は関係ない。
    つまりモノポール間のポテンシャルは$0$である。

    次に$|𝐺⟩_\𝚞{Kitaev}$にチェイン上の演算子
    $∏_{e ∈ γ}σ^z(e)$を掛ければ、
    電荷の対を生成できる。
    $g=0$では電荷はゲージ対称性を破るだけで、エネルギーは$0$である。
    $g > 0$では基底状態に$g²$のオーダーの寄与が加わるため、
    $R$だけ離れた電荷の間のエネルギーは
    \begin{align}
        𝛥E(R) = 2E₀(g) + V(g,R)
    \end{align}
    と書ける。ただし
    \begin{align}
        E₀(g) ∝ g² + O(g⁴),␣
        V(g,R) ∼ A(g)ℯ^{-R/ξ_s(g)}
    \end{align}
    である。
    ただし、gappedな相なので$V(g,R)$の$R$依存性が指数関数になるとした。

    これは電荷が遮蔽されていると捉えることもできる。
    電荷の対を生成して無限遠まで離していく過程で有限のエネルギーしか必要でないので、
    弱結合相を非閉じ込め相と呼ぶ。
\end{frame}
\begin{frame}{\currentname}
    Wilsonループの期待値は$g=0$においては
    \begin{align}
        ⟨𝐺₀|W(∂S)|𝐺₀⟩ = 1
    \end{align}
    である。
    $g > 0$では基底状態は束縛されたモノポール対が希薄気体と考えられる。
    Wilsonループの期待値はWilsonループがモノポール対と交差するごとに$-1$倍される。よってモノポール対の密度を$ρ(g)$とすると
    \begin{align}
        ⟨G|W(∂S)|G⟩ ≈ (1-ρ(g))^{|∂S|} ≈ ℯ^{-ρ(g)|∂S|}
    \end{align}
    と考えられ、Wilsonループの期待値が周長則に従うことがわかる。

    これもちゃんと考えたいなら摂動論を使うのがいいだろう。
    基底状態は
    \begin{align}
        |𝐺⟩ = |𝐺₀⟩ + g²÷{1-|𝐺₀⟩⟨𝐺₀|}{H₀+g²∑_e ⟨𝐺₀|σ^x(e)|𝐺⟩} ∑_e σ^x(e)|𝐺⟩
    \end{align}
    満たす。
\end{frame}
\begin{frame}{\currentname}
    モノポール対が$n$個ある状態を$|n⟩$と書くと、
    モノポール対が希薄であるとき
    \begin{align}
        ÷{1}{H₀+g²∑_e ⟨𝐺₀|σ^x(e)|𝐺⟩}|n⟩
        ≈ ÷{1}{H₀}|n⟩ ≈ ÷{1}{2n}|n⟩
    \end{align}
    としてしまってよい。よって
    \begin{align}
        |𝐺⟩
        &
        ≈ |𝐺₀⟩ + ÷{g²}{2} ∑_e σ^x(e)|𝐺₀⟩
        + ÷{1}{2!} (÷{g²}{2})² ∑_{e,e'}σ^x(e)σ^x(e')
            |𝐺₀⟩
        + ⋯ \∅
        &
        = ∏_e \exp(÷{g²}{2}σ^x(e))|𝐺₀⟩
    \end{align}
    となり、
    \begin{align}
        ⟨𝐺|W(∂S)|𝐺⟩ ≈ ∏_{e ∉ ∂S}\cosh(g²) ∏_{e ∈ ∂S} 1
    \end{align}
    と計算できる。
    ただし、これは発散するので以下のように規格化して、周長則を得る。
    \begin{align}
        ÷{⟨𝐺|W(∂S)|𝐺⟩}{⟨𝐺|𝐺⟩} ≈ ∏_{e ∈ ∂S}÷{1}{\cosh(g²)} = ℯ^{-\log(\cosh(g²))|∂S|}
    \end{align}
\end{frame}

\section*{
    境界条件とトポロジー
}
\begin{frame}{\currentname}
    \begin{align}
        \~Q = ∏_{f ∈ ℱ} τ^x(f) = ∏_{e ∈ ∑_f ∂f} σ^z(e)
    \end{align}
    \begin{align}
        ⟨𝐺|\~Q|𝐺_∞⟩ = 0
    \end{align}
\end{frame}
\begin{frame}{\currentname}
    \begin{align}
        W(γ_i)\~W(\~γ_j) = (-1)^{(γ_i,\~γ_j)} \~W(\~γ_j)W(γ_i)
    \end{align}
    非自明な't Hooft ループの作用はラージゲージ変換とみなせる。
    このもとでゲージ場は

    $g=0$では基底状態は4重縮退している。
\end{frame}
\section{
    物質場を入れた$ℤ₂$ゲージ理論
}
\begin{frame}{\currentname}
    物質場として$ℤ₂$係数$0$-形式場$ψ(v)$を考える。
    ゲージ場$A(e)$と物質場$ψ(v)$に対するゲージ変換は
    \begin{align}
        A(e) → A(e) +\~∂Λ(e),␣
        ψ(v) → (-1)^{Λ(v)} ψ(v)
    \end{align}
    と定義される。物質場に対応するHilbert空間として
    \begin{align}
        ψ(v) = ±1 → |τ^z(v) = ±1⟩
    \end{align}
    を考える。するとゲージ変換は演算子
    \begin{align}
        Q(v) ≔ \~W(\~∂v)τ^x(v)
        = (∏_{e ∈ \~∂v}σ^x(e))τ^x(v)
    \end{align}
    として表される。そこで、ハミルトニアンを
    \begin{align}
        H =&- g ∑_e σ^x(e)
            - ÷{1}{g} ∑_f W(∂f)\∅
            &
            - ÷{1}{λ} ∑_v τ^x(v)
            - λ ∑_{⟨v,v'⟩} τ^z(v)σ^z(⟨v,v'⟩)τ^z(v')
    \end{align}
    とする。各項が$Q(v)$と可換なことに注意。
\end{frame}
\begin{frame}{\currentname}
    \begin{align}
        C(γ(v,v'))
        = τ^z(v)(∏_{e ∈ γ(v,v')σ^z(e)})τ^z(v')
    \end{align}
    \begin{align}
        H =&- g ∑_e σ^x(e)
            - ÷{1}{g} ∑_f W(∂f)\∅
            &
            - ÷{1}{λ} ∑_v \~W(\~∂v)
            - λ ∑_e σ^z(e)
    \end{align}
    \begin{align}
        λ ↔ ÷{1}{g}
    \end{align}
\end{frame}
\section{
    コンパクトQED
}
\begin{frame}{\currentname}
    \begin{align}
        H_\𝚞{CQED}
        = ÷{g}{2}∑_e E(e)² 
        - ÷{1}{g} ∑_f\cos(\~∂A(f))
    \end{align}
    \begin{align}
        U[α] = ℯ^{¡(α,Q)}
    \end{align}
    \begin{align}
        ∂E(v)|\Phys⟩ = 0
    \end{align}
\end{frame}
\end{document}