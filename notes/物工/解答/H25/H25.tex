\providecommand{\main}{../main}
\documentclass[\main/main.tex]{subfiles}
\graphicspath{{../images/}}
\begin{document}
\newpage
\section{2013年(平成25年)}
\subsection*{
  物理学I
}
\subsection*{
  第1問
}\subsubsection*{
  [1]
}
運動項は
\begin{align}
  ÷{1}{2}m₁\.𝒓₁² + ÷{1}{2}m₂\.𝒓₂²
  &
  ÷{1}{2}m₁(\.𝑹-÷{m₂}{m₁+m₂}\.𝒓)²+÷{1}{2}m₂(\.𝑹+÷{m₁}{m₁+m₂}\.𝒓)²\∅
  &
  = ÷{1}{2}(m₁+m₂)\.𝑹² + ÷{1}{2}÷{m₁m₂}{m₁+m₂}\.𝒓²
\end{align}
と変形できる。よって
\begin{align}
  M = m₁+m₂,␣ μ = ÷{m₁m₂}{m₁+m₂}
\end{align}
と定義すると、
\begin{align}
  ℒ = ÷{1}{2}M\.𝑹²+÷{1}{2}μ\.𝒓² - U(𝒓)
\end{align}
となる。
\subsubsection*{
  [2]
}
\begin{align}
  L &= ÷{1}{2}μ\.𝒓² - U(r)\∅
  &
  = ÷{1}{2}μ(\.r𝒆_r + r\.ϕ𝒆_ϕ)² - U(r)\∅
  &
  = ÷{1}{2}μ(\.r² + r²\.ϕ²) - U(r)
\end{align}
\subsubsection*{
  [3]
}
\begin{align}
  μ\"r &= μr\.ϕ² - \𝚍{U(r)}{r} \∅
  &
  = ÷{l²}{μr³} - \𝚍{U(r)}{r} \∅
  &
  = -\𝚍_r(U(r)+÷{1}{2}÷{l²}{μr²})
\end{align}
これは$U(r)$に遠心力ポテンシャルを加えた運動方程式になっている。
\subsubsection*{
  [4]
}
\begin{align}
  μ\.r\"r = \𝚍_t(÷{1}{2}μ\.r²) = -\𝚍_t(U(r)+÷{1}{2}÷{l²}{μr²})
\end{align}
から、
\begin{align}
  E = ÷{1}{2}μ\.r²+U(r)+÷{1}{2}÷{l²}{μr²}
\end{align}
は時間変化しない。
\subsubsection*{
  [5]
}
$𝒍$は$\.𝒓×𝒍$と垂直であり、$𝒓$とも垂直である。よって、
\begin{align}
  𝒍⋅𝑨 = 0
\end{align}
\begin{align}
  𝑨⋅𝒓 &= μ²\.𝒓²𝒓² - μ²(𝒓⋅\.𝒓)²-μkr \∅
  &
  = μ²r⁴\.ϕ² - μkr \∅
  &
  = l²-μkr = Ar\cosα
\end{align}
\begin{align}
  r = ÷{l²}{μk+A\cosα},␣
  ÷{1}{r} = ÷{μk}{l²}(1+÷{A}{μk}\cosα)
\end{align}
\newpage
\subsection*{
  第2問
}
\subsubsection*{
  [1.1]
}
Biot-Savartの法則から
\begin{align}
  B_z = ÷{μ₀}{4𝜋}∫_0^{2𝜋}÷{aI}{(a²+z²)^{3/2}}a\𝑑{θ}
\end{align}
となる。
よって、
\begin{align}
  B_z = ÷{μ₀I}{2}÷{a²}{(a²+z²)^{3/2}}
\end{align}
\subsubsection*{
  [1.2]
}
どの程度厳密にやるのかわからないが、$N$が十分大きいときは、
\begin{align}
  lB_z = μ₀NI,␣ B_z = ÷{μ₀NI}{l}
\end{align}
となる。
\subsubsection*{
  [1.3]
}
\begin{align}
  V = N⋅𝜋a²⋅\𝚍{B_z}{t} = ÷{𝜋μ₀N²a²}{l}\𝚍{I}{t}
\end{align}
\begin{align}
  L = ÷{𝜋μ₀N²a²}{l}
\end{align}
\subsubsection*{
  [2]
}
コンデンサーの軸方向を$z$軸とする。電場は、
\begin{align}
  𝑬 = ÷{v₀}{d}\sin(ωt)𝒆_z
\end{align}
と表される。
また$z$軸正方向から見て反時計回りの方向の磁場を$B(r,t)$とおく。
$r$は$z$軸からの距離である。
$B(r,t)$を半径$r$の円周上で積分すると、
\begin{align}
  2𝜋rB(r,t) = ε₀μ₀𝜋r²÷{ωv₀}{d}\cos(ωt)
\end{align}
となる。よって、
\begin{align}
  B(r,t) = ÷{ωε₀μ₀v₀}{2d} r\cos(ωt)
\end{align}
\subsubsection*{
  [3.1]
}
回路方程式は、
\begin{align}
  ÷{Q(t)}{C} = L\𝚍{I(t)}{t} = L\𝚍^2{Q(t)}{t}
\end{align}
となる。よって$Q,I$の時間変化は単振動となり、
\begin{align}
  Q(t) = CV\cos(ωt),␣ I(t) = -ωCV \sin(ωt),␣
  ω = ÷{1}{√{LC}}
\end{align}
で与えられる。
\subsubsection*{
  [3.2]
}
コノデンサーとソレノイドのエネルギーはそれぞれ
\begin{align}
  ÷{1}{2}CV²\cos²(ωt),␣
  ÷{1}{2}CV²\sin²(ωt)
\end{align}
となる。
全エネルギーは保存しており、
コンデンサーとソレノイドの間でエネルギーが交互に移動していることが分かる。
\subsubsection*{
  [3.3]
}
全エネルギーは$QV/2 → QV = CV²$に変化する。
またキャパシタンスが$C → C/2$と変化するので、電流の振動数が$√2$倍になる。
\newpage
\subsection*{
  物理学II
}
\subsection*{
  第1問
}
\subsubsection*{
  [1]
}
$σ_x$の固有ベクトルおよび固有値は、
\begin{align}
  ÷{1}{√2}⦅1\\±1⦆,␣ ± 1
\end{align}
$σ_y$の固有ベクトルおよび固有値は、
\begin{align}
  ÷{1}{√2}⦅1\\±¡⦆,␣ ± 1
\end{align}
となる。
\subsubsection*{
  [2]
}
\subsubsection*{
  [2.1]
}
$p_y = p_z = 0,~ 𝑬 = (0,0,E)$を代入すると、
\begin{align}
  H = ÷{p_x²}{2m} + H_\𝚞{SO}
  = ÷{p_x²}{2m} - ÷{γ}{ħ}p_xEσ_y
\end{align}
\subsubsection*{
  [2.2]
}
固有関数を
\begin{align}
  ψ_± = ℯ^{¡kx}|y±⟩
\end{align}
とおく。ただし、$|y±⟩$は$σ_y$の固有値$±1$の固有状態である。
エネルギー固有値は、
\begin{align}
  E_± = ÷{ħ²k²}{2m} ∓ γEk
  = ÷{ħ²}{2m}(k∓÷{mγE}{ħ²})² - ÷{mγ²E²}{2ħ²}
\end{align}
となる。
\subsubsection*{
  [2.3]
}
ハミルトニアンは、
\begin{align}
  H = ÷{p_x²}{2m} - ÷{γ}{ħ}p_xEσ_y + ÷{1}{2}gμ_𝐵Bσ_x
\end{align}
となる。
固有エネルギーは
\begin{align}
  ÷{ħ²k²}{2m} ± √{(γEk)²+(÷{gμ_𝐵B}{2})²}
\end{align}
となる。
ただし、ベクトル$𝒗$に対し$𝒗⋅𝝈$からの固有値が$±|𝒗|$となることを用いた。
設問[2.3]と比べると、準位交差していた部分が上下のバンドに分かれている。
\newpage
\subsection*{
  第2問
}
\subsubsection*{
  [1]
}
エネルギーが$E$であるような状態は、
\begin{align}
  E = ÷{1}{2}m(v_{x,1}²+v_{y,1}²+v_{x,2}²+v_{y,2}²)
\end{align}
で与えられる。
よって、ミクロカノニカル分布は4次元の速度の空間上での球面上の一様な分布となる。
その半径は、$√{2E/m}$で与えられる。
\subsubsection*{
  [2]
}
極座標を用いると、
\begin{align}
  S_{n+1}(r)
  &
  = ∫_{-r}^{r} S_n(r\sinθ)÷{\𝑑{(r\cosθ)}}{\sinθ} \∅
  &
  = ∫_{-r}^{r}\𝑑{q}÷{rS_n(√{r²-q²})}{√{r²-q²}}
\end{align}
となる。よって$S₁(r) = 2$からはじめれば、与えられた公式が導かれる。
\subsubsection*{
  [3]
}
\begin{align}
  ∫P(v₁)\𝑑{v₁}
  = ÷{1}{S_{2N}(r)}∫_{-r}^{r}\𝑑{v₁}÷{rS_{2N-1}(√{r²-v₁²})}{√{r²-v₁²}}
\end{align}
と書ける。ここで$r = √{2E/m}$である。$N=2$のとき、
\begin{align}
  P(v₁) ∝ ÷{rS₃(√{2E/m-v₁²})}{√{2E/m-v₁²}}
  ∝ √{÷{2E}{m}-v₁²}
\end{align}
\subsubsection*{
  [4]
}
\begin{align}
  P(v₁) ∝ ÷{rS_{2N-1}(√{2E/m-v₁²})}{√{2E/m-v₁²}}
  ∝ (÷{2Nε}{m}-v₁²)^{N-3/2}
\end{align}
\subsubsection*{
  [5]
}
\begin{align}
  P(v₁) ∝ (1-÷{mv₁²}{2Nε})^{N-3/2} → \exp(-÷{mv₁²}{2ε})
\end{align}
よって$N → ∞$でMaxwell速度分布になる。
\subsubsection*{
  [6]
}
設問[5]の結果を
\begin{align}
  P(v₁) ∝ \exp(-÷{mv₁²}{2𝘬T})
\end{align}
と比較すると、
\begin{align}
  T = ÷{ε}{𝘬}
\end{align}
となる。(1粒子あたりのエネルギーは$ε = 2⋅𝘬T/2$となり妥当。)
\newpage
\subsection*{
  第3問
}
\subsubsection*{
  [1]
}
\begin{align}
  D(𝒓,t) = ÷{4𝜋a²σ}{4𝜋r²} = σ(÷{a}{r})²
\end{align}
\subsubsection*{
  [2]
}
\begin{align}
  ∇×𝑯 = \∂{𝑫}{t}
\end{align}
\subsubsection*{
  [3]
}
\begin{align}
  ∫_{C}𝑯⋅\𝑑{𝒍} = ∫_S \∂{𝑫}{t} \𝑑{S}
\end{align}
\subsubsection*{
  [4]
}
$C$に囲まれる面の、原点Oから見た立体角は、
\begin{align}
  2𝜋(1-\cosθ)
\end{align}
で与えられる。よって、
\begin{align}
  𝛥Ψ = 4𝜋a²σ⋅𝛥{(÷{1-\cosθ}{2})}
  = 2𝜋a²σ\sinθ 𝛥θ
\end{align}
となる。
\begin{align}
  v𝛥t = 𝛥{(d\cotθ)} = -÷{d}{\sin²θ}𝛥θ
\end{align}
より、
\begin{align}
  𝛥Ψ = = -÷{2𝜋a²σv}{d}\sin³θ 𝛥t.
\end{align}
\subsubsection*{
  [5]
}
\begin{align}
  H = ÷{|𝛥Ψ/𝛥t|}{2𝜋d} = ÷{a²σv}{d²}\sin³θ
\end{align}
\subsubsection*{
  [6]
}
\begin{align}
  u(𝒓,t) = ÷{1}{2}μ₀H(𝒓,t)²
\end{align}
\subsubsection*{
  [7]
}
\begin{align}
  &
  ∫u(𝒓,t)\𝑑^3x \∅
  &
  = ÷{1}{2}μ₀a⁴σ²v²⋅2𝜋∫_a^∞ r²\𝑑{r} ∫_{-1}^1 \𝑑{\cosθ}
      ÷{1}{(r\sinθ)⁴}\sin⁶θ \∅
      &
  = 𝜋μ₀a⁴σ²v²∫_a^∞ ÷{\𝑑{r}}{r²}∫_{-1}^1 (1-c²)\𝑑{c}\∅
  &
  = ÷{4𝜋a³}{3}μ₀σ²v²
\end{align}
\subsubsection*{
  [8]
}
微小球の運動に伴う全エネルギーは、
\begin{align}
  ÷{1}{2}m₀v² + ÷{4𝜋a³}{3}μ₀σ²v²
  = ÷{1}{2}(m₀+÷{8𝜋a³μ₀σ²}{3})v²
\end{align}
となる。

物理的意味: 電荷の自己エネルギーが質量に転化されている。
もう少し説明すると、荷電していない半径$a$の球を$q$だけ帯電させるためにかかるエネルギーは
\begin{align}
  𝛥E = ∫_0^q ÷{q'}{4𝜋ε₀a}\𝑑{q'} = ÷{q²}{8𝜋ε₀a}
\end{align}
となる。よって
\begin{align}
  𝛥m = ÷{𝛥E}{c²} = ÷{μ₀q²}{8𝜋a} = 2𝜋a³μ₀σ
\end{align}
係数がずれているので、なにか見落としている気がする。
\newpage
\subsection*{
  第4問
}
\subsubsection*{
  [1]
}
波数の大きさが$k$以下になるような状態数は、
\begin{align}
  N ∝ k^d ∝ ω^d
\end{align}
となる。よって、
\begin{align}
  D(ω) = \𝚍{N}{ω} ∝ ω^{d-1}.
\end{align}
よって(a)3次元の場合は$p=2$、(b)2次元の場合は$p=1$となる。
\subsubsection*{
  [2]
}
フォノンによるエネルギーは
\begin{align}
  E &= ∫_0^∞÷{ħω}{ℯ^{ħω/𝘬T}-1}D(ω)\𝑑{ω}\∅
  &
  ∝ ∫_0^∞ ÷{ω^d\𝑑{ω}}{ℯ^{ħω/𝘬T}-1}\∅
  &
  ∝ T^{d+1}
\end{align}
となる。よって、
\begin{align}
  C = \∂{E}{T} ∝ T^d.
\end{align}
すなわち(a)3次元の場合$q=3$、(b)2次元の場合$q=2$。
\subsubsection*{
  [3]
}
この系は面内の方が弾性力が大きい。
弾性力が大きいほど$ω$も大きいので、(i)が面間、(ii)が面内。
\subsubsection*{
  [4]
}
$ω₀ ≪ ω_a$では全ての方向で分散関係が線形になる。
等エネルギー面は、
\begin{align}
  v_a²k_a² + v_b²(k_{b,1}²+k_{b,2}²) = ω₀² 
\end{align}
で与えられる。これは楕円体の表面である。
ただし面に垂直な波数成分を$k_a$、水平な波数成分を$k_{b,1},k_{b,2}$とし、
それぞれの方向の音速を$v_a,v_b$とおいた。

次に$ω_a ≪ ω₀ ≪ ω_b$では、面に垂直な方向のフォノンによるエネルギーは$ω_a$でほぼ一定になってしまう。よって等エネルギー面は
\begin{align}
  v_b²(k_{b,1}²+k_{b,2}²) = ω₀²-ω_a²
\end{align}
で与えられる。これは円筒である。
\subsubsection*{
  [5]
}
$ω₀ ≪ ω_a$では、等エネルギー面の面積は$ω²$に比例する。
よってこのとき
\begin{align}
  D(ω) ∝ ω²,␣ r = 2.
\end{align}
次に$ω_a ≪ ω₀ ≪ ω_b$では、等エネルギー面の面積は$ω$に比例する。
よってこのとき
\begin{align}
  D(ω) ∝ ω,␣ r = 1.
\end{align}
\subsubsection*{
  [6]
}
$𝘬T ≪ ħω_a$のとき、設問[2]の3次元の場合と同様に
\begin{align}
  C_V ∝ T³,␣ s = 3
\end{align}
が言える。
次に$ħω_a ≪ 𝘬T ≪ ħω_b$のとき、
面に垂直な方向のフォノンによる比熱は無視でき、設問[2]の2次元の結果から、
\begin{align}
  C_V ∝ T²,␣ s = 2
\end{align}
となる。
最後に$ħω_b ≪ 𝘬T$のとき、全ての振動モードの比熱が$𝘬$で一定になるので
\begin{align}
  C_V ∝ T⁰,␣ s = 0
\end{align}
となる。
\end{document}
