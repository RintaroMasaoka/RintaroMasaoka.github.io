\providecommand{\main}{../main}
\documentclass[\main/main.tex]{subfiles}
\graphicspath{{../images/}}
\begin{document}
\newpage
\section{2022年(令和4年)}
\subsection*{
    第1問
}
\subsubsection*{
    [1.1]
}
奇数に対し、
\begin{align}
    ψ_{2l+1}(x) = ÷{1}{√{a}} \cos(\÷{(2l+1)𝜋}{2a}x)
\end{align}
となる。
偶数に対し、
\begin{align}
    ψ_{2l}(x) = ÷{1}{√{a}} \sin(\÷{l𝜋}{a}x)
\end{align}
となる。ただし$l=1,2,…$である。
エネルギーは、
\begin{align}
    E_n = ÷{ħ²}{2m}(÷{n𝜋}{2a})²
\end{align}
\subsubsection*{
    [1.2]
}
設問[1.1]の結果から、$ψ_n(x)$は$\^P$の固有状態であり、固有値は
\begin{align}
    ψ_{2l+1}(x): 1,␣
    ψ_{2l}(x): -1
\end{align}
\subsubsection*{
    [2.1]
}
\begin{align}
    \^M(ψ₁(x)ψ₁(y)) = ψ₁(x)ψ₁(y),␣
    \^M(ψ₂(x)ψ₂(y)) = ψ₂(x)ψ₂(y),␣
\end{align}
より、$Ψ_{1,1}(x,y),Ψ_{2,2}(x,y)$は$\^M$の固有状態で、固有値は$1$。
\subsubsection*{
    [2.2]
}
\begin{align}
   |Ψ_±⟩ ≔ ÷{1}{√2}(Ψ_{2,1}(x,y)±Ψ_{1,2}(x,y))
\end{align}
は$\^M$の固有値$±1$の固有状態になっている。
\subsubsection*{
    [2.3]
}
\begin{align}
    \^M\^L\^M = -\^L
\end{align}
から、
\begin{align}
    ⟨Ψ_±|\^L|Ψ_±⟩ = -⟨Ψ_±|\^M\^L\^M|Ψ_±⟩
    = -⟨Ψ_±|\^L|Ψ_±⟩ = 0
\end{align}
\subsubsection*{
    [2.4]
}
\begin{align}
    \^C₄Ψ_{1,1}(x,y) = ψ₁(-y)ψ₁(x) = ψ₁(x)ψ₁(y)
\end{align}
から、$Ψ_{1,1}(x,y)$は$\^C₄$の固有状態で固有値は$1$。
次に
\begin{align}
    \^C₄Ψ_{2,2}(x,y) = ψ₂(-y)ψ₂(x) = -ψ₂(x)ψ₂(y)
\end{align}
から、$Ψ_{2,2}(x,y)$は$\^C₄$の固有状態で固有値は$-1$。
\subsubsection*{
    [2.5]
}
\begin{align}
    |Ψ'_±⟩ ≔ ÷{1}{√2}(Ψ_{2,1}(x,y)±¡Ψ_{1,2}(x,y))
\end{align}
に対し、
\begin{align}
    \^C₄|Ψ'_±⟩
    &
    = ÷{1}{√2}(Ψ_{2,1}(-y,x)±¡Ψ_{1,2}(-y,x)) \∅
    &
    = ÷{1}{√2}(±¡Ψ_{2,1}(x,y)-Ψ_{1,2}(x,y))
\end{align}
となる。よって$|Ψ'_±⟩$は$\^C₄$の固有値$±¡$の固有状態になっている。
\subsubsection*{
    [2.6]
}
\begin{align}
    ⟨Ψ'_±|\^L|Ψ'_±⟩
    &
    = ÷{±¡}{2}(
        ⟨Ψ_{2,1}|\^L|Ψ_{1,2}⟩
        -⟨Ψ_{1,2}|\^L|Ψ_{2,1}⟩
    )\∅
    &
    = ∓\Im ⟨Ψ_{2,1}|\^L|Ψ_{1,2}⟩\∅
    &
    = ∓2\Im (⟨ψ₂|x|ψ₁⟩⟨ψ₁|p|ψ₂⟩)
\end{align}
ここで、
\begin{align}
    ⟨ψ₂|x|ψ₁⟩
    &
    = ÷{1}{a}∫_{-a}^a x \sin(÷{𝜋x}{a})\cos(÷{𝜋x}{2a})\𝑑{x}\∅
    &
    = ÷{32}{9𝜋²}a,\\
    ⟨ψ₁|p|ψ₂⟩
    &
    = ÷{-¡ħ𝜋}{a²}∫_{-a}^a \cos(÷{𝜋x}{2a})\cos(÷{𝜋x}{a})\𝑑{x} \∅
    &
    = ÷{-4¡ħ}{3a}
\end{align}
から、
\begin{align}
    ⟨Ψ'_±|\^L|Ψ'_±⟩ = ±÷{256ħ}{27𝜋²}
\end{align}
\newpage
\subsection*{
    第2問   
}
\subsubsection*{
    [1]
}
\begin{align}
    x = L\sin θ,\␣
    y = a\cos Ωt + L\cos θ
\end{align}
\subsubsection*{
    [2]
}
\begin{align}
    &
    m\"x = - T\sin θ
    \∅ & 
    m\"y =  mg - T\cos θ
\end{align}
\subsubsection*{
    [3]
}
\begin{align}
    \"x = L(\"θ\cos θ -\.θ²\sin θ),␣
    \"y = - aΩ²\cos Ωt - L(\"θ\sin θ +\.θ²\cos θ)
\end{align}
\subsubsection*{
    [4]
}
問[1]の結果から、運動方程式は
\begin{align}
    &
    mL(\"θ\cos θ -\.θ²\sin θ) = - T\sin θ,
    \∅ & 
    - maΩ²\cos Ωt - mL(\"θ\sin θ +\.θ²\cos θ) = mg - T\cos θ
\end{align}
と書ける。
よって$θ$に関する運動方程式は、
% \begin{align}
%     mL(\"θ\cos²θ-\.θ²\sinθ\cosθ) 
%     + maΩ²\cosΩt\sinθ + mL(\"θ\sin²θ+\.θ²\sinθ\cosθ)
%     = -mg\sinθ.
% \end{align}
% 整理すると、
\begin{align}
    \"θ + ÷{aΩ²}{L}\cosΩt\sinθ = - ÷{g}{L}\sinθ
\end{align}
\subsubsection*{
    [5]
}
$θ$が微小なとき、$\sinθ ≈ θ$として、
\begin{align}
    \"θ + ÷{aΩ²}{L}θ\cosΩt = -÷{g}{L}θ.
\end{align}
となる。
$a = 0$のとき、
\begin{align}
    \"θ = -÷{g}{L}θ.
\end{align}
この解は$θ₀(0)=A,~\.θ₀(0)=0$のとき、
\begin{align}
    θ₀(t) = A\cos(√{÷{g}{L}}t)
\end{align}
で与えられる。固有角振動数は、
\begin{align}
    ω₀ = √{÷{g}{L}}.
\end{align}
\subsubsection*{
    [6]
}
運動方程式において、$a$の1次の係数は以下のようになる。
\begin{align}
    \"θ₁(t) + ÷{AΩ²}{L}\cosΩt\cosω₀t = -÷{g}{L}θ₁(t)
\end{align}
\subsubsection*{
    [7]
}
運動方程式は
\begin{align}
    -\"θ₁(t)-ω₀²θ₁(t) = ÷{AΩ²}{2L}(\cos[(Ω+ω₀)t]+\cos[(Ω-ω₀)t])
\end{align}
と書ける。これに
\begin{align}
    θ₁(t) = u₁\cos[(Ω+ω₀)t]+u₂\cos[(Ω-ω₀)t]
\end{align}
を代入すると、$\cos[(Ω±ω₀)t]$の係数から
\begin{align}
    [(Ω+ω₀)²-ω₀²]u₁ = ÷{AΩ²}{2L},␣
    [(Ω-ω₀)²-ω₀²]u₂ = ÷{AΩ²}{2L}
\end{align}
が成り立っていればよい。よって、
\begin{align}
    u₁ = ÷{AΩ}{2L(Ω+2ω₀)},␣
    u₂ = ÷{AΩ}{2L(Ω-2ω₀)}.
\end{align}
このような$u₁,u₂$が存在する条件は、
\begin{align}
    Ω ≠ ±2ω₀.
\end{align}
である。
$ω₀ = √{g/L}$から、
\begin{align}
    u₁ = ÷{AΩ}{2L(Ω+2√{g/L})},␣
    u₂ = ÷{AΩ}{2L(Ω-2√{g/L})}
\end{align}
\newpage
\subsection*{
    第3問
}
\subsubsection*{
    [1.1]
}
\begin{align}
    𝜋R²L(2𝜋m𝘬T)^{3/2} = h³L\~Z₁,␣
    \~Z₁ = ÷{𝜋R²}{h³}(2𝜋m𝘬T)^{3/2} = γR²(𝘬T)^{3/2}
\end{align}
\subsubsection*{
    [1.2]
}
\begin{align}
    Z_N = ÷{1}{N!}÷{(𝜋R²L)^N}{h^{3N}}(2𝜋m𝘬T)^{3N/2}
    = ÷{L^N}{N!}\~Z₁^N
\end{align}
Stirlingの公式から、
\begin{align}
    F ≈ -N𝘬T\ln(L\~Z₁) + N𝘬T(\ln N - 1) 
\end{align}
よって
\begin{align}
    \~F = -ν𝘬T(\ln\~Z₁-\ln ν + 1)
\end{align}
\subsubsection*{
    [1.3]
}
\begin{align}
    P = -÷{1}{2𝜋R}\∂{F}{R}
    = ÷{ν𝘬T}{2𝜋R\~Z₁(T,R)}\∂_R \~Z₁(T,R)
\end{align}
問[1.1]の結果から、
\begin{align}
    P = ÷{ν𝘬T}{𝜋R²}
\end{align}
\subsubsection*{
    [2.1]
}
\begin{align}
    &
    H₁ - ωM₁ \∅
    &
    = ÷{p_x²+p_y²+p_z²}{2m}-ω(xp_y-yp_x) \∅
    &
    = ÷{1}{2m}(p_x+mωy)²+÷{1}{2m}(p_y-mωx)²+÷{1}{2m}p_z²
    - ÷{mω²}{2}(x²+y²)
\end{align}
となる。この最小値は
\begin{align}
    -÷{mω²R²}{2}.
\end{align}
\subsubsection*{
    [2.2]
}
\begin{align}
    v_x(𝒓) = -ωy,␣v_y(𝒓) = ωx.
\end{align}
\subsubsection*{
    [2.3]
}
\begin{align}
    2𝜋∫_0^R \~r\exp(÷{mω²\~r²}{2𝘬T})\𝑑{\~r}
    = ÷{2𝜋𝘬T}{mω²}[\exp(÷{mω²R²}{2𝘬T})-1]
\end{align}
から
\begin{align}
    ÷{1}{h³\~Z₁}(2𝜋m𝘬T)^{3/2}⋅÷{2𝜋𝘬T}{mω²}[\exp(÷{mω²R²}{2𝘬T})-1]
    = 1
\end{align}
となる。よって、
\begin{align}
    \~Z₁(T,R,ω) = ÷{2γ(𝘬T)^{5/2}}{mω²}[\exp(÷{mω²R²}{2𝘬T})-1]
\end{align}
\subsubsection*{
    [2.4]
}
\begin{align}
    P &= ÷{ν𝘬T}{2𝜋R}\∂_R\ln\~Z₁(T,R,ω)\∅
    &
    = ÷{νmω²/2𝜋}{1-\exp(-mω²R²/2𝘬T)}
\end{align}
$T → 0$では
\begin{align}
    P = ÷{νmω²}{2𝜋}.
\end{align}
\subsubsection*{
    [2.5]
}
$√{x²+y²}=\~r$とする。
\begin{align}
    I &= mN∫\𝑑^3{p}∫_{-L/2}^{L/2}\𝑑{z}∫_0^R 2𝜋\𝑑{\~r} \~r³ρ(𝒓,𝒑)\∅
    &
    = mN⋅÷{𝘬T}{mω}\∂_ω\ln\~Z₁(T,R,ω)\∅
    &
    = mN⋅(-÷{2𝘬T}{mω²} + ÷{R²}{1-\exp(-mω²R²/2𝘬T)})\∅
    &
    = ÷{2Nε}{ω²(1-\exp(-ε/𝘬T))} - ÷{2N𝘬T}{ω²}
\end{align}
\subsubsection*{
    [2.6]
}
\begin{align}
    E = -N\∂_β \ln\~Z₁
    = ÷{5}{2}N𝘬T - ÷{Nε}{1-\exp(ε/𝘬T)}
\end{align}
また別解として、
\begin{align}
    E
    &
    = -÷{1}{2}I(T)ω² + ÷{3}{2}N𝘬T \∅
    &
    = ÷{5}{2}N𝘬T - ÷{Nε}{1-\exp(-ε/𝘬T)} 
\end{align}
と計算しても良い。比熱は
\begin{align}
    C_ω
    = ÷{5}{2}N𝘬-÷{Nε²}{𝘬T²}÷{\exp(-ε/𝘬T)}{(1-\exp(-ε/𝘬T))²}
\end{align}
となる。
高温極限では、
\begin{align}
    C_ω = ÷{5}{2}N𝘬-÷{Nε²}{𝘬T²}÷{1}{(ε/𝘬T)²}
    = ÷{5}{2}N𝘬 - N𝘬 = ÷{3}{2}N𝘬.
\end{align}
低温極限では、
\begin{align}
    C_ω = ÷{5}{2}N𝘬.
\end{align}
\newpage
\subsection*{
    第4問
}
\subsubsection*{
    [1.1]
}
外部電場に対し、
\begin{align}
    ∇⋅𝑬 = -∇²ϕ = -2(a+b+c) = 0
\end{align}
となる。よって$a+b+c=0$。
\subsubsection*{
    [1.2]
}
\begin{align}
    m\𝚍^2{𝒓(t)}{t} = -q∇ϕ(𝒓(t)) + q\𝚍{𝒓(t)}{t}×𝑩
\end{align}
\subsubsection*{
    [1.3]
}
$z$方向の運動方程式は、
\begin{align}
    m\"z = -2cqz
\end{align}
となる。この角振動数は、
\begin{align}
    ω_z = √{÷{2cq}{m}}.
\end{align}
\subsubsection*{
    [1.4]
}
$u = x+¡y$とおくと、運動方程式は
\begin{align}
    m\"u = qcu - ¡qB \.u
\end{align}
と書ける。
\subsubsection*{
    [1.5]
}
$xy$平面内の運動を解く。$u ∝ ℯ^{-¡ωt}$とおくと、
\begin{align}
    mω² - qBω + qc = 0
\end{align}
よって
\begin{align}
    ω = ÷{qB ± √{q²B²-4mqc}}{2m}
\end{align}
となる。束縛運動になるための条件は、
\begin{align}
    qB² - 4mc > 0.
\end{align}
\subsubsection*{
    [2.1]
}
\begin{align}
    m\𝚍^2{x}{t} = -2qαx\cosωt
\end{align}
$τ$についての方程式に直すと
\begin{align}
    \𝚍^2{x}{τ} = ÷{4}{ω²}\𝚍^2{x}{t} = -÷{8qα}{mω²}x\cos2τ
\end{align}
となる。よって
\begin{align}
    λ = ÷{4qα}{mω²}
\end{align}
\subsubsection*{
    [2.2]
}
\begin{align}
    x(τ) = ∑_{n=-∞}^{∞} C_n \cos[(2n+Ω)τ]
\end{align}
とおく。
\begin{align}
    2λ\cos2τ\cos[(2n+Ω)τ]
    = λ\cos[(2(n+1)+Ω)τ] + λ\cos[(2(n-1)+Ω)τ]
\end{align}
から、$C_n$が満たすべき漸化式は、
\begin{align}
    -(2n+Ω)²C_n + λ(C_{n-1}+C_{n+1}) = 0
\end{align}
である。
\subsubsection*{
    [2.3]
}
$0 < λ ≪ 1$のとき、$C_{-1},C₀,C₁$以外の寄与を無視すると、
\begin{align}
    C₁ = ÷{λ}{(Ω+2)²}C₀,␣
    C₀ = ÷{λ}{Ω²}(C_{-1}+C₁),␣
    C_{-1} = ÷{λ}{(Ω-2)²}C₀
\end{align}
となる。よって、
\begin{align}
    ÷{C_{±1}}{C₀} = ÷{λ}{(Ω±2)²}
\end{align}
となる。$Ω = δλ$から、
\begin{align}
    ÷{C_{±1}}{C₀} = ÷{λ}{4},␣ ε = ÷{1}{4}.
\end{align}
この結果を代入すると、
\begin{align}
    1 = ÷{λ}{Ω²}(C_{-1}+C₁) = ÷{λ²}{2Ω²}
\end{align}
となる。よって
\begin{align}
    Ω = ÷{1}{√2}λ,␣ δ = ÷{1}{√2}
\end{align}
となる。
\subsubsection*{
    [2.4]
}
問[2.3]の結果から、
\begin{align}
    x(τ) &
    = C₀(
        \cos÷{λτ}{√2}
        +÷{λ}{4}\cos[(2+÷{λ}{√2})t]
        +÷{λ}{4}\cos[(2-÷{λ}{√2})t]
    )\∅
    &
    = C₀(1+÷{λ}{2}\cos2τ)\cos(÷{λ}{√2}t)
\end{align}
\end{document}