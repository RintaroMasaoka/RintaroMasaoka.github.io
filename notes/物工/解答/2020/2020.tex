\providecommand{\main}{../main}
\documentclass[\main/main.tex]{subfiles}
\graphicspath{{../images/}}
\begin{document}
\newpage
\section{2019年(令和1年)}
\subsection*{
    第1問
}

\subsubsection*{
    [1.1]
}
簡単のため、$x_0 = x_{N+1} = 0$とおく。
運動方程式は
\begin{align}
    m\𝚍^2{t}x_i(t) = -k(x_i-x_{i+1})-k(x_i-x_{i-1})
    = -k(2x_i-x_{i+1}-x_{i-1})
\end{align}
となるので、
\begin{align}
    K = \𝐩(
        2k & -k &    &    &    \\
        -k & 2k & -k &    &    \\
           & -k & 2k &  ⋱ &    \\
           &    &  ⋱ &  ⋱ & -k \\
           &    &    & -k & 2k \\
    )
\end{align}
\subsubsection*{
    [1.2]
}
固有ベクトルは
\begin{align}
    (𝒖_l)_n = \÷{N_l}{2¡}\𝚙(ℯ^{¡q_ln}-ℯ^{-¡q_ln})
\end{align}
と書ける。
この式は、$(𝒖_l)_0 = (𝒖_l)_{N+1} = 0$を満たす。
これに$K$を作用させると、
\begin{align}
    K𝒖_l = 2k𝒖_l - k(ℯ^{¡q_l}+ℯ^{-¡q_l})𝒖_l
    = 2k(1-\cos q_l)
\end{align}
となる。
したがって、固有値は$2k\𝚙(1-\cos q_l)$である。
次に、規格化定数を求める。これは
\begin{align}
    ∑_{l=0}^{N}[(𝒖_l)_n]^2 = 1
\end{align}
となるように定めるべきなので、
\begin{align}
    N_l = \√{\÷{2}{N+1}}
\end{align}
となる。 ただし、ベクトル
\begin{align}
    |l⟩_n =\÷1{\√{N+1}}ℯ^{¡ q_l n},\␣
    |-l⟩_n =\÷1{\√{N+1}}ℯ^{-¡ q_l n}
\end{align}
がどちらも規格化されていることと、これらが直交する事実から、
\begin{align}
    \𝚕|\÷{|l⟩+|-l⟩}{\√2}\𝚛|² = 1
\end{align}
となることを用いた。
\subsubsection*{
    [1.3]
}
運動方程式
\begin{align}
    \𝚍^2{t}𝒙(t) = -\÷K{m}𝒙(t)
\end{align}
から、
\begin{align}
    ∑_l\𝚍^2{t}α_l(t)𝒖_l = -∑_lα_l\÷K{m}𝒖_l
    = -∑_l\÷{2k}{m}(1-\cos q_l)α_l𝒖_l
\end{align}
となる。
$𝒖_l$が完全系を張ることに注意して、
\begin{align}
    \𝚍^2{t}α_l(t) = -\÷{2k}{m}(1-\cos q_l)α_l(t)
\end{align}
を得る。よって
\begin{align}
    ω_l = \√{\÷{2k}{m}(1-\cos q_l)} 
    = 2\√{\÷k{m}}\𝚟|\sin\÷{q_l}{2}|
\end{align}
\subsubsection*{
    [1.4]
}
省略。masslessな線形分散が図示できればOK

\subsubsection*{
    [2.1]
}
釣り合いの式から、
\begin{align}
    𝒇 - K𝒙 = 0
\end{align}
となるので、
\begin{align}
    β_l - mω_l²α_l = 0,\␣
    \÷{α_l}{β_l} = \÷1{mω_l²}
\end{align}
\subsubsection*{
    [2.2]
}
$1/mω_l²$が最大となる$l$を求める。
\begin{align}
    mω_l² = 4k\sin²\÷{q_l}{2}
\end{align}
より、これは$l = 1,N$となるときに最小となる。
\begin{align}
    q_l = \÷{𝜋l}{N+1}
\end{align}
より、$N ≫ 1$では
\begin{align}
    \÷1{mω_1²} ≈ \÷1{4k}\𝚙(\÷{𝜋}{2(N+1)})^{-2}
    = \÷{(N+1)²}{𝜋²k}
\end{align}
である。

\subsubsection*{
    [3.1]
}
$x_n$についての運動方程式の右辺に$-k₀x_n$という項が追加されるので、
\begin{align}
    K → K + k₀I
\end{align}
となる。
固有値は全て$k₀$だけ増加し、固有ベクトルは不変。
\subsubsection*{
    [3.2]
}
\begin{align}
    mω_l² = 2k(1-\cos q_l) + k₀ = 4k\sin²\÷{q_l}2 + k₀
\end{align}
より、
\begin{align}
    ω_l = \√{\÷1{m}\𝚙(4\sin²\÷{q_l}2 + k₀)}
\end{align}
となる。
$N → ∞$において最小の$ω_l$は、$q_l → 0$によって得られるので、
\begin{align}
    ω_\𝚝{min} = \√{\÷{k₀}m}
\end{align}
\subsubsection*{
    [3.3]
}
省略。massiveな分散関係を図示すればOK
\newpage
\subsection*{
    第2問
}
\subsubsection*{
    [1]
}
Maxwell eq.に複素数表示の平面波解を代入すると、
\begin{align}
    𝒌×𝑬 = ω𝑩,\␣
    𝒌×𝑩 = -ωμ₀\~ε𝑬
\end{align}
が得られる。したがって、
\begin{align}
    𝒌×(𝒌×𝑬) = ω𝒌×𝑩 = -ω²μ₀\~ε𝑬
\end{align}
が得られる。指数関数部分を取り除くと、
\begin{align}
    𝒌×(𝒌×𝑬₀) = -ω²μ₀\~ε𝑬₀
\end{align}
を得る。
\subsubsection*{
    [2]
}
[1]の結果を整理すると、
\begin{align}
    𝒌×(𝒌×𝑬₀) = 𝒌(𝒌⋅𝑬₀) - 𝒌²𝑬₀
    = -ω²μ₀\~ε𝑬₀
\end{align}
したがって、
\begin{align}
    \~X = \𝐩(
        ω²μ₀ε₁-k_y²-k_z²  & k_xk_y            & k_xk_z \\
        k_xk_y            & ω²μ₀ε₁-k_x²-k_z²  & k_yk_z \\
        k_xk_z            & k_yk_z            & ω²μ₀ε₂-k_x²-k_y²
    )
\end{align}
と定義すれば、[1]の結果は$\~X𝑬₀ = 𝟎$と書ける。
\subsubsection*{
    [3]
}
$𝒌 = (0,k\sinθ,k\cosθ)^𝑇$を代入すると、
\begin{align}
    \~X =\𝐩(
        ω²μ₀ε₁ - k² & 0                & 0 \\
        0           & ω²μ₀ε₁-k²\cos²θ  & k²\cosθ\sinθ \\
        0           & k²\cosθ\sinθ     & ω²μ₀ε₂-k²\sin²θ
    )
\end{align}
となる。$\~X𝑬₀ = 0$が非自明な解を持つためには、$\det\~X = 0$が必要。
したがって、
\begin{align}
    ω²μ₀ε₁ - k² = 0
\end{align}
または、
\begin{align}
    &
    \𝚙(ω²μ₀ε₁ - k²\cos²θ)\𝚙(ω²μ₀ε₂ - k²\sin²θ)
    -k⁴\cos²θ\sin²θ
    \∅ & 
    = ω⁴μ₀²ε₁ε₂ - k²ω²μ₀(ε₁\sin²θ+ε₂\cos²θ)
    = 0
\end{align}
である。したがってそれぞれの場合に、
\begin{align}
    k₁ = ω\√{μ₀ε₁},\␣
    k₂ = ω\√{\÷{μ₀ε₁ε₂}{(ε₁\sin²θ+ε₂\cos²θ)}}
\end{align}
となる。
\subsubsection*{
    [4]
}
$k = k₁$に対しては明らかに$(1,0,0)^𝑇 ∈ \Ker\~X$だから、
$𝑬₀ = (E₀,0,0)^𝑇$とすればよい。
$k=k₂$に対しては、
\begin{align}
    \𝐩(
        ω²μ₀ε₁-k₂²\cos²θ  & k₂²\cosθ\sinθ \\
        k₂²\cosθ\sinθ     & ω²μ₀ε₂-k₂²\sin²θ
    )
\end{align}
のカーネルの元を見つければよい。
これを仮に$(1,a)^𝑇$とおくと、
\begin{align}
    (ω²μ₀ε₁-k₂²\cos²θ)+ak₂²\cosθ\sinθ = 0
\end{align}
よって、
\begin{align}
    a
    &
    = -\÷{ω²μ₀ε₁k₂^{-2}-\cos²θ}{\cosθ\sinθ}
    \∅ & 
    = -\÷{(ε₁/ε₂)\sin²θ + \cos²θ - \cos²θ}{\cosθ\sinθ}
    \∅ & 
    = -\÷{ε₁\sinθ}{ε₂\cosθ}
\end{align}
となる。規格化すると、
\begin{align}
    𝑬₀ = \÷{E₀}{\√{ε₁²\sin²θ+ε₂²\cos²θ}}
        \𝐩(0\\ε₂\cosθ\\-ε₁\sinθ)
\end{align}
\subsubsection*{
    [5]
}
電場が$x$成分のみを持つのは$k=k₁$の場合。
Poyntingベクトルは
\begin{align}
    𝑺 = 𝑬 × 𝑯
    = 𝑬 × (μ₀𝒌×𝑬)
    = μ₀𝑬²𝒌 - μ₀(𝑬⋅𝒌)𝑬
\end{align}
と書ける。
$k = k₁$の場合、$𝑬⋅𝒌 ∝ 𝑬₀⋅𝒌 = 0$であるから、
第2項が消えて$𝑺 ∝ 𝒌$となる。
つまり光線は入射後も直進する。
\subsubsection*{
    [6]
}
電場が$x$成分のみを持つのは$k=k₂$の場合。
このとき$𝒌 × 𝑬 ∝ 𝒆_x = (1,0,0)^𝑇$だから、
\begin{align}
    𝑺 = 𝑬 × (μ₀𝒌×𝑬) 
    ∝ 𝑬₀ × 𝒆_x
    = \𝐩(0\\ε₁\sinθ\\ε₂\cosθ)
\end{align}
となる。
% \begin{align}
%     𝑬₀²𝒌 = E₀²k\𝐩(0\\\sinθ\\\cosθ)
% \end{align}
% また、
% \begin{align}
%    (𝑬₀⋅𝒌)𝑬₀ = \÷{E₀²k(ε₂-ε₁)\cosθ\sinθ}{ε₁²\sin²θ+ε₂²\cos²θ}
%    \𝐩(0\\ε₂\cosθ\\-ε₁\sinθ)
% \end{align}
% となる。
% したがって、Poyntingベクトル(から指数関数部分を除いたもの)は
% \begin{align}
%     &
%     \÷{μ₀E₀²k}{ε₁²\sin²θ+ε₂²\cos²θ}
%    \𝐩(
%         0\\
%         (ε₁²\sin²θ+ε₂²\cos²θ)\sinθ - ε₂(ε₂-ε₁)\cos²θ\sinθ\\
%         (ε₁²\sin²θ+ε₂²\cos²θ)\cosθ + ε₁(ε₂-ε₁)\cosθ\sin²θ
%     )
%     \∅ & 
%     = \÷{μ₀E₀²k}{ε₁²\sin²θ+ε₂²\cos²θ}
%    \𝐩(
%         0\\
%         ε₁²\sin³θ + ε₂ε₁\cos²θ\sinθ\\
%         ε₂²\cos³θ + ε₁ε₂\cosθ\sin²θ
%     )
%     \∅ & 
%     = μ₀E₀²k
%     \𝐩(
%          0\\
%          ε₁\sinθ\\
%          ε₂\cosθ
%      )
% \end{align}
すなわち、
\begin{align}
    \tan(θ+α) = \÷{ε₁\sinθ}{ε₂\cosθ}
\end{align}
となる。
また、
\begin{align}
    \tan α
    &
    =\÷{\sin(θ+α)\cosθ-\cos(θ+α)\sinθ}{\cos(θ+α)\cosθ+\sin(θ+α)\sinθ}
    \∅ & 
    = \÷{\tan(θ+α)-\tanθ}{1+\tan(θ+α)\tanθ}
\end{align}
より、
\begin{align}
    \tan α
    = \÷{(ε₁/ε₂-1)\tanθ}{1+\tan²θ}
\end{align}
\subsubsection*{
    [7]
}
上下にずれたQが2つ重なって見える。
\newpage
\subsection*{
    第3問
}
\subsubsection*{
    [1]
}
\begin{align}
    Z^{(𝑔)}(V,β,N)
    &
    = \÷1{N!}\÷1{(2𝜋ħ)^{3N}}∫\𝑑x^{3N}\𝑑p^{3N}
        \exp(-∑_{i=1}^{3N}\÷{βp_i²}{2m})
    \∅ & 
    = \÷{V^N}{N!}\÷1{(2𝜋ħ)^{3N}}
        \𝚙(∫\𝑑p\exp(\÷{βp²}{2m}))^{3N}
    \∅ & 
    =\÷{V^N}{N!}\÷1{(2𝜋ħ)^{3N}}\𝚙(\÷{2m𝜋}β)^{3N/2}
    \∅ & 
    =\÷{V^N}{N!}\÷1{(2𝜋ħ)^{3N}}\𝚙(\÷{m}{2𝜋ħ²β})^{3N/2}
\end{align}
\subsubsection*{
    [2]
}
\begin{align}
    Z^{(𝑔)}_𝐺(V,β,μ)
    &
    = ∑_{N=0}^∞Z^{(𝑔)}(V,β,N)ℯ^{^{βμN}}
    \∅ & 
    = \exp(Vℯ^{βμ}\𝚙(\÷{m}{2𝜋ħ²β})^{3/2})
\end{align}
\subsubsection*{
    [3]
}
\begin{align}
    P(β,μ)
    = \÷1{β}\∂V\log Z^{(𝑔)}_𝐺
    =\𝚙(\÷{m}{2𝜋ħ²})^{3/2}ℯ^{βμ}β^{-5/2}
\end{align}
\subsubsection*{
    [4]
}
\begin{align}
    ξ^{(𝑎)}_𝐺
    = 1 + ℯ^{β(ε+μ)}
\end{align}
\subsubsection*{
    [5]
}
\begin{align}
    n_a = \÷{ℯ^{β(ε+μ)}}{1 + ℯ^{β(ε+μ)}}
    =\÷{1}{1 + ℯ^{-β(ε+μ)}}
\end{align}
\subsubsection*{
    [6]
}
\begin{align}
    ℯ^{-βμ} =\𝚙(\÷{m}{2𝜋ħ²})^{3/2} β^{-5/2}P^{-1}
\end{align}
より、
\begin{align}
    n_a
    &
    =\÷1{1+\𝚙(m/2𝜋ħ²)^{3/2}ℯ^{-βε}β^{-5/2}P^{-1}}
    \∅ & 
    =\÷1{1+\𝚙(m/2𝜋ħ²)^{3/2}ℯ^{-ε/\kB T}(\kB T)^{5/2}P^{-1}}
\end{align}
\subsubsection*{
    [7]
}
図示は省略。
$P$を$0$から上げていくと、$n_a$は$0$から単調増加して$1$に漸近する。
$T$を$0$から上げていくと、$n_a$は$1$から減少したあと、
極小値をとってまた$1$に近づいていく。
\newpage
\subsection*{
    第4問
}
\subsubsection*{
    [1]
}
\begin{align}
    \~H = ħω(\^a_x^†\^a_x +\^a_y^†\^a_y + 1)
\end{align}
\subsubsection*{
    [2]
}
\begin{align}
    E = ħω(n_x+n_y+1),\␣
    n_x,n_y ∈ ℤ_{≥0}
\end{align}
\subsubsection*{
    [3]
}
角運動量の次元を表す定数は$ħ$であるから、
\begin{align}
    \^l_z &
    = -\÷{¡ħ}2(\^a_x + \^a_x^†)(\^a_y - \^a_y^†)
        +\÷{¡ħ}2(\^a_y + \^a_y^†)(\^a_x - \^a_x^†)
    \∅ & 
    = ¡ħ(\^a_x\^a_y^†-\^a_x^†\^a_y)
\end{align}
と書ける。ここから
\begin{align}
    [\^H,\^l_z]
    = ¡ħ²ω(-\^a_x\^a_y^†+\^a_x\^a_y^†-\^a_x^†\^a_y+\^a_x^†\^a_y)
    = 0
\end{align}
となる。つまり$\^l_z$は保存量。
\subsubsection*{
    [4]
}
\begin{align}
    \^l_z\^A|l_z⟩
    = ([\^l_z,\^A]+\^A\^l_z)|l_z⟩
    = (α+l_z)\^A|l_z⟩
\end{align}
より、$\^A|l_z⟩≠0$ならば、
$\^A|l_z⟩$は固有値$l_z+α$を持つ$\^l_z$の固有状態となる。
\subsubsection*{
    [5]
}
\begin{align}
    [\^l_z, C\^a_x^† + D\^a_y^†]
    = ¡ħ(C\^a_y^†-D\^a_x^†)
    = α(C\^a_x^†+D\^a_y^†)
\end{align}
より、
\begin{align}
    αC = -¡ħD,\␣ αD = ¡ħC
\end{align}
となる。
したがって$α²C = -¡ħαD = ħ²C$となるから、$α = ±ħ$である。
$α = ħ$に対しては、
\begin{align}
    \^b₁^† = C(\^a_x^† + ¡\^a_y^†),\␣
    \^b₁ = C(\^a_x -¡\^a_y)
\end{align}
となる。さらに$|C|² + |D|² = 1$から
\begin{align}
    \^b₁^† = \÷1{\√2}(\^a_x^† +¡\^a_y^†),\␣
    \^b₁ = \÷1{\√2}(\^a_x -¡\^a_y)
\end{align}
となる。また$α=-ħ$に対して同様に計算すると、
\begin{align}
    \^b₂^† = \÷1{\√2}(\^a_x^† - ¡\^a_y^†),\␣
    \^b₂ = \÷1{\√2}(\^a_x + ¡\^a_y)
\end{align}
となる。
\subsubsection*{
    [6]
}
\begin{align}
    [\^b₁,\^b₁^†] = [\^b₂,\^b₂^†] = 1,\␣
    [\^b₁,\^b₂^†] = [\^b₂,\^b₁^†] = 0
\end{align}
\subsubsection*{
    [7]
}
\begin{align}
    \^H = ħω(\^b₁^†\^b₁+\^b₂^†\^b₂+1),\␣
    \^l_z = ħ(\^b₁^†\^b₁-\^b₂^†\^b₂)
\end{align}
\subsubsection*{
    [8]
}
$\^b₁^†\^b₁$の固有値を$m₁$、$\^b₂^†\^b₂$の固有値を$m₂$とすると、
$N$番目のエネルギー準位については$m₁+m₂ = N$となる。
$\^l_z$の固有値は
\begin{align}
    l_z = ħ(m₁-m₂)
\end{align}
であるから、$(m₁,m₂) = (N,0),(N-1,1),…,(0,N)$を代入して、
\begin{align}
    l_z = ħN,ħ(N-2),…,-ħN
\end{align}
となる。
\subsubsection*{
    [9]
}
エネルギー準位は$\^b₁^†\^b₁$の固有値を$m₁ ∈ ℤ_{≥0}$、
$\^b₂^†\^b₂$の固有値を$m₂ ∈ ℤ_{≥0}$、
スピン量子数を$s = ±1$として、
\begin{align}
    E = ħω(m₁+m₂+1)+ħλs(m₁-m₂)
\end{align}
と書ける。
\end{document}