\providecommand{\main}{../main}
\documentclass[\main/main.tex]{subfiles}
\graphicspath{{../images/}}
\begin{document}
\newpage
\section{2017年(平成29年)}
\subsection*{
  第1問
}
\subsubsection*{
  [1.1]
}
エネルギー保存
\begin{align}
  ÷{1}{2}mv² + mgR\cosθ = mgR
\end{align}
から、
\begin{align}
  v = √{2gR(1-\cosθ)}
\end{align}
\subsubsection*{
  [1.2]
}
動径方向の運動方程式を考えると、重力と遠心力のつりあいから、
\begin{align}
  mg\cosθ_{c1} = ÷{mv²}{R} = 2mg(1-\cosθ_{c1})
\end{align}
したがって、
\begin{align}
  \cosθ_{c1} = ÷{2}{3}
\end{align}
このとき、
\begin{align}
  v_{c1} = √{÷{2}{3}gR}
\end{align}
\subsubsection*{
  [2.1]
}
\begin{align}
  I = 2𝜋∫_{-1}^{1}\𝑑{\cosθ}∫_0^r {r'}²\𝑑{r'} ρ(r'\sinθ)²
\end{align}
\begin{align}
  ρ = ÷{3m}{4𝜋r³}
\end{align}
\begin{align}
  I &
  = ÷{3mr²}{2}∫_{-1}^{1}(1-\cos²θ)\𝑑{\cosθ}∫_0^1 x⁴\𝑑{x}\∅
  &
  = ÷{3mr²}{2}⋅÷{4}{3}⋅÷{1}{5}\∅
  &
  = ÷{2}{5}mr²
\end{align}
\subsubsection*{
  [2.2]
}
剛体球の回転角を$ϕ$とおく。
球全体が剛体に沿って$θ$回転し、
さらに球が滑らないことによる回転を考慮すると、
\begin{align}
  ϕ = θ + ÷{R}{r}θ
\end{align}
となる。
時間微分を取ると、$v = (R+r)\.θ$に注意して、
\begin{align}
  ω = (1+÷{R}{r})\.θ = ÷{v}{r}
\end{align}
を得る。

\subsubsection*{
  [2.3]
}
エネルギー保存則は、
\begin{align}
  ÷{1}{2}mv² + ÷{1}{2}Iω² + mg(R+r)\cosθ = mg(R+r).
\end{align}
これに[2.1],[2.2]の結果を代入すると、
\begin{align}
  ÷{1}{2}mv² + ÷{1}{5}mv²
  = ÷{7}{10}mv²
  = mg(R+r)(1-\cosθ)
\end{align}
よって、
\begin{align}
  v² = ÷{10}{7}g(R+r)(1-\cosθ)
\end{align}
である。
動径方向の運動方程式を考えると、重力と遠心力のつりあいから、
\begin{align}
  mg\cosθ_{c2} = ÷{mv²}{R+r} = ÷{10}{7}mg(1-\cosθ_{c2})
\end{align}
よって、
\begin{align}
  \cosθ_{c2} = ÷{10}{17}
\end{align}
となる。
\subsubsection*{
  [2.4]
}
回転運動にエネルギーが移動する分、剛体球の速度が小さくなり、
結果として遠心力が小さくなるので、$θ_{c2}$は$θ_{c1}$と異なる。
\subsubsection*{
  [3.1]
}
\begin{align}
  v₀ = ÷{P}{m},␣ ω₀ = ÷{P(h-r)}{I} = ÷{5}{2}÷{P(h-r)}{mr²}
\end{align}
\begin{align}
  ω₀ = ÷{v₀}{r}
\end{align}
\begin{align}
  ÷{5}{2}÷{P(h-r)}{mr²} = ÷{P}{mr}
\end{align}
% \begin{align}
%   ÷{5}{2}(h-r) = r
% \end{align}
\begin{align}
  h = ÷{7}{5}r
\end{align}
\subsubsection*{
  [3.2]
}
エネルギー保存則
\begin{align}
  ÷{1}{2}mv²+÷{1}{2}Iω²+mg(R+r)\cosθ
  = ÷{1}{2}mv₀²+÷{1}{2}Iω₀²+mg(R+r)
\end{align}
から、
\begin{align}
  v² = ÷{10}{7}g(R+r)(1-\cosθ) +  ÷{P²}{m²}
\end{align}
となる。また動径方向の運動方程式から、
\begin{align}
  m\cosθ_{c3} = ÷{mv_{c3}²}{R+r}
  = ÷{10}{7}mg(1-\cosθ_{c3}) +  ÷{P²}{m(R+r)}
\end{align}
となる。
よって、
\begin{align}
  ÷{17}{7}\cosθ_{c3} = ÷{10}{7} + ÷{P²/m²}{g(R+r)}
\end{align}
となり、
\begin{align}
  &
  \cosθ_{c3} = ÷{10}{17} + ÷{7}{17}÷{P²/m²}{g(R+r)}\\
  &
  v_{c3} = √{g(R+r)\cosθ_{c3}}
  = √{÷{10}{17}g(R+r)+÷{7}{17}÷{P²}{m²}}
\end{align}
が得られる。
\newpage
\subsection*{
  第2問
}
\subsubsection*{
  [1.1]
}
\begin{align}
  E(r) = \begin{cases}
    ÷{Q₀}{4𝜋ε₀r²} & a ≤ r ≤ b\\
    0 & r ≤ a \𝚚{or} b ≤ r
  \end{cases}
\end{align}
\begin{align}
  ϕ(r) = \begin{cases}
    ÷{Q₀}{4𝜋ε₀}(÷{1}{a}-÷{1}{b}) & r ≤ a \\
    ÷{Q₀}{4𝜋ε₀}(÷{1}{r}-÷{1}{b}) & a ≤ r ≤ b \\
    0 & b ≤ r
  \end{cases}
\end{align}
\subsubsection*{
  [1.2]
}
\begin{align}
  ϕ(b)-ϕ(a) = ÷{Q₀}{4𝜋ε₀}(÷{1}{a}-÷{1}{b})
\end{align}
より、
\begin{align}
  C = ÷{4𝜋ε₀ab}{b-a}
\end{align}
\subsubsection*{
  [1.3]
}
誘電体を入れた場合、静電容量は
\begin{align}
  C' = ÷{4𝜋εab}{b-a}
\end{align}
となる。エネルギーは、
\begin{align}
  ÷{Q₀²}{2C'} = ÷{Q₀²}{8𝜋ε}(÷{1}{a}-÷{1}{b})
\end{align}
\subsubsection*{
  [1.4]
}
2つのキャパシタ
\begin{align}
  C₁ = ÷{4𝜋ε₀a(b-d)}{b-d-a},␣
  C₂ = ÷{4𝜋εb(b-d)}{d}
\end{align}
を直列につなげると考えれば良い。よって、
\begin{align}
  ÷{1}{C} &= ÷{1}{C₁}+÷{1}{C₂}\∅
  &
  = ÷{1}{4𝜋ε₀}(÷{1}{a}-÷{1}{b-d})
  +÷{1}{4𝜋ε}(÷{1}{b-d}-÷{1}{b})\∅
  &
  = ÷{1}{4𝜋ε₀}(
    ÷{1}{a}-÷{1}{b}-÷{d}{b²}
    +÷{ε₀}{ε}÷{d}{b²}
  )
\end{align}
\begin{align}
  C &= ÷{4𝜋ε₀ab}{b-a}[1+(1-÷{ε₀}{ε})÷{ad}{b(b-a)}]\∅
  &
  = ÷{4𝜋ε₀ab}{b-a} + ÷{4𝜋ε₀(ε-ε₀)}{ε}÷{a²}{(b-a)²}d
\end{align}
\subsubsection*{
  [2.1]
}
\begin{align}
  j(r) = σE(r) = ÷{σQ₀}{4𝜋εr²}
\end{align}
電流は、
\begin{align}
  I = 4𝜋r²j = ÷{σQ₀}{ε}.
\end{align}
Joule熱は、
\begin{align}
  VI = ÷{σQ₀²}{4𝜋ε²}(÷{1}{a}-÷{1}{b}).
\end{align}
\subsubsection*{
  [2.2]
}
\begin{align}
  \𝚍{Q}{t} = -I = -÷{σQ₀}{ε}
\end{align}
を解いて、
\begin{align}
  Q(t) = Q₀ℯ^{-σt/ε}
\end{align}
となる。時刻$0$から$t$までに発生したJoule熱は、
\begin{align}
  W(t) &= ÷{σ}{4𝜋ε²}(÷{1}{a}-÷{1}{b})∫_0^t Q(t')²\𝑑{t'}\∅
  &
  = ÷{Q₀²}{8𝜋ε}(÷{1}{a}-÷{1}{b})(1-ℯ^{-2σt/ε})
\end{align}
\subsubsection*{
  [2.3]
}
$t → ∞$とすると$W(t)$は[1.3]で求めた静電エネルギーに漸近していく。
これは静電エネルギーが全てJoule熱として散逸していくことを意味する。

\subsubsection*{
  [3]
}
それぞれの媒質に流れる電流は、
\begin{align}
  j₁ = σ₁E₁(r₀) = ÷{σ₁Q₁}{4𝜋ε₁r₀²},␣
  j₂ = σ₂E₂(r₀) = ÷{σ₂Q₂}{4𝜋ε₂r₀²}
\end{align}
となる。
\begin{align}
  j₁ = j₂ = ÷{I}{4𝜋r₀²}
\end{align}
から
\begin{align}
  ÷{σ₁Q₁}{ε₁} = ÷{σ₂Q₂}{ε₂} = I
\end{align}
となるので、求める電荷の面密度は
\begin{align}
  ÷{Q₂-Q₁}{4𝜋r₀²} = (÷{ε₂}{σ₂}-÷{ε₁}{σ₁})÷{I}{4𝜋r₀²}.
\end{align}
\newpage
\subsection*{
  物理学II
}
\subsection*{
  第1問
}
\subsubsection*{
  [1]
}
\begin{align}
  -÷{ħ²}{2m}(ϕ'(+0)-ϕ'(-0)) = vϕ(0)
\end{align}
に$ϕ(x)=Nℯ^{-ξ|x|}$を代入すると、
\begin{align}
  ÷{ħ²}{2m}⋅2Nξ = vN
\end{align}
よって、
\begin{align}
  ξ = ÷{mv}{ħ²}
\end{align}
次に規格化条件から、
\begin{align}
  N²∫_{-∞}^{∞}ℯ^{-2ξ|x|}\𝑑{x}
  = 2N² ÷{1}{2ξ} = 1.
\end{align}
よって、
\begin{align}
  N = √ξ = √{÷{mv}{ħ²}}
\end{align}
$N = √ξ$。
次に$x ≠ 0$でのSchrödinger方程式から、
\begin{align}
  E = -÷{ħ²}{2m}ξ² = -÷{mv²}{2ħ²}
\end{align}
\subsubsection*{
  [2.1]
}
\begin{align}
  E(c₁,c₂) = ÷{⟨Φ_t|\^H|Φ_t⟩}{⟨Φ_t|Φ_t⟩}
  = ÷{A(c₁²+c₂²)+2Bc₁c₂}{c₁²+c₂²+2Sc₁c₂}
\end{align}
\subsubsection*{
  [2.2]
}
拘束条件として、$⟨Φ_t|Φ_t⟩=1$を課してもよい。
この場合、変分するべきは未定係数$E$を含めた
\begin{align}&
  A(c₁²+c₂²)+2Bc₁c₂-E(c₁²+c₂²+2Sc₁c₂)\∅
  &
  = ⦅c₁&c₂⦆⦅A-E&B-2ES\\B-2ES&A-E⦆⦅c₁\\c₂⦆
\end{align}
である。これを$c₁,c₂$で偏微分すると、
\begin{align}
   ⦅A&B\\B&A⦆⦅c₁\\c₂⦆ = E⦅1&S\\S&1⦆⦅c₁\\c₂⦆
\end{align}
を得る。
\subsubsection*{
  [2.3]
}
条件を以下のように書き換えられる。
\begin{align}
   ⦅A+B&0\\0&A-B⦆⦅c₁+c₂\\c₁-c₂⦆ = E⦅1+S&0\\0&1-S⦆⦅c₁\\c₂⦆
\end{align}
ここから
\begin{align}
  E = ÷{A+B}{1-S},␣ ÷{A-B}{1+S}
\end{align}
であり、それぞれのエネルギーに対応して
\begin{align}
  ⦅c₁\\c₂⦆ ∝ ⦅1\\1⦆,␣⦅1\\-1⦆
\end{align}
となる。よって波動関数は規格化定数を除いて
\begin{align}
  ϕ_s(x₁+R/2)ϕ_s(x₂-R/2) ± ϕ_s(x₁-R/2)ϕ_s(x₁-R/2)
\end{align}
となる。
\subsubsection*{
  [2.4]
}
\begin{align*}
  &
  (ϕ_s(x₁+R/2)ϕ_s(x₂-R/2) + ϕ_s(x₁-R/2)ϕ_s(x₁-R/2))(α(ω₁)β(ω₂)-β(ω₁)α(ω₂))\\
  &
  (ϕ_s(x₁+R/2)ϕ_s(x₂-R/2) - ϕ_s(x₁-R/2)ϕ_s(x₁-R/2))(α(ω₁)β(ω₂)-β(ω₁)α(ω₂))\\
  &
  (ϕ_s(x₁+R/2)ϕ_s(x₂-R/2) - ϕ_s(x₁-R/2)ϕ_s(x₁-R/2))α(ω₁)α(ω₂)\\
  &
  (ϕ_s(x₁+R/2)ϕ_s(x₂-R/2) - ϕ_s(x₁-R/2)ϕ_s(x₁-R/2))β(ω₁)β(ω₂)
\end{align*}
\newpage
\subsection*{
  第2問
}
\subsubsection*{
  [1.1]
}
\begin{align}
  W = ÷{M!}{N!(M-N)!}
\end{align}
となるので、
\begin{align}
  S = 𝘬\ln W
  &
  = 𝘬(M\ln M - N\ln N - (M-N)\ln(M-N))\∅
  &
  = 𝘬N\ln÷{M}{N} + 𝘬(M-N)\ln÷{M}{M-N}
\end{align}
\subsubsection*{
  [1.2]
}
\begin{align}
  p₀ &= -\∂{F₀}{V} = T\∂{S}{V} = ÷{T}{v}\∂{S}{M}\∅
  &
  = ÷{𝘬T}{v}(÷{N}{M}+\ln÷{M}{M-N}+÷{M-N}{M}-1)\∅
  &
  = ÷{𝘬T}{v}\ln÷{M}{M-N}\∅
  &
  = -÷{𝘬T}{v}\ln(1-÷{vN}{V})
\end{align}
これは$ϕ = vN/V ≪ 1$のときに
\begin{align}
  p₀ = ÷{𝘬NT}{V}+÷{𝘬vN²T}{2V²}+⋯
\end{align}
と展開でき、1次では$p_\id$と一致する。
\subsubsection*{
  [2.1]
}
\begin{align}
  F = U-TS =  -÷{1}{2}Mzαϕ² - 𝘬T(Mϕ\ln÷{1}{ϕ} + M(1-ϕ)\ln÷{1}{1-ϕ})
\end{align}
\begin{align}
  μ &= ÷{1}{M}\∂{F}{ϕ} = -αzϕ - 𝘬T(-\lnϕ-1+\ln(1-ϕ)+1)\∅
  &
  = -αzϕ + 𝘬T[\lnϕ-\ln(1-ϕ)]
\end{align}
\subsubsection*{
  [2.2]
}
\begin{align}
  μ'(ϕ) = -αz + 𝘬T[÷{1}{ϕ}+÷{1}{1-ϕ}]
  = -αz + ÷{𝘬T}{ϕ(1-ϕ)}
\end{align}
$μ'(ϕ) > 0$となる温度の条件は、
\begin{align}
  ϕ(1-ϕ) < ÷{𝘬T}{αz}
\end{align}
が任意の$ϕ$について成り立つこと。
よって、
\begin{align}
  ÷{𝘬T}{αz} > ÷{1}{4},␣
  T > ÷{αz}{4𝘬}
\end{align}
\subsubsection*{
  [2.3]
}
等しい$μ$を与える$ϕ$が3つ存在するとき、相共存が起こりうる。
すなわち$μ(ϕ₁)=μ(ϕ₂)$となる$ϕ₁,ϕ₂$が存在し、
$ϕ₁ < ϕ < ϕ₁$となる$ϕ$に対しては系が密度$ϕ₁,ϕ₂$の2つの領域に分かれる。
この$ϕ₁,ϕ₂$はMaxwellの等面積則によって決まる。
\newpage
\subsection*{
  第3問
}
\subsubsection*{
  [1.1]
}
\begin{align}
  &
  \𝚍{N₁}{t} = (A+BW(ω₀))N₂ - BW(ω₀)N₁ \\
  &
  \𝚍{N₂}{t} = BW(ω₀)N₁-(A+BW(ω₀))N₂
\end{align}
\subsubsection*{
  [1.2]
}
定常状態では$\𝚍*{N₁}{t} = \𝚍*{N₂}{t} = 0$から、
\begin{align}
  -(N₁-N₂)BW(ω₀) + AN₂ = 0
\end{align}
よって、
\begin{align}
  W(ω₀) = ÷{AN₂}{B(N₁-N₂)}
\end{align}
となる。また、
\begin{align}
  ÷{N₂}{N₁} = ÷{BW(ω₀)}{A+BW(ω₀)} < 1
\end{align}
\subsubsection*{
  [2.1]
}
一つのモードは波数$𝒌$で特徴づけられ、$𝒌$は
\begin{align}
  𝒌 = ÷{2𝜋}{d}𝒏,␣ 𝒏 ∈ ℤ³
\end{align}
と書ける。$ω = c|𝒌|$から、角周波数が$0$から$ω$までのモードの数$N$は、
偏向を考慮すると、
\begin{align}
  N = 2(÷{d}{2𝜋})³÷{4𝜋(ω/c)³}{3}
  = ÷{d³ω³}{3𝜋²c³}
\end{align}
\subsubsection*{
  [2.2]
}
単位角周波数あたり、単位体積あたりの電磁波のモード数は、
\begin{align}
  D(ω) = ÷{1}{d³}\𝚍{N}{ω} = ÷{ω²}{𝜋²c³}
\end{align}
よって、
\begin{align}
  W_\𝚞{eq}(ω) = ÷{ħω}{ℯ^{ħω/𝘬T}-1}D(ω)
  = ÷{ħω³}{𝜋²c³} ÷{1}{ℯ^{ħω/𝘬T}-1}
\end{align}
\subsubsection*{
  [2.3]
}
\begin{align}
  ÷{A}{B} = ÷{N₁-N₂}{N₂}W(ω₀)
  = (ℯ^{ħω₀/𝘬T}-1)W(ω₀)
\end{align}
\subsubsection*{
  [3.1]
}
\begin{align}
  &
  \𝚍{N₁}{t} = -(P+L)N₁ +(A+L)N₂ + (A'+P)N₃ \\
  &
  \𝚍{N₂}{t} = LN₁-(A+L)N₂+CN₃\\
  &
  \𝚍{N₃}{t} = PN₁ - (A'+P+C)N₃
\end{align}
\subsubsection*{
  [3.2]
}
\begin{align}
  ⦅
    -P-L& A+L&   A'+P \\
       L&-A-L&      C \\
       P&   0&-A'-P-C
  ⦆⦅N₁\\N₂\\N₃⦆ = 0
\end{align}
から、
\begin{align}
   (-C(P+L)-(A'+P)L)N₁+(A'+P+C)(A+L)N₂
\end{align}
よって、
\begin{align}
  ÷{N₂}{N₁} = ÷{(A'+P+C)L+CP}{(A'+P+C)(A+L)}
\end{align}
となる。$N₂/N₁ > 1$となるとき、
\begin{align}
  (A'+P+C)L+CP > (A'+P+C)(A+L)
\end{align}
これを整理すると、
\begin{align}
  P(C-A) > A(A'+C)
\end{align}
となる。よって$C>A$のとき、
$P$の臨界値$P_c$が存在し、
\begin{align}
  P_c = ÷{A(C+A')}{C-A}
\end{align}
となる。
\subsubsection*{
  [3.3]
}
反転分布の条件は、
\begin{align}
  W_p(ω_p) > ÷{A(C+A')}{B_{13}(C-A)}
\end{align}
である。短波長では
\begin{align}
  W_p(ω_p) ≈ ÷{ħω³}{𝜋c³}ℯ^{-ħω/𝘬T}
\end{align}
であり、これは$ω$が大きくなると指数関数的に減衰するため、条件を満たすのが難しくなる。
\newpage
\subsection*{
  第4問
}
\subsubsection*{
  [1.1]
}
$v = v_x + ¡v_y,~ E = E_x + ¡E_y$とおくと、
\begin{align}
  m(\𝚍_t + ÷{1}{τ})v = -eE + ¡eBv.
\end{align}
定常状態では
\begin{align}
  v = -÷{eEτ}{m} + ¡ω_cτv
\end{align}
となる。ただし$ω_c = eB/m$である。
よって、
% これは
% \begin{align}
%   \𝚍_t v &= (÷{¡eB}{m}-÷{1}{τ})v - ÷{eE}{m}\∅
%   &
%   = (¡ω_c-÷{1}{τ})(v + ÷{eEτ/m}{1-¡ω_cτ})
% \end{align}
% と書けるので、
% \begin{align}
%   v = -÷{eEτ/m}{1-¡ω_cτ} + C\exp(¡ω_ct-÷{t}{τ})
% \end{align}
\begin{align}
  J = -nev = ÷{ne²τ/m}{1+(ω_cτ)²}(1+¡ω_cτ)E
\end{align}
これを行列で表すと、
\begin{align}
  𝑱 = \~σ𝑬,␣
  \~σ = ÷{ne²τ/m}{1+(ω_cτ)²}⦅1&-ω_cτ\\ω_cτ&1⦆.
\end{align}
よって
\begin{align}
  σ₀ = ÷{ne²τ}{m}
\end{align}
\subsubsection*{
  [1.2]
}
$j_y = 0$のとき、
\begin{align}
  E = ÷{1-¡ω_cτ}{σ₀}j_x.
\end{align}
よって、
\begin{align}
  R_𝐻 = ÷{E_y}{j_xB}
  = -÷{ω_cτ}{σ₀B}
  = -÷{1}{ne},␣
  σ = σ₀ = ÷{ne²τ}{m}
\end{align}
\subsubsection*{
  [2.1]
}
正孔に対する運動方程式は定常状態の場合、
\begin{align}
  v = ÷{eEτ}{m} - ¡ω_cτv
\end{align}
と書ける。
よって伝導率は
\begin{align}
  ÷{ne²τ/m}{1-¡ω_cτ}+÷{pe²τ/m}{1+¡ω_cτ}
\end{align}
となる。行列で書くと、
\begin{align}
  \~σ = ÷{e²τ/m}{1+(ω_cτ)²}⦅n+p&-(n-p)ω_cτ\\(n-p)ω_cτ&n+p⦆.
\end{align}
\subsubsection*{
  [2.2]
}
\begin{align}
  E &= ÷{m(1+(ω_cτ)²)}{e²τ}⋅÷{1}{(n+p)+¡(n-p)ω_cτ}j_x\∅
  &
  = ÷{m(1+(ω_cτ)²)}{e²τ}⋅÷{(n+p)-¡(n-p)ω_cτ}{(n+p)²+(n-p)²(ω_cτ)²}j_x
\end{align}
\begin{align}
  R_𝐻 &=÷{E_y}{j_xB}
  = -÷{e}{mω_c}⋅÷{m(1+(ω_cτ)²)}{e²τ}
  ⋅÷{(n-p)ω_cτ}{(n+p)²+(n-p)²(ω_cτ)²}\∅
  &
  = ÷{1}{(n-p)e}⋅÷{1+(ω_cτ)²}{(÷{n+p}{n-p})²+(ω_cτ)²}
\end{align}
\begin{align}
  σ = ÷{j_x}{E_x} = ÷{(n+p)e²τ}{m}⋅÷{1+(÷{n-p}{n+p})²(ω_cτ)²}{1+(ω_cτ)²}
\end{align}
\subsubsection*{
  [2.3]
}
問[2.2]の結果から、
\begin{align}
  R_H^{(0)} = -÷{n-p}{e(n+p)²},␣
  R_H^{(∞)} = -÷{1}{e(n-p)}.
\end{align}
よって、
\begin{align}
  n-p = -÷{1}{eR_H^{(∞)}},␣
  (n+p)² = ÷{1}{e²R_H^{(0)}R_H^{(∞)}}
\end{align}
となるから、
\begin{align}
  n = ÷{1}{2e}(÷{1}{√{R_H^{(0)}R_H^{(∞)}}}-÷{1}{R_H^{(∞)}}),␣
  p = ÷{1}{2e}(÷{1}{√{R_H^{(0)}R_H^{(∞)}}}+÷{1}{R_H^{(∞)}})
\end{align}
\subsubsection*{
  [2.4]
}
$p/n = 0$の場合、
\begin{align}
  σ = ÷{j_x}{E_x} = ÷{ne²τ}{m}
\end{align}
で$σ$は一定。$p/n > 0$の場合
\begin{align}
  σ(ω_cτ) = ÷{(n+p)e²τ}{m}⋅÷{1+(÷{n-p}{n+p})²(ω_cτ)²}{1+(ω_cτ)²}
\end{align}
である。これは$ω_cτ$に対して単調減少し、
\begin{align}
  σ(0) = ÷{(n+p)e²τ}{m},␣
  σ(∞) = ÷{(n-p)²e²τ}{(n+p)m} 
\end{align}
となる。
\subsubsection*{
  [2.5]
}
$p=n$のとき、
\begin{align}
  σ(ω_cτ) = ÷{(n+p)e²τ}{m}⋅÷{1}{1+(ω_cτ)²}
\end{align}
から$ω_cτ → ∞$で$σ$は$0$に漸近する。
これは電子・正孔のサイクロトロン運動が激しくなることで、
電場と平行な方向への運動が妨げられることから理解できる。
\end{document}