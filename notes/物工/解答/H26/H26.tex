\providecommand{\main}{../main}
\documentclass[\main/main.tex]{subfiles}
\graphicspath{{../images/}}
\begin{document}
\newpage
\section{2014年(平成26年)}
\subsection*{
  物理学I
}
\subsection*{
  第1問
}
\subsubsection*{
  [1]
}
\begin{align}
  I &= ÷{3m}{4𝜋a³}⋅ 2𝜋∫_{-1}^{1}\𝑑{\cosθ} ∫_0^a \𝑑{r} r⁴\sin²θ \∅
  &
  = ÷{3m}{2a³}⋅÷{4}{3}⋅ ÷{a⁵}{5}\∅
  &
  = ÷{2}{5}ma²
\end{align}
\subsubsection*{
  [2]
}
接地点の速度がゼロになることから、
\begin{align}
  v' + aω' = 0
\end{align}
\subsubsection*{
  [3]
}
\begin{align}
  mv' = P,␣
  I(ω'-ω) = ÷{2}{5}ma²(ω'-ω) = aP
\end{align}
以上の条件から、
\begin{align}
  ÷{2}{5}ma(ω'-ω)-mv'
  = ÷{1}{5}ma(7ω'-2ω) = 0
\end{align}
よって、
\begin{align}
  ω' = ÷{2}{7}ω
\end{align}
\subsubsection*{
  [4]
}
跳ね返り前後の関係は
\begin{align}
  m(v_n-v_{n-1}) = -P_n,␣
  I(ω_n-ω_{n-1}) = -aP_n
\end{align}
である。1つ目の式の$a$倍から2つ目の式を引くと、
\begin{align}
  (Iω_n-amv_n) - (Iω_{n-1}-amv_{n-1}) = 0
\end{align}
を得る。よって
\begin{align}
  l ≔ Iω_n - amv_n
  \label{def of l}
\end{align}
は一定となる。
\subsubsection*{
  [5]
}
反跳が終わったとき、球がすべらずに転がったことから、
\begin{align}
  v_𝑓 + aω_𝑓 = 0.
\end{align}
これを(\ref{def of l})と連立すると、
\begin{align}
  l = Iω_𝑓 - amv_𝑓 = -(÷{I}{a}+am)v_f
  = -÷{7}{5}mav_𝑓
\end{align}
よって、
\begin{align}
  v_𝑓 = -÷{5l}{7ma}
\end{align}
と表せる。
また、$v_𝑓 = 0$となるためには、
\begin{align}
  l = Iω₀ - amv₀ = 0
\end{align}
となればよい。整理すると、
\begin{align}
  v₀ = ÷{2}{5}aω₀
\end{align}
となる。

\newpage
\subsection*{
  第2問
}
\subsubsection*{
  [1]
}
円筒状の領域でGaussの定理を用いると、
\begin{align}
  2𝜋rlE(r) = ÷{λl}{ε₀},␣
  E(r) = ÷{λ}{2𝜋ε₀r}
\end{align}
となる。よって、
\begin{align}
  ϕ(r) = - ∫_{r₀}^{r} \𝑑{r} E(r)
  = -÷{λ}{2𝜋ε₀}\ln ÷{r}{r₀}
\end{align}
\subsubsection*{
  [2]
}
点$(r\cos θ, r\sin θ)$と$(a,0),(b,0)$との間の距離は、
\begin{align}
  √(a²+r²-2ar\cosθ),␣
  √(b²+r²-2br\cosθ)
\end{align}
で与えられるので、
\begin{align}
  ϕ(r,θ)
  = ÷{λ}{4𝜋ε₀}\ln÷{b²+r²-2br\cosθ}{a²+r²-2ar\cosθ}
\end{align}
となる。
\subsubsection*{
  [3]
}
設問[2]の結果から、静電ポテンシャルは
\begin{align}
  ϕ(r,θ)
  = ÷{λ}{4𝜋ε₀}\ln÷{D²+r²-2Dr\cosθ}{d²+r²-2dr\cosθ}
  \label{ϕ}
\end{align}
と書ける。$r=R$においてこれが$θ$に依存しない条件は、
\begin{align}
  ÷{D²+R²}{2DR} = ÷{d²+R²}{2dR}
\end{align}
である。整理すると、
\begin{align}
  D = ÷{R²}{d}
\end{align}
が得られる。これを(\ref{ϕ})に代入すると、
\begin{align}
  ϕ(r,θ) = ÷{λ}{4𝜋ε₀}\ln(
    ÷{R²}{d²}
    ÷{(r/R)²d²+R²-2dr\cosθ}{d²+r²-2dr\cosθ}
  )
\end{align}
となる。
\subsubsection*{
  [4]
}
\begin{align}
  σ &= ε₀(\∂{ϕ}{r})_{r=R}\∅
  &
  = ÷{λ}{4𝜋}[
    ÷{2d²/R-2d\cosθ}{d²+R²-2dR\cosθ}
    -÷{2R-2d\cosθ}{d²+R²-2dR\cosθ}
  ]\∅
  &
  = ÷{λ(d²-R²)}{2𝜋R(R²+d²-2Rd\cosθ)}
\end{align}
\subsubsection*{
  [5]
}
$d/R = a$ とおくと、
\begin{align}
  σ = ÷{λ}{2𝜋R}÷{a²-1}{1-2a\cosθ+a²}
  = -÷{λ}{2𝜋R}(
    1+2∑_{n=1}^∞ a^n\cos nx
  )
\end{align}
これを$r=R$の円周上で積分すると、
\begin{align}
  R∫_0^{2𝜋}σ\𝑑{θ} = -λ
\end{align}
となる。
つまり、$z$軸方向の単位長さあたりの誘起される電荷は$-λ$となる。

\subsubsection*{
  [6]
}
線電荷が$θ=0$の位置にいるとし、
$σ_n$がそれぞれ$θ_n ≔ -ψ+(n-1)𝜋/2$の位置にいるとしてよい。
このとき、
\begin{align}
  ÷{2𝜋Rσ_n}{λ}
  = ÷{-(R²-d²)}{R²+d²-2Rd\cos θ_n}
\end{align}
となる。
$d$が$R$に比べて十分小さいとして、
$d/R$に関して2次の項を無視すると、
\begin{align}
  ÷{2𝜋Rσ_n}{λ}
  = -1-÷{2d}{R}\cos θ_n
\end{align}
となる。ここで、
\begin{align}
  \cosθ₁ = \cosψ,␣
  \cosθ₂ = \sinψ,␣
\end{align}
から、
\begin{align}
  (x₀,y₀) = (d\cosψ,d\sinψ)
  = (
    -÷{R}{2}(1+÷{2𝜋Rσ₁}{λ}),
    -÷{R}{2}(1+÷{2𝜋Rσ₂}{λ})
  )
\end{align}
となる。また、
\begin{align}
  \cosθ₃ = -\cosψ,␣
  \cosθ₄ = -\sinψ
\end{align}
から、
\begin{align}
   σ₁+σ₃ = σ₂+σ₄ = -÷{λ}{𝜋R}
\end{align}
となるので、
\begin{align}
  (x₀,y₀) 
  &
  = (
    -÷{R}{2}(1-÷{2σ₁}{σ₁+σ₃}),
    -÷{R}{2}(1-÷{2σ₂}{σ₂+σ₄})
  )\∅
  &
  = (
    ÷{R}{2}÷{σ₁-σ₃}{σ₁+σ₃},
    ÷{R}{2}÷{σ₂-σ₄}{σ₂+σ₄}
  ).
\end{align}
% \begin{align}
%   ÷{R²}{4}[(÷{2𝜋Rσ₁+λ}{λ})²+(÷{2𝜋Rσ₂+λ}{λ})²] = d²
% \end{align}
% となる。よって、
% \begin{align}
%   d = ÷{R}{2λ}√{(2𝜋R)²(σ₁²+σ₂²)+4𝜋Rλ(σ₁+σ₂)+2λ²}
% \end{align}

\newpage
\subsection*{
  物理学II
}
\subsection*{
  第1問
}
\subsubsection*{
  [1]
}
$ψ_S$と$ψ_A$の節の数はそれぞれ0と1であり、
振動定理から節の数が少ない$ψ_S$ほうが固有エネルギーが低い。

\subsubsection*{
  [2]
}
\begin{align}
  ⟨ψ_L|ψ_R⟩ = ÷{1}{2}(⟨ψ_S|ψ_S⟩-⟨ψ_A|ψ_A⟩) = 0
\end{align}
\begin{align}
  H = ⦅0&-J\\-J&0⦆
\end{align}
\subsubsection*{
  [3]
}
\begin{align}
  ℯ^{-¡Ht/ħ} = ⦅\cosωt&¡\sinωt\\¡\sinωt&\cosωt⦆,␣
  ω = ÷{J}{ħ}
\end{align}
より、$\sin²(Jt/ħ)$
\subsubsection*{
  [4.1]
}
\begin{align}
  |0⟩ = |LL⟩,␣
  |1⟩ = ÷{1}{√2}(|LR⟩+|RL⟩),␣
  |2⟩ = |RR⟩
\end{align}
\begin{align}
  ℋ|0⟩ = ℋ|2⟩ = -J(|LR⟩+|RL⟩) = -√2J|1⟩
\end{align}
\begin{align}
  ℋ|1⟩ = ÷{1}{√2}(2|LL⟩+2|RR⟩) = -√2J(|0⟩+|2⟩)
\end{align}
\subsubsection*{
  [4.2]
}
もとの固有状態で考えれば、$-2J,0,2J$
\subsubsection*{
  [4.3]
}
2つの粒子が独立なことから、
\begin{align}
  P₁(t)
  = 2\sin²÷{Jt}{ħ}\cos²÷{Jt}{ħ} %= ÷{1}{2}\sin²÷{2Jt}{ħ}
\end{align}
\begin{align}
  P₂(t) = \cos⁴÷{Jt}{ħ} %= ÷{1}{4}(1+\cos²÷{2Jt}{ħ})²
\end{align}
% \begin{align}
%   ℋ² = 4J²⦅
%     1/2&0&1/2\\
%     0&1&0\\
%     1/2&0&1/2
%   ⦆
% \end{align}
% から、
% \begin{align}
%   ℯ^{-¡ℋt/ħ}
%   = ÷{1}{2}⦅
%     \cosΩt+1&-√2\sinΩt&\cosΩt-1\\
%     -√2\sinΩt&\cosΩt&-√2\sinΩt\\
%     \cosΩt-1&-√2\sinΩt&\cosΩt+1
%   ⦆
% \end{align}
% ただし、$Ω = 2Jt/ħ$である。よって、
% \begin{align}
%   P₁(t) = ÷{1}{2}\sin²Ωt,␣
%   P₂(t) = ÷{1}{4}(\cos²Ωt -2\cosΩt+1)
% \end{align}
\subsubsection*{
  [5.1]
}
\begin{align}
  ℋ = ⦅
    -A&-√2J&0\\
    -√2J&0&-√2J\\
    0&-√2J&-A
  ⦆
\end{align}
\subsubsection*{
  [5.2]
}
固有方程式は、
\begin{align}
  E³ + 2AE² +(A²-4J²)E - 4AJ²
  &
  = (E+A)(E²+AE-4J²) = 0
\end{align}
(パリティ対称性があるので、$3×3$行列でも楽に対角化できることが保証されている。
つまり、自明な固有ベクトルとして$(1,0,-1)$があるので、
これに$ℋ$を作用させれば少なくとも1つの固有値は求まる。)
よって
\begin{align}
  E = -A,÷{-A±√{A²+16J²}}{2}.
\end{align}
$J²/A$のオーダーまで展開すると、
\begin{align}
  E = -A-÷{4J²}{A},-A,÷{4J²}{A}
\end{align}
\subsubsection*{
  [5.3]
}
摂動がパリティを破っていないことに注意すると、$|1⟩$の成分はない。
よって2個の粒子が複合粒子を作っている状況なので、
設問[3]と同じものを選べば良い。
よってグラフは(b)。また点線が$P₀$、実線が$P₂$、一点鎖線が$P₁$。
周期は基底状態と第一励起状態のエネルギー差を$𝛥E$として、
\begin{align}
  T = ÷{2𝜋}{𝛥E/ħ} = ÷{𝜋ħA}{2J²}.
\end{align}
\subsubsection*{
  [6]
}
$A → -B$とすればよい。グラフは(b)。点線が$P₀$、実線が$P₂$、一点鎖線が$P₁$。
周期は
\begin{align}
  T = ÷{𝜋ħB}{2J²}
\end{align}
\subsubsection*{
  [7]
}
引力の場合と斥力の場合の違いは、
複合粒子状態($|0⟩,|2⟩$)のエネルギーが
そうでない状態($|1⟩$)のエネルギーよりが高いか低いかだけである。
しかし、摂動がパリティを破っていないことから、$|1⟩$の成分はない。
よって2つの場合で差はない。

\newpage
\subsection*{
  第2問
}
\subsubsection*{
  [1]
}
\begin{align}
  Z = ÷{V^N}{h^{3N}N!}(∫_{-∞}^{∞}\𝑑{p}\exp(-÷{p²}{2m𝘬T}))^{3N}
  = ÷{V^N}{N!}(÷{m𝘬T}{2𝜋ħ²})^{3N/2}
\end{align}
\subsubsection*{
  [2]
}
\begin{align}
  U = -\∂_β \ln Z = ÷{3}{2}N𝘬T
\end{align}
\begin{align}
  C = ÷{3}{2}𝘬T
\end{align}
\subsubsection*{
  [3]
}
\begin{align}
  P = 𝘬T\∂_V\ln Z = ÷{N𝘬T}{V}
\end{align}
\subsubsection*{
  [4]
}
相互作用がない場合の分配関数を$Z₀(T,V,N)$と書くことにする。
分配関数は、
\begin{align}
  Z = Z₀(T,V-b,N)\exp(-÷{αN²}{V-b})
\end{align}
となる。ここから、
\begin{align}
  P = 𝘬T\∂_V\ln Z = ÷{𝘬T}{V-b}+÷{α𝘬TN²}{(V-b)²}
\end{align}
\subsubsection*{
  [5]
}
状態方程式は$A'(V)=0$のときに満たされる。
これは3点あるが、力学的に安定となるためには
\begin{align}
  A''(V) = \∂^2{F}{V} = -\∂{P}{V} < 0
\end{align}
が成り立つ必要がある。
よって安定な点は2つの極小点で、極大となる点は不安定。
% \subsubsection*{
%   [6]
% }
% 圧縮率は正だから、$P$を大きくすると、$V$が小さくなり、$P$を小さくすると、$V$が大きくなる。
% \subsubsection*{
%   [7]
% }
\newpage
\subsection*{
  第3問
}
\subsubsection*{
  [1]
}
\begin{align}
  𝑬(z,t) = ℯ^{-k₂z}\Re[\~𝑬₀\exp{¡(k₁z-ωt)}]
\end{align}
の位相速度は、
\begin{align}
  v = ÷{ω}{k₁}
\end{align}
\subsubsection*{
  [2]
}
\begin{align}
  d = ÷{1}{k₂}
\end{align}
\subsubsection*{
  [3]
}
(1),(2)から、
\begin{align}
  \~k² = εμω² + ¡μσω ≕ μω√{ε²ω²+σ²}ℯ^{¡θ}
\end{align}
となる。ここで、
\begin{align}
  \~k
  = ±√{μω√{ε²ω²+σ²}}(√{÷{1+\cosθ}{2}} + ¡√{÷{1-\cosθ}{2}})
\end{align}
と書けるから、
\begin{align}
  &
  k₁ = √{÷{1}{2}μω(√{ε²ω²+σ²}+εω)},\\
  &
  k₂ = √{÷{1}{2}μω(√{ε²ω²+σ²}-εω)}
\end{align}
となる。ただし符号は$k₂ > 0$となるように選択した。
\subsubsection*{
  [4]
}
\begin{align}
  d = √{÷{2}{μω(√{ε²ω²+σ²}-εω)}}
\end{align}
$σ ≫ εω$のとき、
\begin{align}
  d ≈ √{÷{2}{μωσ}} ∝ σ^{-1/2}
\end{align}
$σ ≪ εω$のとき、
\begin{align}
  d ≈ √{÷{4ε}{μσ²}} ∝ σ^{-1}
\end{align}
\subsubsection*{
  [5]
}
$-\∂_t𝑩 = ∇×𝑬$より、
\begin{align}
  ¡ω𝑩 = ¡\~k𝒆_z×𝑬.
\end{align}
よって、位相の遅れは
\begin{align}
  ϕ = \arg(k₁+¡k₂)
\end{align}
$σ/εω → ∞$では$k₁/k₂ → 1$から、
\begin{align}
  ϕ = \arg(1+¡) = ÷{𝜋}{4}
\end{align}
$σ/εω → 0$では$k₂/k₁ → 0$から、
\begin{align}
  ϕ = \arg(1) = 0
\end{align}
\subsubsection*{
  [6]
}
$z=0$において、
\begin{align}
  |𝑺| = ÷{|\~k||\~𝑬₀|²}{μω}\cos(ωt)\cos(ωt+ϕ)
  = ÷{|\~k||\~𝑬₀|²}{2μω}(\cos(2ωt+ϕ)+\cosϕ)
\end{align}
と書ける。よってこの時間平均をとると、
\begin{align}
  ÷{|\~𝑬₀|²|\~k|\cosϕ}{2μω} = ÷{|\~𝑬₀|²k₁}{2μω}
  &
  = ÷{|\~𝑬₀|²}{2}√{÷{1}{2μω}(√{ε²ω²+σ²}+εω)}
\end{align}
\subsubsection*{
  [7]
}
\begin{align}
  Q = 𝑬(𝒓,t)⋅𝑱(𝒓,t)
\end{align}
\subsubsection*{
  [8]
}
$𝑱 = σ𝑬$より、
\begin{align}
  ⟨Q⟩ = ÷{1}{2}σ|\~𝑬₀|²ℯ^{-2k₂z}
\end{align}
\begin{align}
  ∫_0^{+∞}⟨Q⟩\𝑑{z} = ÷{σ|\~𝑬₀|²}{4k₂} = ÷{|\~𝑬₀|²k₁}{2μω}
\end{align}
ここで
\begin{align}
  k₁k₂ = ÷{1}{2}μωσ
\end{align}
を用いた。
よって、境界から導体内へ流入するエネルギーは全てJoule熱として消費される。
\newpage
\subsection*{
  第4問
}
\subsubsection*{
  [1]
}
\begin{align}
  &
  𝒃₁ = 2𝜋÷{\det\𝐯|𝒆₁&-√3a/2\\𝒆₂&a/2|}{\det\𝐯|√3a/2&-√3a/2\\a/2&a/2|}
  = 2𝜋⦅1/√3a\\1/a⦆,\\
  &
  𝒃₂ = 2𝜋÷{\det\𝐯|√3a/2&𝒆₁\\a/2&𝒆₂|}{\det\𝐯|√3a/2&-√3a/2\\a/2&a/2|}
  = 2𝜋⦅-1/√3a\\1/a⦆.
\end{align}
$P(2𝜋/√3a,0),~ Q(0,4𝜋/3a)$
\subsubsection*{
  [2]
}
$k_𝐹$に対して、
\begin{align}
  2⋅𝜋k_𝐹²÷{S}{(2𝜋)²} = 2⋅÷{S}{s}
\end{align}
が成り立つ。$S$は系の面積、$s$は単位胞の面積であり、
\begin{align}
  s = ÷{√3}{2}a²
\end{align}
で与えられる。
よって、
\begin{align}
  k_𝐹 = √{÷{4𝜋}{s}} = √{÷{8𝜋}{√3}}a^{-1}.
\end{align}
同じことを波数空間でも考えられる。
Fermi面の内部の面積とBrillouinゾーンの面積が等しいことから、
\begin{align}
  𝜋k_𝐹² = ÷{(2𝜋)²}{s} = ÷{8𝜋²}{√3a²}.
\end{align}
よって同じ答えを得る。
\subsubsection*{
  [4]
}
\begin{align}
  &
  ⟨ϕ_i^A|ℋ|ψ_𝒌⟩
  = λ_Aϵ_A ℯ^{¡𝒌⋅𝒓_i^A} + λ_Bτ ∑_{𝜹} ℯ^{¡𝒌⋅(𝒓_i^A+𝜹)} \\
  &
  ⟨ϕ_i^B|ℋ|ψ_𝒌⟩
  = λ_Bϵ_B ℯ^{¡𝒌⋅𝒓_i^B} + λ_Aτ ∑_{𝜹} ℯ^{¡𝒌⋅(𝒓_i^B-𝜹)} 
\end{align}
から、
\begin{align}
  ⦅
    ⟨ψ^A_𝒌|ℋ|ψ^A_𝒌⟩&⟨ψ^A_𝒌|ℋ|ψ^B_𝒌⟩\\
    ⟨ψ^B_𝒌|ℋ|ψ^A_𝒌⟩&⟨ψ^B_𝒌|ℋ|ψ^B_𝒌⟩
  ⦆
  = ⦅
          ϵ_A       &τ ∑_{j} ℯ^{¡𝒌⋅𝜹_j}\\
    τ ∑_{j} ℯ^{-¡𝒌⋅𝜹_j}&      ϵ_B
  ⦆
\end{align}
となる。ここで、
\begin{align}
  𝜹₁ = ÷{𝒂₁+2𝒂₂}{3},␣
  𝜹₂ = ÷{𝒂₁-𝒂₂}{3},␣
  𝜹₃ = ÷{-2𝒂₁-𝒂₂}{3}
\end{align}
と定義した。$λ_A,λ_B$が満たす方程式は、
\begin{align}
  ⦅
    ϵ_A & τ ∑_{j} ℯ^{¡𝒌⋅𝜹_j}\\
    τ ∑_{j} ℯ^{-¡𝒌⋅𝜹_j} & ϵ_B
  ⦆⦅λ_A\\λ_B⦆
  = E(𝒌)⦅λ_A\\λ_B⦆
\end{align}
となる。
\subsubsection*{
  [5]
}
まず、
\begin{align}
  Δ(𝒌) ≔ τ ∑_{j} ℯ^{¡𝒌⋅𝜹_j}
\end{align}
とおく。設問[4]で求めた方程式に非自明な解が存在する条件は、
\begin{align}
  E(𝒌)²-(ϵ_A+ϵ_B)E(𝒌)+(ϵ_Aϵ_B-|Δ(𝒌)|²) = 0
\end{align}
となる。よって
\begin{align}
  E(𝒌) = ÷{ϵ_A+ϵ_B}{2} ± √{÷{(ϵ_A-ϵ_B)²}{4}+|Δ(𝒌)|²}
\end{align}
である。正直ここで終わりにしたいが、$|Δ(𝒌)|²$を計算する。
\begin{align}
  |Δ(𝒌)|²
  &
  = τ²[3+2\cos(𝒌⋅𝜹_{23})+2\cos(𝒌⋅𝜹_{31})+2\cos(𝒌⋅𝜹_{12})]\∅
  &
  = τ²(3+2\cos(÷{√3}{2}ak_x+÷{1}{2}ak_y)+2\cos(-÷{√3}{2}ak_x+÷{1}{2}ak_y)+2\cos(ak_y))\∅
  &
  = τ²(3+4\cos(÷{√3}{2}ak_x)\cos(÷{1}{2}ak_y)+2\cos(ak_y))
\end{align}
ただし$𝜹_{ij} = 𝜹_i - 𝜹_j$と略記している。
よって、
\begin{align}
  E(𝒌) = ÷{ϵ_A+ϵ_B}{2}
  ± √{
    ÷{(ϵ_A-ϵ_B)²}{4}
    +τ²(3+4\cos(÷{√3ak_x}{2})\cos(÷{ak_y}{2})+2\cos(ak_y))
  }
\end{align}
\subsubsection*{
  [6]
}
$k_x=0$のとき、
\begin{align}
  E(k_y) &= ÷{ϵ_A+ϵ_B}{2} ± √{
    ÷{(ϵ_A-ϵ_B)²}{4}
    +τ²(1+4\cos(÷{ak_y}{2})+4\cos²(÷{ak_y}{2}))
  }\∅
  &
  = ÷{ϵ_A+ϵ_B}{2} ± √{
    ÷{(ϵ_A-ϵ_B)²}{4}
    +τ²(1+2\cos(÷{ak_y}{2})²)
  }.
\end{align}
ギャップが最小になるのは
$\cos(ak_y/2) = -1/2$のときなので、
\begin{align}
  k_y = ÷{4𝜋}{3a} + 4n𝜋,␣ ÷{8𝜋}{3a} + 4n𝜋,␣
  n ∈ ℤ
\end{align}
のとき。ただし、これらは全て等価であり、Brilloinゾーンに含まれる点としては
\begin{align}
  k_y = ±÷{4𝜋}{3a}
\end{align}
となる。このとき、ギャップは
\begin{align}
  E_g = ϵ_A - ϵ_B
\end{align}
\subsubsection*{
  [7]
}
点Xからの$k_y$のずれを$𝛿k_y$とおく。
\begin{align}
  1+2\cos(÷{2𝜋}{3}+÷{a𝛿k_y}{2})
  = -÷{√3a}{2}𝛿k_y
\end{align}
から、
\begin{align}
  𝑬(𝒌) &≈ ÷{ϵ_A+ϵ_B}{2} ± ÷{(ϵ_A-ϵ_B)}{2}[
    1 + ÷{2τ²}{(ϵ_A-ϵ_B)²}(÷{√3a}{2}𝛿k_y)²
  ]\∅
  &
  = ÷{ϵ_A+ϵ_B}{2} ± [
    ÷{(ϵ_A-ϵ_B)}{2}
    +÷{3a²τ²}{4(ϵ_A-ϵ_B)}𝛿k_y²
  ]
\end{align}
上下のバンドの有効質量は、
\begin{align}
  ±÷{2ħ²(ϵ_A-ϵ_B)}{3a²τ²}
\end{align}
となる。
\subsubsection*{
  [8]
}
$ϵ_A=ϵ_B$とすればよい。
\begin{align}
  E(𝒌) = ϵ_A ± τ\𝚟|1+2\cos(÷{ak_y}{2})|
\end{align}
から、点X付近では、
\begin{align}
  E(𝒌) = ϵ_A ± ÷{√3aτ}{2}\𝚟|𝛿k_y|.
\end{align}
\end{document}