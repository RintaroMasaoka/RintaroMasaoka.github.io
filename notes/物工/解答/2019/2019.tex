\providecommand{\main}{../main}
\documentclass[\main/main.tex]{subfiles}
\graphicspath{{../images/}}
\begin{document}
\newpage
\section{2018年(平成30年)}
\subsection*{
  物理学I
}
\subsection*{
  第1問
}
\subsubsection*{
  [1.1]
}
遠心力と重力の釣り合いは、
\begin{align}
  m÷{v₁²}{R} = ÷{mMG}{R²}.
\end{align}
よって
\begin{align}
  v₁ = √{÷{MG}{R}}
\end{align}
\subsubsection*{
  [1.2]
}
脱出する最小の速度$v₂$に対し、
\begin{align}
  ÷1{2}mv₂² - ÷{mMG}{R} = 0
\end{align}
となる。よって
\begin{align}
  v₂ = √{÷{2MG}{R}}
\end{align}
\subsubsection*{
  [1.3]
}
楕円
\begin{align}
  r = ÷l{1+ε\cos(θ-θ₀)}
\end{align}
において、$θ = θ₀$および$θ = θ₀ + 𝜋$を代入すると、
\begin{align}
  r(θ = θ₀) = ÷l{1+ε},\␣
  r(θ = θ₀ + 𝜋) = ÷l{1-ε}
\end{align}
となる。よって長軸は
\begin{align}
  L
  &
  = ÷l{1+ε} + ÷l{1-ε} = ÷{2l}{1-ε²}
  \∅ & 
  = ÷{2h²}{GM} ⋅ ÷{G²M²m}{2h²(-E)}
  \∅ & 
  = ÷{GMm}{-E}
  \∅ & 
  = ÷{GM}{GM/R - v²/2}
\end{align}
\subsubsection*{
  [2.1]
}
\begin{align}
  2mR₀ω₀² = ÷{2mMG}{R₀²}
\end{align}
\begin{align}
  ω₀ = √{÷{MG}{R₀³}}
\end{align}
\subsubsection*{
  [2.2]
}
\begin{align}
  I = 2⋅m\𝚙(÷l{2})² = ÷{ml²}2
\end{align}
\subsubsection*{
  [2.3]
}
モーメント$𝑵$を地道に計算する。
\begin{align}
  𝑵
  &
  = -÷{mMG}{|𝑹₀+𝒍/2|^3}÷𝒍{2}×\𝚙(𝑹₀+÷𝒍{2})
  +÷{mMG}{|𝑹₀-𝒍/2|^3}÷𝒍{2}×\𝚙(𝑹₀+÷𝒍{2})
  \∅ & 
  = ÷{mMG}{2R₀³}𝒍×𝑹₀\𝚋[
    \𝚙(1-÷{𝑹₀⋅𝒍}{R₀²}+÷{l²}{4R²})^{-3/2}
    -\𝚙(1+÷{𝑹₀⋅𝒍}{R₀²}+÷{l²}{4R²})^{-3/2}
  ]
  \∅ & 
  = ÷{mMGl}{2R₀²}\sinϕ 𝒆_z\𝚋[
    \𝚙(1-÷{l}{R₀}\cosϕ+÷{l²}{4R²})^{-3/2}
    -\𝚙(1+÷{l}{R₀}\cosϕ+÷{l²}{4R²})^{-3/2}
  ]
  \∅ & 
  = ÷{mMGl}{2R₀²}{3l}{R₀}\sinϕ\cosϕ 𝒆_z
  \∅ &
  = ÷1{3} ml²ω₀²\sin 2ϕ 𝒆_z
\end{align}
\subsubsection*{
  [2.4]
}
\begin{align}
  \sin 2ϕ₀ = 0,␣
  ∴ ϕ₀ = 0,÷𝜋{2},𝜋,÷{3𝜋}{2}
\end{align}
\begin{align}
  N = I𝛿{\"ϕ} = ÷2{3}m\cos2ϕ₀ l²ω₀² 𝛿ϕ
\end{align}
$ϕ = 0,𝜋$のまわりでは、
\begin{align}
  𝛿{\"ϕ} = ÷{3ml²ω₀²}{2I} 𝛿ϕ = 3ω₀² 𝛿ϕ
\end{align}
となる。これは不安定である。
次に$ϕ = ÷𝜋{2},÷{3𝜋}{2}$のまわりでは、
\begin{align}
  𝛿{\"ϕ} = -÷{ml²ω₀²}{I} 𝛿ϕ = -3ω₀² 𝛿ϕ
\end{align}
となる。これは安定な調和振動となり、角振動数は$√3ω₀$となる。
\newpage
\subsection*{
  第2問
}
\subsubsection*{
  [1.1]
}
\begin{align}
  ∇×𝑬 = -\∂𝑩{t}
\end{align}
これを半径$r$の上から見て反時計回りの経路で積分すると、
\begin{align}
  ∮𝑬⋅𝑑𝒍 = ∫(∇×𝑬)⋅𝑑𝑺 = -∫\∂B{t}𝑑S
\end{align}
となる。よって、
\begin{align}
  2𝜋rE(t) = -𝜋r²B'(t),␣
  E(t) = -÷1{2}rB'(t)
\end{align}
である。つまり
\begin{align}
  E(t) = -𝜋fμ₀H₀r\cos(2𝜋ft)
\end{align}
\subsubsection*{
  [1.2]
}
$ω = 2𝜋f,B₀ = μ₀H₀$とおくと、
\begin{align}
  E(t) = -÷1{2}rωB₀\cos(ωt)
\end{align}
である。
単位体積あたりに発生するJoule熱は、
\begin{align}
  p(r) = ÷{𝑬²}{ρ} = ÷{ω²B₀²}{4ρ}r²\cos²(ωt)
\end{align}
で計算できる。これを積分して、
\begin{align}
  P = 2𝜋L∫_0^R rp(r)𝑑r
  = 2𝜋L÷{ω²B₀²}{4ρ}÷{R⁴}{4}\cos²(ωt)
\end{align}
を得る。
$ω = 2𝜋f$および$B₀=μ₀H₀$を代入すると、
\begin{align}
  P = ÷{𝜋³μ₀²H₀²f²LR⁴}{2ρ}\cos²(2𝜋ft)
\end{align}
となる。
\subsubsection*{
  [1.3]
}
\begin{align}
  ÷{⟨P⟩}{C}/\si{K} &
  = ÷{𝜋³μ₀²H₀²f²LR⁴}{4ρC}\∅
  &
  = ÷{2^{4+6}×10^{-14+8+2-3-8}×𝜋⁵}{2^{2+2-1}×10^{-8}}\∅
  &
  = 128𝜋⁵×10^{-7}
\end{align}
である。
\begin{align}
  𝜋⁵ ≈ 3⁵⋅1.05⁵ ≈ 243⋅1.25 ≈ 300
\end{align}
から、
\begin{align}
  ÷{⟨P⟩}{C} ≈ \SI{4e-3}{K}
\end{align}
\subsubsection*{
  [2.1]
}
系の対称性から$B_x$について計算すればよい。
Biot-Savartの法則より、
\begin{align}
  B_x
  = ∫_0^{2𝜋}÷{μ₀}{4𝜋}÷{Ia²\𝑑θ}{(a²+x²)^{3/2}}
  = ÷{μ₀I}2÷{a²}{(a²+x²)^{3/2}}
\end{align}
\subsubsection*{
  [2.2]
}
\begin{align}
  B(x)
  = ÷{μ₀Ia²}2\𝚙(
    ÷1{(a²+(b+x)²)^{3/2}}
    - ÷1{(a²+(b-x)²)^{3/2}}
  )
\end{align}
これを$x$について展開する。
まず、$B(0) = 0$である。
また、
\begin{align}
  B'(0) = -2 ⋅ ÷{μ₀Ia}2 ⋅ ÷3{2}÷{2b}{(a²+b²)^{5/2}}
  = ÷{3μ₀Iab}{(a²+b²)^{5/2}}
\end{align}
である。
\begin{align}
  B(x) = -B(-x)
\end{align}
から$B''(0)=-B''(0)=0$なので、
\begin{align}
  B(x) = -÷{3μ₀Ia²b}{(a²+b²)^{5/2}}x + 𝒪(x³)
\end{align}
と書ける。

\subsubsection*{
  [2.3]
}
電流が同じ向きの場合、
\begin{align}
  B(x)
  = ÷{μ₀Ia²}2\𝚙(
    ÷1{(a²+(b+x)²)^{3/2}}
    + ÷1{(a²+(b-x)²)^{3/2}}
  )
\end{align}
となる。
()内第1項を無次元化したものを$x$について展開すると、以下のようになる。
\begin{align}
  \𝚙(÷{a²+b²+2bx+x²}{a²+b²})^{-3/2}
  &
  = \𝚙(1+÷{2bx+x²}{a²+b²})^{-3/2}
  \∅ & 
  = 1 - ÷3{2}⋅÷{2bx+x²}{a²+b²}
  + ÷{15}{8}⋅÷{4b²x²+ 𝒪(x³)}{(a²+b²)²}
  \∅ & 
  = 1 - ÷{3bx}{a²+b²}
  + ÷3{2}⋅÷{4b²-a²}{(a²+b²)²}x² + 𝒪(x³)
\end{align}
これに$x → -x$とした式を足し上げ、定数倍することで$B(x)$が得られる。
したがって、$B(x)$の$x²$に比例する項が消える条件は、
\begin{align}
  4b² - a² = 0
\end{align}
とである。$a,b>0$から求める条件は
\begin{align}
  a = 2b
\end{align}
である。
\newpage
\subsection*{
  物理学II
}
\subsection*{
  第1問
}
\subsubsection*{
  [1.1]
}
まず波動関数の連続性から
\begin{align}
  ψ(ε) = ψ(-ε)
\end{align}
が成り立つ。$ε$は微小な正の数である。
次に、Schrödinger方程式
\begin{align}
  -÷{ħ²}{2m}\𝚍^2{ψ(x)}{x} + U(x)ψ(x) = Eψ(x)
\end{align}
を$-ε$から$ε$まで積分することで、
\begin{align}
  -÷{ħ²}{2m}\𝚙(ψ'(ε)-ψ'(-ε))
  + αψ(0) = 0
\end{align}
を得る。

\subsubsection*{
  [1.2]
}
境界条件から
\begin{align}&
  1+r = t \\
  &
  ÷{¡kħ²}{2m}\𝚙(t+r-1) = α(1+r)
\end{align}
となる。整理すると、
\begin{align}&
  1+r = t \\
  &
  ¡(t+r-1) = 2C(1+r)
\end{align}
となる。これを解くと、
\begin{align}
  r = ÷C{¡-C},␣
  t = ÷¡{¡-C}.
\end{align}
\subsubsection*{
  [2.1]
}
\begin{align}
  Ψ(x₁,x₂)
  &
  ∝ \det\𝐩(
    tψ₊(x₁)+rψ₋(x₁) & rψ₊(x₁) + tψ₋(x₁) \\
    tψ₊(x₂)+rψ₋(x₂) & rψ₊(x₂) + tψ₋(x₂)
  )
  \∅ & 
  = (t²-r²)\det\𝐩(
    tψ₊(x₁) & tψ₋(x₁) \\
    tψ₊(x₂) & tψ₋(x₂)
  )
  \∅ & 
  = (t²-r²)(ψ₊(x₁)ψ₋(x₂)-ψ₋(x₁)ψ₊(x₂))
\end{align}
よって2粒子は反対方向に散乱される。

\subsubsection*{
  [2.2]
}
対称な場合、
\begin{align}
  Ψ(x₁,x₂)
  &
  ∝ 2tr(ψ₊(x₁)ψ₊(x₂)+ψ₊(x₁)ψ₋(x₂))\∅
  &
  +(t²+r²)(ψ₊(x₁)ψ₋(x₂)+ψ₋(x₁)ψ₊(x₂))
\end{align}
\begin{align}
  tr = ÷{¡C}{(¡-C)²}
\end{align}
は、$α → 0$または$α → ∞$で$0$になる。
よって、このとき2粒子は反対方向に散乱される。
また$C=¡$すなわち、
\begin{align}
  α = α₀ = ÷{¡ħ²k}{m} = ¡ħ√{÷{2E}{m}}
\end{align}
のとき、2粒子は必ず同方向に散乱される。
\subsubsection*{
  [3]
}
$x<0$での波動関数を$ℯ^{¡kx}$、
$0<x<L$での波動関数を$Aℯ^{¡kx}+Bℯ^{-¡kx}$、
$L<x$での波動関数を$ℯ^{¡k(x-L+δ)}$とおく。
接続条件は
\begin{align}
  &
  A+B = 1,\\
  &
  ¡(A-B-1) = 2C,\\
  &
  Aℯ^{¡kL}+Bℯ^{-¡kL} = ℯ^{¡kδ},\\
  &
  ¡(ℯ^{¡kδ} - Aℯ^{¡kL} + Bℯ^{-¡kL}) = 2Cℯ^{¡kδ}
\end{align}
となる。上の2式から
\begin{align}
  A = 1-¡C,␣
  B = ¡C
\end{align}
となる。また下の2式から
\begin{align}
  Aℯ^{¡k(L-δ)} = 1+¡C,␣
  Bℯ^{-¡k(L+δ)} = -¡C
\end{align}
となる。よって
\begin{align}
  ℯ^{2¡kL}
  = ÷{B}{A}⋅÷{Aℯ^{¡k(L-δ)}}{Bℯ^{-¡k(L+δ)}}
  = ÷{¡C}{1-¡C}⋅÷{1+¡C}{-¡C} = ÷{1+¡C}{-1+¡C}.
\end{align}
よって
\begin{align}
  L = ÷{1}{2k¡}\log÷{1+¡C}{-1+¡C}
\end{align}
\newpage
\subsection*{
  第2問
}
\subsubsection*{
  [1]
}
van der Waalsの状態方程式
\begin{align}
  P = ÷{n𝘬T}{1-bn} - an²
\end{align}
について考える。
まず
\begin{align}
  (\∂P{n})_T = 0 \𝚚{and} (\∂^2P{n})_T = 0
\end{align}
となる点$T_c, n_c$を求める。
\begin{align}
  (\∂P{n})_T &
  = ÷{𝘬T}{1-bn} + ÷{bn𝘬T}{(1-bn)²} - 2an \∅ 
  &
  = ÷{𝘬T}{(1-bn)²} - 2an = 0
\end{align}
\begin{align}
  (\∂^2P{n})_T
  = ÷{2b𝘬T}{(1-bn)³} - 2a = 0
\end{align}
よって、
\begin{align}
  ÷{2b}{1-bn} = ÷1{n},␣
  n_c = ÷1{3}÷1{b}.
\end{align}
また
\begin{align}
  T_c = ÷{2an_c}{𝘬}(1-bn_c)² = ÷8{27}÷{a}{𝘬b}.
\end{align}
このとき
\begin{align}
  P_c = ÷{n_c𝘬T_c}{1-bn_c}-an_c²
  = ÷{4a}{27b²}- ÷a{9b²} = ÷1{27}÷{a}{b²}.
\end{align}
\subsubsection*{
  [2]
}
\begin{align}
  K_T &
  = -÷1{V}(\∂{V}{P})_T \∅
  &
  = [-V(\∂{P}{V})_T]^{-1}\∅
  &
  = [n(\∂{P}{n})_T]^{-1}\∅
  &
  = (÷{n𝘬T}{(1-bn)²} - 2an²)^{-1}
\end{align}
$n = n_c = 1/3b$のとき、
\begin{align}
  K_T =÷1{n_c}(÷{9}{4}𝘬T - ÷{2a}{3b})^{-1}
  = ÷4{9}÷1{n_c𝘬(T-T_c)}
\end{align}
となる。
\subsubsection*{
  [3]
}
\begin{align}
  Z₀(T,V,N) &
  = ÷1{N!} ÷{V^N}{h^{3N}}
    [∫_{-∞}^{∞}\𝑑{p}\exp(-÷1{𝘬T}÷{p²}{2m})]^{3N} \∅
  &
  = ÷1{N!}÷{V^N}{h^{3N}}(2m𝘬T)^{3N/2}
\end{align}
\begin{align}
  F₀(T,V,N) &= -𝘬T\ln Z₀(T,V,N) \∅
  &
  = -N𝘬T(
    \ln ÷{V}{h³} -  \ln N + 1 + ÷{3}{2}\ln(2m𝘬T)
  )
\end{align}
\subsubsection*{
  [4]
}
\begin{align}
  A ≈ (÷{V-Nv}{V})^N[
    1 - ÷{4𝜋}{V} ∫_l^∞ r²÷{u(r)}{𝘬T}\𝑑{r}
  ]^{N(N-1)/2}
\end{align}
\begin{align}
  ÷{4𝜋}{V} ∫_l^∞ r²÷{u(r)}{𝘬T}\𝑑{r}
  &
  = -÷{4𝜋ε}{𝘬TV} ∫_l^∞ ÷{l⁶}{r⁴}\𝑑{r} \∅
  &
  = - ÷{4𝜋εl³}{3𝘬TV} \∅
  &
  = -÷ε{𝘬T}÷{2v}{V}
\end{align}
\begin{align}
  \ln A &≈ N\ln(÷{V-Nv}{V})
  + ÷{N(N-1)}{2}\ln(1 + ÷{ε}{𝘬T}÷{2v}{V}) \∅
  &
  ≈ N\ln(÷{V-Nv}{V}) + ÷{ε}{𝘬T}÷{vN²}{V}
\end{align}
\begin{align}
  F = -N𝘬T(
    \ln ÷{V-Nv}{Nh³} + 1 + ÷{3}{2}\ln(2m𝘬T)
    + ÷{ε}{𝘬T}÷{vN}{V}
  )
\end{align}
\subsubsection*{
  [5]
}
\begin{align}
  P = -\∂F{V}
  &
  =÷{N𝘬T}{V-Nv} - ÷{vεN²}{V²}\∅
  &
  = ÷{n𝘬T}{1-vn} - εvn²
\end{align}
\subsubsection*{
  [6]
}
\begin{align}
  b = v,␣ a = εv
\end{align}
\newpage
\subsection*{
  第3問
}
\subsubsection*{
  [1.1]
}

\begin{align}
  E^𝑖(z,t) = E₀^𝑖\exp[-¡(÷{ωn₀}{c}z + ωt)]
\end{align}
\begin{align}
  H^𝑖(z,t) = H₀^𝑖\exp[-¡(÷{ωn₀}{c}z + ωt)]
\end{align}
Maxwellの方程式
\begin{align}
  ∇×𝑬 = -\∂{𝑩}{t} = -μ₀\∂{𝑯}{t}
\end{align}
より、
\begin{align}
  \∂{E_y}{z} = -μ₀\∂{H_x}{t}
\end{align}
が成り立つ。
\begin{align}
  -¡÷{ωn₀}{c}E₀^𝑖 = ¡ωμ₀H₀^𝑖
\end{align}
\begin{align}
  H₀^𝑖 = - ÷{n₀}{cμ₀}E₀^𝑖
\end{align}
\subsubsection*{
  [1.2]
}
\begin{align}
  E₀^𝑖 + E₀^𝑟 = E₀^𝑡,␣
  H₀^𝑖 + H₀^𝑟 = H₀^𝑡
\end{align}
\subsubsection*{
  [1.3]
}
境界条件は以下のように書き直せる。
\begin{align}
  &
  E₀^𝑖 + E₀^𝑟 = E₀^𝑡 \\
  &
  ÷{n₀}{cμ₀}(E₀^𝑖-E₀^𝑟) = ÷{n_𝑔}{cμ₀} E₀^𝑡
\end{align}
これを解くと、
\begin{align}
  &
  E₀^𝑖 = ÷1{2}(1+÷{n_𝑔}{n₀})E₀^𝑡 \\
  &
  E₀^𝑟 = ÷1{2}(1-÷{n_𝑔}{n₀})E₀^𝑡
\end{align}
となる。振幅反射係数$r₀ = E₀^𝑟/E₀^𝑖$は、
\begin{align}
  r₀ = ÷{E₀^𝑟}{E₀^𝑖} = ÷{n₀-n_𝑔}{n₀+n_𝑔}
\end{align}
となる。

\subsubsection*{
  [2.1]
}
\begin{align}
  E_l(z,t) = {
    E⁻_l\exp[-¡÷{ωn_l}{c}(z-z_l)]
    + E⁺_l\exp[¡÷{ωn_l}{c}(z-z_l)]
  }\exp(-¡ωt)
\end{align}
\begin{align}
  \∂{E_y}{z} = -μ₀\∂{H_x}{t}
\end{align}
\begin{align}
  H^±_l = ∓÷{n_l}{cμ₀}E^±_l
\end{align}
\subsubsection*{
  [2.2]
}
第$l$層の厚みを$d_l = z_{l-1} - z_l$とする。
$z=z_{l-1}$における境界条件は
\begin{align}
  E_l(z_{l-1},t) = E_{l-1}(z_{l-1},t),␣
  H_l(z_{l-1},t) = H_{l-1}(z_{l-1},t),␣
\end{align}
で与えられる。電場の境界条件は、
\begin{align}
  E⁻_l\exp[-¡÷{ωn_ld_l}{c}] 
  + E⁺_l\exp[¡÷{ωn_ld_l}{c}]
  = E⁻_{l-1}+E⁺_{l-1}
\end{align}
となる。磁場の境界条件は、
\begin{align}
  - n_lE⁻_l\exp[-¡÷{ωn_ld_l}{c}] 
  + n_lE⁺_l\exp[¡÷{ωn_ld_l}{c}]
  = -n_{l-1}E⁻_{l-1}+n_{l-1}E⁺_{l-1}
\end{align}
となる。よって、$Δ_l = n_lωd_l/c$とおくと、
\begin{align}
  ÷{E⁻_{l-1}-E⁺_{l-1}}{E⁻_{l-1}+E⁺_{l-1}}
  = ÷{n_l}{n_{l-1}}⋅
    ÷{E⁻_lℯ^{-¡Δ_l} - E⁺_lℯ^{¡Δ_l}}
      {E⁻_lℯ^{-¡Δ_l} + E⁺_lℯ^{¡Δ_l}}
\end{align}
すなわち、
\begin{align}
  α_l = ÷{n_l}{n_{l-1}}
\end{align}
\subsubsection*{
  [2.3]
}
% $r_l = E⁺_l/E⁻_l$と定義すると、境界条件は
% \begin{align}
%   ÷{1-r_{l-1}}{1+r_{l-1}}
%   = ÷{n_l}{n_{l-1}}⋅
%     ÷{ℯ^{-¡Δ_l} - r_lℯ^{¡Δ_l}}
%       {ℯ^{-¡Δ_l} + r_lℯ^{¡Δ_l}}
% \end{align}
% と表せる。
% 1次分数変換が行列で表せることを思い出すと、以下のように整理できる。
% \begin{align}
%   ⦅-r_{l-1}&1\\r_{l-1}&1⦆
%   ∝ ⦅n_l&0\\0&n_{l-1}⦆
%     ⦅-ℯ^{¡Δ_l}r_l&ℯ^{-¡Δ_l}\\ℯ^{¡Δ_l}r_l&ℯ^{-¡Δ_l}⦆
% \end{align}
% \begin{align}
%   ⦅-r_{l-1}&1\\r_{l-1}&1⦆
%   &
%   ∝ ⦅n_l&0\\0&n_{l-1}⦆⦅r_l&1\\-r_l&1⦆ \∅
%   &
%   ∝ ⦅n_l&0\\0&n_{l-1}⦆⦅0&1\\1&0⦆⦅-r_l&1\\r_l&1⦆
% \end{align}
$Δ_l = 𝜋/2$のとき、
\begin{align}
  ÷{E⁻_{l-1}-E⁺_{l-1}}{E⁻_{l-1}+E⁺_{l-1}}
  = ÷{n_l}{n_{l-1}}⋅
    ÷{E⁻_l + E⁺_l}{E⁻_l - E⁺_l}
\end{align}
である。
屈折率$n_𝐿,n_𝐻$の層を空気側から$LHLH⋯LH$と交互に重ねていく場合に、
漸化式を繰り返し用いると、
\begin{align}
  ÷{E⁻_{2N+1} + E⁺_{2N+1}}{E⁻_{2N+1} - E⁺_{2N+1}}
  &
  = ÷{n_𝐻}{n_𝑔}⋅
    ÷{E⁻_{2N} - E⁺_{2N}}{E⁻_{2N} + E⁺_{2N}} \∅
  &
  = ⋯ = ÷{n₀n_𝐻^{2N}}{n_𝑔n_𝐿^{2N}}⋅
        ÷{E⁻₀-E⁺₀}{E⁻₀+E⁺₀}
\end{align}
となる。$E⁺_{2N+1} = 0$から
\begin{align}
  ÷{1-r₁}{1+r₁}
  = ÷{n_𝑔n_𝐿^{2N}}{n₀n_𝐻^{2N}},␣
  r₁ = ÷{E⁺₀}{E⁻₀}
\end{align}
となる。したがって、
\begin{align}
  r₁ = ÷{n₀n_𝐻^{2N}-n_𝑔n_𝐿^{2N}}{n₀n_𝐻^{2N}+n_𝑔n_𝐿^{2N}}
\end{align}
であり、反射防止条件$r₁ ≤ 0$は
\begin{align}
  n₀n_𝐻^{2N} ≤ n_𝑔n_𝐿^{2N},␣
  ÷{n_𝐿}{n_𝐻} ≥ (÷{n₀}{n_𝑔})^{1/2N}
\end{align}
となる。
\newpage
\subsection*{
  第4問
}
\subsubsection*{
  [1.1]
}
化学ポテンシャル$μ$は内部エネルギー$U(S,V,N)$
($S$はエントロピー、$V$は体積、$N$は粒子数)
に対し
\begin{align}
  \𝑑{U} = T\𝑑{S} - p\𝑑{V} + μ\𝑑{N}
\end{align}
によって定義される。($T$は温度、$p$は圧力)

2つの系の間で粒子のやりとりが許され、
化学ポテンシャルが高い方から低い方へ粒子が流れる。
今の場合、金属板$𝑀$と半導体板$𝑆$で化学ポテンシャルが異なったために電荷が移動した。
\subsubsection*{
  [1.2]
}
微小な電荷$𝛿q$が$M$から$S$へ移動するときの静電エネルギーの変化は
\begin{align}
  ÷{ρd}{ε₀}𝛿q
\end{align}
で与えられる。
一方電荷の移動によって獲得される化学ポテンシャルの差分は
\begin{align}
  (W-ϕ_𝑠)÷{𝛿q}{e}
\end{align}
となる。これらが釣り合うことから、
\begin{align}
  ρ = ÷{ε₀}{ed}(W-ϕ_𝑠).
\end{align}
% まず、移動した電子の数は
% \begin{align}
%   𝛥n = ÷{ρA}{e}
% \end{align}
% と表される。ここから、系のエネルギーは
% \begin{align}
%   U &= ∫÷1{2}εE²\𝑑{V} + (ϕ_𝑠-W)𝛥n\∅
%   &
%   = ÷{Ad}{2ε₀}ρ² + (ϕ_𝑠-W)÷{A}{e}ρ \∅
%   &
%   = ÷{Ad}{2ε₀}(ρ-÷{ε₀}{ed}(W-ϕ_𝑠))² -÷{ε₀A}{2e²d}(W-ϕ_𝑠)²
% \end{align}
% となる。これを最小にする$ρ$が実現されるので、
% \begin{align}
%   ρ = ÷{ε₀}{ed}(W-ϕ_𝑠)
% \end{align}
\subsubsection*{
  [1.3]
}
$M,S$の間の力は、
\begin{align}
  Aρ⋅÷{ρ}{2ε₀} = ÷{ε₀A}{2e²d²}|W-ϕ_𝑠|²
\end{align}
\subsubsection*{
  [1.4]
}
\begin{align}
  ρ(t) &
  = ÷{ε₀}{ed}(W-ϕ_𝑠)(1 + ÷δ{d}\sin Ωt)^{-1}\∅
  &
  ≈ ÷{ε₀}{ed}(W-ϕ_𝑠)(1 - ÷δ{d}\sin Ωt)
\end{align}
から、電流は
\begin{align}
  I(t) = A\𝚍{ρ}{t} = -÷{ε₀Aδ}{ed²}(W-ϕ_𝑠)Ω\cos Ωt
\end{align}
となる。よって、
\begin{align}
  I₀ = ÷{ε₀Aδ}{ed²}|W-ϕ_𝑠|Ω
\end{align}
\subsubsection*{
  [2.1]
}
$Δ_𝑠$は電子を$S$から$M$へ動かすときに必要なエネルギーなので、
化学ポテンシャルの差に一致する。すなわち、
\begin{align}
  Δ_𝑠 = W - ϕ_𝑠
\end{align}
\subsubsection*{
  [2.2]
}
接合面近傍の伝導電子の数は電圧を$V$として、$ℯ^{eV/𝘬T}$に比例する。
よって$V<0$の領域では$0<ℯ^{eV/𝘬T}<1$から電流は小さく、
$V>0$の領域では$ℯ^{eV/𝘬T}>1$となり電流は大きくなる。
\end{document}