\providecommand{\main}{../main}
\documentclass[\main/main.tex]{subfiles}
\graphicspath{{../images/}}
\newcommand{\ex}{_\mathup{ex}}
\begin{document}
\newpage
\section{2016年(平成28年)}
\subsection*{
  物理学I
}
\subsection*{
  第1問
}
\subsubsection*{
  [1.1]
}
\begin{align}
  kv₀t₀ = μmg,␣
  t₀ = ÷{μmg}{kv₀}
\end{align}
\subsubsection*{
  [1.2]
}
\begin{align}
  m\"x_B(t) = k(v₀(t+t₀)-2x_B(t))-÷{2}{3}μmg
  = -2kx_B(t) + kv₀t + ÷{1}{3}kv₀t₀
\end{align}

\subsubsection*{
  [1.3]
}
運動方程式は
\begin{align}
  m\𝚍^2_t(x_B(t)-÷{v₀t}{2}-÷{v₀t₀}{6})
  = -2k(x_B(t)-÷{v₀t}{2}-÷{v₀t₀}{6})
\end{align}
と書ける。
よって$x_B(0)=\.x_B(0)=0$から、$ω = √{2k/m}$とおくと
\begin{align}
  x_B(t) = ÷{v₀t}{2} + ÷{v₀t₀}{6}
  - ÷{v₀t₀}{6}\cosωt
  -÷{v₀}{2ω}\sinωt
\end{align}
となる。
\subsubsection*{
  [1.4]
}
$ωt₀ ≪ 1$のとき、
\begin{align}
  x_B(t₀) &
  = ÷{v₀}{2}(t₀-÷{\sinωt₀}{ω}) + ÷{v₀t₀}{6}(1-\cosωt₀)\∅
  &
  = ÷{v₀t₀}{12}ω²t₀² \∅
  &
  = ÷{kv₀t₀³}{6m}
\end{align}
となる。
$t=t₀$での$A,B$間の力は
\begin{align}
  kx_B(t₀) = ÷{μmg}{12}ω²t₀² ≪ μmg
\end{align}
となり、$A$は静止している。
\subsubsection*{
  [2.1]
}
\begin{align}
  \𝚍^2{U}{x} = k₁[1-+3(÷{x}{l})²]
\end{align}
より、$x=0$まわりでばね定数$-k₁$、$x=±l$でばね定数$2k₁$の復元力が働く。
ここから運動方程式を書き下すと、
\begin{align}&
  m\𝚍^2_t⦅q_A\\q_B\\q_C⦆
  = ⦅
    -k₀-2k₁&k₀&0\\
    k₀&-2k₀+k₁&k₀\\
    0&k₀&-k₀-2k₁
  ⦆⦅q_A\\q_B\\q_C⦆
\end{align}
となる。
次にこれを対角化するが、系の対称性$A ↔ C$に注意する。
ここから固有ベクトルは
\begin{align}
  ⦅1\\0\\-1⦆,␣⦅a\\b\\a⦆
\end{align}
という形を取る。1つ目の固有値は$-k₀$である。
次に2つ目に対して、
\begin{align}
  ⦅
    -k₀-2k₁&k₀&0\\
    k₀&-2k₀+k₁&k₀\\
    0&k₀&-k₀-2k₁
  ⦆⦅a\\b\\a⦆
  = ⦅(-k₀-2k₁)a+k₀b\\2k₀a+(-2k₀+k₁)b\\(-k₀-2k₁)a+k₀b⦆
\end{align}
から、ブロック対角行列
\begin{align}
  ⦅
    -k₀-2k₁&k₀\\
    2k₀&-2k₀+k₁
  ⦆
\end{align}
の固有値を求めればよい。この固有値は
\begin{align}
  λ²+(3k₀+k₁)λ+(3k₀k₁-2k₁²)=0,␣
  λ = ÷{3k₀+k₁±√{9k₀²-6k₀k₁+9k₁²}}{2}
\end{align}
となる。以上により固有振動数は
\begin{align}
  ω = √{÷{k₀}{m}},␣√{÷{3k₀+k₁±√{9k₀²-6k₀k₁+9k₁²}}{2m}}
\end{align}
\subsubsection*{
  [2.2]
}
振動は不安定になりうるのは
\begin{align}
  ω = √{÷{3k₀+k₁-√{9k₀²-6k₀k₁+9k₁²}}{2m}}
\end{align}
である。その条件は$ω² < 0$であり、したがって
\begin{align}
  (3k₀+k₁)²-(9k₀²-6k₀k₁+9k₁²)
  = 12k₀k₁ - 8k₁² < 0.
\end{align}
よって
\begin{align}
  k₁ > ÷{3}{2}k₀ = k_c
\end{align}
\subsubsection*{
  [2.3]
}
$k₁ = 4k_c/9 = 2k₀/3$のとき、
\begin{align}
  ω² = ÷{k₀}{2m}⋅(3+÷{2}{3}-√{9-4+4}) = ÷{k₀}{3m}
\end{align}
である。また[2.1]のブロック対角行列は
\begin{align}
  k₀⦅
    -7/3&1\\
    2&-4/3
  ⦆
\end{align}
となる。この行列の固有値$-k₀/3$に対応する固有ベクトルは
$(a,b)=(1,2)$で与えられる。よって
\begin{align}
  q_A:q_B:q_C = 1:2:1
\end{align}
\newpage
\subsection*{
  第2問
}
\subsubsection*{
  [1]
}
\begin{align}
  ϕ(P) = ÷{q}{4𝜋ε₀}(÷{1}{r₊}-÷{1}{r₋})
\end{align}
\subsubsection*{
  [2]
}
\begin{align}
  ÷{1}{r_±} = ÷{1}{r}(1∓÷{d\cosθ}{r}+÷{d²}{4r²})^{-1/2}
  = ÷{1}{r} ± ÷{d\cosθ}{2r²}
\end{align}
より、
\begin{align}
  ϕ(P) = ÷{qd}{4𝜋ε₀}÷{\cosθ}{r²}
\end{align}
\subsubsection*{
  [3]
}
\begin{align}
  𝑬 = -∇ϕ
  &
  = -÷{qd}{4𝜋ε₀}(𝒆_r\∂_r+÷{𝒆_θ}{r}\∂_θ)÷{\cosθ}{r²}\∅
  &
  = ÷{p}{4𝜋ε₀r³}(2\cosθ𝒆_r+\sinθ𝒆_θ)
\end{align}
\subsubsection*{
  [4]
}
$𝒑₁ ∝ 𝒆_z$とする。

(a) $(θ₁,θ₂)=(𝜋/2,0)$のとき、$𝒑₂$には$𝒆_θ = -𝒆_z$方向の電場がかかり、時計回りに回転する。

(b) $(θ₁,θ₂)=(𝜋/2,-𝜋/2)$のとき、$𝒑₂$には$𝒆_θ = -𝒆_z$方向の電場がかかり、安定釣り合いの状態で回転しない。
\subsubsection*{
  [5]
}
\begin{align}
  U &= -÷{p₁}{4𝜋ε₀l³}𝒑₂⋅(2\cosθ₁𝒆_r+\sinθ₁𝒆_{θ₁})\∅
  &
  =  -÷{p₁p₂}{4𝜋ε₀l³}(2\cosθ₁\cosθ₂-\sinθ₁\sinθ₂)\∅
  &
  =  -÷{p₁p₂}{8𝜋ε₀l³}(3\cos(θ₁+θ₂)+\cos(θ₁-θ₂))
\end{align}
\subsubsection*{
  [6]
}
$θ₁+θ₂,θ₁-θ₂$が独立なことから、
\begin{align}
  θ₁+θ₂ = 2m𝜋
  θ₁-θ₂ = 2n𝜋,
  m,n ∈ ℤ
\end{align}
のときに最も安定。よって
\begin{align}
  (θ₁,θ₂) = (0,0),(𝜋,𝜋)
\end{align}
このとき
\begin{align}
  U = -÷{p₁p₂}{2𝜋ε₀l³}
\end{align}
\subsubsection*{
  [7]
}
問[5]の結果で$p₁=-p₂=p,l = 2h,θ₁=𝜋-α,θ₂=α$とすればよい。すなわち、
\begin{align}
  U' = ÷{p²}{64𝜋ε₀h³}(3\cos𝜋+\cos(𝜋-2α))
  = -÷{p²}{64𝜋ε₀h³}(\cos(2α)+3)
\end{align}
\subsubsection*{
  [8]
}
$\cos(2α)=1$のときに最も安定。よって
\begin{align}
  α = 0,𝜋
\end{align}

\newpage
\subsection*{
  物理学II
}
\subsection*{
  第1問
}
\subsubsection*{
  [1]
}
ポテンシャルから境界条件
\begin{align}
  ψ(a/2) = ψ(-a/2) = 0
\end{align}
が課される。一方、Schrödinger方程式から
\begin{align}
  ψ(x) = A\sin(kx)+B\cos(kx),␣(A,B = \𝚞{const.})
\end{align}
となるので、規格化された固有関数は
\begin{align}
  ψ_n(x) = \begin{cases}
    √{÷{2}{a}}\cos(÷{n𝜋x}{a}) & (n = 1,3,5,…)\\
    √{÷{2}{a}}\sin(÷{n𝜋x}{a}) & (n = 2,4,6,…)
  \end{cases}
\end{align}
となる。エネルギー固有値は
\begin{align}
  E_n = ÷{ħ²}{2m}(÷{n𝜋}{a})²,␣(n = 1,2,3,…)
\end{align}
で与えられる。
\subsubsection*{
  [2]
}
あり得る波動関数は以下の4つ。
\begin{gather}
  g(x₁)g(x₂)\\
  e(x₁)e(x₂)\\
  ÷{1}{√2}(g(x₁)e(x₂)+e(x₁)g(x₂))\\
  ÷{1}{√2}(g(x₁)e(x₂)-e(x₁)g(x₂))
\end{gather}
\subsubsection*{
  [3]
}
まず$S=1$となる固有状態は以下の3つがある。
\begin{align}
  |↑⟩₁|↑⟩₂,␣
  |↓⟩₁|↓⟩₂,␣
  ÷{1}{√2}(|↑⟩₁|↓⟩₂+|↓⟩₁|↑⟩₂)
\end{align}
次に$S=0$となる固有状態は以下の1つだけである。
\begin{align}
  ÷{1}{√2}(|↑⟩₁|↓⟩₂-|↓⟩₁|↑⟩₂)
\end{align}
これ以外の基底はない。
\subsubsection*{
  [4]
}
まず軌道が対称でスピンが反対称な固有状態として、以下の3つがある。
\begin{align}
  &
  Ψ₁ = ÷{1}{√2}g(x₁)g(x₂)(|↑⟩₁|↓⟩₂-|↓⟩₁|↑⟩₂),\\
  &
  Ψ₂ = ÷{1}{√2}e(x₁)e(x₂)(|↑⟩₁|↓⟩₂-|↓⟩₁|↑⟩₂),\\
  &
  Ψ₃ = ÷{1}{2}(g(x₁)e(x₂)+e(x₁)g(x₂))(|↑⟩₁|↓⟩₂-|↓⟩₁|↑⟩₂).
\end{align}
それぞれのエネルギー固有値は、
\begin{align}
  E₁ = ÷{ħ²𝜋²}{ma²},␣
  E₂ = ÷{4ħ²𝜋²}{ma²},␣
  E₃ = ÷{5ħ²𝜋}{2ma²}
\end{align}
となる。
次に軌道が反対称でスピンが対称な固有状態として、以下の3つがある。
\begin{align}
  &
  Ψ₄ = ÷{1}{√2}(g(x₁)e(x₂)-e(x₁)g(x₂))|↑⟩₁|↑⟩₂,\\
  &
  Ψ₅ = ÷{1}{√2}(g(x₁)e(x₂)-e(x₁)g(x₂))|↓⟩₁|↓⟩₂\\
  &
  Ψ₆ = ÷{1}{2}(g(x₁)e(x₂)-e(x₁)g(x₂))(|↑⟩₁|↓⟩₂+|↓⟩₁|↑⟩₂)
\end{align}
それぞれのエネルギー固有値は、
\begin{align}
  E₄=E₅=E₆=÷{5ħ²𝜋²}{2ma²}
\end{align}
となる。
\subsubsection*{
  [5]
}
$g(x)$が偶関数、$e(x)$が奇関数であることに注意すると、
\begin{align}
  ∫_{-∞}^{∞}xg(x)²\𝑑{x}
  = ∫_{-∞}^{∞}xe(x)²\𝑑{x}
  = ∫_{-∞}^{∞}x²e(x)g(x)\𝑑{x}
  = 0
\end{align}
が言える。このことに気をつけると、$⟨(x₁-x₂)²⟩$は以下のように計算できる。
\begin{align}
  &
  ⟨Ψ₁|(x₁-x₂)²|Ψ₁⟩
  = 2A, \\
  &
  ⟨Ψ₂|(x₁-x₂)²|Ψ₂⟩
  = 2B.
\end{align}
次に、
\begin{align}
  &
  ⟨Ψ₃|(x₁-x₂)²|Ψ₃⟩\∅
  &
  = ÷{1}{2}∫\𝑑{x₁}\𝑑{x₂}(x₁-x₂)²
    (g(x₁)²e(x₂)²+2g(x₁)e(x₁)g(x₂)e(x₂)+e(x₁)²g(x₂)²) \∅
  &
  = ÷{1}{2}(A+B-4C²+A+B)\∅
  &
  = A+B-2C².
\end{align}
$Ψ₄,Ψ₅,Ψ₆$についてはスピン部分が期待値に寄与しないので、
\begin{align}
  &
  ⟨Ψ₄|(x₁-x₂)²|Ψ₄⟩ = ⟨Ψ₅|(x₁-x₂)²|Ψ₅⟩ = ⟨Ψ₆|(x₁-x₂)²|Ψ₆⟩\∅
  &
  =  ÷{1}{2}∫\𝑑{x₁}\𝑑{x₂}(x₁-x₂)²
  (g(x₁)²e(x₂)²-2g(x₁)e(x₁)g(x₂)e(x₂)+e(x₁)²g(x₂)²)\∅
  &
  = A+B+2C².
\end{align}
\subsubsection*{
  [6]
}
1次の摂動エネルギーは
\begin{align}
  ΔE = ⟨Ψ|V₀x₁x₂|Ψ⟩
\end{align}
と表される。これを各状態について求めると、
\begin{align}
  &
  ⟨Ψ₁|V₀x₁x₂|Ψ₁⟩ = ⟨Ψ₂|V₀x₁x₂|Ψ₂⟩ = 0 \∅
  &
  ⟨Ψ₃|V₀x₁x₂|Ψ₃⟩ = V₀C² \∅
  &
  ⟨Ψ₄|x₁x₂|Ψ₄⟩ = ⟨Ψ₅|x₁x₂|Ψ₅⟩ = ⟨Ψ₆|x₁x₂|Ψ₆⟩ = -V₀C²
\end{align}
となる。したがって、$Ψ₁,…,Ψ₆$のエネルギーは、
\begin{align}
  ÷{ħ²𝜋²}{ma²},÷{4ħ²𝜋²}{ma²},÷{5ħ²𝜋²}{2ma²}+V₀C²,
  ÷{5ħ²𝜋²}{2ma²}-V₀C²,÷{5ħ²𝜋²}{2ma²}-V₀C²,÷{5ħ²𝜋²}{2ma²}-V₀C²
\end{align}
となる。
\newpage
\subsection*{
  第2問
}
\subsubsection*{
  [1]
}
終点の座標$X$は、
\begin{align}
  X = a(n⁺-n⁻)
\end{align}
と表される。
分子が取りうる状態の数は$2^n$である。
また$X = na$のとき、分子が取りうる状態の数は$N!/n⁺!n⁻!$である。
\begin{align}
  n⁺ = ÷{N+n}{2},␣
  n⁻ = ÷{N-n}{2}
\end{align}
なので、$X=na$となる確率$P(n)$は
\begin{align}
  P(n) =  ÷{1}{2^n}÷{N!}{((N+n)/2)!((N-n)/2)!}
\end{align}
で与えられる。
分子のエントロピーは、
\begin{align}
  S = 𝘬\ln 2^n = n𝘬\ln 2
\end{align}
である。
\subsubsection*{
  [2]
}
終点を固定したとき、分子のエントロピーは
\begin{align}
  S &= 𝘬\ln(÷{N!}{n⁺!n⁻!})\∅
  &
  ≈ 𝘬(N\ln N - n⁺\ln n⁺ - n⁻ \ln n⁻)\∅
  &
  = 𝘬n⁺\ln÷{N}{n⁺} + 𝘬n⁻\ln÷{N}{n⁻}
\end{align}
となる。
全ての状態のエネルギーが等しいので、内部エネルギーは定数であり、
エントロピーは温度依存しない。
これらのことに注意すると、
\begin{align}
  τ &= (\∂{F}{X})_T = -T\∂{S}{X}\∅
  &
  = -T(\∂{n⁺}{X}\∂{S}{n⁺}+\∂{n⁻}{X}\∂{S}{n⁻})\∅
  &
  = -÷{𝘬T}{2a}(\ln÷{N}{n⁺}-\ln÷{N}{n⁻})\∅
  &
  = ÷{𝘬T}{2a}\ln ÷{n⁺}{n⁻}\∅
  &
  = ÷{𝘬T}{2a}\ln ÷{aN+X}{aN-X}.
\end{align}
また$X ≪ Na$のとき、
\begin{align}
  τ = ÷{𝘬TX}{a²N}
\end{align}
\subsubsection*{
  [3]
}
$β = 1/𝘬T$とおく。分配関数は、
\begin{align}
  Z(β) = ∑_{n⁻=0}^{N} ÷{N!}{n⁻!(N-n⁻)!}ℯ^{βκ(N-n⁻)}ℯ^{-βκn⁻}
  = (ℯ^{βκ}+ℯ^{-βκ})^N
\end{align}
である。ここから$⟨E⟩,⟨X⟩$は、
\begin{align}
  &
  ⟨E⟩ = -\∂_β\ln Z(β) = -Nκ\tanh÷{κ}{𝘬T}\∅
  &
  ⟨X⟩ = -÷{a}{κ}⟨E⟩ = Na\tanh÷{κ}{𝘬T}
\end{align}
と表される。次に$|κ| ≪ 𝘬T$のとき、
\begin{align}
  ⟨E⟩ = -÷{Nκ²}{𝘬T},␣
  ⟨X⟩ = ÷{Naκ}{𝘬T}
\end{align}
となる。
\subsubsection*{
  [4]
}
$⟨E²⟩=Z''(β)/Z(β)$から、
\begin{align}
  \∂^2_β \ln Z(β) = ÷{Z''(β)}{Z(β)}-÷{Z'(β)²}{Z(β)²}
  = ⟨E²⟩ - ⟨E⟩²
\end{align}
となる。よって
\begin{align}
  ⟨E²⟩-⟨E⟩² = ÷{Nκ²}{\cosh²(κ/𝘬T)}
\end{align}
となる。ここから
\begin{align}
  ⟨X²⟩-⟨X⟩² = ÷{a²}{κ²}(⟨E²⟩-⟨E⟩²) = ÷{Na²}{\cosh²(κ/𝘬T)}
\end{align}
となる。
\newpage
\subsection*{
  第3問
}
\subsubsection*{
  [1]
}
電子の運動方程式は、
\begin{align}
  m\"𝒖 = -e𝑬\ex -e\.𝒖×𝑩\ex - mω₀²𝒖
\end{align}
である。$u_x,u_y$についての方程式は
\begin{align}
  -mω²u_x = -eE_x + ¡eωu_yB - mω₀²u_x \\
  -mω²u_y = -eE_y - ¡eωu_xB - mω₀²u_x
\end{align}
となる。
\begin{align}
  ⦅u_x\\u_y⦆
  = ⦅m(ω²-ω₀²)&¡eω\\-¡eω&m(ω²-ω₀²)⦆^{-1}⦅eE_x\\eE_y⦆
\end{align}
から、
% ここで、$u_± = u_x±¡u_y,~E_± = E_x±¡E_y$とおくと、
% \begin{align}
%   -mω²u_± = -eE_± ± eωBu_± - mω₀²u_±
% \end{align}
% となる。よって、
% \begin{align}
%   u_± = ÷{eE_±}{m(ω²-ω₀²) ± eωB}
% \end{align}
% となる。
$u_x,u_y$は、
\begin{align}
  &
  u_x = ÷{u₊+u₋}{2}
  = ÷{em(ω²-ω₀²)E_x-¡e²ωBE_y}{m²(ω²-ω₀²)²-e²ω²B²},\∅
  &
  u_y = ÷{u₊-u₋}{2¡}
  = ÷{¡e²ωBE_x + em(ω²-ω₀²)E_y}{m²(ω²-ω₀²)²-e²ω²B²}.
\end{align}
\subsubsection*{
  [2]
}
$\~ε𝑬\ex - 𝑬\ex = -ne𝒖/ε₀$から、
\begin{align}
  &
  ε_{xx} = 1-÷{ne²m(ω²-ω₀²)/ε₀}{m²(ω²-ω₀²)²-e²ω²B²}\∅
  &
  γ = ÷{ne³ωB/ε₀}{m²(ω²-ω₀²)²-e²ω²B²}
\end{align}
となる。
\begin{align}
  m²(ω₀²-ω²)² - e²ω²B²
  &
  = m²ω₀⁴((1-÷{ω²}{ω₀²})² - ÷{ω²}{ω₀²}÷{e²B²}{m²})\∅
  &
  ≈ m²ω₀⁴ > 0
\end{align}
から、
\begin{align}
  ε_{xx}  > 1,␣γ > 0
\end{align}
である。
\subsubsection*{
  [3]
}
\begin{align}
  𝒌×(𝒌×𝑬₀) = (𝑬₀⋅𝒌)𝒌 - |𝒌|²𝑬₀
  = -μ₀ε₀\~εω²𝑬₀
\end{align}
よって、
\begin{align}
  (𝑬₀⋅𝒌)𝒌 - |𝒌|²𝑬₀ + (÷{ω}{c})²\~ε𝑬₀ = 0.
\end{align}
\subsubsection*{
  [4]
}
\begin{align}
  &
  -k_z²E_{0,x} + (÷{ω}{c})²(ε_{xx}E_{0,x}+¡γE_{0,y}) = 0 \∅
  &
  -k_z²E_{0,y} + (÷{ω}{c})²(-¡γE_{0,x}+ε_{xx}E_{0,x}) = 0
\end{align}
から
\begin{align}
  E_{0,x}² + E_{0,y}² = 0
\end{align}
となる。
よって$E_{0,y}/E_{0,x} = ±¡$であり、
\begin{align}
  k_± = ÷{ω}{c}√{ε_{xx}±γ}
\end{align}
となる。
また、
\begin{align}
  E_{0,z}k_z² - k_z²E_{0,z} + (÷{ω}{c})²ε_{z,z}E_{0,z} = 0
\end{align}
から$E_{0,z} = 0$となるので、
\begin{align}
  𝑬_{0±} = ÷{E}{√2}⦅1\\∓¡\\0⦆
\end{align}
\subsubsection*{
  [5]
}
\begin{align}
  &
  \Re E_{+,x}(z=0,t) = ÷{E}{√2}\cos(ωt),\\
  &
  \Re E_{+,y}(z=0,t) = -÷{E}{√2}\sin(ωt)
\end{align}
\subsubsection*{
  [6]
}
\begin{align}
  ⦅E\\0\\0⦆ = ÷{𝑬₊ + 𝑬₋}{√2}
\end{align}
から、
\begin{align}
  𝑬(l,t)
  &
  = ÷{𝑬₊}{√2}ℯ^{¡(k₊l-ωt)} + ÷{𝑬₋}{√2}ℯ^{¡(k₋l-ωt)}\∅
  &
  =  [÷{𝑬₊}{√2}ℯ^{¡(k₊-k₋)l}+÷{𝑬₋}{√2}ℯ^{-¡(k₊-k₋)l}]
    \exp(¡÷{k₊+k₋}{2}l - ¡ωt)\∅.
\end{align}
よって、
\begin{align}
  𝑭 = ÷{𝑬₊}{√2}ℯ^{¡(k₊-k₋)l/2}+÷{𝑬₋}{√2}ℯ^{-¡(k₊-k₋)l/2}
  = ⦅E\cos((k₊-k₋)l/2)\\E\sin((k₊-k₋)l/2)\\0⦆
\end{align}
偏向面の回転角は、
\begin{align}
  θ = ÷{(k₊-k₋)l}{2}
\end{align}
\subsubsection*{
  [7]
}
反射する光に対しては$𝑩\ex → -𝑩\ex$として同様の議論をすれば、回転角が求まる。
このとき$γ → -γ,~k_± → k_∓$となることと、
回転角の符号の取り方が逆になることに注意すると、
反射するときの回転角は$θ$に等しい。
よって全体の回転角は、
\begin{align}
  2θ = (k₊-k₋)l
\end{align}
\newpage
\subsection*{
  第4問
}
\subsubsection*{
  [1]
}
\begin{align}
  E = eV = ÷{ħ²k²}{2m} = ÷{h²}{2mλ²}
\end{align}
よって
\begin{align}
  V = ÷{h²}{2meλ²}
  % = ÷{6.6²}{2×9.1×1.6}×10^4\si{V}
  = \SI{1.5e4}{V}
\end{align}
\subsubsection*{
  [2]
}
\begin{align}
  A(𝑲)
  = f∑_{n_x=0}^{N_x-1}ℯ^{-¡aK_xn_x}
      ∑_{n_y=0}^{N_y-1}ℯ^{-¡aK_yn_y}
\end{align}
よって
\begin{align}
  L(K,N) = ∑_{n=0}^{N-1}ℯ^{-¡aKn}
\end{align}
とすれば、
\begin{align}
  A(𝑲,N_x,N_y) = fL(K_x,N_x)L(K_y,N_y)
\end{align}
と表せる。
\begin{align}
  L(K,1) &= 1,\∅
  L(K,2) &= 2ℯ^{-¡aK/2}\cos(aK/2),\∅
  L(K,3) &= ℯ^{-¡aK}(1+2\cos(aK))
\end{align}
\subsubsection*{
  [3]
}
\begin{align}
  𝒂₁ = a⦅√3/2\\1/2⦆,␣
  𝒂₂ = a⦅-√3/2\\1/2⦆
\end{align}
\begin{align}
  𝒃₁ = ÷{2𝜋}{a}⦅1/√3\\1⦆,␣
  𝒃₂ = ÷{2𝜋}{a}⦅-1/√3\\1⦆
\end{align}
散乱の条件は、
\begin{align}
  𝒌'-𝒌 = m₁𝒃₁+m₂𝒃₂+c𝒆_z,␣ |𝒌| = |𝒌'|
\end{align}
である。散乱角$θ$は、
\begin{align}
  |m₁𝒃₁+m₂𝒃₂| = 2|𝒌|\sinθ = ÷{λ}{4𝜋}\sinθ
\end{align}
によって定義される。よって
\begin{align}
  θ = \sin^{-1}÷{λ|m₁𝒃₁+m₂𝒃₂|}{4𝜋}
\end{align}
となる。
$|m₁𝒃₁+m₂𝒃₂|$は小さい順から
\begin{align}
  ÷{4𝜋}{√3a},÷{4𝜋}{a},÷{8𝜋}{√3a},…
\end{align}
となるので、
\begin{align}
  θ₁ = \sin^{-1}÷{λ}{√3a},␣
  θ₂ = \sin^{-1}÷{λ}{a},␣
  θ₁ = \sin^{-1}÷{2λ}{√3a}
\end{align}
図示は省略する。
逆格子点を原点からの距離ごとに図示すれば良い。
\subsubsection*{
  [4]
}
\begin{align}
  𝑨(𝑲) = f(1+÷12ℯ^{-¡𝑲⋅(2𝒂₁+𝒂₂)/3})
      L(K_x,N_x)L(K_y,N_y)
\end{align}
ここで
\begin{align}
  ÷{2𝒂₁+𝒂₂}{3}⋅(m₁𝒃₁+m₂𝒃₂) = ÷{2𝜋}{3}(2m₁+m₂)
\end{align}
から、相対的な強度は
\begin{align}
  \𝚟|1+2ℯ^{-¡𝑲⋅(2𝒂₁+𝒂₂)/3}|²
  = 5+4\cos(÷{2𝜋}{3}(2m₁+m₂))
\end{align}
となる。よって
\begin{align}
  \cos 0 = 1, ␣ \cos÷{2𝜋}{3} = \cos÷{4𝜋}{3} = -÷{1}{2}
\end{align}
から$θ₁,θ₂,θ₃$のピークの強度比は$3:9:3 = 1:3:1$となる。
\subsubsection*{
  [5]
}
各層からの散乱光が打ち消し合うため。
このときピークが現れる$c$が離散化される。
積層面を傾けていくことはビームを傾けることと等価であり、
このとき$c$が変化していくが、
$c$が離散化された値に横切るときにピークが復活する。
\end{document}