\providecommand{\main}{../main}
\documentclass[\main/main.tex]{subfiles}
\graphicspath{{../images/}}
\begin{document}
\newpage
\section{2021年(令和3年)}
\subsection*{
  第1問
}
\subsubsection*{
  [1.1]
}
\begin{align}
  \^H = -μ⦅B₀&B₁ℯ^{-¡ωt}\\B₁ℯ^{¡ωt}&-B₀⦆\
  = ÷{ħ}{2}⦅ω₀&ω₁ℯ^{-¡ωt}\\ω₁ℯ^{¡ωt}&-ω₀⦆
\end{align}
である。また$α(t),β(t)$は
\begin{align}
  ¡\𝚍_t⦅α(t)\\β(t)⦆
  = ÷{1}{2}⦅ω₀&ω₁ℯ^{-¡ωt}\\ω₁ℯ^{¡ωt}&-ω₀⦆⦅α(t)\\β(t)⦆
\end{align}
を満たす。
\subsubsection*{
  [1.2]
}
$a(t)=α(t)ℯ^{¡ω₀t/2},~b(t)=β(t)ℯ^{-¡ω₀t/2}$に対し、
\begin{align}
  ¡\𝚍_t⦅a(t)\\b(t)⦆
  &
  = ÷{1}{2}⦅ℯ^{¡ω₀t/2}&0\\0&ℯ^{-¡ω₀t/2}⦆
  ⦅0&ω₁ℯ^{-¡ωt}\\ω₁ℯ^{¡ωt}&0⦆⦅α(t)\\β(t)⦆\∅
  &
  = ÷{ω₁}{2}⦅
    0&ℯ^{-¡(ω-ω₀)t}\\
    ℯ^{¡(ω-ω₀)t}&0
  ⦆⦅a(t)\\b(t)⦆
\end{align}
よって、
\begin{align}
  A(t) = ℯ^{-¡(ω-ω₀)t}.
\end{align}
\subsubsection*{
  [2.1]
}
$b(t) = 1$と近似すると、
\begin{align}
  \𝚍{a}{t} = -÷{¡ω₁}{2}ℯ^{-¡(ω-ω₀)t}
\end{align}
となり、$a(0)=0$から
\begin{align}
  a(t) = ÷{ω₁}{2(ω-ω₀)}[ℯ^{-¡(ω-ω₀)t}-1]
\end{align}
となる。よって、
\begin{align}
  P₁ = |a(τ)|² &= ÷{ω₁²}{(ω-ω₀)²}\sin²÷{(ω-ω₀)τ}{2}
\end{align}
\subsubsection*{
  [2.2]
}
$P₁(ω)$は$ω = ω₀$で最大値$ω₁²τ²/4$をとる。
またこの近傍で$P₁(ω)=0$となる角振動数は
\begin{align}
  \sin(÷{(ω-ω₀)τ}{2}) = 0,␣ ω = ω₀ ± ÷{2𝜋}{τ}
\end{align}
で与えられる。
\subsubsection*{
  [3.1]
}
問[2.1]と同様な議論から、$T < t < T+τ$において
\begin{align}
  a(t) = ÷{ω₁}{2(ω-ω₀)}[ℯ^{-¡(ω-ω₀)t}-C]
\end{align}
と書ける。$C$は定数である。
ここで、磁場$𝑩₀$を印加したとき$a(t),b(t)$が不変なことから
\begin{align}
  a(T) = a(τ) = ÷{ω₁}{2(ω-ω₀)}[ℯ^{-¡(ω-ω₀)τ}-1]
\end{align}
となる。よって
\begin{align}
  a(t) &
  = ÷{ω₁}{2(ω-ω₀)}[
    ℯ^{-¡(ω-ω₀)t}-ℯ^{-¡(ω-ω₀)T}+ℯ^{-¡(ω-ω₀)τ}-1
  ]
\end{align}
となる。
\begin{align}
  a(T+τ) = ÷{ω₁}{2(ω-ω₀)}(ℯ^{-¡(ω-ω₀)T}+1)(ℯ^{-¡(ω-ω₀)τ}-1)
\end{align}
から、
\begin{align}
  P₂ = |a(T+τ)|²
  = ÷{4ω₁²}{(ω-ω₀)²}\cos²÷{(ω-ω₀)T}{2}\sin²÷{(ω-ω₀)τ}{2}
\end{align}
\subsubsection*{
  [3.2]
}
$P₂(ω)$は$ω=ω₀$で最大となる。
また$ω=ω₀$の近傍で$P₂(ω)=0$となるのは、$T ≫ τ$に注意すると、
\begin{align}
  \cos÷{(ω-ω₀)T}{2} = 0
\end{align}
のとき。よって、
\begin{align}
  Ω₁ = ω₀-÷{𝜋}{T},␣
  Ω₂ = ω₀+÷{𝜋}{T}
\end{align}
となる。
\subsubsection*{
  [3.3]
}
\begin{align}
  𝛥Ω = Ω₂ - Ω₁ = ÷{2𝜋}{T}
\end{align}
精度を改善するには、$T$を大きくすれば良い。
\subsubsection*{
  [4.1]
}
\begin{align}
  ÷{2𝜋}{T} = ÷{λL}{l},␣
  ÷{4𝜋}{τ} = ÷{2λL}{D}
\end{align}
と対応づけられる。よって$l$が$T$に、$D$が$τ$に対応する。
\subsubsection*{
  [4.2]
}
時刻$t$で測定すると、ほとんど$|-⟩$が測定される。
よって時刻$t$においてを$|-⟩$を初期状態としてよい。
このとき位相を除いて問[2.2]と同じ結果になる。
つまり干渉が消える。

\newpage
\subsection*{
  第2問
}
\subsubsection*{
  [1.1]
}
\begin{align}
  m\"x_n = -k(2x_n-x_{n+1}-x_{n-1})
\end{align}
\subsubsection*{
  [1.2]
}
$x_n(t) = ℯ^{¡qn}c_q(t)$を[1.1]の結果に代入すると、
\begin{align}
  \"c_q(t) = -÷{2k}{m}(1-\cos q)c_q
  = -÷{4k}{m}\sin²÷{q}{2}
\end{align}
となる。
\subsubsection*{
  [1.3]
}
\begin{align}
  ω_q = √{÷{4k}{m}}\abs{\sin÷{q}{2}}
\end{align}
\subsubsection*{
  [1.4]
}
$q=0$のとき、$c_q(t)$についての微分方程式およびその解は、
\begin{align}
  \"c₀(t) = 0,␣ c₀(t) = At+B
\end{align}
となる。ただし$A,B$は定数である。
\subsubsection*{
  [2.1]
}
$|n| > 0$については[1.1]と同様。すなわち、
\begin{align}
  m\"x_n = -k(2x_n-x_{n+1}-x_{n-1})
\end{align}
となる。$n=0$については、
\begin{align}
  M\"x₀ = -k(2x₀-x₁-x₋₁)
\end{align}
となる。
\subsubsection*{
  [2.2]
}
$n ≤ -2$あるいは$n ≥ +1$に対しては、運動方程式は[1.1]と同様である。
これに$x_n(t) = ℯ^{¡qn-¡ω_qt}$を代入すると、[1.3]と同様な議論から
\begin{align}
  ω_q = √{÷{4k}{m}}\abs{\sin÷{q}{2}}
\end{align}
となる。また$x_n(t) = ℯ^{-¡qn-¡ω_qt}$を代入しても同じ結果を得る。
さらに$x_n(t)$としてこれらの線形結合をとっても問題ない。
よって(1)式に対して$ω_q$の表式は[1.3]と同じである。
\subsubsection*{
  [2.3]
}
$n=0,-1$に対する運動方程式
\begin{align}&
  -M\"x₀ = -k(2x₀-x₁-x₋₁)\\
  &
  -m\" x₋₁ = -k(2x₋₁-x₀-x₋₂)
\end{align}
に(1)式を代入すると、
\begin{align}&
  (2k-Mω_q²)T_q = k(ℯ^{¡q}T_q+ℯ^{-¡q}+R_qℯ^{¡q})\\
  &
  (2k-mω_q²)(ℯ^{-¡q}+R_qℯ^{¡q}) = k(T_q+ℯ^{-2¡q}+R_qℯ^{2¡q})
\end{align}
となる。整理すると、
\begin{align}&
  (2-ℯ^{¡q}-÷{4M}{m}\sin²÷{q}{2})T_q - ℯ^{¡q}R_q
  = ℯ^{-¡q},\\
  &
  T_q - R_q = 1
\end{align}
となる。
\subsubsection*{
  [2.4]
}
[2.3]の2つ目の式から$R_q = T_q-1$となる。
これを1つ目の式に代入すると、
\begin{align}
  (2-2ℯ^{¡q}-÷{4M}{m}\sin²÷{q}{2})T_q = ℯ^{-¡q}-ℯ^{¡q}
\end{align}
よって、
\begin{align}
  T_q &
  = ÷{ℯ^{-¡q}-ℯ^{¡q}}{2-2ℯ^{¡q}-(M/m)(2-ℯ^{¡q}-ℯ^{-¡q})}\∅
  &
  = ÷{ℯ^{-¡q}-ℯ^{¡q}}{2(1-M/m)+(M/m)ℯ^{-¡q}-(2-M/m)ℯ^{¡q}}
\end{align}
\subsubsection*{
  [2.5]
}
$M=m$では
\begin{align}
  T_q = ÷{ℯ^{-¡q}-ℯ^{¡q}}{ℯ^{-¡q}-ℯ^{¡q}} = 1
\end{align}
これは$M=m$では波が全て透過し、反射が起こらないことを表している。
次に、$M=+∞$では、
\begin{align}
  T_q = 0
\end{align}
となる。これは透過が起こらず、全反射が起こることを表している。
\subsubsection*{
  [3.1]
}
\begin{align}
  \det[
    ⦅T_q&R_q\\R_q&T_q⦆-ℯ^{-¡qL}⦅1&0\\0&1⦆
  ] = 0
\end{align}
となるから、
\begin{align}
  &
  ℯ^{-2¡qL} - 2T_qℯ^{-¡qL} + T_q²-R_q²\∅
  &
  = (ℯ^{-¡qL}-T_q-R_q)(ℯ^{-¡qL}-T_q+R_q)= 0
\end{align}
となる。
よって、
\begin{align}
  ℯ^{-¡qL} = T_q ± R_q = 2T_q-1,1
\end{align}
\subsubsection*{
  [3.2]
}
$M=m$の場合、$T_q=1,~R_q=0$となる。このとき、
\begin{align}
  ℯ^{-¡qL} = 1
\end{align}
から
\begin{align}
  q = ÷{2l𝜋}{L},␣ l ∈ ℤ
\end{align}
となる。このとき、
\begin{align}
  ω_q = √{÷{4k}{m}}\abs{\sin÷{l𝜋}{L}}
\end{align}
\subsubsection*{
  [3.3]
}
$M=2m$の場合、
\begin{align}
  T_q = ÷{ℯ^{-¡q}-ℯ^{¡q}}{-2+2ℯ^{-¡q}}
\end{align}
となる。よって、
\begin{align}
  ℯ^{¡qL} = 1,
\end{align}
または、
\begin{align}
  ℯ^{¡qL} = 2T_q-1 = ÷{1-ℯ^{¡q}}{-1+ℯ^{-¡q}} = ℯ^{¡q}.
\end{align}
ここから
\begin{align}
  q = ÷{2l𝜋}{L},÷{2lπ}{L-1},␣ l ∈ ℤ
\end{align}
となる。また、
\begin{align}
  ω_q = √{÷{4k}{m}}\abs{\sin÷{l𝜋}{L}},␣√{÷{4k}{m}}\abs{\sin÷{l𝜋}{L-1}}
\end{align}
である。$0 < q < 𝜋$から基準振動の総数は、
\begin{align}
  0 < l < ÷{L}{2},␣ 0 < l < ÷{L-1}{2}
\end{align}
となる整数$l$の数の和で与えられる。よって
\begin{align}
  (÷{L}{2}-1)+(÷{L}{2}-1) = L-2.
\end{align}
これは$L$よりも2個モードが少ない。
1個のモードは並進運動の自由度を表す。
もう1個のモードは$n=0$の質点のまわりに束縛されている。
\subsubsection*{
  [3.4]
}
$M → ∞$の場合、固定端の問題と同じ。
$T_q=0$から、
\begin{align}
  ℯ^{¡qL} = ±1
\end{align}
よって
\begin{align}
  q = ÷{l𝜋}{L},␣ l ∈ ℤ
\end{align}
であり、振動モードの数は$L$個。(固定端だから並進運動できない。)
また、
\begin{align}
  ω_q = √{÷{4k}{m}}\abs{\sin÷{l𝜋}{2L}}
\end{align}

\newpage
\subsection*{
  第3問
}
\subsubsection*{
  [1]
}
\begin{align}
  N_{AB} = N_A⋅z⋅÷{N_B}{N} = Nzx(1-x)
\end{align}
\begin{align}
  E = NVzx(1-x)
\end{align}
\subsubsection*{
  [2]
}
\begin{align}
  S &= 𝘬\log÷{N!}{N_A!N_B!}\∅
  &
  = -𝘬 N_A\log ÷{N_A}{N} - 𝘬 N_B \log ÷{N_B}{N}\∅
  &
  = -𝘬N[x\log x + (1-x)\log(1-x)]
\end{align}
\subsubsection*{
  [3]
}
\begin{align}
  F = E-TS = NVzx(1-x) + 𝘬NT[x\log x + (1-x)\log(1-x)]
\end{align}
\subsubsection*{
  [4]
}
\begin{align}
  f(x) = Vzx(1-x)+𝘬T[x\log x + (1-x)\log(1-x)]
\end{align}
から、$f(x)$極値をとる$x$では
\begin{align}
  f'(x) = -Vz(2x-1) + 𝘬T(\log x -\log(1-x)) = 0
\end{align}
となる。よって、
\begin{align}
  ÷{zV}{𝘬T}(2x-1) = g(x) = \log÷{x}{1-x}
\end{align}
\subsubsection*{
  [5]
}
両辺の$x=1/2$における傾きが一致するとき、
\begin{align}
  ÷{2zV}{𝘬T} = [÷{1}{x}+÷{1}{1-x}]_{x=1/2} = 4
\end{align}
となる。よって、
\begin{align}
  T_{c0} = ÷{zV}{2𝘬}
\end{align}
\subsubsection*{
  [6]
}
両辺はどちらも$x=1/2$を軸として奇関数になっている。
また$g(0)=-g(1)=∞$である。
$T<T_{c0}$では(2)式の解が3つ現れ、$T>T_{c0}$では(2)式の解は1つだけである。
これらを図示すれば良い。
\subsubsection*{
  [7]
}
\begin{align}
  f''(x) &= -2zV + 𝘬T[÷{1}{x}+÷{1}{1-x}] \∅
  &
  = -2zV + ÷{𝘬T}{x-x²} = 0
\end{align}
から、
\begin{align}
  x² - x + ÷{𝘬T}{2zV} = 0.
\end{align}
よって
\begin{align}
  x = ÷{1}{2}±√{÷{1}{4}-÷{𝘬T}{2zV}},␣
  δ₁ = √{÷{1}{4}-÷{𝘬T}{2zV}}
\end{align}
\subsubsection*{
  [8]
}
省略。
$f'(0)=-∞,f'(1)=∞$に気を付ける。(対数発散なので、グラフにすると大したことない。)
$f(x)$の極値が$1/2-δ₀,0,1/2+δ₀$であり、極値の間に$1/2±δ₁$がある。
\subsubsection*{
  [9]
}
よくある、$T < T_{c0}$で二価な関数が$T=T_{c0}$で合流してゼロになるようなグラフを
$δ₀,δ₁$のそれぞれについて描く。
$δ₀(T_{c0})=δ₁(t_{c0})=0$、$δ₀(T=0)=δ₁(T=0)=1/2$および
$0 < T < T_{c0}$で$|δ₀(T)| > |δ₁(T)|$となることに注意する。
\subsubsection*{
  [10]
}
\begin{align}
  sx₁+(1-s)x₂ = x₀
\end{align}
より、
\begin{align}
  s= ÷{x₂-x₀}{x₂-x₁}
\end{align}
\subsubsection*{
  [11]
}
\begin{align}
  f^* = sf(x₁)+(1-s)f(x₂)
  = ÷{x₂-x₀}{x₂-x₁}f(x₁)+÷{x₁-x₀}{x₁-x₂}f(x₂)
  % = Vz[sx₁(1-x₁)+(1-s)x₂(1-x₂)]+
  % 𝘬T[sx₁\log x₁ + s(1-x₁)\log(1-x₁)+(1-s)x₂\log x₂+(1-s)(1-x₂)\log(1-x₂)]
\end{align}
\subsubsection*{
  [12]
}
$x₁=x₀-δ,x₂=x₀+δ$とすると、
\begin{align}
  f^* &
  = ÷{1}{2}f(x₀-δ)+÷{1}{2}f(x₀+δ)\∅
  &
  = ÷{1}{2}(f(x₀)-δf'(x₀)+÷{δ²}{2}f''(x₀))
  + ÷{1}{2}(f(x₀)+δf'(x₀)+÷{δ²}{2}f''(x₀))+𝒪(δ³)\∅
  &
  = f(x₀) + ÷{1}{2}δ²f''(x₀)+𝒪(δ³)
\end{align}
となる。よって$f^* > f(x₀)$となるための条件は、
\begin{align}
  \∂^2{f}{x}\𝙴_{x=x₀} > 0
\end{align}
\subsubsection*{
  [13]
}
高温側が(I)、$δ₀$と$δ₁$に囲まれている領域が(II)、$δ₁$に囲まれている領域が(III)。

\newpage
\subsection*{
  第4問
}
\subsubsection*{
  [1.1]
}
\begin{align}
  m\"𝒙(t) = -q𝑬₀ℯ^{-¡ωt} - mω₀²𝒙(t) - mγ\𝚍{𝒙(t)}{t}
\end{align}
\subsubsection*{
  [1.2]
}
$𝒙 = 𝒙₀\exp(-¡ωt)$とすると、
\begin{align}
  -mω²𝒙₀ = -q𝑬₀ - mω₀²𝒙₀ + ¡mγω𝒙₀
\end{align}
よって、
\begin{align}
  𝒙₀ = ÷{-q𝑬₀}{m[(ω₀²-ω²)-¡γω]}
\end{align}
\subsubsection*{
  [1.3]
}
\begin{align}
  𝑷₀ = -Nq𝒙₀ = ÷{Nq²}{m[(ω₀²-ω²)-¡γω]}𝑬₀ = ϵ₀χ𝑬₀
\end{align}
よって、
\begin{align}
  χ = ÷{Nq²}{ϵ₀m}÷{(ω₀²-ω²)+¡γω}{(ω₀²-ω²)²+γ²ω²}
\end{align}
である。
\begin{align}
  χ_R = ÷{Nq²}{ϵ₀m}⋅÷{ω₀²-ω²}{(ω₀²-ω²)²+γ²ω²},␣
  χ_I = ÷{Nq²}{ϵ₀m}⋅÷{γω}{(ω₀²-ω²)²+γ²ω²}
\end{align}
\subsubsection*{
  [1.4]
}
$χ_R$の符号は$ω < ω₀$で正、$ω > ω₀$で負となる。
また$|ω - ω₀| ∼ γ$を境にゼロに減衰していく。
\subsubsection*{
  [2.1]
}
\begin{align}
  E₀\exp(¡k\sinθ₀x)+E_m\exp(¡K_{mx}x) = E_t\exp(¡K_{tx}x)
\end{align}
\subsubsection*{
  [2.2]
}
[2.1]の結果が任意の$x$で成り立つことから、
\begin{align}
  K_{mx} = K_{tx} = k\sinθ₀
\end{align}
\subsubsection*{
  [2.3]
}
波動方程式は、
\begin{align}
  (∇²-ϵ₀μ₀\∂^2_t)(𝑬₀+𝑬_m) = 𝟎,␣
  (∇²-(1+χ)ϵ₀μ₀\∂^2_t)𝑬_t = 𝟎.
\end{align}
ここから、
\begin{align}
  k² = K_{mx}²+K_{mz}² = ϵ₀μ₀ω²,␣
  K_{tx}²+K_{tz}² = (1+χ)ϵ₀μ₀ω²
\end{align}
\subsubsection*{
  [2.4]
}
\begin{align}
  K_{mz} = k\cosθ₀
\end{align}
\subsubsection*{
  [2.5]
}
$θ₀ = θ_c$において、
\begin{align}
  K_{tz}² = (1+χ)ϵ₀μ₀ω² - k²\sin²θ_c = k²(1+χ-\sin²θ_c)
\end{align}
となる。よって、
\begin{align}
  θ_c = \arcsin(√{1+χ})
\end{align}
である。$θ₀ > θ_c$のとき、$z$軸負方向に伝播していくことに注意して、
\begin{align}
  K_{tz} = -k√{1+χ-\sin²θ₀}.
\end{align}
$θ₀ < θ_c$のとき、$z$軸負方向に減衰していくことに注意して、
\begin{align}
  K_{tz} = ¡k√{\sin²θ₀-(1+χ)}
\end{align}
\subsubsection*{
  [2.6]
}
誘電体内での振幅は$ℯ^{¡K_{tz}z} = ℯ^{-k√{\sin²θ₀-(1+χ)}z}$から
\begin{align}
  z₀ = ÷{1}{k√{\sin²θ₀-(1+χ)}}
  = ÷{1}{k√{-χ-\cos²θ₀}}
\end{align}
これは$\cos²θ₀ → -χ$で発散する。
\end{document}