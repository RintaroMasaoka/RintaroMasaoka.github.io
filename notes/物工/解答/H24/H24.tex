\providecommand{\main}{../main}
\documentclass[\main/main.tex]{subfiles}
\graphicspath{{../images/}}
\begin{document}
\newpage
\section{2012年(平成24年)}
\subsection*{  
  物理学I
}
\subsection*{
  第1問
}
\subsubsection*{
  [1]
}
$𝑭$は図1を前方から見て角度$7𝜋/6$の向き。
$𝑭'$は角度$𝜋$の向き。
Bへのモーメントの釣り合いから
\begin{align}
  F - F' = 0
\end{align}
\subsubsection*{
  [2]
}
BとCの間の垂直抗力の大きさを$T$とおく。
Cの$y$方向の釣り合いから、
\begin{align}
  2(÷{1}{2}F+÷{√3}{2}T) = F + √3T = mg.
\end{align}
またBの$x$方向の釣り合いから、
\begin{align}
  -F' - ÷{√3}{2}F + ÷{1}{2}T = 0.
\end{align}

これらを$F' = F$と連立すると、
\begin{align}
  F = ÷{mg}{4+2√3} = ÷{2-√3}{2}mg,␣
  T = ÷{1}{2}mg
\end{align}
が得られる。
またBと床との間の垂直抗力$T' = 3/2mg$である。
円柱が静止する条件は、
\begin{align}
  ÷{F}{T} = 2-√3< μ,␣
  ÷{F'}{T'} = ÷{2-√3}{3} < μ'.
\end{align}
\subsubsection*{
  [3]
}
\begin{align}
  I = ÷{m}{𝜋a²l}⋅2𝜋l∫_0^a r³\𝑑{r} = ÷{1}{2}ma²
\end{align}
\subsubsection*{
  [4]
}
運動方程式は、
\begin{align}&
  m\𝚍{v}{t} = ÷{I}{a}\𝚍{ω}{t} = -μ''mg
\end{align}
で与えられる。これを解くと、
\begin{align}
  v = v₀ - μ''gt,␣
  ω = ω₀ - ÷{2μ''g}{a}t
\end{align}
となる。$v=0$となるときの角速度は、
\begin{align}
  ω₀ -  ÷{2μ''g}{a}⋅÷{v₀}{μ''g}
  = ω₀ - ÷{2v₀}{a}.
\end{align}
\subsubsection*{
  [5]
}
滑らずに転がりはじめるのは$v+aω = 0$となるとき。
このときの速さ$|v|$は、
\begin{align}
 |v| = -(v₀ - μ''g⋅÷{v₀+aω₀}{3μ''g})
 = ÷{aω₀-2v₀}{3}
\end{align}
\subsubsection*{
  [6]
}
$v+aω = 0$となる時間が$T$に到達した後、
$S$に到達するまでの間になければならない。
$T,S$に到達する時間はそれぞれ
\begin{align}
  ÷{v₀}{μ''g},␣÷{2v₀}{μ''g}
\end{align}
であり、$v+aω = 0$となる時間は、
\begin{align}
  t = ÷{v₀+aω₀}{3μ''g}
\end{align}
で与えられるから、求める条件は
\begin{align}
  ÷{v₀}{μ''g} < ÷{v₀+aω₀}{3μ''g} < ÷{2v₀}{μ''g}
\end{align}
である。
整理すると、
\begin{align}
  ÷{2v₀}{a} < ω₀ < ÷{5v₀}{a}
\end{align}
となる。
\newpage
\subsection*{
  第2問
}
\subsubsection*{
  [1]
}
\begin{align}
  ÷{1}{2}mv² = eϕ(x)
\end{align}
より、
\begin{align}
  v(x) = √{÷{2eϕ(x)}{m}}
\end{align}
\subsubsection*{
  [2]
}
連続の式から
\begin{align}
  \∂{ρ(t,x)}{t} = -\∂{i(t,x)}{x}
\end{align}
である。定常状態では左辺はゼロなので、$\𝚍*{i}{x} = 0$となる。
よって$i(x)$は定数となり、$n(x)v(x)>0$から$i = -i₀(i₀>0)$となる。
\subsubsection*{
  [3]
}
\begin{align}
  ε₀\𝚍^2{ϕ(x)}{x} = ÷{i₀}{v(x)}
  = i₀√{÷{m}{2e}}ϕ(x)^{-1/2}
\end{align}
よって
\begin{align}
  A = ÷{i₀}{ε₀}√{÷{m}{2e}},␣ α = -÷{1}{2}.
\end{align}
\subsubsection*{
  [4]
}
設問[3]の方程式
\begin{align}
  \𝚍^2{ϕ}{x} =  Aϕ^{-1/2}
\end{align}
を$ϕ(0)=0$のもとで解く。$ϕ = Bx^β$とおくと、
\begin{align}
  Bβ(β-1)x^{β-2} = AB^{-1/2}x^{-β/2}
\end{align}
から、
\begin{align}
  β = ÷{4}{3},␣
  B = (÷{9}{4}A)^{2/3}
\end{align}
とすればよい。
\begin{align}
  ϕ(x) = (÷{9i₀}{4ε₀})^{2/3}(÷{m}{2e})^{1/3}x^{4/3}
\end{align}
\subsubsection*{
  [5]
}
\begin{align}
  V = ϕ(L) = (÷{9i₀}{4ε₀})^{2/3}(÷{m}{2e})^{1/3}L^{4/3}
\end{align}
\newpage
\subsection*{
  物理学II
}
\subsection*{
  第1問
}
\subsubsection*{
  [1]
}
$\^X ∝ \^a + \^a^†$から$⟨n|q\^X|n±1⟩$のみが値をもつ。
つまり$|n±1⟩$に遷移しうる。
\subsubsection*{
  [2]
}
\begin{align}
  ⟨n|λ\^X⁴|n⟩
  &
  = λ(÷{ħ}{2mω₀})²⟨n|(\^a+\^a^†)⁴|n⟩\∅
\end{align}
ここで
\begin{align}
  &
  ⟨n|(\^a+\^a^†)⁴|n⟩\∅
  &
  = (n+1)(n+2)+(n+1)²+(n+1)n+n(n+1)+n²+n(n-1)\∅
  &
  = 6n²+6n+3
\end{align}
から、$|n⟩$のエネルギーの変化は、
\begin{align}
  3λ(2n²+2n+1)(÷{ħ}{2mω₀})²
\end{align}
となる。
\subsubsection*{
  [3]
}
$|n⟩'$には$|n-4⟩,|n-2⟩,|n⟩,|n+2⟩,|n+4⟩$の成分が含まれているので、\\ 
$⟨n|'(\^a+\^a^†)|8⟩' ≠ 0$ となる$n$は、
\begin{align}
  n = 3,5,7,9,11,13.
\end{align}
このような$|n⟩'$に遷移しうる。
\subsubsection*{
  [4]
}
\begin{align}
  [\^H₀,\^a] = -ħω₀\^a,␣
  [\^H₀,\^a^†] = ħω₀\^a^†,␣
\end{align}
より、
\begin{align}
  \^X(t) &= ℯ^{¡H₀t/ħ}\^X(0)ℯ^{-¡H₀t/ħ}\∅
  &
  = √{÷{ħ}{2mω₀}}(ℯ^{-¡ω₀t}\^a + ℯ^{¡ω₀t}\^a^†)\∅
  &
  = \cos(ω₀t)\^X(0) + ÷{1}{mω₀}\sin(ω₀t)\^P(0)
\end{align}
\subsubsection*{
  [5]
}
$⟨n|\^X(0)|n⟩ = ⟨n|\^P(0)|n⟩ = 0$から、設問[4]の結果より
\begin{align}
  ⟨n|\^X(t)|n⟩ = 0
\end{align}
となる。
位置の期待値が有限の振幅で振動する状態の例としては、
コヒーレント状態
\begin{align}
  |α⟩ = ℯ^{-|α|²/2}∑_{n=0}^∞ ÷{α^n}{√{n!}}|n⟩,␣
  α ∈ ℂ
\end{align}
が挙げられる。これは$\^a$の固有値$α$の固有状態なので、
\begin{align}
  ⟨α|\^X(t)|α⟩
  = √{÷{ħ}{2mω₀}}(ℯ^{-¡ω₀t}α + ℯ^{¡ω₀t}α^*)
\end{align}
となり、振動する。
\newpage
\subsection*{
  第2問
}
\subsubsection*{
  [1]
}
\begin{align}
  f_\𝚞{BE}(E) = ÷{1}{ℯ^{(E-μ)/𝘬T}-1}
\end{align}
$E-μ = 0$でこれは発散するので、$μ < 0$でなければいけない。
\subsubsection*{
  [2]
}
$f_\𝚞{BE}(E)$は$μ$についての増加関数であり、
$-μ$を大きくしていくと指数関数的に減衰する。
\subsubsection*{
  [3]
}
エネルギーが$E$以下となるような$𝒑$の、運動量空間における体積は
\begin{align}
  ÷{4𝜋}{3}(2mE)^{3/2}
\end{align}
で与えられる。よってエネルギーが$E$以下の状態数は
\begin{align}
  N(E) = (÷{L}{2𝜋ħ})³÷{4𝜋}{3}(2mE)^{3/2}
  = ÷{4V}{3√𝜋}(÷{mE}{2𝜋ħ²})^{3/2}
\end{align}
となる。状態密度は、
\begin{align}
  D(E) = \𝚍{N}{E} = ÷{2V}{√𝜋}(÷{m}{2𝜋ħ²})^{3/2}E^{1/2}
\end{align}
\subsubsection*{
  [4]
}
$N_𝑡$は$μ=0$のときに最大。よって、$D(E) = AE^{1/2}$とおくと、
\begin{align}
  N_𝑡^\𝚞{max} &= A∫_0^∞ ÷{E^{1/2}}{ℯ^{βE}-1}\𝑑{E}\∅
  &
  = Aβ^{-3/2}Γ(3/2)ζ(3/2)\∅
  &
  = ζ(3/2)V(÷{m𝘬T}{2𝜋ħ²})^{3/2}
\end{align}
これは$T → 0$でゼロになる。
\subsubsection*{
  [5]
}
\begin{align}
  ζ(3/2)÷{V}{ħ³}(÷{m𝘬T_𝑐}{2𝜋})^{3/2} = N
\end{align}
\begin{align}
  T_𝑐 = ÷{2𝜋ħ²}{m𝘬}(÷{1}{ζ(3/2)}÷{N}{V})^{2/3}
\end{align}
\subsubsection*{
  [6]
}
$E > 0$に対する分布は$μ=0$とした$f_\𝚞{BE}(E)$で与えられ、
さらに$E=0$の準位に$N-N_𝑡^\𝚞{max}$個の粒子がいる。
\subsubsection*{
  [7]
}
\begin{align}
  E &= A∫_0^∞÷{E^{3/2}}{ℯ^{βE}-1}\𝑑{E}\∅
  &
  = Aβ^{5/2}Γ(5/2)ζ(5/2)\∅
  &
  = ÷{3V}{2}ζ(5/2)(÷{m}{2𝜋ħ²})^{3/2}(𝘬T)^{5/2}
\end{align}
\begin{align}
  C = ÷{15V}{4}ζ(5/2)(÷{m𝘬T}{2𝜋ħ²})^{3/2}
\end{align}
\subsubsection*{
  [8]
}
\begin{align}
  T_𝑐/\si{K} &
  = ÷{2𝜋ħ²}{m𝘬}(÷{1}{ζ(3/2)}÷{N}{V})^{2/3}\∅
  &
  = ÷{2𝜋×1.1²}{6.7×1.4}×10^{-18}
  ×(÷{1.5}{2.61×6.7}×10^{29})^{2/3}\∅
  &
  ≈ 8^{2/3} = 4
\end{align}
(ちゃんと計算すると$3$なのだけど、計算機使わないと無理では?)
\subsubsection*{
  [9]
}
ヘリウム原子間の相互作用
\newpage
\subsection*{
  第3問
}
\subsubsection*{
  [1]
}
屈折が起こらない条件を求めれば良い。光線が屈折角$θ'$で屈折するとき、
\begin{align}
  ÷{k₀n₂\sinθ}{k₀n₁\sinθ'} = 1
\end{align}
が成り立つ。このような$\sin θ'$が存在しない条件は、
\begin{align}
  \sinθ' = ÷{β}{k₀n₁} > 1
\end{align}
よって$β$の取りうる範囲は、
\begin{align}
  k₀n₁ < β < k₀n₂
\end{align}
\subsubsection*{
  [2]
}
Maxwell方程式から、
\begin{align}
  &
  ¡βE_y = -¡μ₀ωH_x, \\
  &
  \𝚍{E_y}{x} = -¡μ₀ωH_z, \\
  &
  ¡βH_y = 0, \\
  &
  -¡βH_x-\𝚍{H_z}{x} = ¡ϵ_jωE_y, \\
  &
  \𝚍{H_y}{x} = 0.
\end{align}
となる。整理すると、
\begin{align}
  &
  \𝚍{E_y}{x} = -¡μ₀ωH_z \\
  &
  \𝚍{H_z}{x} = ¡(÷{β²}{μ₀ω}-ϵ_jω)E_y \\
  &
  H_x = -÷{β}{μ₀ω}E_y,␣ H_y = 0
\end{align}
\subsubsection*{
  [3]
}
$H_z(x) = 0$と仮定する。
このとき領域IIで
\begin{align}
  E_y = ÷{μ₀ω}{¡(β²-ε₂μ₀ω²)}\𝚍{H_z}{x} =  0
\end{align}
が成り立つ。ただし分母が
\begin{align}
  β² - ε₂μ₀ω² = β² - (k₀n₂)² ≠ 0
\end{align}
となることに注意する。
ここから$H_x = 0$も言えるので、電磁場の全ての成分がゼロになってしまう。
よって$H_z(x) ≠ 0$
\subsubsection*{
  [4]
}
設問[2]で求めた方程式から
\begin{align}
  \𝚍^2{E_y}{x} = -¡μ₀ω\𝚍{H_z}{x}
  = (β² - ϵ_jμ₀ω²)E_y
\end{align}
となるので、
\begin{align}
  \𝚍^2{E_y(x)}{x} + (k₀²n_j²-β²)E_y(x) = 0,␣
  j = \begin{cases}
    1 & (|x| > d) \\
    2 & (|x| < d)
  \end{cases}
\end{align}
\subsubsection*{
  [5]
}
領域$|x| < d$における$E_y(x)$を
\begin{align}
  E_y(x) = a\cos (kx),␣
  k = √{k₀²n₂²-β²}
\end{align}
とおく。
また$x > d$における$E_y(x)$を
\begin{align}
  E_y(x) = bℯ^{-κx},␣
  κ = √{β²-k₀²n₁²}
\end{align}
とおく。接続条件から
\begin{align}
  &
  a\cos(kd) = cℯ^{-κd} \∅
  &
  -ka\sin(kd) = -κcℯ^{-κd}
\end{align}
よって、
\begin{align}
  k\tan(kd) = κ
\end{align}
となる。よって
\begin{align}
  A(β) = kd = √{k₀²n₂²-β²}d,␣
  B(β) = ÷{κ}{k} = √{÷{β²-k₀²n₁²}{k₀²n₂²-β²}}
\end{align}
\subsubsection*{
  [6]
}
\begin{align}
  F(β) = √{k₀²n₂²-β²}d -\arctan(√{÷{β²-k₀²n₁²}{k₀²n₂²-β²}})
\end{align}
である。$β → k₀n₁$とすると
\begin{align}
  F(β) = √{n₂²-n₁²}k₀d > 0
\end{align}
となる。また$β → k₀n₂$とすると
\begin{align}
  F(β) = -÷{𝜋}{2} < 0
\end{align}
となる。よって、$F(β) = 0$となる点が少なくとも1つ存在する。
方程式を満たす$β$が一つだけ存在する条件は、
\begin{align}
  √{n₂²-n₂²}k₀d < 𝜋
\end{align}
である。
このとき$E_y(x)$の概形としては、山が一つでき、$|x| > d$で減衰していく。
\newpage
\subsection*{
  第4問
}
\subsubsection*{
  [1]
}
運動方程式は
\begin{align}
  &
  m\"u_n = C₁(v_n-u_n) + C₂(v_{n-1} - u_n),\\
  &
  m\"v_n = C₂(u_{n+1} - v_n) + C₁(u_n-v_n).
\end{align}
解として
\begin{align}
  u_n = uℯ^{-¡ωt+¡kan},~ v_n = vℯ^{-¡ωt+¡ka(n+1/2)}
\end{align}
を仮定すると、
\begin{align}
  &
  ⦅
    mω²-C₁-C₂&C₁ℯ^{¡ka/2}+C₂ℯ^{-¡ka/2}\\
    C₂ℯ^{¡ka/2}+C₁ℯ^{-¡ka/2}&mω²-C₁-C₂
  ⦆⦅u\\v⦆
  = ⦅0\\0⦆
\end{align}
となる。この方程式が非自明な解をもつためには、
\begin{align}
  mω² = C₁+C₂ ± |C₁ℯ^{¡ka/2}+C₂ℯ^{-¡ka/2}|
\end{align}
であればよい。計算上のテクニックとしては、Pauli行列の線形結合
\begin{align}
  ⦅0&α\\α^*&0⦆
\end{align}
の固有値が$±|α|$であることを覚えておくと計算が簡便になる。
よって、
\begin{align}
  ω = √{÷{C₁+C₂ ± √{C₁²+C₂²+2C₁C₂\cos(ka)}}{m}}
\end{align}
\subsubsection*{
  [2]
}
長波長での分散関係は、
\begin{align}
  mω² &= C₁+C₂ ± √{(C₁+C₂)²-C₁C₂(ka)²}\∅
  &
  = C₁+C₂ ± (C₁+C₂ - ÷{C₁C₂(ka)²}{2(C₁+C₂)})
\end{align}
から、
\begin{align}
  ω_𝑎 = √{÷{C₁C₂}{2m(C₁+C₂)}}ak,␣
  ω_𝑜 = √{÷{2(C₁+C₂)}{m}}
        - ÷{C₁C₂(ka)²}{4√{m(C₁+C₂)³}}
\end{align}
ただし$ω_𝑎$は音響フォノンを表し、$ω_𝑜$は光学フォノンを表す。
よって群速度は
\begin{align}
  \𝚍{ω_𝑎}{k} = √{÷{C₁C₂}{2m(C₁+C₂)}}a,␣
  \𝚍{ω_𝑜}{k} = -÷{C₁C₂a²}{2√{m(C₁+C₂)³}}k
\end{align}
となる。
\subsubsection*{
  [3]
}
音響フォノンは$u$と$v$が同じ向きに振動し、
光学フォノンは$u$と$v$が反対向きに振動する。
\subsubsection*{
  [4]
}
まず$𝑲_𝑖,𝑲_𝑓,𝒒_𝐵$の関係は
\begin{align}
  𝑲_𝑖 = 𝑲_𝑓 + 𝒒_𝐵
\end{align}
で与えられる。
$ω_𝑖 ≫ ω_𝐵$のとき、$ω_𝑖 ≈ ω_𝑓$から$|𝑲_𝑖| ≈ |𝑲_𝑖|$である。
よって、
\begin{align}
  |𝒒_𝐵|² 
  &
  = |𝑲_𝑖|²+|𝑲_𝑓|² - 2|𝑲_𝑖||𝑲_𝑓|\cosθ \∅
  &
  ≈ 2|𝑲_𝑖|²(1-\cosθ)
\end{align}
となる。よって主要な寄与をとると、
\begin{align}
  |𝒒_𝐵| = 2|𝑲_𝑖|\sin÷{θ}{2}
\end{align}
\subsubsection*{
  [5]
}
\begin{align}
  ÷{ω_𝐵}{v_𝑎} = ÷{2nω_𝑖}{c}\sin÷{θ}{2}
\end{align}
\subsubsection*{
  [6]
}
\begin{align}
  v_𝑎 &= ÷{c}{2n}÷{ω_𝐵}{ω_𝑖}\∅
  &
  = ÷{c}{2n}÷{λ_𝑖}{λ_𝐵}\∅
  &
  = ÷{3.0×680×3.0×10^{8-9-2}}{2×2.4}\∅
  &
  ≈ \SI{1.3}{m/s}
\end{align}
\subsubsection*{
  [7]
}
$k = 𝜋/a$のとき、
\begin{align}
  ω_𝑎 = √{÷{2C₂}{m}},␣
  ω_𝑜 = √{÷{2C₁}{m}},␣
\end{align}
となる。
エネルギー保存から、
\begin{align}
  ÷{ħ}{2M_𝑛}(÷{2𝜋}{λ_{𝑖j}})² - ÷{ħ}{2M_𝑛}(÷{2𝜋}{λ_𝑓})² = √{÷{2C_j}{m}}
\end{align}
よって
\begin{align}
  C_j = ÷{2m𝜋⁴ħ²}{M_𝑛²}(÷{1}{λ_{𝑖j}}-÷{1}{λ_𝑓})²
\end{align}
\subsubsection*{
  [8]
}
\begin{align}
  a &= √{2m(÷{1}{C₁}+÷{1}{C₂})}\∅
  &
  = ÷{M_𝑛}{𝜋²ħ}√{(÷{1}{λ_{𝑖1}}-÷{1}{λ_𝑓})^{-2}+(÷{1}{λ_{2j}}-÷{1}{λ_𝑓})^{-2}}
\end{align}
\end{document}
