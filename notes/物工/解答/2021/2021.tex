\providecommand{\main}{../main}
\documentclass[\main/main.tex]{subfiles}
\graphicspath{{../images/}}
\begin{document}
\newpage
\section{2020年(令和2年)}
\subsection*{
    第1問
}
$ħ = 1$とおく。 

\subsubsection*{
    [1]
}
\begin{align}
    \^U(t-t₀) = ℯ^{-¡H(t-t₀)}
\end{align}

\subsubsection*{
    [2]
}
\begin{align}
    (ℯ^{A})^†
    &
    =\𝚙(1 + A +\÷{A²}{2} + ⋯)^†
    \∅ & 
    = 1 + A^† +\÷{(A^†)²}{2} + ⋯
    \∅ & 
    = ℯ^{A^†}
\end{align}
より、
\begin{align}
    \^U(t-t₀)^† = ℯ^{¡\^H^†(t-t₀)}
    = ℯ^{¡\^H(t-t₀)}
    =\^U(t-t₀)^{-1}
\end{align}
よって$\^U$はユニタリー。

\subsubsection*{
    [3]
}
設問[2]の結果から、
\begin{align}
    ⟨ψ(t)|ψ(t)⟩
    = ⟨ψ(t₀)|\^U(t-t₀)^†\^U(t-t₀)|ψ(t₀)⟩
    = ⟨ψ(t₀)|ψ(t₀)⟩
\end{align}
で時間に依存しない。

\subsubsection*{
    [4]
}
\begin{align}
    \^U(τ) = ℯ^{-¡a\^σ_zτ} =\cos(aτ)\^1 -¡\sin(aτ)\^σ_z
\end{align}

\subsubsection*{
    [5]
}
\begin{align}
    \^U(t) = ⦅
        1&0\\
        0&ℯ^{¡ϕt/τ}
    ⦆
\end{align}
とすると、
\begin{align}
    ¡\𝚍_t\^U(t)
    = ⦅0&0\\0&(-ϕ/τ)ℯ^{¡ϕt/τ}⦆
    = ⦅0&0\\0&-ϕ/τ⦆\^U(t)
\end{align}
となるから、
\begin{align}
    \^H =  ⦅0&0\\0&-ϕ/τ⦆
\end{align}

\subsubsection*{
    [6]
}
$\^U_\𝚞{NOT}(τ) =\^σ_x$としたい。ここで、
\begin{align}
    ℯ^{-¡𝜋/2}ℯ^{¡𝜋\^σ_x/2} =\^σ_x
\end{align}
に注目すると、
\begin{align}
    \^U(t) = \exp(- ÷{¡𝜋t}{2τ}\^1+÷{¡𝜋t}{2τ}\^σ_x)
\end{align}
とすれば良い。よって、
\begin{align}
    \^H = \÷{𝜋}{2τ}(\^1-\^σ_x)
    =\÷𝜋{2τ}\𝐩(1&-1\\-1&1)
\end{align}
とすればよい。

\subsubsection*{
    [7]
}
$(\sinθ\cosϕ\^σ_x +\sinθ\sinϕ\^σ_y +\cosθ\^σ_z)² = 1$より、
\begin{align}
    &
    \exp(
        -¡aτ[\sinθ\cosϕ\^σ_x +\sinθ\sinϕ\^σ_y +\cosθ\^σ_z]
        -¡bτ\^1
    )
    \∅ & 
    = ℯ^{-¡b}\𝚙(\cos(aτ)\^1 -¡\sin(aτ)[\sinθ\cosϕ\^σ_x +\sinθ\sinϕ\^σ_y +\cosθ\^σ_z])
\end{align}

\subsubsection*{
    [8]
}
\begin{align}
    \^U_𝐻 = ÷{\^σ_z+\^σ_x}{√2}
\end{align}
より、[6]と同じ議論によって、
\begin{align}
   \^H =\÷{𝜋}{2τ}\𝙱{
        \^1-\÷1{\√2}\𝐩(
            1&1\\
            1&-1
        )
   }
   =\÷{𝜋}{2\√2 τ}\𝐩(
        √2-1 & -1 \\
        -1 & √2-1
   )
\end{align}

\subsubsection*{
    [9]
}
\begin{align}
    \^H =\𝐩(
        0&0&0&   0 \\
        0&0&0&   0 \\
        0&0&0&   0 \\
        0&0&0&-ϕ/τ\
    )
\end{align}

\subsubsection*{
    [10]
}
\begin{align}
    \^H = \÷{𝜋}{2τ}\𝐩(
        0&0& 0& 0 \\
        0&0& 0& 0 \\
        0&0& 1&-1 \\
        0&0&-1& 1
    )
\end{align}
\newpage
\subsection*{
    第2問
}
\subsubsection*{
    [1]
}
質量、長さ、時間の次元をそれぞれ$M,L,T$で表す。
左辺の次元は
\begin{align}
    ÷{M}{L³}L³ ÷{L}{T²} = ÷{ML²}{T²}
\end{align}
となり、右辺の次元は
\begin{align}
    ÷{ML}{T²L³}L³ = ÷{ML²}{T²}
\end{align}
となるから両辺の次元は一致する。
\subsubsection*{
    [2]
}
下面で接する2つの体積素の間での作用・反作用の法則から、
下面に働く応力は$-𝒑^{(3)}(𝒙'',t)$となる。
\subsubsection*{
    [3]
}
体積素が$x_k$軸に直交する2つの面から受ける$x₁$方向の力の合計は
\begin{align}
    p_1^{(k)}(𝒙+÷{𝛥x_k}{2}𝒆_x)-p_1(𝒙-÷{𝛥x_k}{2}𝒆_x)
    ≈ \∂{p_1^{(k)}}{x_k}
\end{align}
となる。よってこれを$k=1,2,3$について足し合わせると
\begin{align}
    F₁(𝒙,t) = ∑_{k=1,2,3} \∂{p₁^{(k)}}{x_k}
\end{align}
を得る。
\subsubsection*{
    [4]
}
\begin{align}
    ρ\∂^2{u_j}{t} = F_j &
    = ∑_{k,m,n}÷{1}{2}c_{jkmn}∂_k(∂_nu_m+∂_mu_n)\∅
    &
    = ∑_{k}(λ∂_j∂_ku_k+μ∂_k(∂_ju_k+∂_ku_j))\∅
\end{align}
から、
\begin{align}
    ρ\∂^2{u_j}{t} = μ∇²𝒖 + (λ+μ)∇(∇⋅𝒖).
\end{align}
よって$A = λ+μ$
\subsubsection*{
    [5]
}
$𝒖(𝒙,t) = 𝒖₀\exp(¡𝒌⋅𝒙-¡ωt)$とおく。
縦波の場合、$𝒌 ∝ 𝒖₀$から
\begin{align}
    -ρω²𝒖₀ = -μ𝒌²𝒖₀-(λ+μ)𝒌(𝒌⋅𝒖₀),␣
    ρω² = (λ+2μ)k².
\end{align}
よって位相速度は
\begin{align}
    v = ÷{ω}{k} = √{÷{λ+2μ}{ρ}}
\end{align}
となる。次に横波の場合、$𝒌⋅𝒖₀ = 0$から
\begin{align}
    ρω²𝒖₀ = -μ𝒌².
\end{align}
よって位相速度は
\begin{align}
    v = ÷{ω}{k} = √{÷{μ}{ρ}}
\end{align}
\subsubsection*{
    [6]
}
\begin{align}
    𝒖_{t0} = u_t\𝐩(-\cos α_t \\ \sin α_t),\␣
    𝒖'_{t0} = u_t'\𝐩(\cos α_t' \\ \sin α_t')
\end{align}
\begin{align}
    𝒌_t = k_t\𝐩(\sin α_t \\ \cos α_t),\␣
    𝒌_t' = k_t'\𝐩(\sin α_t' \\ -\cos α_t')
\end{align}
とおける。境界条件から、
\begin{align}
    p_1^{(2)} = μ\𝚙(\∂{u₁}{x₂}+\∂{u₂}{x₁}) = 0
\end{align}
である。これを計算すると、
\begin{align}
    &
    \∂{u₁}{x₂}+\∂{u₂}{x₁}\∅
    &
    = -¡u_tk_t\cos2α_tℯ^{¡k_tx\sinα_t-¡ω_tt}
        -¡u_t'k_t'\cos2α_t'ℯ^{¡k_t'x\sinα_t'-¡ω_tt}
    = 0
\end{align}
$u_t,k_t,u_t',k_t'$は全てゼロではないので、$x,t$によらずこれがゼロになるためには、
\begin{align}
    ω_t = ω_t',␣
    k_t'\sin α_t' = k_t\sin α_t
\end{align}
が必要。$ω_t = ω_t'$から$k_t = k_t'$なので、
\begin{align}
    ω_t = ω_t',␣ α_t = α_t'
\end{align}
が分かる。

\subsubsection*{
    [7]
}
\begin{align}
    p_2^{(2)} = λ∇⋅𝒖 + 2μ\∂{u₂}{x₂} = 0
\end{align}
とする。
横波なので、$∇⋅𝒖 = 0$としてよく、
\begin{align}
    \∂{u₂}{x₂} = \sin α_t\cos α_t(u_t' - u_t) = 0
\end{align}
となる。$α_t ≠ 0, 𝜋$から、$u_t' = u_t$が分かる。
この結果を$p_1^{(2)} = 0$に代入すると、
\begin{align}
    \cos2α_t = 0
\end{align}
が分かる。よって求める角度は$α_t = 𝜋/4$。

\subsubsection*{
    [8]
}
縦波について、
\begin{align}
    𝒖_l'(x,t) = \Re[𝒖_{l0}'\exp(¡(𝒌_l'⋅𝒙-ω_l't))]
\end{align}
とおく。
\begin{align}
    𝒖_{l0}' = u_l'⦅
        \sin α_l'\\
        -\cos α_l'
    ⦆,␣
    𝒌_l' = k_l'⦅
        \sin α_l'\\
        -\cos α_l'
    ⦆
\end{align}
とおける。境界条件から
\begin{align}
    λ k_l'u_l'ℯ^{¡k_l'\sin(α_l')x-¡ω_l't} + 2μ\sin α_t\cos α_t(u_t' - u_t)ℯ^{¡k_t\sin(α_t)x-¡ω_tt}
     = 0
\end{align}
$k_l' ≠ 0, u_l' ≠ 0$であり、上の式が任意の$x$について成り立つことから、
\begin{align}
    k_l'\sin α_l' = k_t \sin α_t
\end{align}
となる。
\newpage
\subsection*{
    第3問
}
\subsubsection*{
    [1.1]
}
固有状態は、
\begin{align}
    ψ(𝒙) = ÷1{√{L³}}\exp(¡𝒌⋅𝒙)
\end{align}
で表される。ここで
\begin{align}
    (÷L{2𝜋})³ 𝒌 ∈ ℤ³
\end{align}
であるから、運動量空間の単位体積あたりの固有状態の数は、
\begin{align}
    (÷{L}{2𝜋ħ})³
\end{align}
\subsubsection*{
    [1.3]
}
$ϵ = p²/2m$から、
\begin{align}
    (÷{L}{2𝜋ħ})³⋅4𝜋p²\𝑑{p} = D(ϵ)\𝑑{ϵ}
\end{align}
より、
\begin{align}
    D(ϵ) = 4𝜋m√{2mϵ}(÷{L}{2𝜋ħ})³
    = ÷2{√𝜋}(÷{mL²}{2𝜋ħ²})^{3/2}ϵ^{1/2}.
\end{align}
\subsubsection*{
    [1.3]
}
\begin{align}
    ∫_0^∞ ÷{ϵ^{1/2}}{ℯ^{ϵ/T}-1}\𝑑{ϵ}
    &
    = T^{3/2}∫_0^∞ ÷{x^{1/2}}{ℯ^x-1}\𝑑{x}\∅
    &
    = ÷{√𝜋}{2}ζ(÷3{2})T^{3/2}
\end{align}
したがって、
\begin{align}
    ζ(÷3{2})(÷{mL²T_c}{2𝜋ħ²})^{3/2} = N
\end{align}
\begin{align}
    T_c = ÷{2𝜋ħ²}{m}ζ(÷3{2})^{-2/3} (÷N{L³})^{2/3}
\end{align}
\subsubsection*{
    [1.4]
}
\begin{align}
    N-N₀ = ζ(÷3{2})(÷{mL²T}{2𝜋ħ²})^{3/2}
    = (÷{T}{T_c})^{3/2}N
\end{align}
\begin{align}
    N₀ = [1-(÷T{T_c})^{3/2}]N
\end{align}
[1.5] 2次元の場合、
\begin{align}
    D(ϵ) = D = \𝚞{const.}
\end{align}
と書けるから、
\begin{align}
    N
    &
    = ∫_0^∞ ÷{D}{ℯ^{(ϵ-μ)/T}-1}\𝑑{ϵ}\∅
    &
    = ∫_0^∞ ÷{Dℯ^{-(ϵ-μ)/T}}{1-ℯ^{-(ϵ-μ)/T}}\𝑑{ϵ}\∅
    &
    = [DT\ln(1-ℯ^{-(ϵ-μ)/T})]_{ε=0}^∞\∅
    &
    = - DT\ln(1-ℯ^{μ/T})
\end{align}
となる。これは$μ → -0$とすればいくらでも大きくなるので、BECは起こらない。

\subsubsection*{
    [2.1]
}
エネルギーが$ϵ$以下になる状態数を$N(ϵ)$と書くと、
\begin{align}
    ħω\𝚍{N(ϵ)}{ϵ} ≈ N(ϵ+ħω)-N(ϵ) = ÷{ϵ}{ħω} + 2 ≈ ÷{ϵ}{ħω} 
\end{align}
である。ただし、$ϵ/ħω ≫ 1$を仮定し、$ϵ/ħω$の0次の項を無視した。
ここから、
\begin{align}
    D(ϵ) = ÷{ϵ}{(ħω)²}
\end{align}
となる。

\subsubsection*{
    [2.2]
}
$T = T_c$において、
\begin{align}
    ∫_0^∞÷{ϵ/(ħω)²}{ℯ^{ϵ/T_c}-1}\𝑑{ϵ} = N
\end{align}
が成り立つ。与えられた公式から、
\begin{align}
    ∫_0^∞÷{ϵ/(ħω)²}{ℯ^{ϵ/T_c}-1}\𝑑{ϵ}
    &
    = (÷{T_c}{ħω})²∫_0^∞÷{x}{ℯ^{x}-1}\𝑑{x}\∅
    &
    = ÷{𝜋²}{6} (÷{T_c}{ħω})² = N
\end{align}
となるので、
\begin{align}
    T_c = ÷{√6ħω}{𝜋}N^{1/2}
\end{align}
\subsubsection*{
    [2.3]
}
$ω ∝ L₀^{-1}$より、
\begin{align}
    ωN^{1/2} ∝ ÷{N^{1/2}}{L₀} = \𝚞{const.}
\end{align}
を満たせば良い。

\subsubsection*{
    [2.4]
}
\begin{align}
    N - N₀
    = ÷{𝜋²}{6} (÷{T}{ħω})²
    = (÷{T}{T_c})² N
\end{align}
より、
\begin{align}
    N₀ = [1-(÷T{T_c})²]N
\end{align}
\newpage
\subsection*{
    第4問
}
\subsubsection*{
    [1]
}
\begin{align}
    φ(𝒓) &= ÷{q(t)}{4𝜋ε₀}[
        (r²-lr\cosθ+÷{l²}{4})^{-1/2}
        -(r²+lr\cosθ+÷{l²}{4})^{-1/2}
    ]\∅
    &
    = ÷{q(t)}{4𝜋ε₀}÷l{r²}\cosθ
\end{align}
$q(t)l = αE₀$
\begin{align}
    φ(𝒓) = ÷{αE₀\cosθ}{4𝜋ε₀}{0⋅÷1{r}+1⋅÷1{r²}}
\end{align}
\begin{align}
    C = 0,␣ D = 1
\end{align}
\subsubsection*{
    [2]
}
\begin{align}
    r_± ≈ r - ÷l{2}\cosθ
\end{align}
より、
\begin{align}
    q(t-r_±/c)l
    &
    ≈ αE₀\cos(ω₀t - ÷{ω₀r}{c} ± ÷{lω₀\cosθ}{2c})\∅
    &
    ≈ αE₀\cos(ω₀(t-÷r{c})) ∓ αE₀\sin(ω₀(t-÷r{c}))÷{lω₀\cosθ}{2c}
\end{align}
これを代入すると、
\begin{align}
    φ(𝒓,t) ≈ 
    ÷{αE₀\cosθ}{4𝜋ε₀}[
        \cos(ω₀(t-÷r{c}))÷{1}{r²}
        -\sin(ω₀(t-÷r{c}))÷{ω₀}{cr}
    ]
\end{align}
\begin{align}
    F = - ÷{ω₀}{c}\sin(ω₀(t-÷r{c})),␣
    G = \cos(ω₀(t-÷r{c}))
\end{align}
\subsubsection*{
    [3]
}
\begin{align}
    𝑨(𝒓,t)
    &
    = ÷1{4𝜋ε₀c²r}\𝚍{p(t-÷r{c})}{t}𝒆_z\∅
    &
    = -÷{α₀E₀ω₀}{4𝜋ε₀c²r}\sin(ω₀(t-÷r{c}))
        (\cosθ𝒆_r -\sinθ𝒆_θ)
\end{align}
\subsubsection*{
    [4]
}
$1/r²$に比例する項を無視すると、
\begin{align}
    ∇÷{F\cosθ}{r}
    ≈ ÷{\cosθ}{r}∇F
    = ÷{ω₀²\cosθ}{c²r}\cos(ω₀(t-÷r{c}))𝒆_r
\end{align}
となる。よって、
\begin{align}
    ∇φ = ÷{α₀E₀ω₀²\cosθ}{4𝜋ε₀c²r}\cos(ω₀(t-÷r{c}))𝒆_r
\end{align}
\subsubsection*{
    [5]
}
\begin{align}
    \∂{𝑨}{t} =
    -÷{α₀E₀ω₀²}{4𝜋ε₀c²r}\cos(ω₀(t-÷r{c}))
    (\cosθ𝒆_r -\sinθ𝒆_θ)
\end{align}
から、
\begin{align}
    𝑬 = -∇ϕ-\∂{𝑨}{t} 
    = ÷{α₀E₀ω₀²\sinθ}{4𝜋ε₀c²r}\cos(ω₀(t-÷r{c}))𝒆_θ
\end{align}
次に$𝑩$を求める。$1/r²$に比例する項を無視すると、
\begin{align}
    𝑩 = ∇×𝑨
    &
    ≈ -÷{α₀E₀ω₀}{4𝜋ε₀c²r}∇\sin(ω₀(t-÷r{c}))
        × (\cos𝒆_r-\sinθ𝒆_θ)\∅
        &
    = ÷{α₀E₀ω₀²\sinθ}{4𝜋ε₀c³r}\cos(ω₀(t-÷r{c}))𝒆_ϕ
\end{align}
\subsubsection*{
    [6]
}
\begin{align}
    𝑺 = ε₀c²𝑬×𝑩
    = ÷{(α₀E₀ω₀²)²}{16𝜋²ε₀c³r}÷{\sin²θ}{r²}\cos²(ω₀(t-÷r{c}))𝒆_r
\end{align}
$|𝑺|$は$ϕ$によらず、$θ$に対しては
\begin{align}
    |𝑺| ∝ \sin²θ
\end{align}
となる。
\subsubsection*{
    [7]
}
\begin{align}
    α(t) = α₀ + α₁\cosω_vt,␣
    α'(t) = -ω_vα₁\sinω_vt
\end{align}
から、
\begin{align}
    \∂{𝑨}{t} &
    = ÷1{4𝜋ε₀c²r}\𝚍^2{p(t-r/c)}{t}𝒆_z\∅
    &
    = -÷{α(t)E₀ω₀²}{4𝜋ε₀c²r}\cos(ω₀(t-÷r{c}))𝒆_z \∅
    &␣
       -÷{2α'(t)E₀ω₀}{4𝜋ε₀c²r}\sin(ω₀(t-÷r{c}))𝒆_z \∅
       &␣
       +÷{α''(t)E₀}{4𝜋ε₀c²r}\cos(ω₀(t-÷r{c}))𝒆_z
\end{align}
よって、
\begin{align}
    𝑬 = ÷{E₀}{4𝜋ε₀c²r}&[
        α(t)ω₀²\sinθ\cos(ω₀(t-÷r{c}))𝒆_θ \𝚛.\∅ 
        & ␣\𝚕.
        +2α'(t)ω₀\sin(ω₀(t-÷r{c}))𝒆_z
        -α''(t)\cos(ω₀(t-÷r{c}))𝒆_z
    ]
\end{align}
これは
\begin{align}
    𝑬
    &
    = ÷{α(t)E₀ω₀²\sinθ}{4𝜋ε₀c²r}\cos(ω₀(t-÷r{c}))𝒆_θ \∅
    &␣
    + ÷{α₁E₀ω_v}{4𝜋ε₀c²r}[
        -2ω₀\sin(ω_vt)\sin(ω₀(t-÷r{c}))
        +ω_v\cos(ω_vt)\cos(ω₀(t-÷r{c}))
    ]𝒆_z
\end{align}
と書けるから、散乱された電磁波の角振動数は$ω₀,ω₀±ω_v$である。
\end{document}