\providecommand{\main}{../main}
\documentclass[\main/main.tex]{subfiles}
\graphicspath{{../images/}}
\begin{document}
\newpage
\section{2011年(平成23年)}
\subsection*{
  第1問
}
\subsubsection*{
  [1]
}
\begin{itemize}
  \item (i) 糸が伸びない・たるまないため張力は仕事をしない。
  \item (ii) 張力の方向が原点$O$ではなく点$Q$を向いているため
\end{itemize}
\subsubsection*{
  [2]
}
\begin{align}
  𝒓 = a⦅\cosφ\\\sinφ⦆+(l₀-aφ)⦅-\sinφ\\\cosφ⦆
\end{align}
\begin{align}
  𝒗 &= \𝚍{𝒓}{t} 
  = a\𝚍{φ}{t}⦅-\sinφ\\\cosφ⦆-a\𝚍{φ}{t}⦅-\sinφ\\\cosφ⦆
    -(l₀-aφ)\𝚍{φ}{t}⦅\cosφ\\\sinφ⦆\∅
    &
  = -(l₀-aφ)\𝚍{φ}{t}⦅\cosφ\\\sinφ⦆
\end{align}
\begin{align}
  L &= m\det(𝒓,𝒗) = -m(l₀-aφ)²\𝚍{φ}{t}\det⦅-\sinφ&\cosφ\\\cosφ&\sinφ⦆\∅
  &
  = m(l₀-aφ)²\𝚍{φ}{t}
\end{align}
\subsubsection*{
  [3]
}
\begin{align}
  𝒗² = (l₀-aφ)²(\𝚍{φ}{t})² = v₀²
\end{align}
より、
\begin{align}
  \𝚍{φ}{t} = ÷{v₀}{l₀-aφ}
\end{align}
根号を取る際に$\𝚍*{φ}{t}>0$から正符号を採用した。これを解くと、
\begin{align}
  &
  (l₀-aφ)\𝑑{φ} = v₀\𝑑{t}\∅
  &
  ÷1{2}aφ²-l₀φ+v₀t+C=0
\end{align}
$t=0$で$φ=0$となることから$C=0$であり、
\begin{align}
  φ = ÷{l₀}{a}-√{÷{l₀²}{a²}-÷{2v₀t}{a}}
\end{align}
となる。ただし$t=0$のときに$φ=0$となるように第2項の符号を決めた。

次に、$l₀-aφ=0$となる時刻$τ$は
\begin{align}
  ÷{l₀²}{a²} - ÷{2v₀τ}{a} = 0,␣
  τ = ÷{l₀²}{2v₀a}
\end{align}
で与えられる。

\subsubsection*{
  [4]
}
質点に働く力は張力のみなので、
\begin{align}
  𝑻 &= m\𝚍^2{𝒓}{t}
  = -v₀\𝚍{φ}{t}⦅-\sinφ\\\cosφ⦆\∅
  &
  = -÷{mv₀²}{l₀-aφ}⦅-\sinφ\\\cosφ⦆
\end{align}
\subsubsection*{
  [5]
}
モーメントは
\begin{align}
  N &= \det(𝒓,𝑻)
  = -÷{mv₀²a}{l₀-aφ}\det⦅\cosφ&-\sinφ\\\sinφ&\cosφ⦆\∅
  &
  = -÷{mv₀²a}{l₀-aφ}.
\end{align}
一方
\begin{align}
  \𝚍{L}{t} = \𝚍_t(mv₀(l₀-aφ)) = -mv₀a\𝚍{φ}{t}
  = -÷{mv₀²a}{l₀-aφ}
\end{align}
となって、$N=\𝚍*{L}{t}$が成り立つ。

\newpage
\subsection*{
  第2問
}
\subsubsection*{
  [1.1]
}
散乱を受けてから$t$経った後の電子の速度は
\begin{align}
  v = ÷{qE}{m}t
\end{align}
よって、
\begin{align}
  v_a = ÷{1}{2τ}∫_0^{2τ}÷{qE}{m}t\𝑑{t}
  = ÷{qEτ}{m}
\end{align}
\subsubsection*{
  [1.2]
}
電流は、
\begin{align}
  IL = qv_a⋅n𝜋a²L
\end{align}
を満たすので、
\begin{align}
  I = ÷{𝜋a²q²nτ}{mL}V,␣ V = EL
\end{align}
となる。
よってOhmの法則が成り立つ。
また
\begin{align}
  ÷{I}{𝜋a²} = σE
\end{align}
から
\begin{align}
  σ = ÷{q²nτ}{m}
\end{align}
\subsubsection*{
  [1.3]
}
Joule熱は
\begin{align}
  J = σE² = ÷{q²nτ}{m}E²
\end{align}
である。一方電子が単位時間、単位体積あたりに失うエネルギーは
\begin{align}
  n⋅÷{1}{2τ}⋅÷{1}{2}m(÷{qE}{m}2τ)²
  = ÷{q²nτ}{m}E²
\end{align}
となって、両者は一致する。

\subsubsection*{
  [2.1]
}
\begin{align}
  ∇×𝑯 = 𝒋
\end{align}
から、
\begin{align}
  2𝜋rH(r) = 𝜋r²|𝒋|
\end{align}
よって、
\begin{align}
  H(r) = ÷{1}{2}r|𝒋| = ÷{1}{2}σrE
\end{align}
となる。よって、
\begin{align}
  𝑺 = 𝑬×𝑯 = ÷{1}{2}σrE²𝒆_z×𝒆_θ
  = -÷{1}{2}σrE²𝒆_r
\end{align}
となる。$I/𝜋a² = σE$から、
\begin{align}
  𝑺 = -÷{I²}{2𝜋²a⁴σ}r𝒆_r
\end{align}
\subsubsection*{
  [2.2]
}
$C$に流入するPoyntingベクトルの総量は、
\begin{align}
  ÷{1}{2}σrL²⋅2𝜋rL = 𝜋r²LσE²
\end{align}
一方、$C$内での単位時間あたりのJoule熱は、
\begin{align}
  σE² = 𝜋r²L
\end{align}
よって両者は一致する。
この系では流れ込んだエネルギーと同量のエネルギーがJoule熱に変わる。
したがって、Joule熱を含めればエネルギー保存則が成り立っている。

\subsubsection*{
  [2.3]
}
\begin{align}
  &
  j(r) = σ(r)E \\
  &
  I = ∫_0^a 𝜋rj(r)\𝑑{r}
  = 𝜋E∫_0^arσ(r)\𝑑{r}
\end{align}
より、
\begin{align}
  j(r) = ÷{σ(r)I}{𝜋∫_0^arσ(r)\𝑑{r}}
\end{align}
次に、磁場は
\begin{align}
  H(r) = ÷{E}{2r}∫_0^r r'σ(r')\𝑑{r'}
\end{align}
となるから、
\begin{align}
  |𝑺| = ÷{E²}{2r}∫_0^r r'σ(r')\𝑑{r'}.
\end{align}
$C$に流れ込むエネルギーの総量は、
\begin{align}
  |𝑺|⋅2𝜋L &
  = 2𝜋L⋅÷{E²}{2r}∫_0^r r'σ(r')\𝑑{r'} \∅
  &
  = 𝜋LE²∫_0^r r'σ(r')\𝑑{r'}
\end{align}
一方、Joule熱は
\begin{align}
  L⋅𝜋∫_0^r r'σ(r')\𝑑{r'}
\end{align}
となって両者は一致する。
\newpage
\subsection*{
  物理学II
}
\subsection*{
  第1問
}
\subsubsection*{
  [1]
}
まず、
\begin{align}
  J_z = ⦅
    1&0&0\\
    0&0&0\\
    0&0&-1
  ⦆
\end{align}
である。また
\begin{align}
  J_+ = √2⦅
    0&1&0\\
    0&0&1\\
    0&0&0
  ⦆,␣
  J_- = √2⦅
    0&0&0\\
    1&0&0\\
    0&1&0
  ⦆
\end{align}
より、
\begin{align}
  J_x = ÷{1}{√2}⦅
    0&1&0\\
    1&0&1\\
    0&1&0
  ⦆,␣
  J_y = ÷{¡}{√2}⦅
    0&-1&0\\
    1&0&-1\\
    0&1&0
  ⦆
\end{align}
\subsubsection*{
  [2]
}
Hamiltonianを行列表示すると、
\begin{align}
  H_𝑀 = -γ𝑱⋅𝑩
  = -γ⦅
    B&0&0\\
    0&0&0\\
    0&0&-B
  ⦆
\end{align}
となる。固有値は$0,±γB$。

\subsubsection*{
  [3]
}
$x$軸方向に磁場$𝑩_\𝚞{RF} = (B_\𝚞{RF}\cosωt,0,0)$を掛ける。
ただし、$ħω=γB$とする。
通常の相互作用表示の扱いだと、
\begin{align}
  H_I
  &
  = -γB_\𝚞{RF} ℯ^{-¡ωtJ_z}J_xℯ^{¡ωtJ_z}\∅
  &
  = -γB_\𝚞{RF}ℯ^{-¡ωt\ad(J_z)}J_x\∅
  &
  = -γB_\𝚞{RF}(\cos(ωt)J_x+\sin(ωt)J_y)
\end{align}
だが、ここでは$ωt ≪ 1$として、
\begin{align}
  H_I ≈ H_\𝚞{RF} = -γB_\𝚞{RF}J_x
\end{align}
と近似する。このとき相互作用表示の状態$|Ψ(t)⟩$は、
\begin{align}
  |Ψ(t)⟩ = ℯ^{-¡H_It}|Ψ(0)⟩
  = ℯ^{¡γB_\𝚞{RF}tJ_x}|1_z⟩
\end{align}
となる。ここで
\begin{align}
  \exp[÷{¡θ}{√2}⦅0&1&0\\1&0&1\\0&1&0⦆]
  =÷{1}{2} ⦅
     \cosθ+1&√2¡\sinθ& \cosθ-1\\
    √2¡\sinθ&  2\cosθ&√2¡\sinθ\\
     \cosθ-1&√2¡\sinθ& \cosθ+1
  ⦆
\end{align}
より、
\begin{align}
  |Ψ(t)⟩
  = ÷{\cos(γB_\𝚞{RF}t)+1}{2}|{1_z}⟩
  +÷{¡\sin(γB_\𝚞{RF}t)}{√2}|{0_z}⟩
  +÷{\cos(γB_\𝚞{RF}t)-1}{2}|{-1_z}⟩
\end{align}
% \begin{align}
%   |{-1_x}⟩ &= ÷{1}{2}|{-1_z}⟩
%               -÷{1}{√2}|{0_z}⟩
%               +÷{1}{2}|{1_z}⟩,\\
%   |{1_x}⟩ &= ÷{1}{2}|{-1_z}⟩
%             + ÷{1}{√2}|{0_z}⟩
%             + ÷{1}{2}|{1_z}⟩,\\
%   |{0_x}⟩ &= -÷{1}{√2}|{-1_z}⟩+÷{1}{√2}|{1_z}⟩
% \end{align}
% \begin{align}
%   |1_z⟩ = ÷{1}{2}|{-1_x}⟩+÷{1}{√2}|{0_x}⟩+÷{1}{2}|{1_x}⟩
% \end{align}
% から、
% \begin{align}
%   |Ψ(t)⟩_I
%   = 
% \end{align}
% \begin{align}
%   ℯ^{J_x} = \cosh J_x + \sinh J_x
% \end{align}
% \begin{align}
%   J_x² = 2⦅1&0&1\\0&2&0\\1&0&1⦆
% \end{align}

\subsubsection*{
  [4]
}
\begin{align}
  H_Q = A(3J_x² - J²) = 3AJ_x² - 2A
\end{align}
と書ける。ここで、
\begin{align}
  J_x² = ÷{1}{2}⦅1&0&1\\0&2&0\\1&0&1⦆
\end{align}
から、
\begin{align}
  ⟨1|H_Q|1⟩ = -÷{1}{2}A,␣
  ⟨0|H_Q|0⟩ = 2A,␣
  ⟨-1|H_Q|-1⟩ = -÷{1}{2}A
\end{align}
となる。
\newpage
\subsection*{
  第2問
}
\subsubsection*{
  [1]
}
\begin{align}
  Z_A = (Z_{A1})^N
\end{align}
\begin{align}
  Z_{A1} = ÷{L}{2𝜋ħ}∫_{-∞}^∞\𝑑{p}\exp(-÷{βp²}{2m})
  = √{÷{m𝑘T}{2𝜋ħ²}}L
\end{align}
\begin{align}
  Z_A = (÷{L}{λ})^N
\end{align}
\subsubsection*{
  [2]
}
\begin{align}
  F(T,L) = -𝑘T\ln Z_A = 𝑘NT\ln(÷{L}{λ})
\end{align}
\subsubsection*{
  [3]
}
\begin{align}
  μ = \∂{F}{N} = -𝑘T\ln(÷{L}{λ})
\end{align}
\subsubsection*{
  [4]
}
\begin{align}
  E = -\∂_β\ln Z_{A1} = ÷{𝑘T}{2}
\end{align}
\begin{align}
  \exp(÷{μ-E}{𝑘T})
  = ÷{λ}{L}ℯ^{-1/2} ≪ 1
\end{align}
よって
\begin{align}
  ÷{λ}{L} ≪ 1
\end{align}
であればよい。
$λ$は長さの次元を持つ量で、ある温度における状態の典型的な波長を表す。

\subsubsection*{
  [5]
}
\begin{align}
  Z_B = ÷{1}{N!}Z_A(L-Nd) = ÷{1}{N!}(÷{L-Nd}{λ})^N
\end{align}
\begin{align}
  F = -N𝑘T[\ln(÷{L-Nd}{λ}) -\ln N + 1]
\end{align}
状態方程式は、
\begin{align}
  P = -\∂F{L} = ÷{N𝑘T}{L-Nd}
\end{align}
となる。気体$A$と違って$L → Nd$で圧力が発散するため、気体を$Nd$以下に圧縮できない。
\newpage
\subsection*{
  第3問
}
\subsubsection*{
  [1]
}
\begin{align}
  ÷{-n²ω²}{c²}+k_x²-κ₂² = 0
\end{align}
\begin{align}
  κ₂ = √{k_x²-÷{n²ω²}{c²}}
\end{align}
\begin{align}
  k_x > ÷{n²ω²}{c²}
\end{align}
\subsubsection*{
  [2]
}
\begin{align}
  𝒌⋅𝑬₁ = k_xE_{1,x} + k₁E_{1,z} = 0
\end{align}
また
\begin{align}
  𝒌×𝑬₁ = ωμ₀𝑯₁
\end{align}
から、
\begin{align}
  -k_xE_{1,z}+k₁E_{1,x} = ωμ₀H_{1,y}
\end{align}
となる。これらを連立して
\begin{align}
  ÷{k_x²+k₁²}{k₁}E_{1,x} = ÷{n²ω²/c²}{k₁}E_{1,x} = ωμ₀H_{1,y}
\end{align}
\begin{align}
  ÷{E_{1,x}}{H_{1,y}} = ÷{k₁c²μ₀}{n²ω} = ÷{k₁}{ϵ₁ω}
\end{align}
となる。つぎに同様の議論を媒質2について行えば、
\begin{align}
  ÷{E_{2,x}}{H_{2,y}} = ÷{k₂}{ϵ₂ω}
\end{align}
となる。
\subsubsection*{
  [3]
}
\begin{align}
  ϵ₁μ₀ω² = k_x² - κ₁²,␣
  ϵ₂μ₀ω² = k_x² - κ₂²,␣
\end{align}
また境界条件
\begin{align}
  E_{1,x} = E_{2,x},␣
  H_{1,x} = H_{2,x}
\end{align}
\begin{align}
  ÷{κ₁}{ϵ₁ω} = ÷{κ₂}{ϵ₂ω}
\end{align}
\begin{align}
  ÷{κ₁²}{κ₂²} = ÷{ϵ₁²}{ϵ₂²} = ÷{k_x²-ϵ₁μ₀ω²}{k_x²-ϵ₂μ₀ω²}
\end{align}
よって、
\begin{align}
  k_x
  &
  = √{÷{-ϵ₁²μ₀ω²/ϵ₂+ϵ₁μ₀ω²}{-ϵ₁²/ϵ₂²+1}}\∅
  &
  = √{÷{ϵ₁ϵ₂μ₀ω²}{ϵ₂+ϵ₁}}\∅
  &
  = √{÷{|ϵ₂|ϵ₀μ₀ω²}{|ϵ₂|-ϵ₁}}
\end{align}
% 境界条件は、
% \begin{align}
%   &
%   E_{1,x} = E_{2,x},␣
%   H_{1,x} = H_{2,x}\\ &
%   ϵ₁E_{1,z} = ϵ₂E_{2,z},␣
%   H_{1,z} = H_{2,z},␣
% \end{align}
% である。
% \begin{align}
%   ÷{E_{1,x}}{H_{1,y}}
% \end{align}
\subsubsection*{
  [4]
}
プリズムと金属の間の真空領域では、[1]から電場は$z$方向について指数関数的に減衰し、振動する電場成分は取り除かれる。
これによって[3]の電場を励起することができる。
\newpage
\subsection*{
  第4問
}
\subsubsection*{
  [1]
}
周期的境界条件から、
\begin{align}
  ℯ^{¡kMa} = 1,␣k ∈ ÷{2𝜋}{Ma}ℤ
\end{align}
\subsubsection*{
  [2]
}
\begin{align}
  E(k) = ÷{1}{M}[M(E_A+2β)+Mγ(ℯ^{¡ka}+ℯ^{-¡ka})]
  = E_A+2β+2γ\cos(ka)
\end{align}
\subsubsection*{
  [3]
}
波数が小さい方がエネルギーも小さくなるので、$γ < 0$。
また結晶の安定性から$β < 0$。
\subsubsection*{
  [4]
}
\begin{align}
  ÷{1}{ħ²}\∂^2{E}{k} = -÷{2γa²}{ħ²}\cos(ka)
\end{align}
を用いると、電子と正孔の有効質量はそれぞれ、
\begin{align}
  m_e^* = ÷{ħ²}{-2γa²},␣
  m_h^* = ÷{ħ²}{2γa²}
\end{align}
\subsubsection*{
  [5]
}
ハミルトニアン
\begin{align}
  ⟨ϕ_A(j)|H|ϕ_A(j)⟩ = E_A+2β,\\
  &
  ⟨ϕ_A(j)|H|ϕ_A(j±1)⟩ = γ,\\
  &
  ⟨ϕ_B(j)|H|ϕ_B(j)⟩ = E_B,\\
  &
  ⟨ϕ_A(j)|H|ϕ_B(j)⟩ = 2δ,
\end{align}
の対角化をすればよい。波動関数を
\begin{align}
  |Ψ_k⟩ = ∑_j ℯ^{¡kja}(c_k|ϕ_A(j)⟩+d_k|ϕ_B(j)⟩)
\end{align}
とすると、$c_k,d_k$に対して
\begin{align}
  ⦅
    E_A+2β+2γ\cos(ka) & 2δ \\
    2δ & E_B
  ⦆⦅c_k\\d_k⦆ = E(k)⦅c_k\\d_k⦆ 
\end{align}
が成り立つ。
\subsubsection*{
  [6]
}
まず$δ=0$のとき、2つのバンドの固有値は
\begin{align}
  E(k) = E_A+2β+2γ\cos(ka),␣ E_B
\end{align}
となる。
次に$δ ≠ 0$のとき、
\begin{align}
  E(k)
  = ÷{E_A+E_B}{2} + β + γ\cos(ka)
  ± √{(÷{E_A-E_B}{2}+β+γ\cos(ka))²+4δ²}
\end{align}
となる。
バンド図としては、$δ=0$で2つのバンドが交わる点において、
$δ ≠ 0$では縮退が解ける。
\end{document}