\providecommand{\main}{../main}
\documentclass[\main/main.tex]{subfiles}
\graphicspath{{../images/}}
\begin{document}
\newpage
\section{2015年(平成27年)}
\subsection*{
  物理学I
}
\subsection*{
  第1問
}
\subsubsection*{
  [1]
}
Lagrangianは
\begin{align}
  L = ÷{1}{2}m₁l₁²\.θ₁² + m₁gl₁\cosθ₁
\end{align}
なので、運動方程式は
\begin{align}
  m₁l₁²\"θ₁ = -m₁gl₁\sinθ₁
\end{align}
となる。$θ₁ ≪ 1$の場合、
\begin{align}
  \"θ₁ ≈ -÷{g}{l₁}θ₁
\end{align}
から運動は単振動となり、各周波数は$ω₀ = √{g/l₁}$で与えられる。
\subsubsection*{
  [2]
}
単振動の復元力が重力から来ており、$m₁$に比例するため、$ω₀$は$m₁$に依存しない。
撃力を加えた場合、振幅が$1/m₁$に比例する。
\subsubsection*{
  [3]
}
$ϕ = θ₁+θ₂$とおく。質点1,2の座標は
\begin{align}
  ⦅x₁\\y₁⦆ = ⦅l₁\sinθ₁\\-l₁\cosθ₁⦆,␣
  ⦅x₂\\y₂⦆ = ⦅l₁\sinθ₁+l₂\sinϕ\\-l₁\cosθ₁-l₂\cosϕ⦆
\end{align}
となる。ここで$θ₁,ϕ$を微小に動かすことを考えると、
\begin{align}
  &
  \𝑑{y₁} = l₁\sinθ₁\𝑑{θ₁}\\
  &
  \𝑑{x₂} = l₁\cosθ₁\𝑑{θ₁} + l₂\cosϕ\𝑑{ϕ}\\
  &
  \𝑑{y₂} = l₁\sinθ₁\𝑑{θ₁} + l₂\sinϕ\𝑑{ϕ}
\end{align}
となる。拘束力は仕事をしないことに注意して、力の釣り合いから、
\begin{align}
  F\𝑑{x₂}-m₁g\𝑑{y₁}-m₂g\𝑑{y₂} = 0
\end{align}
となる。$\𝑑{θ₁},\𝑑{ϕ}$の係数から
\begin{align}
  &
  Fl₁\cosθ₁-(m₁+m₂)gl₁\sinθ₁ = 0,\\
  &
  Fl₂\cosϕ-m₂gl₂\sinϕ = 0.
\end{align}
よって、
\begin{align}
  θ₁ = \arctan÷{F}{(m₁+m₂)g},␣
  ϕ = \arctan÷{F}{m₂g}
\end{align}
となる。$θ₂$については、
\begin{align}
  θ₂ = \arctan÷{F}{m₂g}-\arctan÷{F}{(m₁+m₂)g}
\end{align}
である。
\subsubsection*{
  [4]
}
\begin{align}
  L &
  = ÷{1}{2}ml²(\.θ₁²+(\.θ₁²+\.ϕ²+2\cos(θ₁-ϕ)\.θ₁\.ϕ))
     +mgl(\cosθ₁ + (\cosθ₁+\cosϕ))\∅
  &
  = ÷{1}{2}ml²(2\.θ₁²+(\.θ₁+\.θ₂)²+2\cos(θ₁-ϕ)\.θ₁(\.θ₁+\.θ₂))+mgl(2\cosθ₁+\cos(θ₁+θ₂))
\end{align}
\subsubsection*{
  [5]
}
$θ₁,ϕ$について2次の項を無視すると、
\begin{align}
  &
  \𝚍_t\∂{L}{\.θ₁} = 2ml²\"θ₁ + ml²\"ϕ \\
  &
  \∂{L}{θ₁} = -2mglθ₁\\
  &
  \𝚍_t\∂{L}{\.ϕ} = ml²\"ϕ + ml²\"θ₁\\
  &
  \∂{L}{ϕ} = -mglϕ
\end{align}
となる。よって、運動方程式は
\begin{align}
  ⦅2&1\\1&1⦆⦅\"θ₁\\\"ϕ⦆ = -÷{g}{l}⦅2&0\\0&1⦆⦅θ₁\\ϕ⦆
\end{align}
と書ける。$θ₁,θ₂$についてまとめると、
\begin{align}
  ⦅3&1\\2&1⦆⦅\"θ₁\\\"θ₂⦆ = -÷{g}{l}⦅2&0\\1&1⦆⦅θ₁\\θ₂⦆
\end{align}
\subsubsection*{
  [6]
}
解として$θ_i = A_iℯ^{-¡ωt}$と仮定すると、
\begin{align}
  ⦅2ω²-2ω₀²&ω²\\ω²&ω²-ω₀²⦆⦅A₁\\A₁+A₂⦆ = ⦅0\\0⦆
\end{align}
となる。ただし$ω₀ = g/l$である。
よって、非自明な解が存在するためには、
\begin{align}
  ω⁴-4ω₀²ω²+2ω₀⁴ = 0
\end{align}
でなければならない。よって
\begin{align}
  ÷{ω²}{ω₀²} = 2±√2
\end{align}
となる。よってこの系は基準振動
\begin{align}
  ω = ω₀√{2±√2}
\end{align}
をもつ。
\newpage
\subsection*{
  第2問
}
\begin{align}
  B_x = B₀αz,␣
  B_y = 0,␣
  B_z = B₀(1+αx)
\end{align}
\subsubsection*{
  [1]
}
\begin{align}
  Φ = B₀∫\𝑑{S}(1+αx)
  = 𝜋a²B₀(1+αX)
\end{align}
% \begin{align}\cosθ\sin2θ/2 (\sin3θ+\sinθ)/4
%   𝑨 = ⦅-B₀y(1+αx)\\0\\B₀αyz⦆
% \end{align}
% とおける。$(X+a\cosθ,a\sinθ,0)$に対し、
% \begin{align}
%   𝑨 = ⦅-B₀a\sinθ(1+αX+αa\cosθ)\\0\\0⦆
% \end{align}
% \begin{align}
%   Φ
%   = B₀a²∫_0^{2𝜋}\sin²θ(1+αX+αa\cosθ)\𝑑{θ}
%   = ÷{𝜋a²B₀}{2}(1+αX)
% \end{align}
\subsubsection*{
  [2]
}
磁束の時間変化は
\begin{align}
  \𝚍{Φ}{t} = 𝜋a²B₀αV = IR
\end{align}
となる。よって
\begin{align}
  I = ÷{𝜋a²B₀αV}{R}
\end{align}
\subsubsection*{
  [3]
}
加えた仕事とJoule熱が等しくなるので、
\begin{align}
  RI² = FV,␣ F = ÷{(𝜋a²B₀α)²V}{R}
\end{align}
\subsubsection*{
  [4]
}
時計回りの電流を$I$とおく。Lorentz力は、
\begin{align}
  𝑰×𝑩 = ⦅I\sinθ\\-I\cosθ\\0⦆×⦅0\\0\\B₀b(t)(1+αx)⦆
  = ⦅-I\cosθB₀b(t)(1+αx)\\-I\sinθB₀b(t)(1+αx)\\0⦆
\end{align}
から、
\begin{align}
  a∫_0^{2𝜋}𝑰×𝑩\𝑑{θ} = -𝜋a²B₀αb(t)I⦅1\\0\\0⦆
\end{align}
である。よって
\begin{align}
  I = ÷{1}{R}\𝚍{Φ}{t} =  ÷{𝜋a²B₀}{R}\𝚍_t[b(t)(1+αX)]
\end{align}
から、運動方程式は
\begin{align}
  m\𝚍^2{X}{t} = -÷{(𝜋a²B₀)²α}{R}b(t)\𝚍_t[b(t)(1+αX)]
\end{align}
となる。
\begin{align}
  λ = ÷{(𝜋a²B₀α)²}{mR}
\end{align}
\subsubsection*{
  [5]
}
$X$を
\begin{align}
  X = ∑_{k=0}^∞ a_kt^k
\end{align}
と展開する。
$X(0)=0,~X'(0)=0$から$a₀=a₁=0$。また運動方程式
\begin{align}
  α\𝚍^2{X}{t} = -÷{λt}{τ²}[1+αX+αt\𝚍{X}{t}]
\end{align}
の$t^0$の係数から、
\begin{align}
  2αa₂ = 0
\end{align}
よって$a₀=a₁=a₂=0$。
\subsubsection*{
  [6]
}
運動方程式の$t^1$の係数から
\begin{align}
  6αa₃ = -÷{λ(1+αa₀)}{τ²}
\end{align}
よって$a₀=0$から
\begin{align}
  a₃ = -÷{λ}{6ατ²}
\end{align}
となる。$X = a₃t³$という近似のもとで、
\begin{align}
  V₀ = X'(τ) = 3a₃τ² = -÷{λ}{2α}
\end{align}
\subsubsection*{
  [7]
}
$t ≥ τ$での運動方程式は
\begin{align}
  α\𝚍^2{X}{t} = -λα\𝚍{X}{t}
\end{align}
であるから、$X(τ)=X₀,X'(τ)=V₀$を初期条件として、
\begin{align}
  X'(τ+t) = V₀ℯ^{-λt},␣
  X(τ+t) = X₀ + ÷{V₀}{λ}(1-ℯ^{-λt})
\end{align}
となる。よって
\begin{align}
  X(t → ∞) = X₀+÷{V₀}{λ} = -÷{λτ}{6α} - ÷{1}{2α}
\end{align}
\newpage
\subsection*{
  物理学II
}
\subsection*{
  第1問
}
\subsubsection*{
  [1]
}
\begin{align}
  [\^N,\^a] = [\^a^†,\^a]\^a = -\^a
\end{align}
である。$\^N$の固有値$n+1$の固有状態$|n+1⟩$に対し、
\begin{align}
  \^N(\^a|n+1⟩) = (\^a\^N-\^a)|n+1⟩ = n\^a|n+1⟩
\end{align}
から$\^a|n+1⟩$は$\^N$の固有値$n$の固有状態になる。
また、
\begin{align}
  ‖\^a|n+1⟩‖ = ⟨n+1|\^N|n+1⟩ = n+1
\end{align}
となるから、
\begin{align}
  \^a|n+1⟩ = √{n+1}|n⟩
\end{align}
と書ける。
また、
\begin{align}
  ‖\^a|0⟩‖ = ⟨0|\^N|0⟩ = 0
\end{align}
から$\^a|0⟩=0$。
\subsubsection*{
  [2]
}
\begin{align}
  \^a|Ψ(α)⟩ = ℯ^{-|α|²/2}∑_{n=1}^∞÷{α^n}{√{(n-1)!}}|n-1⟩
  = α|Ψ(α)⟩
\end{align}
\subsubsection*{
  [3]
}
\begin{align}
  ⟨\^N⟩ = |α|²,␣⟨\^H⟩ = ħω(|α|²+÷{1}{2})
\end{align}
\subsubsection*{
  [4]
}
\begin{align}&
  ⟨\^x⟩ = √{÷{ħ}{2mω}}⟨\^a+\^a^†⟩ = √{÷{ħ}{2mω}}(α+α^*),\\
  &
  ⟨\^p⟩ = -¡√{÷{mħω}{2}}⟨\^a-\^a^†⟩ = -¡√{÷{mħω}{2}}(α-α^*),\\
  &
  ⟨\^x²⟩ = ÷{ħ}{2mω}⟨\^a²+2\^a^†\^a+\^a^{†2}+1⟩
  = ÷{ħ}{2mω}((α+α^*)²+1),\\
  &
  ⟨\^p²⟩ = -÷{mħω}{2}⟨\^a²-2\^a^†\^a+\^a^{†2}-1⟩
  = -÷{ħ}{2mω}((α-α^*)²-1).
\end{align}
よって、
\begin{align}
  𝛥x = √{÷{ħ}{2mω}},␣
  𝛥p = √{÷{mħω}{2}}
\end{align}
であり、
\begin{align}
  𝛥x𝛥p = ÷{ħ}{2}
\end{align}
\subsubsection*{
  [5]
}
\begin{align}
  ℯ^{-¡\^Ht/ħ} = ℯ^{-¡\^N ωt-¡ωt/2}
\end{align}
から、
\begin{align}
  ℯ^{-¡\^Ht/ħ}|Ψ(α₀)⟩
  &
  = ℯ^{-¡ωt/2}ℯ^{-|α₀|²/2}∑_{n=0}^∞÷{α₀^n}{√{n!}}ℯ^{-¡nωt}|n⟩\∅
  &
  = ℯ^{-¡ωt/2}|Ψ(α₀ℯ^{-¡ωt})⟩.
\end{align}
よって、
\begin{align}
  α(t) = Aℯ^{-¡(ωt-θ)}
\end{align}
\subsubsection*{
  [6]
}
\begin{align}&
  ⟨\^x⟩_t = √{÷{2ħ}{mω}}\Re α(t)
  = √{÷{2ħ}{mω}}A\cos(ωt-θ),\\
  &
  ⟨\^p⟩_t = √{2mħω}\Im α(t) = -√{2mħω}A\sin(ωt-θ)
\end{align}
これは調和振動子の古典的運動に一致するので、準古典的状態と呼ぶのは妥当である。
\subsection*{
  第2問
}
\subsubsection*{
  [1]
}
Maxwellの関係式から
\begin{align}
  (\∂{T}{P})_S
  = (\∂{V}{S})_P
  = (\∂{V}{T})_P(\∂{S}{T})_P^{-1}
  = ÷{TVα}{C_P}
\end{align}
\subsubsection*{
  [2]
}
各領域のエンタルピーの微小変化は
\begin{align}
  \𝑑{H_i} = T\𝑑{S_i} + V\𝑑{P_i},␣ i = 1,2.
\end{align}
である。
高圧領域と低圧領域のどちらでも
$\𝑑{P_i}=0$であり、断熱過程であることから$\𝑑{S₁}+\𝑑{S₂}=0$なので、
\begin{align}
  \𝑑{H} = \𝑑{H₁} + \𝑑{H₂} = 0
\end{align}
となる。
\subsubsection*{
  [3]
}
\begin{align}
  (\∂{T}{P})_H
  &
  = -(\∂{H}{P})_T(\∂{H}{T})_P^{-1}\∅
  &
  = -÷{1}{C_P}[(\∂{G}{P})_T+T(\∂{S}{P})_T]\∅
  &
  -÷{1}{C_P}[V-T(\∂{V}{T})_P]\∅
  &
  = ÷{V(αT-1)}{C_P}
\end{align}
$C_P > 0,~V>0$から、
\begin{align}
  (\∂{T}{P})_H = ÷{V(αT-1)}{C_P} < ÷{TVα}{C_P} = (\∂{T}{P})_S
\end{align}
理想気体の場合、
\begin{align}
  α = ÷{1}{V}(\∂{V}{T})_P = ÷{1}{T}
\end{align}
から、Joule-Thomson係数は、
\begin{align}
  (\∂{T}{P})_H = 0
\end{align}
\subsubsection*{
  [4]
}
$PV$を$b/V$について1次まで展開すると、
\begin{align}
  PV = ÷{RT}{1-b/V} - ÷{a}{V}
  = RT + (RT-÷{a}{b})÷{b}{V} 
\end{align}
よって
\begin{align}
  B(T) = 1-÷{a}{bRT}.
\end{align}
ボイル温度は$B(T)=0$となる温度である。
このとき$PV = RT + 𝒪(b²/V²)$となり、気体は理想気体に近くなる。

\subsubsection*{
  [5]
}
\begin{align}
  V ≈ ÷{RT}{P} + (1-÷{a}{bRT})b
\end{align}
と近似すると、
\begin{align}
  αT = ÷{T}{V}(\∂{V}{T})_P
  &
  = ÷{RT}{PV} + ÷{a}{RTV} \∅
  &
  ≈ 1 - (1-÷{a}{bRT})÷{b}{V} + ÷{a}{RTV}.
\end{align}
よって、
\begin{align}
  (\∂{T}{P})_H = ÷{1}{C_P}(÷{2a}{RT}-b)
\end{align}
この符号が変化する温度は、
\begin{align}
  T_\𝚞{inv} = ÷{2a}{Rb}
\end{align}
\subsubsection*{
  [6]
}
$a$の寄与を無視すると、
体積$V$で排除体積$b$がある気体は、体積が$V-b$の理想気体として考えられる。
このとき体積の膨張$V₁ → V₂,~ V₂ > V₁$に対して
\begin{align}
  ÷{V₂-b}{V₁-b}
\end{align}
は$b$の増加関数である。
すなわち、理想気体で考えると$b$が大きくなるほど激しい膨張をすることになる。
それに対応して温度は低くなる。

\newpage
\subsection*{
  第3問
}
\subsubsection*{
  [1]
}
Lorentz力を無視すると、
\begin{align}
  \"𝒙 = -ω₀²𝒙 - ÷{e}{m}𝑬
\end{align}
\subsubsection*{
  [2]
}
$𝑷 = -eN𝒙$から、
\begin{align}
  \"𝑷 = -ω₀²𝑷 + ÷{e²N}{m}𝑬
\end{align}
\subsubsection*{
  [3]
}
% \begin{align}
%   𝒙 = -÷{e}{m(ω₀²-ω²)}𝑬
% \end{align}
% \begin{align}
  % 𝒋 = -÷{e²N}{m(ω₀²-ω²)}\∂{𝑬}{t}
% \end{align}
% \begin{align}
  % ∇×𝑩 = ε₀μ₀[1+÷{e²N}{mε₀(ω₀²-ω²)}]\∂{𝑬}{t}
% \end{align}
\begin{align}
  P = ÷{e²N}{m(ω₀²-ω²)}E
\end{align}
\subsubsection*{
  [4]
}
\begin{align}
  εE = (ε₀+÷{e²N}{m(ω₀²-ω²)})E
\end{align}
\begin{align}
  k² = (ε₀+÷{e²N}{m(ω₀²-ω²)})μ₀ω²
\end{align}
\subsubsection*{
  [5]
}
\begin{align}
  ÷{k}{ω} = √{ε₀μ₀(1+÷{e²N}{ε₀m(ω₀²-ω²)})}
\end{align}
根号の中身は$ω₀$および
\begin{align}
  ω₁ ≔ √{ω₀² + ÷{e²N}{ε₀m}}
\end{align}
を境に符号を変えるので、$0 < ω < ω₀,~ω₁ < ω$では
\begin{align}
  k = √{ε₀μ₀(1+÷{e²N}{ε₀m(ω₀²-ω²)})}ω,␣
\end{align}
となり、$ω₀ < ω < ω₁$では
\begin{align}
  k = ¡√{ε₀μ₀(÷{e²N}{ε₀m(ω²-ω₀²)}-1)}
\end{align}
となる。
グラフには発散があるが、粘性項を無視する近似をしているので、仕方ない。
\subsubsection*{
  [6]
}
Lorentz力を無視すると、
\begin{align}
  m\"𝒙 = -e𝑬
\end{align}
\subsubsection*{
  [7]
}
[5]の結果で$ω₀ = 0$とすると、
\begin{align}
  ÷{k}{ω} = √{ε₀μ₀(1-÷{e²N}{ε₀mω²})}
\end{align}
よって
\begin{align}
  ω_p = √{÷{e²N}{ε₀m}}
\end{align}
とすると、$ω < ω_p$のときに$k$は虚数になる。
\subsubsection*{
  [8]
}
誘電体に対しては電磁波は透過する。
金属に対しては電磁波は完全に反射し、金属内の電磁場は指数関数的に減衰する。
\newpage
\subsection*{
  第4問
}

\subsubsection*{
  [1.1]
}
\begin{align}
  N(ε) = ÷{L³}{(2𝜋)³}÷{4𝜋}{3}∏_i √{÷{2m_iε}{ħ²}}
\end{align}
\begin{align}
  ρ(ε) = ÷{1}{L³}\𝚍{N}{ε} = ÷{√{2m₁m₂m₃ε}}{2𝜋²ħ³}
\end{align}
\subsubsection*{
  [1.2]
}
\begin{align}
  \𝚍_t⦅k_x\\k_y\\k_z⦆
  &
  % = -e⦅
  %     0&B_z&0\\
  %     -B_z&0&0\\
  %     0&-0&0
  %   ⦆⦅
  %     1/m₁&0&0\\
  %     0&1/m₂&0\\
  %     0&0&1/m₃
  %   ⦆⦅k_x\\k_y\\k_z⦆\∅
  %   &
  = ⦅
    0&-eB/m₂&0\\
    eB/m₁&0&0\\
    0&0&0
  ⦆⦅k_x\\k_y\\k_z⦆
\end{align}
\begin{align}
  \𝚍_t⦅√{m₂}k_x\\√{m₁}k_y\\k_z⦆
  &
  =÷{eB}{√{m₁m₂}} ⦅
    0&-1&0\\
    1&0&0\\
    0&0&0
  ⦆⦅√{m₂}k_x\\√{m₁}k_y\\k_z⦆
\end{align}
\begin{align}
  ω_c = ÷{eB}{√{m₁m₂}}
\end{align}
\begin{align}
  √{m₂}k_x = √{m₂}k₀\cos(ω_ct)
\end{align}
より、
\begin{align}
  k_y = √{÷{m₂}{m₁}}k_{0x}\sin(ω_ct),␣
  k_z = k_{0z}
\end{align}
\subsubsection*{
  [1.3]
}
\subsubsection*{
  [1.4]
}
\begin{align}
  (x,y) = ÷{ħk_{0x}}{√{m₁}ω_c}
  (÷{1}{√{m₁}}\sin(ω_ct),  ÷{1}{√{m₂}}(1-\cos(ω_ct)))
\end{align}
\begin{align}
  z = ÷{ħk_{0z}}{m₃}t
\end{align}

\begin{align}
  ϵ(𝒌) = ±γ√{k_x²+k_y²}
\end{align}
\subsubsection*{
  [2.1]
}
\begin{align}
  N = ÷{4𝜋}{3}(÷{ϵ}{γ})³
\end{align}
\begin{align}
  ρ(ϵ) = ÷{4𝜋}{γ³}ϵ²
\end{align}
\subsubsection*{
  [2.2]
}
\begin{align}
  𝒗 = ÷{1}{ħ}\𝚍{ϵ}{𝒌} = ±÷{γ}{ħ}÷{𝒌}{|𝒌|}
\end{align}
\begin{align}
  ħ\𝚍{𝒌}{t} = ∓÷{γe}{ħ}÷{𝒌×𝑩}{|𝒌|}
\end{align}
$|𝒌|=k₀=\𝚞{const.}$となる解を求めると、
\begin{align}
  k ≔ k_x + ¡k_y = k₀ℯ^{¡ω_ct},␣
  ω_c  = ±÷{γeB}{ħ²k₀}
\end{align}
\subsubsection*{
  [2.3]
}
$w = x+¡y$とおくと、
\begin{align}
  \𝚍{w}{t} &= ±÷{γ}{ħ}÷{k}{|k|} = ±÷{γ}{ħ}ℯ^{¡ω_ct}
\end{align}
\begin{align}
  w = ∓÷{¡γ}{ħω_c}(ℯ^{¡ω_ct}-1) = -÷{ħk₀}{eB}(ℯ^{¡ω_ct}-1)
\end{align}
\end{document}