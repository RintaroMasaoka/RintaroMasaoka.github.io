\documentclass[12pt]{ltjsarticle}
\usepackage{amsmath,ascmac,amssymb,mathtools,siunitx,diffcoeff,inputenc}
\usepackage{mygraphics}
\graphicspath{{../images/}}
\usepackage[backend=biber]{biblatex}
\addbibresource{../ref/ref.bib}
\usepackage[oldfont,exchangeupit]{unicommand}
\usepackage{subfiles}

%% options %%
\renewcommand{\headfont}{\bfseries}
% \numberwithin{equation}{section}
% \renewcommand*\abstractname{}
% \setlength{\parindent}{0pt}
\begin{document}
\title{多様体上の量子力学}
\author{政岡凜太郎}
\maketitle

$ℝ^d$上の量子力学に対し、運動量演算子は
\begin{align}
    [p_i, x_j] = -¡δ_{ij}
\end{align}
を満たす。
この定義を一般の多様体に拡張したい。
まず位置座標を$x ∈ ℳ$として、一般の状態は
\begin{align}
    |ψ⟩ = ∫_ℳ \𝑑{\vol} ψ(x) |x⟩
\end{align}
と書かれるとする。
$ℳ$上の座標$(x¹,…,x^d)$に対して、運動量演算子は
\begin{align}
    p_i|x⟩ = ¡÷{∂}{∂x^i}|x⟩
\end{align}
と定義すれば良さそう。
ただし問題として、座標$(x¹,…,x^d)$は$ℳ$全域で定義されるとは限らない。
次に座標の取り方によらない定義として、
\begin{align}
    ⟨x|p_f|ψ⟩ = -¡f^i(x)÷{∂}{∂x^i}⟨x|ψ⟩
\end{align}
を考える。正準交換関係は
\begin{align}
    [g(x), p_f] = ¡f^i÷{∂g}{∂x^i}
\end{align}
もう一つの解決策として、外微分を用いて
\begin{align}
    p ≔ -¡𝑑
\end{align}
とする。
このとき波動関数の空間は関数だけでなく微分形式も含む。(超対称量子力学)


\begin{align}
    ⟨ξ|η⟩ = ∫ξ^*∧★η
\end{align}
\begin{align}
    ★(η_{i₁⋯i_m}𝑑x^{i₁}∧⋯∧𝑑x^{i_m})
    = ÷{√{|g|}}{}
\end{align}
\end{document}