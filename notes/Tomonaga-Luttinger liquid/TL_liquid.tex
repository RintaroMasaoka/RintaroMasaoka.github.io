\providecommand{\main}{.}
\documentclass[9pt, dvipdfmx]{beamer}
\usepackage{bxdpx-beamer}
\usepackage{minijs}
\renewcommand{\kanjifamilydefault}{\gtdefault}
\usetheme{metropolis}
\usecolortheme{spruce}
\setbeamercolor{block title}{fg = white, bg = green!50!blue}
\setbeamercolor{block body}{fg = black, bg = white!90!black}
\setbeamercolor{block title example}{fg = white, bg = green!40!yellow!60!blue}
\setbeamercolor{block body example}{fg = black, bg = white!90!black}
\setbeamercolor{block title alerted}{fg = white, bg = green!30!yellow!60!blue}
\setbeamercolor{block body alerted}{fg = black, bg = white!90!black}
\setbeamercolor{itemize item}{fg=green!50!black}
\usefonttheme{professionalfonts}
\usepackage{amsmath,amsfonts,bm,graphicx,tikz,mathtools,physics,float}
\graphicspath{{./images/}}
\usepackage{hyperref}
\usepackage{pxjahyper}
\hypersetup{
    setpagesize=false,
    bookmarksnumbered=true,
    bookmarksopen=true,
    colorlinks=true,
    linkcolor=green!50!blue!60!white!65!black,
    citecolor=green!50!blue!80!black,
    urlcolor=green!60!yellow!75!red!85!blue
}
\newcommand{\del}{\partial}
\newcommand{\dblpi}{(2\pi)}
\newcommand{\kB}{k_\mathrm{B}}
\numberwithin{equation}{section}
\usepackage{subfiles}
\begin{document}
\begin{frame}{}
    \title{Jacobiの三重積と朝永-Luttinger模型}
    \author{政岡凜太郎}
    \titlepage
\end{frame}
\begin{frame}{Jacobiの三重積}
    こんな式がある。
    \begin{block}{Jacobiの三重積}
        \begin{align}
            \sum_{n\in \mathbb{Z}} q^{n(n+1)/2}z^n
            = \prod_{k=1}^{\infty}
                (1-q^{k})(1+q^{k}z)(1+q^{k-1}z^{-1})
        \end{align}
    \end{block}
    % 左辺では、$q^0,q^1,q^3,q^6,\ldots$に比例する項しか出てこない。
    % 右辺で$q^2$に掛かる係数を書き出してみると、
    % \begin{align}
    %     (1+z^{-1})(-1+z+z^{-1}-z+1-z^{-1}) = 0
    % \end{align}
    % となってキャンセルしている。
    % 同様のことが$q^4,q^5,q^7,\ldots$で起こる。
    これはもともとJacobiがテータ関数を研究する中で見つけたものである。

    複素関数論を使った証明も知られているが、ここでは組み合わせ論を使った証明をする。
\end{frame}
\subfile{1/1.tex}
\subfile{2/2.tex}
\subfile{3/3.tex}
\end{document}