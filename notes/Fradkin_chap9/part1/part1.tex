\providecommand{\main}{../main}
\documentclass[\main/main.tex]{subfiles}
\graphicspath{{../images/}}
\begin{document}
\section{
    ホモロジー・コホモロジー
}
\begin{frame}{チェイン}
    グラフ\footnote{
        正確にはCW複体というべき。
    }$G$の頂点の集合を$𝒱(G)$、
    辺の集合を$ℰ(G)$、
    面の集合を$ℱ(G)$と書く。
    より高次のセルは考えない。

    $𝒱(G),ℰ(G),ℱ(G)$の要素を基底とする形式的なベクトル空間を考え、
    それぞれ$C₀(G),C₁(G),C₂(G)$と書く。
    $C_p(G)$の元を$p$-チェインと呼ぶ。
    
    バウンダリー作用素$∂$は$p$-チェインに対して
    その境界の$(p-1)$-チェインを返す線形写像で、
    \begin{align}
        ∂² = 0
    \end{align}
    を満たしている。
    $C_p(G)$および$∂$の作用はグラフに関する全ての情報をもつ。
\end{frame}
\begin{frame}{コチェイン}
    双対グラフ$G^*$を
    \begin{align}
        𝒱(G^*) = ℱ(G),␣ ℰ(G^*) = ℰ(G),␣ℱ(G^*) = 𝒱(G)
    \end{align}
    によって構成する。また
    \begin{align}
        C₀(G^*) = C₂(G),␣
        C₁(G^*) = C₁(G),␣
        C₂(G^*) = C₀(G),␣
    \end{align}
    と対応付ける。$C_p(G^*)$の元を$p$-コチェインと呼ぶ。
    \footnote{
        今の構成ではベクトル空間とその双対空間を同一視しているので、
        チェインとコチェインは同じものである。
        ただし少数の部分を除いて大体サイクルなものはチェインと呼び、
        大体コサイクルなものをコチェインと呼ぶことにする。
    }
    
    コバウンダリー作用素$\~∂$を
    \begin{align}
        \~∂ = ∂^𝑇
    \end{align}
    によって定義する。(Stokesの定理)
    
    $\~∂$は双対格子におけるバウンダリー作用素とみなせる。
    明らかに$\~∂² = 0$が成り立つ。
\end{frame}

\begin{frame}{(コ)サイクル・(コ)バウンダリー・(コ)ホモロジー}
    サイクルは
    \begin{align}
        ∂C = 0
    \end{align}
    を満たすチェインのことである。バウンダリーは
    \begin{align}
        C = ∂S
    \end{align}
    と書けるチェインのことである。
    $∂² = 0$からバウンダリーはサイクルである。
    サイクルにおいてバウンダリーの差異を無視した類をホモロジー類という。

    コサイクルは
    \begin{align}
        \~∂a = 0
    \end{align}
    を満たすコチェインのことである。コバウンダリーは
    \begin{align}
        a = \~∂λ
    \end{align}
    と書けるコチェインのことである。
    $\~∂² = 0$からコバウンダリーはコサイクルである。
    コサイクルにおいてコバウンダリーの差異を無視した類をコホモロジー類という。
\end{frame}

\begin{frame}{正方格子の場合}
    一般的に語ったが、正方格子しか使わないので、正方格子の場合を考える。
    ベクトル空間の基底は
    \begin{align}
        {v_i},{e_{ij}},{f_{ijkl}}
    \end{align}
    である。
    添字どうしの位置関係は文脈で読み取ってほしい。
    また
    \begin{align}
        e_{ji} = -e_{ij},␣
        f_{σ(i)σ(j)σ(k)σ(l)} = \sign(σ)f_{ijkl}
    \end{align}
    を課す。
    $∂$の作用は
    \begin{align}
        ∂v_i = 0,␣ ∂e_{ij} = v_i - v_j,␣
        ∂f_{ijkl} = e_{ij} + e_{jk} + e_{kl} + e_{li}
    \end{align}
    である。これを転置すると、
    \begin{align}
        \~∂v_i = e_{ij} + e_{ik} + e_{il} + e_{im},␣
        \~∂e_{ij} = f_{ijkl} - f_{mnji},␣
        \~∂f_{ijkl} = 0
    \end{align}
    のようになる。

    正方格子における$1$-サイクルは、ご想像の通り。
    $1$-コサイクルは双対格子での$1$-サイクルを元の格子に戻してきたものである。
    もちろん$\~∂$のカーネルと考えてもいい。
\end{frame}

\begin{frame}{正方格子の場合}
    次に成分表示の見方をする。
    グラフ上の$0$-形式、$1$-形式、$2$-形式
    \begin{align}
        ϕ₀ = ∑_{v ∈ 𝒱} ϕ₀(v)v,␣
        ϕ₁ = ∑_{e ∈ ℰ} ϕ₁(e)e,␣
        ϕ₂ = ∑_{f ∈ ℱ} ϕ₂(f)f
    \end{align}
    を考える。これらに$\~∂$を作用させると
    \begin{align}&
        \~∂ϕ₀ = ∑_{e_{ij} ∈ ℰ} (ϕ₀(v_i) - ϕ₀(v_j)) e_{ij},\∅
        &
        \~∂ϕ₁ = ∑_{f_{ijkl} ∈ ℱ} 
        (ϕ₁(e_{ij})+ϕ₁(e_{jk})+ϕ₁(e_{kl})+ϕ₁(e_{li})) f_{ijkl},\∅
        &
        \~∂ϕ₂ = 0
    \end{align}
    となって、ほとんど外微分である。
    よって格子間隔を小さくする極限で
    \begin{align}
        \~∂ → 𝑑,␣ ∂ → δ = (-1)^p ★^{-1}𝑑★
    \end{align}
    となる。
\end{frame}

\begin{frame}{コホモロジーの構成}
    コホモロジー群はホモロジー群の双対空間としても実現できる。

    ホモロジー類の代表元としてサイクル$C$をとってくる。
    内積を
    \begin{align}
        (v,v') = δ_{v,v'},␣
        (e,e') = δ_{e,e'},␣
        (f,f') = δ_{f,f'}
    \end{align}
    によって定める。
    $C$とコチェイン$a$との内積$(a,C)$が$C$の取り方に依らないためには、
    \begin{align}
        (a,∂S) = (\~∂a,S) = 0
    \end{align}
    が必要である。よって$a$はコサイクルである。
    一方$a$をコバウンダリー$\~∂λ$だけ変形すると、
    \begin{align}
        (a+\~∂λ,C) = (a,C) + (a,∂C) = (a,C)
    \end{align}
    となるから、$a$にコバウンダリーを足しても線形写像として同じものが得られる。

    ホモロジー群の双対空間はコホモロジー群となる。
\end{frame}
\section{
    量子ダイマー模型
}

\begin{frame}{\currentname}
    ダイマー配位$D ∈ 𝒟(G)$とは、グラフ$G$の辺集合の部分集合で、
    互いに共通の頂点を持たず、
    かつ全ての頂点を覆うようなもののことである。
    \footnote{
        ダイマー配位が存在するためには頂点の総数$|𝒱(G)|$が偶数である必要がある。
    }

    ダイマー配位はグラフ$G$上の$ℤ₂$係数$1$-チェインであって
    \begin{align}
        ∂D = ∑_{v_i ∈ 𝒱(G)} v_i
    \end{align}
    となるものとしても定義できる。
    ここで$𝒱(G)$はグラフ$G$の頂点集合である。
    $v_i$との内積をとると、
    \begin{align}
        (v_i,∂D) = (\~∂v_i,D) = 1
    \end{align}
    となる。これは教科書の(9.2)式に対応する。

    2つのダイマー配位$D₀,D₁ ∈ 𝒟(G)$に対して
    \begin{align}
        ∂(D₁-D₀) =  0
    \end{align}
    が成り立つ。
    したがって、1つの基準$D₀$を決めてしまえば任意の$D ∈ 𝒟(G)$は$1$-サイクルとしても表せる。
\end{frame}
\begin{frame}{\currentname}
    量子ダイマー模型のハミルトニアンは
    \begin{align}
        H_\𝚞{QDM} = H_\𝚞{res} + H_\𝚞{diag}
    \end{align}
    である。$H_\𝚞{res}$は共鳴項
    \begin{align}
        H_\𝚞{res} = \=J ∑_{\𝚞{plaquettes}}(
            |{=}⟩⟨‖| + |‖⟩⟨{=}|
        )
    \end{align}
    であり、$H_\𝚞{diag}$は対角項
    \begin{align}
        H_\𝚞{diag} = V ∑_{\𝚞{plaquettes}}(
            |{=}⟩⟨{=}| + |‖⟩⟨‖|
        )
    \end{align}
    である。
    $|V| ≫ |J|$の場合、基底状態を求めることは容易である。
    $V < 0$の場合、figure 8.12 (a) のように全てのダイマーが平行になる。
    $V > 0$の場合、figure 8.12 (b) のようにダイマーが互い違いに並ぶ。
\end{frame}
\begin{frame}{\currentname}
    $J/V=-1$をRokhsar - Kivelson (RK) pointといい、
    この点で厳密な基底状態が構成できる。
    具体的には
    \begin{align}
        |Ψ_{\𝚞{sRVB}}⟩ = ∑_{D ∈ 𝒟(G)} |D⟩
    \end{align}
    とする。つまり全てのダイマー配位を等しい重みで足し合わせた状態である。
    これは$J/V=-1$の場合に基底状態となっている。
    \begin{align}
        H_\𝚞{QDM}|Ψ_\𝚞{sRVB}⟩ = 0
    \end{align}
    $|Ψ_\𝚞{sRVB}⟩$は規格化されておらず、
    \begin{align}
        ‖Ψ_\𝚞{sRVB}‖² = Z_\𝚞{dimer}
    \end{align}
    となる。$Z_\𝚞{dimer}$は古典ダイマー模型の分配関数である。
    ダイマー基底について対角な演算子$𝒪$をもってくると
    \begin{align}
        ÷{⟨Ψ_\𝚞{sRVB}|𝒪|Ψ_\𝚞{sRVB}⟩}{‖Ψ_\𝚞{sRVB}‖²}
        = ÷{1}{Z_\𝚞{dimer}}∑_{D ∈ 𝒟(G)}⟨D|𝒪|D⟩
        = ⟨𝒪⟩_\𝚞{dimer}
    \end{align}
    となり、期待値が古典ダイマー模型の相関関数にマップできる。
\end{frame}
\begin{frame}{\currentname}
    正方格子の古典ダイマー模型の相関関数では、2つの平行な辺の間の相関関数が
    \begin{align}
        G(R) ∝ ÷{1}{R²}
    \end{align}
    となるらしい。
    これはcriticalな振る舞いになっている。

    三角格子の場合、相関関数は指数関数的に減衰する。
\end{frame}
\section*{
    格子$U(1)$ゲージ理論
}
\begin{frame}{\currentname}
    グラフの各辺に$\U(1) ≅ S¹$上の自由粒子のHilbert空間を考え、
    全ての積について直積をとる。
    \begin{align}
        ℋ = ⨂_{e ∈ ℰ} ℋ_e,␣
        ℋ_e ≅ ℋ_{\U(1)}
    \end{align}
    このとき各辺の自由度は角度変数$0 ≤ A ≤ 2𝜋$か
    、角運動量$l ∈ ℤ$で特徴づけられる。
    $e$上の角度演算子を$A(e)$、角運動量演算子を$E(e)$と書く。
    また$E(e)$の固有状態を$|l(e)⟩ ∈ ℋ_e$と書く。
    
    $A(e), E(e)$は正準交換関係
    \begin{align}
        [A(e),E(e')] = ¡δ_{e,e'}
    \end{align}
    を満たす。交換関係から明らかに
    \begin{align}
        ℯ^{-¡m\ad a}E = E + m
        \label{eq: adjoint action of a}
    \end{align}
    が成り立つ。
        $|0⟩$から一般の$|l⟩$は
    \begin{align}
        |l⟩ = ℯ^{¡ma}|0⟩
    \end{align}
    によって構成される。
    これは$|l⟩$の波動関数表示だが
    (\ref{eq: adjoint action of a})からもわかる。
\end{frame}
\begin{frame}{\currentname}
    $A = ∑_{e ∈ ℰ} A(e)e$に対してゲージ変換は
    \begin{align}
        A → A + \~∂α
    \end{align}
    と定義される。
    これを演算子として表現すると、
    \begin{align}
        ℯ^{¡(\~∂α,E)}
        = ℯ^{¡(α,Q)},␣
        Q(v) ≔ ∂E(v) = ∑_{e ∈ \~∂v} E(e)
    \end{align}
    となる。
    ただし$E = ∑_e E(e)e,~ Q = ∑_v Q(v)v$である。
    $Q(v)$と交換する演算子はゲージ不変な物理量となる。
    まず$[Q(v),E(e)] = 0$なので、$E(e)$はゲージ不変な物理量である。
    電磁場$\~∂A$やWilsonループ
    \begin{align}
        W(Γ) = ℯ^{(A,Γ)},␣ (∂Γ = 0)
    \end{align}
    もゲージ不変である。
    $Γ = ∂f$と取ったものを足し合わせてみると、
    \begin{align}
        ∑_f W(∂f)
        &
        = 1 + ∑_{f ∈ ℱ} (A,∂f) + ∑_{f ∈ ℱ}(A,∂f)² + ⋯ \∅
        &
        = 1 + ‖\~∂A‖²
    \end{align}
    となる。これはMaxwell作用$∫\𝑑{A}∧★\𝑑{A}$の離散的な対応物になっている。
\end{frame}
\begin{frame}{\currentname}
    量子ダイマー模型が$U(1)$ゲージ理論として書けることを示そう。
    まず、各辺の自由度を$l = 0,1$に限るために、
    \begin{align}
        H_\𝚞{dimer}
        = ÷{1}{2k}∑_e E(e)(E(e)-1)
    \end{align}
    を加えて$k → 0$とする。
    ここで2部グラフを仮定して、
    辺の向きは常に副格子$A$から副格子$B$への向きにとる。

    ダイマー配位の拘束条件は
    \begin{align}
        ∂E = Q = ∑v_B - ∑v_A
    \end{align}
    となる。
    $E$を電場とみなすならば、
    この条件はstaggerdな背景電荷を表しているとみなせる。
    ここから
    \begin{align}
        H_\𝚞{dimer}
        &
        = ÷{1}{2k}(∑_{e}E(e)²- ∑_{v_A}(E,\~∂v_A))\∅
        &
        = ÷{1}{2k}(‖E‖² - ÷{|𝒱|}{2})
    \end{align}
    と書ける。
\end{frame}
\begin{frame}{\currentname}
    次に、共鳴項は
    \begin{align}
        H_\𝚞{res} = 2\=J∑_f \cos F(f),␣
        F = \~∂A
    \end{align}
    と書ける。また対角項は
    \begin{align}
        H_\𝚞{diag} &
        = V∑_{f_{ijkl}}(
            E(e_{ij})E(e_{kl})
            + E(e_{jk})E(e_{li})
        )\∅
        &
        = ÷{V}{2} ∑_f (E,∂f)² - V ∑_e E(e)²\∅
        &
        = ÷{V}{2} ‖\~∂E‖² - ÷{V|𝒱|}{2}
    \end{align}
    と書ける。
    ただし$k → 0$の場合に$E(e)$がダイマー配位をとることを前提にして、
    \begin{align}
        E(e_{ij})E(e_{jk}) = E(e_{jk})E(e_{kl}) = ⋯ = 0,␣
        ∑_e E(e) = ÷{|𝒱|}{2}
    \end{align}
    とおいた。
\end{frame}
\section{
    古典ダイマー模型
}
\begin{frame}{\currentname}
    せっかく量子をやるのだから、古典ダイマー模型についても少し触れておく。

    グラフ$G$上のダイマー配位$D$に対し、作用を
    \begin{align}
        ℯ^{-S[D,w]} = ∏_{e_{ij} ∈ D} w_{ij}
    \end{align}
    で定める。ここで$w$は辺上に定められた正値の重みである。
    ダイマー模型の分配関数は
    \begin{align}
        Z[w] = ∑_{D ∈ 𝒟(G)}ℯ^{-S[D,w]}
    \end{align}
    で与えられる。
    $w_{ij}$を行列と考えれば、
    $Z[w]$はハフニアンと呼ばれる量に一致する。
    \begin{align}
        Z[w] = \haf w
        = ÷{1}{𝒩} ∑_{σ ∈ S_{2n}}w_{σ(1)σ(2)}w_{σ(3)σ(4)}⋯w_{σ(2n-1)σ(2n)}
    \end{align}
    ここで$|𝒱(G)| = 2n$とおいた。
    $𝒩$は数え上げの重複を除く因子であり、
    分配関数において定数倍には興味がないので省略した。

    ハフニアンはとても計算しにくい量なので、パフィアンに変換したくなってくる。
\end{frame}
\begin{frame}{\currentname}
    辺$e_{ij} ∈ ℰ(G)$に対し、向き付け
    \begin{align}
       ε_{ij} = ±1,␣ ε_{ji} = -ε_{ij}
    \end{align}
    を考える。
    これを用いて、
    \begin{align}
        a_{ij} ≔ ε_{ij}w_{ij}
    \end{align}
    と定義する。
    $a_{ij}$は構成から反対称行列である。
    向きのついたダイマー模型の分配関数は
    \begin{align}
        Z'[w,ε] = \Pf a
        = ÷{1}{𝒩}∑_{σ ∈ S_{2n}}
            \sign(σ)a_{σ(1)σ(2)}a_{σ(3)σ(4)}⋯a_{σ(2n-1)σ(2n)}
    \end{align}
    と表される。
    \alert{$G$が平面グラフの場合、}$Z'[w,ε] = Z[w]$となるような向き付け$ε$が常に存在することが知られている。
    すると、$(\Pf a)² = \det a$から分配関数の計算は$a$の固有値問題に帰着する。
    あるいは、
    \begin{align}
        \Pf a = ∫\𝒟{χ} ℯ^{χ_ia_{ij}χ_j}
    \end{align}
    と書いて自由なマヨラナフェルミオンに帰着する。
\end{frame}
\begin{frame}{\currentname}
    良い向き付け$ε$について、少し解像度を高めておこう。
    ただし、具体的に構成する方法については省略する。
    \footnote{
        気になる人はFisher-Kasteleyn-Temperleyのアルゴリズムで検索してみて。
    }
    満たすべき条件は
    \begin{align}
        \sign(σ)ε_{σ(1)σ(2)}⋯ε_{σ(2n-1)σ(2n)} = \const
    \end{align}
    となることである。ここで、
    \begin{align}&
        σ'(1) = σ(2), σ'(2) = σ(3), …, σ'(2l) = σ(1)\\
        &
        σ'(i) = σ(i) ␣(i > 2l)
    \end{align}
    となるような新たな置換$σ'$を考えてみよう。
    $\sign(σσ'^{-1}) = -1$から
    \begin{align}
        ε_{σ(1)σ(2)}ε_{σ(2)σ(3)}⋯ε_{σ(2l-1)σ(2l)}ε_{σ(2l)σ(1)} = -1
    \end{align}
    が成り立たなければならない。
    そこで、グラフの任意の面$f ∈ ℱ$について
    \begin{align}
        ∏_{e ∈ ∂f} ε(e) = -1
        \label{Z2 gauge invariance for classical dimer}
    \end{align}
    を課す。実は条件としてこれで十分である。
    \footnote{
        $g ≠ 0$の閉曲面上ではちょっと事情は複雑になる。
    }
\end{frame}
\begin{frame}{\currentname}
    次に同値な向き付けの概念を導入しよう。

    行列$a$の$i$行成分を全て反転すると、
    各々のダイマー配位の寄与は全て$-1$倍される。
    したがって同じ統計力学系が得られる。

    言い換えると、ある頂点に対して接する全ての辺の向きを反転させても系は不変である。
    この操作は$ℤ₂$ゲージ変換になる。まず
    \begin{align}
        ε_{ij} = ℯ^{¡𝜋K_{ij}}
    \end{align}
    と書く。2つの向き付け$ε'_{ij},ε_{ij}$の比が
    \begin{align}
        ε'_{ij}/ε_{ij} = ℯ^{¡𝜋(K'_{ij}-K_{ij})}
        = ℯ^{¡𝜋(δ_{ki} - δ_{kj})}
    \end{align}
    となるとき、分配関数が不変だと言っている。

    ここで$δ_{ki}-δ_{kj}$はコバウンダリー$\~∂v_k$の成分表示である。
    さらに、このようなコバウンダリーをいくつ足しても変わらないのだから、
    $K-K₀$を$ℤ₂$コホモロジー類の元、あるいは$ℤ₂$ゲージ場とみなすことができる。
\end{frame}
\begin{frame}{\currentname}
    同値でないゲージ場の数を数えてみよう。
    \begin{itemize}
        \item ゲージ場の自由度は$|ℰ(G)|$個
        \item 満たすべき条件は$|ℱ(G)|$個
        \item 同値なゲージ場への変換が$|𝒱(G)|-1$個
    \end{itemize}
    である。
    ここですべての頂点について$ℤ₂$ゲージ変換を行うと元のゲージ場に戻ってくることに注意。
    よって平面グラフでは
    \begin{align}
        2^{|ℰ(G)| - |ℱ(G)| - |𝒱(G)| + 1} = 2^0 = 1
    \end{align}
    個だけ同値でないゲージ場が存在する。
    種数が$g ≠ 0$の閉曲面では非自明なゲージ場が可能なことも、予想できるだろう。
    \footnote{
        その場合も古典ダイマー模型との間に色々と面白い対応があるのだが、
        スピン構造やArf不変量といった概念が出てきて、いまいち分かっていない。
    }

    ここらで撤退して、本筋に戻る。
\end{frame}
\end{document}