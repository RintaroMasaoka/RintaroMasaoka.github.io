\documentclass[8pt,unicode,xcolor=svgnames]{beamer}

% beamer settings %
\usepackage{luatexja}
\usepackage[ipaex]{luatexja-preset}% IPAex font
\renewcommand{\kanjifamilydefault}{\gtdefault}
\usetheme[
    outer/progressbar=foot,
    % outer/numbering=none
]{metropolis}
\usefonttheme{professionalfonts}
\setbeamercolor{background canvas}{bg=white}
\setbeamercolor{frametitle}{fg=white,bg=Teal}
\setbeamercolor{progress bar}{fg=Teal,bg=white!90!black}
\setbeamercolor{colorbox}{bg=Teal!10}
\setbeamercolor{block body}{bg=white!90!black}
\setbeamercolor{block title}{fg=white,bg=DarkGreen}
\setbeamercolor{block title example}{fg=white,bg=DarkSeaGreen}
\setbeamercolor{block title alerted}{fg=white,bg=DarkKhaki}
\setbeamercolor{itemize item}{fg=ForestGreen}
\makeatletter
\newcommand*{\currentname}{\@currentlabelname}
\setlength{\metropolis@titleseparator@linewidth}{2pt}
\setlength{\metropolis@progressonsectionpage@linewidth}{2pt}
\setlength{\metropolis@progressinheadfoot@linewidth}{2pt}
\makeatother

% packages %
\usepackage{amsmath,amssymb,mathtools,diffcoeff}
\usepackage[oldfont,exchangeupit]{unicommand}

% graphics %
\usepackage{graphicx,float,tikz}
\newcommand\Includegraphics[2][]{\vcenter{\hbox{\includegraphics[#1]{#2}}}}
\graphicspath{{../images/}}

% biblatex %
\usepackage[backend=biber]{biblatex}
\addbibresource{../ref/ref.bib}

% tcolorbox %
\usepackage{tcolorbox}
\tcbuselibrary{skins,theorems}
\tcbset{
    enhanced,
    frame hidden,
    sharp corners,
    colback=Teal!10,
    boxsep=0pt,
    fonttitle=\bfseries
}

% hyperref %
\usepackage{hyperref}
\hypersetup{
    unicode,
    setpagesize=false,
    bookmarksnumbered=true,
    bookmarksopen=true,
    colorlinks=true,
    linkcolor=DarkCyan!80!black,
    citecolor=Chocolate,
    urlcolor=Chocolate
}

\usepackage{subfiles}

% options %
\numberwithin{equation}{section}

\begin{document}
\section{14.6 Non-abelian quantum Hall states}
\begin{frame}{\currentname}
    エニオンの位置の交換は組み紐の群の表現として表せる。
    今まではabelianな表現を考えてきたが、以降はnon-abelianな表現に対応するエニオンを考える。

    有効理論としてはnon-abelianなゲージ場に対するChern-Simons理論が現れる。
    $(2+1)$次元のChern-Simons理論は$2$次元の共形場理論と深い関係をもつことが知られている。
    大雑把には
    \begin{align}
        Z(A,\_A) = ∑_α \_Ψ_α(\_A) Ψ_α(A)
    \end{align}
    という形。$Z(A,\_A)$はCFTの外場つきの分配関数。$Ψ_α(A)$はChern-Simons理論の基底状態であり、かつCFTのconformal blockになっている。
    波動関数は時空の境界条件を定めた経路積分によって表せるから、Chern-Simonsの境界にCFTが住んでいるとみなせる。

    さらに、Laughlin状態などの試行状態もCFTと関係づけることができる。
    これは試行状態の系統的な導出を可能にする。
\end{frame}
\subsection*{14.6.1 Conformal field theory and quantum Hall wave functions}
\begin{frame}{\currentname}
    Laughlin状態
    \begin{align}
        Ψ_m(z₁,…,z_N) = ∏_{i<j}(z_i-z_j)^m
        \exp(-∑_{i=1}^N ÷{|z_i|²}{4l₀²})
    \end{align}
    を改めて考察する。
    注目するべき点はこの波動関数が電子間のポテンシャルの情報を含んでいないことである。
    Laughlin状態はポテンシャルに依存して決まるものではなく、波動関数の長距離での性質を記述する普遍的な性質を記述するものである。
    普遍的な性質を理解する上で2D CFTによる記述がわかりやすい。
    唐突だが2次元Euclid空間上で定義されたカイラルボソン$φ(z)$を考えると、Laughlin状態は
    \begin{align}
        Ψ_m(z₁,…,z_N) = \⟨(∏_{i=1}^N ℯ^{¡√mφ(z_i)})\exp(-¡∫\𝑑^2{z'}√mρ₀φ(z'))\⟩
        \label{Laughlin/CFT}
    \end{align}
    と書ける。$φ(z)/√m$が何らかの位相の自由度であるとし、$φ(z)$と$φ(z) + 2π√m$を同一視する。
    プラズマアナロジーの文脈ではLaughlin状態は一様な背景電荷の中にある電子系を表していたが、今の場合電荷がvertex operatorに対応している。
\end{frame}
\begin{frame}{\currentname}
    フリーボソンについての復習をしておく。作用は
    \begin{align}
        S = ÷1{8π}∫ \𝑑^2x ∂_μϕ∂^μϕ
    \end{align}
    とする。相関関数は
    \begin{align}
        ⟨ϕ(z,\_z)ϕ(w,\_w)⟩ = -\ln|z-w|²
    \end{align}
    である。vertex operatorの相関関数は
    \begin{align}
        ⟨ℯ^{¡αϕ(z,\_z)}ℯ^{¡βϕ(w,\_w)}⟩ = \begin{cases}
            0 & (α+β ≠ 0) \\
            |z-w|^{2αβ} & (α+β = 0)
        \end{cases}
    \end{align}
    となる。カイラルボソンは$ϕ(z,\_z) = φ(z) + \_φ(\_z)$と分けて正則部分を取り出すことで得られる。
    相関関数は以下の通り。
    \begin{align}&
        ⟨φ(z)φ(w)⟩ = -\ln(z-w) \\
        &
        ⟨ℯ^{¡αφ(z)}ℯ^{¡βφ(w)}⟩ =  \begin{cases}
            0 & (α+β ≠ 0) \\
            (z-w)^{αβ} & (α+β = 0)
        \end{cases}
    \end{align}
\end{frame}
\begin{frame}{\currentname}
    (\ref{Laughlin/CFT})が成り立つことを確認しよう。
    多点($∞$点)の相関関数を求めなければならないが、Wickの定理
    \begin{align}
       \⟨ ∏_i ℯ^{A_i}\⟩ = \exp(∑_{i<j}⟨A_iA_j⟩)
    \end{align}
    を使えばうまくいく。ここで$A_i$は生成消滅演算子の線形結合で表される演算子。
    証明は省略する。
    これを適用すると、
    \begin{align}&
        \⟨(∏_{i=1}^N ℯ^{¡√mφ(z_i)})\exp(-¡∫\𝑑^2{z'}√mρ₀φ(z'))\⟩ \∅
        &
        = \exp(
            -m∑_{i<j}⟨φ(z_i)φ(z_j)⟩
            +m∑_i ∫\𝑑^2{z'}ρ₀⟨φ(z_i)φ(z')⟩
            + \const
        ) \∅
        &
        = ∏_{i<j}(z_i-z_j)^m \exp( -m ∑_i ∫\𝑑^2{z'}ρ₀ \ln(z_i-z') + \const).
    \end{align}
\end{frame}
\begin{frame}{\currentname}
    まず$\ln(z_i-z_j)$の虚部を無視して$\ln|z_i-z_j|$でについての積分
    \begin{align}
        -m∫\𝑑^2{z'} ρ₀\ln |z_i-z'|
    \end{align}
    を考える。これは電荷密度が$ρ₀$で分布するときに位置$z_i$での静電ポテンシャルを求める問題に等価である。
    ここで$ρ₀$が半径$R$の円盤上で一様に分布しているとしよう。
    (この仮定はLaughlin状態の性質から妥当である。)
    すると位置$z_i$における電場の大きさは
    \begin{align}
        mπρ₀|z_i|²⋅÷1{|z_i|} = πρ₀|z_i|
    \end{align}
    で与えられる。よってこれを積分することで
    \begin{align}
        -m∫\𝑑^2{z'} ρ₀\ln |z_i-z'| = -÷1{2}πmρ₀|z_i|² + \const
    \end{align}
    となる。$ρ₀ = 1/2πml₀²$を代入すると(\ref{Laughlin/CFT})を再現する。

    $\ln (z_i-z_j)$の虚部についての積分は至る所に分岐点を含むためill-definedである。
    そこで波動関数について至る所で特異ゲージ変換を行ったと考えて位相を単純に無視すると、対称ゲージにおける波動関数が得られる。
\end{frame}
\begin{frame}{\currentname}
    CFTによる定式化の強力な点は準粒子励起を系統的に扱えることである。
    点状の準粒子励起に対し、準粒子励起から十分離れた点での波動関数は依然としてCFTで表されるので、準粒子が波動関数にもたらす変化はCFTにおける何らかの演算子で表されるだろう。
    したがってCFTにおける演算子の分類 $=$ スペクトル分解によって準粒子励起を分類することができる。
    さらに、プライマリー演算子$O(w)$に対応する励起$|Ψ_O(w)⟩$があれば$w^n∂_w|Ψ_O(w)⟩$によってディセンダント演算子に対応する励起が構成されるので、プライマリー演算子のみを考えればよい。
    電子はvertex operator $V_m(z) ≔ ℯ^{¡√mφ(z)}$に対応するので、
    準粒子として
    \begin{align}
        V₁(z) ≔ ℯ^{¡φ(z)/√m}
    \end{align}
    を考える。すると同様の議論によって
    \begin{align}&
        Ψ^+_m(w₁,…,w_M,z₁,…,z_N)\∅
        &
        = \⟨∏_{i=1}^M ℯ^{¡φ(w_i)/√m}∏_{i=1}^N ℯ^{¡√mφ(z_i)}\exp(-∫\𝑑^2{z'}√mρ₀φ(z'))\⟩\∅
        &
        = ∏_{i<j}(w_i-w_j)^{1/m} ∏_{i,j} (z_i-w_j) ∏_{i<j}(z_i-z_j)^m
        \exp(-÷1{4l₀²}∑_{i=1}^N|z_i|²-÷1{4ml₀²}∑_{i=1}^n|w_i|²)
        \label{Laughlin state with quasi hole}
    \end{align}
    が得られる。
\end{frame}
\begin{frame}{\currentname}
    準粒子励起の位置$w_i, w_j$を連続的に動かして反時計回りに入れ替えると、位相$ℯ^{¡π/m}$を獲得する。
    この操作は組み紐に対応付けられるから、粒子の交換は状態空間に作用する組み紐群の表現になっている。
    今の場合、準粒子の位置を定めると状態が一意に定まるため、可換な一次元表現しかあり得ない。
    
    組み紐群の表現はモノドロミー行列として実現される。
    モノドロミー行列は、正則関数によって張られる線形空間作用する行列で、正則関数を閉曲線に沿って解析接続することで得られる。
    モノドロミー行列は一般に基本群の表現となるので、変数$\{w_i\}$についてのモノドロミー行列は組み紐群の表現になっている。
\end{frame}
\begin{frame}{\currentname}
    次に準粒子励起を近づけることで、複合粒子を作ることを考えよう。
    CFTの文脈ではこれは演算子積展開(OPE)そのものである。
    vertex operator $V_p(z) = ℯ^{¡pφ(z)/√m}$のOPEは
    \begin{align}
        V_p(z)V_q(w) ∼ |z-w|^{2pq/m} V_{p+q}(w) + ⋯
    \end{align}
    で与えられる。
    $φ/√m$が周期$2π$でコンパクト化されていることから、$p,q ∈ ℤ$である。
    ここで準粒子を$m$個集めると電子またはホールになるため、電子・ホールの寄与を除けば$p$と$q$をfuseすると$[p+q]_m$になる。
    となる。ここで$[p]_m$は$p\bmod m$を表す。
    これは$V_m = 𝟙$とおくことに対応する。
    このときconpactified bosonは有理共形場理論(RCFT)になる。
    RCFTとは有限個のプライマリー場をもつCFTのこと。
\end{frame}
\begin{frame}{\currentname}
    カイラルボソンにはvertex operatorの他にもプライマリー場$∂φ(z)$がある。
    これは内部空間の並進対称性に付随するカレントとみなせ、$J(z) ≔ ¡∂φ(z)/√m$と書くことにする。
    $J(z)$と$V_p(w)$のOPEは
    \begin{align}
        J(z)V_p(w) ∼ ÷{p}{m}÷{V_p(w)}{z-w} + ⋯
    \end{align}
    となる。
    これは$V_p$の電荷が$p/m$であることを意味している。
    ただし、これは仮想的な位相の自由度$φ(z)$に結合する電荷であり、物理的な電荷はvertex operatorからではなく背景から提供されることに注意。
    電荷の大きさが物理的な準粒子励起の電荷と一致することは$φ(z)$とphysicalなゲージ場との関係を示唆する。
\end{frame}

\subsection*{14.6.2 Conformal blocks}
\begin{frame}{\currentname}
    non-abelianな量子Hall系を導入する前に、non-abelianな統計性について説明しておく。
    $N$粒子の波動関数を$ψ_{p;i₁,…,i_N}(z₁,…,z_N)$と書く。
    ここで$i₁,…,i_N$は各粒子に割り当てられた量子数を表す。
    また$p$は系全体で定義されるような量子数を表す。
    ここで連続的に粒子の位置を交換したとき、
    \begin{align}&
        ψ_{p;i₁,…,i_s,…,i_r,…,i_N}(z₁,…,z_s,…,z_r,…,z_N) \∅
        &
        = ∑_q B_{pq}[i₁,…,i_N]ψ_{q;i₁,…,i_r,…,i_s,…,i_N}(z₁,…,z_r,…,z_s,…,z_N)
    \end{align}
    となり、$B_{pq}$が組み紐群の多次元表現になっている解き、粒子はnon-abelian fractional statisticsをもつという。

    ここでも我々はCFTによる枠組みを用いることにする。
    このとき組み紐群の表現は共形ブロックのモノドロミー群によって得られる。

    相関関数について、
    \begin{align}
        ⟨O₁(z₁,\_z₁) ⋯ O_n(z_n,\_z_n)⟩
        = ∑_p C_p ℱ_p(\{z\})\_ℱ_p(\{\_z\})
    \end{align}
    のように正則部分と反正則部分への分解を行ったとき、$ℱ_p(\{z_i\})$を共形ブロックと呼ぶ。
    次にもう少し丁寧に共形ブロックについて説明する。
\end{frame}
\begin{frame}{\currentname}
    まずCFTの相関関数についての復習。
    プライマリー場$ϕ_i$と書く。
    1点関数は$ϕ₀=𝟙$として$⟨ϕ_i⟩ = δ_{i0}$である。
    次に2点関数は
    \begin{align}
        ⟨ϕ_i(z₁,\_z₁)ϕ_j(z₂,\_z₂)⟩ = ÷{g_{ij}}{z_{12}^{h_i+h_j}\_z_{12}^{\_h_i+\_h_j}}
    \end{align}
    となる。ここで$(h_i,\_h_i)$は演算子$ϕ_i$の共形ウェイト。
    場を線形結合によって再定義することで常に$g_{ij} = δ_{ij}$とできる。
    3点関数は
    \begin{align}
        ⟨ϕ₁(z₁,\_z)ϕ₂(z₂,\_z₂)ϕ₃(z₃,\_z₃)⟩ 
        = C_{123}÷{1}{z_{12}^{h₁+h₂-h₃}z_{23}^{h₂+h₃-h₁}z_{31}^{h₃+h₁-h₂}} × \cc
        \label{3point func}
    \end{align}
    既に演算子のスケールを決めてしまったので$C_{123}$に冗長な自由度はない。
    4点関数は
    \begin{align}
        ⟨ϕ₁(z₁,\_z)ϕ₂(z₂,\_z₂)ϕ₃(z₃,\_z₃)ϕ₄(z₄,\_z₄)⟩ 
        = f(η,\_η)∏_{i<j}^4 (z_{ij}^{h/3-h_i-h_j}\_z_{ij}^{\_h/3-\_h_i-\_h_j}).
    \end{align}
    ここで$η ≔ z₁₂z₃₄/z₁₂z₂₄$は交差比で、$f(η,\_η)$は任意の関数である。
    3点関数までは正則部分と反正則部分の積に分かれるが、4点関数はそうなっていない。
\end{frame}
\begin{frame}{\currentname}
    次にOPEについて。プライマリー場$ϕ₁, ϕ₂$のOPEは
    \begin{align}&
        ϕ₁(z₁,\_z₂)ϕ₂(z₂, z₂) \∅
        &
        = ∑_p C_{12}^p (z_{12}^{h_p-h₁-h₂}
        ∑_{\{k\}}β_{12p}^{\{k\}} z_{12}^{k₁+⋯+k_N}
        \ad L_{-k₁} ⋯ \ad L_{-k_N}  × \cc ) ϕ_p(z₂,\_z₂)
        \label{OPE}
    \end{align}
    となる。係数$C_{12}^p$は(\ref{3point func})の係数に一致する。
    さらに3点関数の微分から係数$β_{12p}^{\{k\}}$を求めることができる。
    よってディセンダントに対する係数は共形不変性から一意に定まっている。


    CFTの状態空間はプライマリー状態とそのディセンダントによって張られる。
    一つのプライマリー状態とそのディセンダントをまとめて共形族(Verma module)という。
    プライマリー場$ϕ_p$から構成される共形族への射影を$Π_p$と書くと、
    OPEによる計算は相関関数に$∑_pΠ_p$を挿入する操作と捉えられる。
    4点関数をOPEを用いて計算すると
    \begin{align}
        ⟨ϕ₁ϕ₂ϕ₃ϕ₄⟩ = ∑_p ⟨ϕ₁ϕ₂Π_pϕ₃ϕ₄⟩
        = ∑_p C_p ℱ_p(\{z\})\_ℱ_p(\{\_z\})
    \end{align}
    と書ける。正則部分と反正則部分に分かれることはOPEの具体形(\ref{OPE})から分かる。
    (おおもとはVirasoro代数が正則部分と反正則部分のテンソル積で表されること。)
    $ℱ_p(\{z_i\})$を共形ブロックと呼ぶ。
\end{frame}
\begin{frame}{\currentname}
    OPE係数$C_{ijk}$はCFTの相関関数についての詳細な情報を含むが、完全な記述は複雑であるから、$C_{ijk}$がゼロか否かだけを気にしたものをフュージョン則よぶ。
    プライマリー場を$ϕ_i$とすると、フュージョン則は
    \begin{align}
        ϕ_i ★ ϕ_j = ∑_k N_{ij}^k ϕ_k
    \end{align}
    と書かれる。$ϕ_i$と$ϕ_j$のOPEに$ϕ_k$で指定される共形族が現れれば$N_{ij}^k = 1$であり、そうでなければ$N_{ij}^k = 0$である。
    $N_{ij}^k > 2$は同じ共形ウェイトをもつ等価なプライマリー演算子が複数出てくる場合。
    フュージョン則は
    \begin{align}
        ϕ_i ★ ϕ_j = ϕ_j ★ ϕ_i,␣ ϕ_i ★ (ϕ_j ★ ϕ_k) = (ϕ_i ★ ϕ_j) ★ ϕ_k
    \end{align}
    を満たす。対応して、
    \begin{align}
        N_{ij}^k = N_{ji}^k,␣
        ∑_l N_{il}^m N_{jk}^l = ∑_n N_{ij}^n N_{nk}^m
    \end{align}
    となる。
\end{frame}
\begin{frame}{\currentname}
    $n$点関数についてOPEを繰り返すことで
    \begin{align}
        ⟨ϕ_{i₁}⋯ϕ_{i_n}⟩
        &
        = ∑_{\{p\}} ⟨ϕ_{i₁}ϕ_{i₂}Π_{p₁}ϕ_{i₃}Π_{p₂}ϕ_{i₄}⋯ϕ_{n-2}Π_{p_{n-3}}ϕ_{n-1}ϕ_n⟩ \∅
        &
        = ∑_{\{p\}}C_{\{p\}}ℱ_{\{p\}}(\{z\})\_ℱ_{\{p\}}(\{\_z\})
    \end{align}
    と表せる。$ℱ_{\{p\}}(\{z\})$が共形ブロック。

    理由はさておき、我々は$ℱ_{\{p\}}(\{z_i\})$をエニオンの波動関数に同定しようとしている。
    そのときエニオンの状態空間の次元は$ℱ_{\{p\}}(\{z_i\})$の個数である。
    一般のCFTではプライマリー場が無限に存在するため共形ブロックも無限に存在するが、RCFTでは状態空間に非自明な拘束条件が加わることでプライマリー場が有限個に限られる。
\end{frame}
\begin{frame}{\currentname}
    共形ブロックの個数を数えるだけならば、OPEを使わなくともフュージョン則で十分である。
    \begin{align}
        \dim ℋ = ∑_{\{p\}}N_{i₁i₂}^{p₁}N_{p₁i₃}^{p₂}N_{p₂i₄}^{p₃}⋯N_{p_{n-2}i_n}^{p_{n-1}}
    \end{align}
    同様の計算によって一般のトポロジーの空間における共形ブロックの個数を数えることができる。
    相関関数に$Π_p$を挿入する操作は空間のパンツ分解と考えられる。
    共形族の添字$p$について縮約をとる操作は空間を張り合わせることに対応している。
    よって例えばトーラス上の$1$点の共形ブロックの数は
    \begin{align}
        \dim ℋ = ∑_p N^p_{ip}
    \end{align}
    となる。
    曲面にハンドルをくっつける操作は$∑_i(⋯)_i ∑_p N^p_{ip}$によって行えるので、任意の曲面に対してフュージョン則だけから$n$点の共形ブロックの個数が数えられる。
\end{frame}
% \subsection*{14. 6. 3}
\begin{frame}{\currentname}
    共形場理論による定式化によってトーラス上の量子Hall流体を考えてみよう。
    そのためにトーラス上のCFTを考える。
    トーラスの周期を$ω₁,ω₂ ∈ ℂ$としよう。
    $ω₂/ω₁ = τ$をモジュラーパラメーターという。
    トーラスの自由度は$τ$のみである。(さらにモジュラー不変性から$\SL(2,ℤ)/\{±\}$の同値関係が入る。)
    一般性を失わずに$ω₁ = L ∈ ℝ$とおく。

    $H$をシリンダー時空におけるハミルトニアン、$P$を運動量演算子とする。
    $ω₂$軸方向のハミルトニアンは
    \begin{align}
        ÷{\Im τ}{|τ|} H₀ - ÷{\Re τ}{|τ|}¡P
    \end{align}
    となる。
    $H=(2π/L)(L₀+\_L₀-c/12),~P = (2π¡/L)(L₀-\_L₀)$を代入すると分配関数は
    \begin{align}
        Z &= \Tr \exp(-H\Im τ + ¡P \Re τ) \∅
        &
        = \Tr \exp(-π(τ-\_τ)(L₀+\_L₀-÷{c}{12})-π(τ+\_τ)(L₀-\_L₀) ) \∅
        &
        = \Tr \exp(-2πτ(L₀-÷{c}{24}) + 2π\_τ(\_L₀ -÷{c}{24}) ) \∅
        &
        = \Tr q^{¡(L₀-c/24)} \Tr \_q^{¡(\_L₀-c/24)},␣
        (q = ℯ^{2π¡τ}).
    \end{align}
    % $Z$を$q, \_q$のべき展開として表すと、$q^h \_q^{\_h}$の係数は共形ウェイト$h,\_h$をもつ状態の次元となる。
    共形族$i$に対する指標を
    \begin{align}
        ψ_i ≔\Tr(Π_i q^{¡L₀-c/24}),␣
        \_ψ_i ≔ \Tr(Π_i \_q^{¡\_L₀-c/24})
    \end{align}
    で定義する。
    ただしトレースはそれぞれ正則な空間と反正則な空間でとるとする。
    分配関数は$Z = ∑_i ψ_i\_ψ_i$と表される。
    以降は正則部分のみを考える。
\end{frame}
\begin{frame}{\currentname}
    次にトーラスはモジュラー変換
    \begin{align}
        T: τ ↦ τ+1,␣
        S: τ ↦ -÷1{τ}
    \end{align}
    についての不変性をもつ。
    モジュラー変換に対する指標の変換性は
    \begin{align}
        T: ψ_i ↦ \exp[2π¡(h_i-÷c{24})]ψ_i,␣
        S: ψ_i ↦ ∑_j 𝒮_i^j ψ_j
    \end{align}
    と表される。
    $T$についての変換は指標の定義から自明である。
    % $S² = \id$より行列$𝒮$は$𝒮² = 1$を満たす。
    % 分配関数のモジュラー不変性から
    % \begin{align}
    %     ∑_{i,j,k} \_ψ_i (S_j^i)^* S_j^k ψ_k = ∑_i \_ψ_iψ_i
    % \end{align}
    % となる。よって
    $𝒮_i^j$はモジュラーS行列と呼ばれる。
    % \begin{align}
    %     ϕ_j(b)ψ_i = ∑_k N_{ij}^k ψ_k
    % \end{align}
    モジュラーS行列はフュージョン行列を対角化することが知られている。
    具体的には
    \begin{align}
        N_{ij}^k = ∑_n ÷{𝒮_j^n 𝒮_i^n (𝒮^{-1})_n^k}{𝒮₀^n}
    \end{align}
    で、これはVerlinde公式と呼ばれる。
\end{frame}

\section{14. 7 The spin-singlet Halperin states}
\begin{frame}{\currentname}
    スピン一重項Halperin $(n+1, n+1, n)$状態もCFTの相関関数として表せる。
    波動関数
    \begin{align}
        Ψ_{(n+1,n+1,n)}(\{z_i^↑\},\{z_i^↓\})
        = & ∏_{i<j}(z_i^↑ - z_j^↑)^{n+1}
        (z_i^↓ - z_j^↓)^{n+1}
        ∏_{i,j}(z_i^↑ - z_j^↓)^n \∅
        &
        ×\exp(-÷1{4l₀²}∑_i(|z_i^↑|² + |z_i^↓|²))
    \end{align}
    をsemionのLaughlin状態
    \begin{align}
        Ψ^{(π/2)}_m(\{z_i^↑\}, \{z_i^↓\})
        = & ∏_{i<j}(z_i^↑ - z_j^↓)^{n+1/2}(z_i^↓ - z_j^↓)^{n+1/2}(z_i^↑ - z_j^↓)^{n+1/2} \\
        &
        ×\exp(-÷1{4l₀²}∑_i(|z_i^↑|²+|z_i^↓|²))
    \end{align}
    と因子
    \begin{align}
        Ψ_{singlet}(\{z_i^↑\}, \{z_i^↓\})
        = ∏_{i<j}÷{(z_i^↑ - z_j^↓)^{1/2}(z_i^↓ - z_j^↓)^{1/2}}{(z_i^↑ - z_j^↓)^{1/2}}
    \end{align}
    に分ける。まず
    \begin{align}
        Ψ^{(π/2)}_m(\{z_i^↑\}, \{z_i^↓\})
        = \⟨∏_{i=1}^N ℯ^{¡√{n+÷1{2}}φ(z_i)}\exp(-∫\𝑑^2{z'}√{n+÷1{2}}ρ₀φ(z'))\⟩_{\U(1)_k}
    \end{align}
\end{frame}
\begin{frame}{\currentname}
    ここで$⟨⋅⟩_{\U(1)_k}$はlevel-$k$のchiral bosonにおける期待値を表す。
    また
    \begin{align}
        Ψ_{singlet}(\{z_i^↑\},\{z_i^↓\})
        = \⟨V_{+1/2}(z₁^↑)⋯V_{+1/2}(z_{N/2}^↑)
        V_{-1/2}(z₁^↓)⋯V_{^1/2}(z_{N/2}^↓)
        \⟩_{\SU(2)₁}
    \end{align}
    である。
    ただし$V^±$はWess--Zumino--Witten(WZW)理論におけるprimary spin-1/2 multipletである。
    WZW理論について、少し補足しておく。
    WZW理論はカレント$J^a(z)$と、カレントが満たすaffine Kac--Moody代数
    \begin{align}&
        J^a(z)J^b(w) ∼ ÷{kδ^{ab}}{(z-w)²} + ÷{¡{f^{ab}}_cJ^c(w)}{z-w}, \\
        &
        [J_m^a,J_n^b] = kmδ^{ab}δ_{m+n,0} + ¡{f^{ab}}_cJ^c_{m+n}
    \end{align}
    によって特徴づけられる。
    ただし${f^{ab}}_c$はLie代数$𝔤$の構造定数である。
    affine Kac--Moody代数はゼロ次の交換関係として$𝔤$を含む。
\end{frame}
\begin{frame}{\currentname}
    $𝔤 = \su(2)$とする。
    最高ウェイト状態$|j⟩$は以下を満たす。
    \begin{align}&
        J₀³|j⟩ = j|j⟩,␣
        J₀⁺|j⟩ = 0,␣
        J_n^a|j⟩ = 0\𝚚{for} n > 0.
    \end{align}
    $|j⟩$に$J₀⁻,J_n^a$をかけていくことで一般の状態が構成される。
    理論のユニタリティを要請することで、$|j⟩$に制限がかかる。
    まず$\su(2)$の交換関係による制限から$j ∈ ÷1{2}ℤ_{≥0}$である。
    このとき$(J₀⁻)^{2j+1}|j⟩$がヌルベクトルとなる。
    さらに追加のヌルベクトルとして$(J⁺_{-1})^{k+1-2j}|j⟩$がある。
    ヌルベクトルはこれら2つとそのディセンダントによって尽きることが知られている。
    $(J⁺_{-1})^{k+1-2j}|j⟩$のノルムは公式
    \begin{align}
        ⟨j|(J₁⁻)^N(J_{-1}⁺)^N|j⟩ = ∏_{n=1}^N n(k+1-n-2j)
    \end{align}
    から分かる。これは$[J₁⁻, J_{-1}⁺] = k - 2J₀³,~J₀³(J_{-1}⁺)^N|j⟩ = (j+N)(J_{-1}⁺)^N|j⟩$から帰納法によって以下のように証明できる。
    \begin{align}
        ⟨j|(J₁⁻)^N(J_{-1}⁺)^N|j⟩
        &
        = ∑_{m=0}^{N-1}⟨j|(J₁⁻)^{N-1}(J_{-1}⁺)^{m}(k-2J₀³)(J_{-1}⁺)^{N-1-m}|j⟩ \∅
        &
        = ∑_{m=0}^{N-1}(k-2j-2N+2+2m)⟨j|(J₁⁻)^{N-1}(J_{-1}⁺)^{N-1}|j⟩ \∅
        &
        = N(k+1-N-2j)⟨j|(J₁⁻)^{N-1}(J_{-1}⁺)^{N-1}|j⟩.
    \end{align}
\end{frame}
\begin{frame}{\currentname}
    もし$j > k/2$ならば、
    \begin{align}
        ⟨j|J₁⁻J_{-1}⁺|j⟩ = k-2l < 0
    \end{align}
    となる。よって$j ≤ k/2$である必要がある。これを満たす$j$が有限個しか存在しないことに注意。

    実はaffine Kac--Moody代数の中にはVirasoro代数が隠れている。具体的には
    \begin{gather}
        T(z) = γ(J^aJ^a)(z) \\
        L_m = γ∑_n{:J^a_nJ^a_{m-n}:}
        = γ(∑_{n≤-1}J_n^aJ_{m-n}^a + ∑_{n≥0}J_{m-n}^aJ_n^a)
    \end{gather}
    はVirasoro代数を満たす。これを菅原構成法という。
    $γ$はLie代数から決まる定数で、ここでは定義を省略する。
    また中心電荷$c$もLie代数から定まる。
    
    この構成において最高ウェイト状態はプライマリー状態である。
    最高ウェイト状態のことをWZWプライマリー状態と呼ぶ。
    % $\SU(2)_k$ WZW理論はWZWプライマリー状態が有限個しかないという意味で、RCFTである。
\end{frame}
\begin{frame}{\currentname}
    次にフュージョン則について考える。
    フュージョン則はWZWプライマリー状態に対する概念として同様に定義される。
    トーラス上の$\SU(2)_k$ WZW理論に対してモジュラーS行列を求め、Verinde公式を用いると
    \begin{align}
        ϕ_{j₁} ★ ϕ_{j₂} = ϕ_{|j₁-j₂|} + ⋯ + ϕ_{j_{max}}.
    \end{align}
    が得られる。
    ここで$j_{max} = \min (j₁+j₂, k-j₁-j₂)$である。
    特に$\SU(2)₁$については、
    \begin{align}
        [1/2] ★ [1/2] = [0]
    \end{align}
    となる。よってfusion channelは1つだけであり、共形ブロックは常に1つである。
    よって$\SU(2)$はnon-abelianにもかかわらず、組み紐群の表現はabelianになる。
    実は$\SU(2)₁$はchiral bosonと以下のように対応する。
    \begin{align}
        J³(z) ∼ ¡∂ϕ(z),␣ J^±(z) ∼ ℯ^{±¡√2ϕ(z)}
    \end{align}
    また
    \begin{align}
        V_{±1/2}(z) ∼ \exp(±÷{¡}{√2}ϕ(z))
    \end{align}
    である。
\end{frame}
\section{14.8 Moore-Read states and their generalizations}
\begin{frame}{\currentname}
    MooreとReadはMajorana fermion(Ising CFT)に対応する以下のような状態を考えた
    \begin{align}
        Ψ_{MR}(z₁,…,z_N) = \Pf(÷1{z_i-z_j})∏_{i<j}(z_i-z_j)^n \exp(-÷1{4l₀²}∑_i |z_i|²).
    \end{align}
    これをMoore--Read状態と呼ぶ。
    この状態はlowest Landau levelに属し、占有率は$ν = 1/n$である。
    波動関数がfermionの統計性を満たすためには、$n$は偶数である必要がある。
    Moore--Read状態はMajorana fermion $ψ(z)$によって
    \begin{align}
        Ψ_{MR}(\{z_i\}) = ⟨ψ(z₁)⋯ψ(z_N)⟩_{Ising}×⟨(∏_{i=1}^Nℯ^{¡√n}ϕ(z_i))\exp(-∫\𝑑^2z'√nρ₀ϕ(z'))⟩_{\U(1)_n}
    \end{align}
    と書ける。ここで、Majorana fermionに対するWickの定理から
    \begin{align}
        ⟨ψ(z₁)⋯ψ(z_N)⟩ = \Pr ⟨ψ(z_i)ψ(z_j)⟩ = \Pf(÷1{z_i-z_j})
    \end{align}
    となることを用いた。
\end{frame}
% \begin{frame}{\currentname}
%     Majorana fermionをプライマリー場にもつRCFTとして、Ising CFTがある。
%     Ising CFT のchiralなプライマリー場は$𝕀 = ϕ_{(1,1)}(z), ψ = ϕ_{(1,2)}(z), σ = ϕ_{(2,2)}(z)$の3つで、それぞれの共形ウェイトは$h = 0, 1/2, 1/16$である。
%     $ψ$はカイラルフェルミオンであり、$σ$はスピン場である。
% \end{frame}
\begin{frame}{\currentname}
    Majorana fermionをプライマリー場にもつRCFTとして、Ising CFTがある。
    Ising CFT のchiralなプライマリー場は$ϕ_{(1,1)}(z), ϕ_{(1,2)}(z), ϕ_{(2,2)}(z)$の3つで、それぞれの共形ウェイトは$h = 0, 1/2, 1/16$である。
    以下の演算子を考える。
    \begin{align}
        𝕀 &= ϕ_{(1,1)}(z) ⊗ ϕ_{(1,1)}(\_z)\\
        ψ(z) &= ϕ_{(1,2)}(z) ⊗ ϕ_{(1,1)}(\_z) \\
        σ(z,\_z) &= ϕ_{(2,2)}(z) ⊗ ϕ_{(2,2)}(\_z) \\
        μ(z,\_z) &= ϕ_{(2,2)}(z) ⊗ ϕ_{(2,2)}(\_z)
        % ε(z,\_z) &= ϕ_{(1,2)}(z) ⊗ ϕ_{(1,2)}(\_z)
    \end{align}
    $ψ$はカイラルフェルミオンであり、$σ$はスピン場、$μ$は双対スピン場である。
    $σ$と$μ$は次元は等しいがOPE係数が異なる。
    フェルミオン場とスピン場のOPEは
    \begin{align}&
        ψ(z)σ(w,\_w) = ÷{ℯ^{¡π/4}}{√2(z-w)^{1/2}}μ(w,\_w) + ⋯ \\
        &
        ψ(z)μ(w,\_w) = ÷{ℯ^{-¡π/4}}{√2(z-w)^{1/2}}σ(w,\_w) + ⋯
    \end{align}
    である。
    % \footnote{
    %     ミニマルモデルでは正則成分の共形ウェイトとフュージョンが定められているが、正則成分と反正則成分の組み合わせ方には自由度が残されていることに注意。
    % }
\end{frame}
\begin{frame}{\currentname}
    Moore-Read状態において電子は$ψ(z)ℯ^{¡√nϕ(z)}$に対応する。
    準粒子励起が電子に対して局所的になることを要請すると、以下の励起があり得る。
    \begin{enumerate}
        \item 恒等演算子: $𝕀$
        \item $σ$粒子(non-abelion, half-vortex): $σ(z)ℯ^{¡ϕ(z)/2√n}$
        \item Majorana fermion: $ψ(z)$
        \item Laughlin準粒子(vortex): $ℯ^{¡ϕ(z)/√n}$
    \end{enumerate}
    ここで$σ(z)$単体では$ψ(z)$とのmutual statisticsが非自明になるので、vertex operator$ℯ^{¡ϕ(z)/2√n}$をつけることで位相を相殺した。
\end{frame}
\begin{frame}{\currentname}
    相関関数$⟨σ⋯σ χ⋯χ⟩$が非ゼロになるためには$σ$粒子、$χ$粒子は常に偶数個必要である。
    $σ$粒子が2つある場合、波動関数は
    \begin{align}
        Ψ_{MR}^{2qh}(\{z_i\})
        &
        = ⟨σ(η₁)σ(η₂)ψ(z₁)⋯ψ(z_N)⟩_{Ising} \∅
        &
        ×\⟨ℯ^{¡ϕ(η₁)/2√n}ℯ^{¡ϕ(η₂)/2√n}∏_{i=1}^Nℯ^{¡√nϕ(z_i)}\exp(-∫\𝑑^2{z'}√nρ₀ϕ(z'))\⟩_{\U(1)_n}
    \end{align}
    である。このとき、Pfaffianを以下のように変えれば良い。
    \begin{align}
        \Pf(÷1{z_i-z_j}) ↦ \Pf(÷{(z_i-η_i)(z_j-η₂)+(i ↔ j)}{z_i - z_j})
    \end{align}
    $σ$粒子が4つある場合は
    \begin{align}
        \Pf(÷1{z_i-z_j})
        &
        ↦ \Pf(÷{(z_i-η₁)(z_i-η₂)(z_j-η₃)(z_j-η₄)+(i ↔ j)}{z_i-z_j}) \∅
        &
        ≕ \Pf_{(12)(34)}
    \end{align}
    とする。これらの計算は大変。
\end{frame}
\begin{align}&
    Ψ^{4qh}_{[0]} = ÷{(η₁₃η₂₄)^{1/4}}{(1+√{1-x})^{1/2}}(Ψ_{(13)(24)} + √{1-x}Ψ_{(14)}{(23)}) \\
    &
    Ψ^{4qh}_{[1/2]} = ÷{(η₁₃η₂₄)^{1/4}}{(1-√{1-x})^{1/2}}(Ψ_{(13)(24)} - √{1-x}Ψ_{(14)}{(23)})
\end{align}
$x = η₁₂η₃₄/η₁₃η₂₄$
\begin{align}
\end{align}
\end{document}