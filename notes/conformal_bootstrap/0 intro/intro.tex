\providecommand{\main}{../main}
\documentclass[\main/main.tex]{subfiles}
\graphicspath{{../images/}}
\begin{document}
\setcounter{section}{-1}
\section{
    まえがき
}
\textbf{はじめに、この記事は学部生の勉強ノートに過ぎず、専門家の査読を受けたものではないため内容の正確性は保証できないことを注意しておく。}

臨界現象において、臨界指数という量が同じ次元と対称性をもつ理論モデル・実験系で同じ値をとる、という\textbf{普遍性(universality)}が知られている。
微視的には全く異なる系が、巨視的な性質に注目する限りは同じ振る舞いを見せるというのは自然(あるいは多体系)の神秘の1つである。

臨界現象の普遍性の解明は20世紀の物理学の一大テーマであったようだが、現在ではWilsonの\textbf{くりこみ群}の視点から理解されている。
くりこみ変換とは系のスケール変換に場の理論的な補正(自由度の粗視化)を加えたものである。
臨界現象の普遍性は臨界点にある異なる理論が粗視化によって同じ固定点に漸近していく振る舞いとして理解される。
固定点は無限に系を粗視化したものであるから、本質的に無限の自由度を含む場の理論となる。
また固定点はくりこみ変換によって不変であり、その帰結としてスケール不変性を有する。

一方で、多くの物理的な臨界点はスケール不変性よりも大きな対称性である\textbf{共形不変性}をもつ\textbf{共形場理論}になることが知られている。
共形不変性とは、ざっくり言うと場所によって倍率の異なるような局所的なスケール変換についての対称性である。
もちろん大域的なスケール不変性があるからといって局所的なスケール不変性があるとは言えない。
つまり、共形不変性は臨界現象に現れる非自明な対称性である。

共形ブートストラップは、微視的なモデルの構造を全く仮定せず、共形不変性(と空間次元や対称性など基本的な情報)のみを仮定して、共形場理論の性質をどこまで探れるかという1つの試みである。
結果はどうだったかと言うと、大きな成果として、\textbf{3次元Ising模型の臨界指数がモンテカルロ法を超える高精度で数値的に決定されている。}
つまり、共形不変性だけから臨界現象を解くことができるということである。
そもそも臨界現象は普遍性をもつのだから、モデルの詳細によらない記述はあって当たり前、という考えもあるかもしれない。
しかし、共形不変性を仮定することで具体的に計算できるというのは驚くべきことである。

この記事では場の理論とくりこみ群についての基本的な事項から始め、共形場理論を導入し、共形ブートストラップの概略を説明する。
記事を作成するにあたって、主に
\cite{simmonsduffin2016tasi}
、
\cite{Nakayama_2019}
を参考にした。また
\cite{Rychkov_2017}
や
\cite{Hikita_2020}
も参考にした。
特に図に関しては
\cite{simmonsduffin2016tasi}
を大いに参考にした。
詳細で正確な議論についてはこれらの文献を参照してもらえればいいと思う。
ネットでアクセスできる日本語の情報としては、
\href{https://research.kek.jp/people/hamada/conformal%20field%20theory.pdf}{浜田賢二さんの解説書}
がある。
テクニカルな面は筆者の能力の問題で、あまり載せられていないのだが、少しでも関連する内容に興味を持って、より専門的な文献にあたってもらえたなら、非常に喜ばしく思う。

読者としては物理学科の学部3年生以上を想定しており、統計力学と量子力学についての基本的な知識を仮定している。
相転移と臨界現象についてのある程度の知識があった方が、わかりやすいと思われる。
例えば
\cite{takahasi2017}
や
\cite{goldenfeld2018lectures}
、
\cite{cardy1996scaling}
などを参照してほしい。
場の量子論についての前提知識は仮定していないが、この記事の解説で足りない分は適宜文献を参照してほしい

\newpage
\subsection*{
    構成
}
この記事の構成と各章について参考にした文献を挙げておく。

共形場理論を扱うための準備として、1章では場の理論について説明している。
ただし、経路積分による場の理論の枠組みの概略を述べるに留めている。具体的な計算について触れる余裕はなかった。
場の理論の教科書は山程あるが、物性物理学の立場からの解説として例えば
\cite{altland2010condensed}
がある。
また和書では
\cite{Sakamoto_2020I}
、
\cite{Sakamoto_2020II}
を紹介しておく。
また2章は共形場理論のモチベーションを説明するために、くりこみ群について軽く説明している。
詳細は
\cite{altland2010condensed}
、
\cite{goldenfeld2018lectures}
、
\cite{cardy1996scaling}
などを参照してほしい。

3章では共形変換と共形不変性を具体的に定義している。
共形不変性が古典的に何を意味するかを4章のNoetherの定理で議論する。
また場の量子論において共形不変性から導かれるWard-Takahashi恒等式を続く5章で議論する。
6章では共形不変性によって相関関係に課される制限を導く。
4章から6章は
\cite{Nakayama_2019}
、
\cite{Hikita_2020}
、\cite{francesco2012conformal}
を参考にした。
\cite{francesco2012conformal}
は共形場理論に関する様々な話題が詳細に載っている。
ただし2次元の共形場理論が主で、本記事が対象とする一般の次元の共形場理論についてはあまり詳しくないようである。

7章から9章で共形ブートストラップのための道具を準備し、
10章で共形ブートストラップの概略を紹介して終わる。
これらの章については
\cite{simmonsduffin2016tasi}
、
\cite{Nakayama_2019}
、
\cite{Rychkov_2017}
を参考にしている。
7章では動径量子化を導入し、共形場理論の重要な性質である状態・演算子対応について説明する。
8章ではユニタリティから導かれる条件を求めている。
9章では相関関数を具体的に計算する際に用いられ、また共形場理論を指定するデータとしての役割をもつ演算子積展開を導入する。
10章では以上の道具を用いて共形場理論のデータをどのように決定するのかについて簡単に紹介する。

\subsection*{
    記法等
}
\begin{itemize}
    \item $ħ=c=\kB=1$とする自然単位形を用い、物理量の次元は質量次元で表すことにする。
    \item $∑_{μ=1}^d A_μB^μ$を単に$A_μB^μ$と書くアインシュタインの規約を用いる。
    \item 空間座標について、$x$は$d$次元の座標を表し、$𝒙$は$d-1$次元の座標を表すことにする。
    \item $x^2 ≔ x_μx^μ,~x⋅y ≔ x_μy^μ$
    という略記を断りなく用いる。
\end{itemize}
\end{document}