\providecommand{\main}{../main}
\documentclass[\main/main.tex]{subfiles}
\graphicspath{{../images/}}
\begin{document}
\section{
    Noetherの定理
}
%------------------------------------------------%
\subsection{
    Euler-Lagrange方程式
}
分配関数$Z = ∫\𝒟ϕ \e^{-S[ϕ]}$において主要な寄与は、$S[ϕ]$が最小となるような$ϕ$からの寄与である。
これを$ϕ_\t{classical}$と書く。
ここでパラメーター$ħ$を入れて、
\begin{align}
    Z = ∫\𝒟ϕ \e^{-S[ϕ]/ħ}
\end{align}
と書き直す。
統計力学の場合、温度$T$が$ħ$に対応する。
$ħ → 0$の極限をとると、$ϕ_\t{classical}$の寄与のみが生き残る。これを古典極限という。

$ϕ_\t{classical}$が満たすべき方程式を求めよう。
作用が局所的なLagrangian $ℒ$によって
\begin{align}
    S[ϕ] = ∫\d^d{x}ℒ(ϕ(x),∂_μϕ(x))
\end{align}
と書かれるとする。
場を$ϕ(x) → ϕ(x) + δϕ(x)$と微小変化させたとき、
作用の変化は
\begin{align}
    δS
    &
    = ∫\d^d{x}\Q(
        \∂{ℒ}{ϕ}δϕ
        +\∂{ℒ}{∂_μϕ}∂_μδϕ
    )
    \∅ &
    = ∫\d^d{x}\Q(
        \∂{ℒ}{ϕ}
        -∂_μ\∂{ℒ}{∂_μϕ}
    )δϕ
    + ∫\d^d{x}∂_μ\Q(\∂{ℒ}{∂_μϕ}δϕ)
\end{align}
となる。第2項は表面項であり、境界で$δϕ=0$とおけば無視できる。
このような任意の場$δϕ$に対して$δS=0$となるとき、
\begin{align}\tcboxmath{
        \δ{S}{ϕ} ≔ \∂{ℒ}{ϕ}-∂_μ\∂{ℒ}{∂_μϕ} = 0
}\end{align}
が成り立つ。
この式を\textbf{Euler-Lagrange方程式}、または単に運動方程式と呼ぶ。
古典場$ϕ_\t{classical}(x)$は作用の停留点になっているので、運動方程式を満たす。

\subsection{
    Noetherの定理
}
座標変換$x^μ → {x'}^μ = x^μ - ε^μ(x)$を考える。
スカラー場$ϕ(x)$の変換性を、
\begin{align}
    ϕ'(x') - ϕ(x) = 0
\end{align}
によって定義すると、$∂_μϕ(x)$の変換性は、
\begin{align}
    ∂'_μ ϕ'(x') - ∂_μ ϕ(x)
    = \Q(\∂{x^ν}{x'^μ} - δ^ν_μ)∂_νϕ(x)
    =  ∂_με^ν∂_ν ϕ
\end{align}
と計算される。また変換による積分測度の変化は
\begin{align}
    \d^d{x'}
    = \det(\∂{x'}{x})\d^d{x}
    = (1 -  ∂_μ ε^μ)\d^d{x}
\end{align}
となる。したがって作用の変化は以下のように与えられる。
\begin{align}
    δS
    &
    = ∫_{Ω'}\d^d{x'}ℒ(ϕ'(x'),∂'_μϕ'(x'))
        - ∫_{Ω}\d^d{x}ℒ(ϕ(x),∂_μϕ(x))
    \\ &
    = ∫_{Ω}\d^d{x}\Q(
        \∂{ℒ}{∂_μϕ}∂_νϕ - δ^μ_ν ℒ
    )∂_μ ε^ν
\end{align}
第1項は$∂_μϕ(x)$による寄与で第2項はJacobianによる寄与である。
したがって、
\begin{align}\tcboxmath{
    δS = ∫\d^d{x} {T^μ}_ν ∂_μ ε^ν,
    \␣
    {T^μ}_ν ≔ \∂{ℒ}{∂_μϕ}∂_νϕ - δ^μ_ν ℒ
    \label{field: delta_S by T}
}\end{align}
と書ける。${T^μ}_ν$を\textbf{(正準)エネルギー運動量テンソル}と呼ぶ。$∂_μ{T^μ}_ν$を計算すると、
\begin{align}
    ∂_μ{T^μ}_ν
    &
    = ∂_μ\∂{ℒ}{∂_μϕ}∂_νϕ
    +\∂{ℒ}{∂_μϕ}∂_ν∂_μϕ - ∂_νℒ
    \∅ &
    = ∂_μ\∂{ℒ}{∂_μϕ}∂_νϕ -\∂{ℒ}{ϕ}∂_νϕ
    \∅ &
    = -\δ{S}{ϕ}∂_νϕ
    \label{field: ∂_μ T^μ_ν}
\end{align}
となる。したがって部分積分から、
\begin{align}\tcboxmath{
        δS = ∫\d^d{x}\Q[
            \δ{S}{ϕ}ε^ν∂_νϕ
            + ∂_μ({T^μ}_νε^ν)
        ]
}\end{align}
と変形できる。
特定の座標変換に対して$δS = 0$となる場合、$\δ*{S}{ϕ} = 0$のもとで
\begin{align}
    ∂_μ({T^μ}_ν ε^ν) = 0
    \␣ \t{(classical)}
\end{align}
が成り立つ。
(classical)は運動方程式を課した上での等号であることを意味する。
これをNoetherの定理と呼び、${T^μ}_ν ε^ν$を\textbf{Noetherカレント}と呼ぶ。
また積分形で書いて
\begin{align}
    Q(∂B) = 0
    \␣\t{(classical)},
    \␣
    Q(Σ) ≔ -∫_{Σ}\d^{d-1}{S_μ}{T^μ}_νε^ν
\end{align}
となる。
ここで$∂B$は余次元$1$ ($d-1$次元)の閉曲面である。
$Q(Σ)$を$Σ$上の\textbf{Noetherチャージ}と呼ぶ。
同じ境界を持つ曲面$Σ_1,Σ_2$によって$∂B = Σ_1-Σ_2$と表せるとき、
\begin{align}
    Q(Σ_1) = Q(Σ_2)
    \␣ \t{(classical)}
\end{align}
が成り立つ。
この式は曲面の境界を保つ連続的な変形に対し、Noetherチャージが不変であることを意味している。

\subsection{
    共形変換に対するNoetherの定理
}
以下で具体的に共形変換に対するNoetherの定理を見ていく。
以下の議論は全て\textbf{無次元のスカラー場}に対する議論であり、後で一般化する際にいくつかの修正がされることを注意しておく。
\subsubsection*{
    並進
}
一様な並進$ε^μ(x) = a^μ$を考える。作用の変化は
\begin{align}
    δS = ∫\d^d{x}{T^μ}_ν ∂_μ a^ν = 0
\end{align}
となる。
今の場合$ℒ(ϕ,∂_μϕ)$が$x$に陽に依存しないことを仮定しており、並進対称性は自動的に満たされている。

作用が座標変換$x^μ → x^μ - a^μ$に対して不変であるとき、Noetherの定理から$\δ*{S}{ϕ}=0$のもとで$a^ν ∂_μ {T^μ}_ν = 0$が成り立つ。
$a^ν$は任意であるから
\begin{align}
    ∂_μ{T^μ}_ν = 0 \␣(\t{classical})
\end{align}
となる。
並進に対応するNoetherチャージは運動量$P_ν$であり、
\begin{align}
    P_ν(∂B) ≔ -∫_{∂B}\d^{d-1}{x_μ}{T^μ}_ν(x) = 0
\end{align}
が成り立つ。
\subsubsection*{
    回転
}
作用が回転対称性 (Lorentz対称性) をもつとき、反対称テンソル$b^{μν} = -b^{μν}$に対して座標変換
\begin{align}
    x^μ → x^μ - b^{νμ}x_ν
\end{align}
は作用を不変に保つ。
このとき、(\ref{field: delta_S by T})から
\begin{align}
    δS 
    = b^{νρ}∫\d^d{x}{T^μ}_ν ∂_μ x_ρ
    = \f{1}{2}b^{νρ}∫\d^d{x}(T_{ρν}-T_{νρ})
    = 0
\end{align}
が成り立つ。$b^{νρ}$は反対称ならば任意であり、積分領域も任意にとることができるので、
\begin{align}
    T_{ρν}-T_{νρ} = 0
\end{align}
が成り立つ。
つまり作用の回転対称性は、エネルギー運動量テンソルが対称テンソルとなることを意味する。
また、Noetherカレントは$b^{νρ}x_ρ{T^μ}_ν$となり、
\begin{align}
    ∂_μ(x_ρ{T^μ}_ν - x_ν{T^μ}_ρ) = 0
    \␣\t{(classical)}
\end{align}
が成り立つ。
\subsubsection*{
    スケール変換
}
次に、スケール変換
\begin{align}
    x^μ → (1-λ)x^μ
\end{align}
に対して作用が不変であるとする。
このとき、(\ref{field: delta_S by T})から
\begin{align}
    δS
    = λ∫\d^d{x}{T^μ}_ν ∂_μ x^ν
    = λ∫\d^d{x}{T^μ}_μ = 0
\end{align}
となる。積分領域は任意にとることができるから、
\begin{align}
    {T^μ}_μ = 0
\end{align}
が成り立つ。つまり系のスケール対称性はエネルギー運動量テンソルがトレースレスになることを意味する。
またNoetherカレントは$x^ν{T^μ}_ν$となる。Noetherの定理から
\begin{align}
    ∂_μ(x^ν{T^μ}_ν) = 0\␣\t{(classical)}
\end{align}
が成り立つ。
\subsubsection*{
   特殊共形変換
}
特殊共形変換$x^ν → x^ν-c_ρ(2x^νx^ρ-x^2δ^{νρ})$に対し、作用の変化は以下のようになる。
\begin{align}
    δS
    &
    = c_ρ∫\d^dx T^{μν}∂_μ(2x_νx^ρ-x^2δ^ρ_ν)
    \∅ &
    = 2c_ρ∫\d^dx\Q(
        x^ρ{T^μ}_μ
        +x_ν T^{ρν}
        -x_μT^{μρ}
    )
    \∅ &
    = 2c_ρ∫\d^dx\Q(
        x^ρ{T^μ}_μ
        +x_μ(T^{ρμ}-T^{μρ})
    )
\end{align}
作用が回転対称かつスケール不変ならば、エネルギー運動量テンソルは対称かつトレースレスとなるので、特殊共形変換に対して作用は不変である。
したがって共形不変性が成り立つ。
つまり無次元のスカラー場では、(回転対称性を仮定した上で)スケール不変性は共形不変性に拡大する。

% \subsection{
%     内部対称性のNoetherカレント
% }
% 内部空間の対称性についても見ておく。
% $n$成分の場$𝝓(x)$に対する微小変換
% \begin{align}
%     𝝓(x) → 𝝓'(x) = 𝝓(x) + ε^a(x)G_a𝝓(x)
% \end{align}
% を考える。
% ここで$G_a$は$n×n$行列であり、内部対称性を表すLie代数の生成子である。
% 例えば、2成分複素場$(ϕ₁,ϕ₂)$の$SU(2)$対称性を考える場合、Pauli行列$σ_a$を用いて
% \begin{align}
%     𝝓'(x) = 𝝓(x) + ε^a(x)\i σ_a 𝝓(x)
%     \␣ (a = 1,2,3)
% \end{align}
% とすればよい。
% 作用の変化は、
% \begin{align}
%     δS &
%     = ∫\d^d{x}\Q(
%         \∂{ℒ}{𝝓}ε^aG_a𝝓
%         +\∂{ℒ}{∂_μ 𝝓}∂_μ(ε^aG_a𝝓)
%     )
%     \∅ &
%     = ∫\d^d{x}\Q(
%         \δ{S}{𝝓}ε^aG_a𝝓
%         + ∂_μ\Q(\∂{ℒ}{∂_μ 𝝓}ε^aG_a𝝓)
%     )
%     \label{field: δS by inner space transformation}
% \end{align}
% によって与えられる。
% ここで、$ε^a(x)$が空間に依存しない定数であるとし、このとき任意の積分領域に対し$δS = 0$であると仮定する。これは系のグローバル対称性を意味する。
% すると、$ε^a(x) = ε^a$を積分の外に出すことができ、
% \begin{align}
%     δS
%     = ε^a∫\d^d{x}\Q(
%         \δ{S}{𝝓}G_a𝝓
%         + ∂_μ\Q(\∂{ℒ}{∂_μ 𝝓}G_a𝝓)
%     ) = 0
% \end{align}
% となる。積分領域は任意にとれるので、恒等式として、
% \begin{align}
%     \δ{S}{𝝓}G_a𝝓 + ∂_μ\Q(\∂{ℒ}{∂_μ 𝝓}G_a𝝓) = 0
%     \label{field: internal Noether identity}
% \end{align}
% が成り立つ。したがって、運動方程式$\δ*{S}{𝝓}=0$のもとで連続の式
% \begin{align}
%     ∂_μ j_a^μ(x) = 0, \␣
%     j_a^μ(x) = \∂{ℒ}{∂_μ 𝝓}G_a𝝓
% \end{align}
% が成り立つ。$j_a^μ(x)$をNoetherカレントと呼ぶ。
% 恒等式(\ref{field: internal Noether identity})を用いると、$ε^a(x)$が定数とは限らない場合は
% \begin{align} \tcboxmath{
%     δS = ∫\d^d{x} j_a^μ(x)∂_με^a(x)
% }\end{align}
% と書ける。

% 内部対称性に対するNoetherカレントの例として、$SO(2)$対称性を持つLagrangian
% \begin{align}
%     ℒ(ϕ₁,ϕ₂,∂_μϕ₁,∂_μϕ₂)
%     &
%     = \f{1}{2}\Q[(∂ϕ₁)²+(∂ϕ₂)²] -\f{1}{2}m²(ϕ₁²+ ϕ₂²)
% \end{align}
% を考える。グローバルな微小変換
% \begin{align}
%     \M(ϕ₁\\ϕ₂) → \Q(1+ε\M(0&-1\\1&0))\M(ϕ₁\\ϕ₂)
% \end{align}
% に対し、$ℒ$は不変である。よってNoetherカレントは
% \begin{align}
%     j^μ(x) =\M(∂^μϕ₁&∂^μϕ₂)\M(0&-1\\1&0)\M(ϕ₁\\ϕ₂) =  ϕ₁∂^μϕ₂ - ϕ₂∂^μϕ₁
% \end{align}
% となり、$\δ*{S}{ϕ₁}=\δ*{S}{ϕ₂}=0$のもとで$∂_μj^μ(x) = 0$が成り立つ。また$ε$が空間に依存する場合、
% \begin{align}
%     δS = ∫\d^d{x}j^μ(x)∂_με(x)
% \end{align}
% と書ける。

\subsection{
    一般の場合のNoetherカレント
}
座標変換と内部空間の変換が同時に起こる場合を考える。
$n$成分の場$𝝓$を考え、微小な座標変換
\begin{align}
    {x'}^μ - x^μ = ε^μ(x) ≔ ε^a(x)X_ax^μ
\end{align}
とそれに伴う内部空間の微小変換
\begin{align}
    𝝓'(x') - 𝝓(x) = G𝝓(x) ≔ ε^a(x)G_a(x)𝝓(x),
\end{align}
を考える。
ただし、$X_a = X^μ_a(x) ∂_μ$は微分演算子であり、$G_a$は$n×n$行列である。
$∂_μ𝝓(x)$の変換性は
\begin{align}
    ∂'_μ𝝓'(x') - ∂_μ𝝓(x)
    = ∂_με^ν∂_ν𝝓(x) + ∂_μ(G𝝓(x))
\end{align}
となる。
$G_a$による作用の変化は、
\begin{align}
    ∫\d^d{x}\Q(
        \∂{ℒ}{𝝓}ε^aG_a𝝓
        +\∂{ℒ}{∂_μ 𝝓}∂_μ(ε^aG_a𝝓)
    )
    = ∫\d^d{x}\Q(
        \δ{S}{𝝓}ε^aG_a𝝓
        + ∂_μ\Q(\∂{ℒ}{∂_μ 𝝓}ε^aG_a𝝓)
    )
    \label{field: δS by inner space transformation}
\end{align}
と計算される。これに(\ref{field: delta_S by T})と同様の寄与を加えると、作用の変化は
\begin{align}\tcboxmath{
        δS = ∫\d^d{x}\Q(
            {T^μ}_ν∂_με^ν
            +\δ{S}{𝝓}G𝝓 + ∂_μj^μ
        )
    \label{field: general δS formulation}
}\end{align}
となる。
ここで、${T^μ}_ν$と$j^μ$を以下のように定義した。
\begin{align}
    {T^μ}_ν ≔ \∂{ℒ}{∂_μ𝝓}∂_ν𝝓 - δ^μ_ν ℒ,
    \␣
    j^μ ≔ \∂{ℒ}{∂_μ𝝓}G𝝓
\end{align}
(\ref{field: ∂_μ T^μ_ν})と同様に
\begin{align}
    {T^μ}_ν∂_με^ν
    &
    = -∂_μ{T^μ}_νε^ν + ∂_μ({T^μ}_νε^ν)
    \∅ &
    =\δ{S}{𝝓}ε^ν∂_ν𝝓+∂_μ({T^μ}_νε^ν)
\end{align}
と変形できるので、
\begin{align}
    δS
    &
    = ∫\d^d{x}\Q[
        {\δ{S}{𝝓}}(ε^ν∂_ν+G)𝝓
        + ∂_μ({T^μ}_νε^ν+j^μ)
    ]
    \∅ &
    ≔ ∫\d^d{x}\Q(
        \δ{S}{𝝓}δ𝝓
        + ∂_μ J^μ
    )
    \∅ &
    ≔ ∫\d^d{x}\Q(
        \δ{S}{𝝓}ε^aδ𝝓_a
        + ∂_μ(ε^a J_a^μ) 
    )
\end{align}
となる。
ここでグローバルな変換$ε^a(x) = ε^a$に対し、$δS = 0$となると仮定すると、恒等式
\begin{align}
    {\δ{S}{𝝓}}δ𝝓_a+ ∂_μ J_a^μ = 0
    \label{field: gauge Noether identity}
\end{align}
が得られる。
ここから$\δ*{S}{𝝓} = 0$のもとで連続の式$∂_μ J_a^μ = 0$が成り立つ。
また定数とは限らない一般の$ε^a(x)$に対し、作用の変化は
\begin{align}\tcboxmath{
        δS
        = ∫\d^d{x}J_a^μ(x)∂_με^a(x),\␣
        J_a^μ(x)
        = {T^μ}_νX_a^ν + j_a^μ
}\end{align}
と書ける。

\subsection{
    エネルギー運動量テンソルの改良(対称化)
}
回転(Lorentz)変換
$ ε^μ = b^{μν}x_ν,~b^{νμ}=-b^{μν}$
に対し、
\begin{align}
    𝝓'(x') = 𝝓(x) + \f{1}{2}b^{μν}𝒮_{μν}𝝓(x)
\end{align}
と変換する$n$成分場$𝝓$を考える。
ただし、$𝒮_{μν}$は$μν$の入れ替えについて反対称な$n×n$行列である。
\begin{align}
    {x^ρ}_{μν} = x_νδ^ρ_μ-x_μδ^ρ_ν,
\end{align}
とおくと、$ε^ρ = \f{1}{2}b^{μν}{x^ρ}_{μν}$と書けるから、以下のカレントが得られる。
\begin{align}
    {J^ρ}_{μν}
    &
    = {T^ρ}_σ{x^σ}_{μν} + \∂{ℒ}{∂_ρ𝝓}𝒮_{μν}𝝓
    \∅ &
    = x_ν{T^ρ}_μ-x_μ{T^ρ}_ν + {\∂{ℒ}{∂_ρ𝝓}}𝒮_{μν}𝝓
\end{align}
第3項を${s^ρ}_{μν}$とおくことにする。
\begin{align}
    {s^ρ}_{μν} ≔ {\∂{ℒ}{∂_ρ𝝓}}𝒮_{μν}𝝓.
\end{align}
これはいままで$j_a^μ$と書いていたものと同じものである。

スピンを考慮すると、スカラー場のときと比べて(\ref{field: general δS formulation})の第2項、第3項が加わっているためにエネルギー運動量テンソルは対称テンソルにはならない。
そこで、エネルギー運動量テンソルを以下のように定義し直す。
\begin{align}\tcboxmath{
    T^\rm{B}_{μν} = T_{μν} - ∂^ρB_{ρμν},\␣
    B_{ρμν} = \f{1}{2}(s_{ρμν}+s_{μνρ}-s_{νρμ})
}\end{align}
これはBelinfanteのエネルギー運動量テンソルと呼ばれる。
また、改良された回転に対するNoetherカレントを
\begin{align}\tcboxmath{
    J^\rm{B}_{ρμν}
    = x_νT^\rm{B}_{ρμ} - x_μT^\rm{B}_{ρν}
}\end{align}
と定義する。
$s_{μνρ}$の定義から$s_{μρν}=-s_{μνρ}$が成り立つので、
\begin{align}
    B_{μρν}
    &
    = \f{1}{2}(s_{μρν}+s_{ρνμ}-s_{νμρ})
    \∅ &
    = \f{1}{2}(-s_{μνρ}-s_{ρμν}+s_{νρμ}) = -B_{ρμν}
\end{align}
が分かる。したがって、
\begin{align}
    ∂^μT^\rm{B}_{μν} = ∂^μT_{μν} - ∂^μ∂^ρB_{ρμν} = ∂^μT_{μν}
\end{align}
となる。
ここから$∂^ρJ^\rm{B}_{ρμν}$を計算すると、
\begin{align}
    ∂^ρJ^\rm{B}_{ρμν}
    &
    = T^\rm{B}_{νμ}-T^\rm{B}_{μν}
    + x_ν∂^ρT^\rm{B}_{ρν} - x_μ∂^ρT^\rm{B}_{ρμ}
    \∅ &
    = T_{νμ}-T_{μν}+∂^ρ s_{ρμν} + x_ν∂^ρT_{ρν} - x_μ∂^ρT_{ρμ}
    \∅ &
    = ∂^ρJ_{ρμν}
\end{align}
となる。したがって$J_{ρμν}$の代わりに$J^\rm{B}_{ρμν}$を用いても問題ない。
$T^\rm{B}_{μν}$が対称テンソルとなることは、
\begin{align}
    T^\rm{B}_{νμ}-T^\rm{B}_{μν}
    &
    = T_{νμ}-T_{μν}+∂^ρ s_{ρμν}
    \∅ &
    = ∂^ρ{J^ρ}_{μν} - x_ν∂^ρT_{ρν} + x_μ∂^ρT_{ρμ}
    \∅ &
    = 0 \␣ (\t{classical})
\end{align}
から分かる。ただし最後の等式では運動方程式を課した。

例として、電磁場のLagrangian
\begin{align}
    ℒ = -\f{1}{4}F_{μν}F^{μν}
\end{align}
を考える。ここで$F_{μν}=∂_μA_ν-∂_νA_μ$である。Lagrangianは
\begin{align}
    ℒ = -\f{1}{2}(η_{μν}η_{ρσ}-η_{μσ}η_{νρ})∂^μA^ρ∂^νA^σ
\end{align}
とも書ける。
第2項のために$A^μ$を$d$個の独立したスカラーとはみなせず、$A^μ$にベクトルとしての変換性を課さなければ、作用の回転対称性を満たすことができない。
実際、エネルギー運動量テンソルを求めると、
\begin{align}
    T_{μν} = -F_{μρ}∂_νA^ρ - δ_{μν}ℒ
\end{align}
であり、対称になっていない。
そこで、エネルギー運動量テンソルを改良する。結果だけ書くと、
\begin{align}
    T^\rm{B}_{μν}
    &
    = T_{μν} - ∂^ρ(F_{ρμ}A_ν)
    = -F_{μρ}F^{νρ} - δ_{μν}ℒ - A_ν ∂^ρF_{ρμ}
\end{align}
であり、運動方程式$∂^ρF_{ρμ}=0$のもとで$T^\rm{B}_{μν}$が対称テンソルになっていることが分かる。
さらに、$d=4$では運動方程式のもとでトレースレスになる。
後で議論するが、これは共形不変性を意味する。

\subsection{
    エネルギー運動量テンソルの改良(トレースレス化)
}
スケール変換$ε^μ = x^μ$に対し、
\begin{align}
    ϕ'(x') = ϕ(x) + Δϕ(x) \␣ (Δ ∈ ℝ)
\end{align}
と変換する場を考える。これは$ϕ$が質量次元$Δ$をもつことを意味する (長さの質量次元は$-1$である)。
スケール変換に対するNoetherカレントは
\begin{align}
    J^μ = x^ν{T^μ}_ν + V^μ,\␣
    V^μ = Δϕ\∂{ℒ}{∂_μϕ}
\end{align}
と定義される。$V^μ$はvirialカレントと呼ばれる。
$Δ ≠ 0$のとき、運動方程式のもとで
\begin{align}
    {T^μ}_μ = -∂_μV^μ \␣ (\t{classical})
\end{align}
となり、エネルギー運動量テンソルがトレースレスにならないことがわかる。

そこで、エネルギー運動量テンソルを改良してトレースレスにすることを考える。
$T_{μν}$を以下のように改良しよう。
ただし、これは既に対称テンソルに改良されているとする。
\begin{align}\tcboxmath{\begin{aligned}
    \~T_{μν}
    &
    = T_{μν} + \f{1}{d-2}(∂_μ∂_ρL^ρ_ν+∂_ν∂_ρL^ρ_μ-∂_ρ∂^ρL_{μν}-δ_{μν}∂_ρ∂_σL^{ρσ})
    \\ & \␣
    +\f{1}{(d-2)(d-1)}(δ_{μν}∂_ρ∂^ρL^σ_σ-∂_μ∂_νL^σ_σ)
    \label{traceless improvement}
\end{aligned}}\end{align}
$L_{μν}$は
\begin{align}
    ∂^μ∂^νL_{μν} = -∂_μV^μ = T^μ_μ
    \␣ (\t{classical})
    \label{field: definition of f}
\end{align}
を満たすような対称テンソルである。
このような$L_{μν}$が必ず存在するとは限らないが、ここでは$L_{μν}$が見つかったとする。
また改良されたNoetherカレントは
\begin{align}\tcboxmath{
    \~J^μ = x^ν\~T^μ_ν
}\end{align}
と定義される。

(\ref{traceless improvement})は複雑な形をしていて面食らうかもしれないが、これでうまくいくことは簡単に確認できる。
$\~T_{μν}$は対称テンソルであり、$∂_μ\~T^μ_ν = ∂_μT^μ_ν$を満たす。
さらに、
\begin{align}
    \~T^μ_μ = T^μ_μ+∂_μV^μ
\end{align}
となる。
したがって、
\begin{align}
    ∂_μ\~J^μ = \~T^μ_μ + x_ν∂_μT^μ_ν
    = ∂_μ J^μ
\end{align}
となる。また、
\begin{align}
    \~T^μ_μ = ∂_μJ^μ - x^ν∂_μT^μ_ν
    = 0 \␣ (\t{classical})
\end{align}
が成り立つ。
ただし最後の等式では運動方程式を課した。特に$L_{μν}=δ_{μν}L$と書ける時、
\begin{align}
    \~T_{μν}
    = T_{μν} + \f{1}{(d-1)}(∂_μ∂_ν-δ_{μν}∂_ρ∂^ρ)L
    \label{traceless improvement (unitary theory)}
\end{align}
となる。

例として、スカラー場のLagrangian
\begin{align}
    ℒ = \f{1}{2}∂_μϕ∂^μϕ - U(ϕ)
\end{align}
を考える。$S = ∫\d^d{x}ℒ$が無次元量となるためには、$ϕ(x)$の質量次元は$Δ = (d-2)/2$でなければならない。また、$U(ϕ)$の質量次元が$d$になるためには
\begin{align}
    U(ϕ) = λϕ^{d/Δ} = λϕ^{2d/(d-2)}
\end{align}
でなければならない。ただし、$d=2$の場合は$ϕ$が無次元となるために、$U(ϕ)$は任意関数となる。$d≠2$の場合、運動方程式は
\begin{align}
    ∂_μ∂^μϕ = -\f{d}{Δ}λϕ^{\f{d}{Δ}-1} = -\f{2d}{d-2}λϕ^{\f{d+2}{d-2}}
\end{align}
となる。ここで、エネルギー運動量テンソル
\begin{align}
    T_{μν}
    &
    = \∂{ℒ}{∂^μϕ}∂_νϕ - δ_{μν}ℒ
    \∅ &
    = ∂_μϕ∂_νϕ - \f{1}{2}δ_{μν}∂^μϕ∂_νϕ + δ_{μν}U(ϕ)
\end{align}
は明らかにトレースレスではない。
(いまはスピン0の場を考えているので、対称テンソルにはなっている。)
これを改良しよう。
\begin{align}
    V^μ = Δϕ\∂{ℒ}{∂_μϕ} = Δϕ∂^μϕ
\end{align}
から、$L_{μν}$は
\begin{align}
    ∂^μ∂^ν L_{μν} =- ∂_μ V^μ = Δ∂_μ(ϕ∂^μϕ) = \f{Δ}{2}∂^μ∂^ν(δ_{μν}ϕ²)
\end{align}
を満たす。
したがって、$L_{μν} = δ_{μν} Δϕ²/2 = δ_{μν}(d-2)ϕ²/4$とおけるので、
(\ref{traceless improvement (unitary theory)})から
\begin{align}
    \~T_{μν}
    &
    = T_{μν} + \f{d-2}{4(d-1)}(∂_μ∂_ν-δ_{μν}∂_ρ∂^ρ)ϕ²
    \∅ &
    = ∂_μϕ∂_νϕ - \f{1}{2}δ_{μν}∂_ρϕ∂^ρϕ + δ_{μν}U(ϕ) + \f{d-2}{4(d-1)}(δ_{μν}∂_ρ∂^ρ-∂_μ∂_ν)ϕ²
\end{align}
となる。運動方程式を用いると、これがトレースレスであることを示せる。

\subsection{
    特殊共形変換
}
特殊共形変換に対する場の変換性はプライマリー場に制限しなければ複雑な形となる。
多くの理論は作用に微分項を含むため、特殊共形変換に対するNoetherカレントを直接構成するのは大変である。
しかし、改良されたエネルギー運動量テンソルを用いると、
\begin{align}
    {J^μ}_ν
    ≔\~T^{μρ}(2x_νx_ρ-x^2δ_{νρ})
    = 2x_νx_ρ\~T^{μρ}- x^2\~T^μ_ν
\end{align}
と簡単に定義できる。
% \begin{align}
%     {J^μ}_ν
%     = 2x_νx_ρT^{μρ}-x^2T^μ_ν
%     - 2x_ν∂_ρL^{μρ}+2L^μ_ν
%     + \t{(surface term)}
% \end{align}
この微分を計算すると、
\begin{align}
    ∂_μ{J^μ}_ν
    = 2x_ν\~T^μ_μ + 2x^μ(\~T_{νμ}-\~T_{μν})
    = 0 \␣ (\t{classical})
\end{align}
が成り立つので、保存則が成り立つ。
ここから特殊共形変換についての対称性が分かる。
ここで、回転対称性とスケール不変性から自動的に共形不変性が導かれるように見えるが、そうではない。
そもそも、エネルギー運動量テンソルをトレースレスに改良する際に、
\begin{align}
    T^μ_μ = ∂^μ∂^νL_{μν}\␣ (\t{classical})
\end{align}
となる$L_{μν}$が存在することを仮定していた。
この仮定はスケール不変性が共形不変性に拡大するための十分条件となっている。

\subsection{
    エネルギー運動量テンソルの別定義
}
Euclid不変な理論に対し、改良されたエネルギー運動量テンソル $T_{μν}$は添字の交換について対称であるから、座標変換$x^μ → {x'}^μ = x^μ - ε^μ$に対し、
\begin{align}
    δS
    &
    = \f{1}{2}∫\d^d{x}T_{μν}(∂^με^ν+∂^νε^μ)
    = \f{1}{2}∫\d^d{x}T_{μν}δg_{μν}
\end{align}
と書ける。
ここで、作用を一般の計量$g_{μν}$まで拡張して
\begin{align}
    S[ϕ] = ∫\d^d{x}ℒ(ϕ,∂_μϕ) → 
    S[g,ϕ] = ∫\d^d{x}\√g ℒ(g_{μν},ϕ,∂_μϕ)
\end{align}
と変更する。
ここで$\√g ≔ \√{\det g_{μν}}$である。
すると、一般座標変換に対して不変となるような作用を構成でき、
\begin{align}
    \f{1}{2}∫\d^d{x}\√g T^{μν}δg_{μν} + ∫\d^d{x}\δ{S}{g_{μν}}δg_{μν} = 0
\end{align}
が成り立つ。ここから
\begin{align}\tcboxmath{
        T^{μν} = -\f{2}{\√g}\δ{S}{g_{μν}}
}\end{align}
が得られる。この式をエネルギー運動量テンソルの定義とすることができる。

ここで、作用$S[g,ϕ]$にRicciテンソル$R_{μν}$に比例する項を加えても平坦な計量の理論は不変であることに注目する。つまり、作用を
\begin{align}
    S → S + ∫\d^dx\√g R_{μν}(x)\~L^{μν}(x)
\end{align}
と変えてもよい。
この自由度を用いてエネルギー運動量テンソルを改良することができる。
実はこれが(\ref{traceless improvement})がやっていることである。
もしエネルギー運動量テンソルがトレースレスに改良できたとすると、共形変換に対し、
\begin{align}
    δS
    = \f{1}{2}∫\d^dx T_{μν}(∂^με^ν+∂^νε^μ)
    = \f{1}{d}∫\d^dx T^μ_μ ∂_ρε^ρ
    = 0
\end{align}
となる。2個目の等号では共形Killing方程式を用いた。
したがって共形不変性が成り立つ。
\end{document}