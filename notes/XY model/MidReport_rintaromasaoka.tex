\documentclass[12pt]{ltjsarticle}
\usepackage{amsmath,ascmac,amssymb,mathtools,siunitx,diffcoeff,inputenc}
\usepackage{mygraphics}
\graphicspath{{../images/}}
\usepackage[backend=biber]{biblatex}
\addbibresource{../ref/ref.bib}
\usepackage[oldfont,exchangeupit]{unicommand}
\usepackage{subfiles}

\usepackage{amsthm}


%% options %%
\renewcommand{\headfont}{\bfseries}
% \numberwithin{equation}{section}
% \renewcommand*\abstractname{}
% \setlength{\parindent}{0pt}

\DeclareMathOperator{\arccot}{\mathrm{arccot}}
\begin{document}
\title{中間レポート}
\author{政岡凜太郎}
\maketitle

\section*{第4問}

ホモロジー・コホモロジーっぽい言葉で議論してみる。わかりにくかったら申し訳ない。

\subsection*{境界演算子・余境界演算子・ラプラシアン}
$Λ$を周期境界条件を課した$L × L$の正方格子とする。
$Λ$の頂点、辺、面の集合をそれぞれ$𝑉(Λ), 𝐸(Λ), 𝐹(Λ)$とおく。
頂点$v ∈ 𝑉(Λ)$を形式的な基底ベクトルとみなすことで、$v$上に定義された変数$ϕ_v$はベクトル$ϕ$として表現できる:
\begin{align}
    ϕ = ∑_{v ∈ 𝑉(Λ)}ϕ_v v,␣ ϕ_v = (v,ϕ).
\end{align}
辺や面の上に定義された変数も同様にベクトルとして表現できる。

次に境界演算子と余境界演算子を導入する。
面の頂点が反時計回りに$v₁,v₂,v₃,v₄$であるとき、その面(に対応する基底ベクトル)を$[v₁v₂v₃v₄]$と書く。
また頂点$v₁$から頂点$v₂$に向かう辺を$[v₁v₂]$と書き、$[v₂v₁] = -[v₁v₂]$とする。
境界演算子$∂$は
% \footnote{
%     境界演算子は$∂$と書くことが多いが、余微分演算子との対応から$∂$と書いた。
% }
\begin{align}&
    ∂[v₁v₂v₃v₄] = [v₁v₂] + [v₂v₃] + [v₃v₄] + [v₄v₁] \\
    &
    ∂[v₁v₂] = v₂ - v₁ \\
    &
    ∂v = 0
\end{align}
を満たす線形演算子として定義される。
$∂$は$∂² = 0$を満たす。
すなわち境界の境界はゼロである。
また余境界演算子$𝑑$を$𝑑 ≔ ∂^𝑇$によって定義する。
\footnote{
    余境界演算子が外微分に対応することから$𝑑$という表記を用いている。
}
転置は$v ∈ 𝑉(Λ),~e ∈ 𝐸(Λ),~f ∈ 𝐹(Λ)$を基底とした成分表示において行う。
余境界演算子は$𝑑² = (∂²)^𝑇 = 0$を満たす。
ラプラシアンは$Δ ≔ -(𝑑+∂)² = -(𝑑∂+∂𝑑)$と定義される。
\footnote{
    これは通常の定義とは符号が逆である。この定義では$Δ$は不定値になり、$∇²$と直接対応する。
}

\subsection*{Kosterlitz--Thouless転移}

準備が終わったので、XY模型の話をする。
流れとしては、まず大雑把にXY模型の高温展開と低温展開を扱う。次に
XY模型を近似するようなVillain模型を導入し、それが2次元のCoulombガスと等価であることを示し、そこで相転移を議論する。

\subsubsection*{高温展開}
$Λ$の頂点$v$上に定義された角度変数$θ_v ∈  ℝ/(2πℤ)$を考える。
また
\begin{align}
    θ ≔ ∑_{v ∈ 𝑉(Λ)}θ_v v
\end{align}
とする。
XY模型のハミルトニアンは
\begin{align}
    βH[θ] = -K ∑_{e ∈ 𝐸(Λ)}\cos (∂e,θ)
\end{align}
で与えられる。ここで$K > 0$は結合定数である。$K ≪ 1$のとき、分配関数を
\begin{align}
    Z &= ∫\𝒟{θ} ∑_{n=0}^∞÷{1}{n!}(÷{K}{2}∑_{e ∈ 𝐸(Λ)}(ℯ^{¡(∂e,θ)}+ℯ^{-¡(∂e,θ)}))^n
\end{align}
と書き、$K$についての最低次のみを取り出す。ここで
\begin{align}
    ÷{1}{2π}∫\𝑑{θ}ℯ^{¡nθ} = δ_{n0}
\end{align}
から、展開したときに指数関数の肩が相殺するものだけが積分の結果生き残る。
辺$e$からの寄与が符号含めて$Γ_e$回現れたとすると、寄与がノンゼロになる条件は$∑_e Γ_e ∂e = ∂Γ = 0$である。よって分配関数は閉曲線についての足し合わせになる。
ただし、$K$の最低次をとる場合は$Z ≈ 1$としてよい。
次に相関関数
\begin{align}&
    ⟨\cos(θ_v - θ_{v'})⟩
    = ⟨ℯ^{¡(θ_v-θ_{v'})}⟩ \∅
    &
    = ÷1{Z}∫\𝒟{θ}ℯ^{¡(θ_v-θ_{v'})} ∑_{n=0}^∞÷{1}{n!}(÷{K}{2}∑_{e ∈ 𝐸(Λ)}(ℯ^{¡(∂e,θ)}+ℯ^{-¡(∂e,θ)}))^n
\end{align}
を考えると、生き残るのは$v$と$v'$をつなぐような経路である。
$K$の最低次では、これは$v$と$v'$をつなぐ最短経路である。
よって
\begin{align}
    ⟨\cos(θ_v - θ_{v'})⟩
    &
    ∼ (÷{K}{2})^{|𝒙_v - 𝒙_{v'}|} \∅
    &
    = \exp(-\ln ÷{2}{K}⋅|𝒙_v - 𝒙_{v'}|)
\end{align}
となる。よって高温では相関関数は指数関数に減衰することが分かる。

\subsubsection*{低温展開}
次に$K ≫ 1$の場合を考える。その場合は$θ$のゆらぎが小さいため
\begin{align}
    ∑_{e ∈ 𝐸(Λ)}\cos(∂e,θ)
    &
    = ∑_{e ∈ 𝐸(Λ)} \cos(e,𝑑θ) \∅
    &
    ≈ ∑_{e ∈ 𝐸(Λ)}(1 - ÷{1}{2}(e,𝑑θ)²) \∅
    &
    = |𝐸(Λ)| - ÷{1}{2}(𝑑θ,𝑑θ) \∅
    &
    = |𝐸(Λ)| - ÷{1}{2}(θ,Δθ)
\end{align}
としてよい。また$θ ∈ ℝ$としてよい。
よって分配関数は定数を除いて
\begin{align}
    Z = ∫ \𝒟{θ} \exp(-÷{K}{2}(θ,Δθ))
\end{align}
となる。このとき相関関数は
\begin{align}
    ⟨\cos(θ_{v}-θ_{v'})⟩
    = ⟨ℯ^{¡(θ_{v}-θ_{v'})}⟩ 
    = ÷1{Z}∫ \𝒟{θ} \exp(¡(θ_{v}-θ_{v'})-÷{K}{2}(θ,Δθ))
\end{align}
となる。ここで$u ≔ v - v'$とおく。また
\begin{align}
    θ' = θ + ÷{¡}{K}Δ^{-1}u
\end{align}
とおく。すると、
\begin{align}
    -÷{K}{2}(θ,Δθ) + ¡(u,θ)
    &
    = -÷{K}{2}(θ',Δθ') - ÷{1}{2K}(u,Δ^{-1}u)
\end{align}
となる。よって
\begin{align}
    ⟨\cos(θ_{v}-θ_{v'})⟩
    &
    = ⋅÷1{Z}∫ \𝒟{θ'} \exp(-÷{K}{2}(θ',Δθ')- ÷{1}{2K}(u,Δ^{-1}u)) \∅
    &
    = \exp(- ÷{1}{2K}(u,Δ^{-1}u)) \∅
    &
    = \exp(÷{1}{2K}(Δ^{-1}_{vv'}+Δ^{-1}_{v'v}) - ÷{1}{2K}(Δ^{-1}_{vv}+Δ^{-1}_{v'v'}))
\end{align}
である。$Δ^{-1}_{vv}$は$|𝒙_v-𝒙_{v'}|$に依存せず、定数倍がかかるだけなので無視する。
$Δ^{-1}_{vv'}$は長距離では連続的なラプラシアンに対するGreen関数に置き換えられるため、
\begin{align}
    Δ^{-1}_{vv'} ≈ -÷{1}{2π}\ln |𝒙_v - 𝒙_{v'}|
\end{align}
となる。よって、
\begin{align}
    ⟨\cos(θ_{v}-θ_{v'})⟩ ∼ |𝒙_v - 𝒙_{v'}|^{-1/(2πK)}
\end{align}
となる。したがって低温では相関関数がべき乗則で減衰することが分かる。
さらにべきの指数が結合定数に依存するという特徴的な振る舞いが見てとれる。


\subsubsection*{Villain模型とCoulombガス}
この節では、もう少し詳しく模型を解析する。
ただしXY模型をそのまま扱うのは難しいので、以下のようにポテンシャルを変更する。
\begin{align}
    ℯ^{K\cos x} → ℯ^{K} ∑_{l ∈ ℤ}ℯ^{-÷{K}{2}(x+2πl)²}
    \label{XY_Villain}
\end{align}
\begin{figure}[H]
    \centering
    \includegraphics[width=0.5\hsize]{XY_Villain.pdf}
    \caption{$K=1$における式(\ref{XY_Villain})の左辺(青)と右辺(赤)の比較}
\end{figure}
このように変更してもXY模型と同じ普遍類に属する模型が得られると考えられる。
分配関数は
\begin{align}
    Z_{Villain} = ∫\𝒟{θ} ∑_{\{l_e\}}\exp(-÷{K}{2}∑_{e ∈ 𝐸(Λ)}((𝑑θ)_e+2πl_e)²)
\end{align}
となる。ただし$θ_v ∈ [-π,π)$とする。これはVillain模型と呼ばれる。
次に、Villain模型の分配関数を$A ≔ 𝑑θ + 2πl$によって記述することを考える。
エネルギーは $βH = ÷{K}{2}‖A‖²$で与えられるが、$A$に課される拘束条件を求める必要がある。
定義から$A$は$Λ$上の任意の閉曲線$Γ$
\footnote{
    ここで閉曲線は辺を係数$1$で足し合わせたものとして考えている。
}
に対し
\begin{align}
    ∮_Γ A ≔ (Γ,A)
    &
    = (Γ,𝑑θ) + 2π(Γ,l) \∅
    &
    = (∂Γ,θ) + 2π(Γ,l) \∅
    &
    = 2π(Γ,l) ∈ 2πℤ
\end{align}
を満たす。逆に、この性質が成り立つならば$A$を線積分することで$θ$が求まる。
よって拘束条件はこれで十分である。$Γ = ∂f,~ f ∈ 𝐹(Λ)$とすれば
\begin{align}
    ∮_{∂f}A
    &
    ≔ (∂f,A) 
    = (f,𝑑A) ∈ 2πℤ
\end{align}
となる。
今$Λ$はトーラスであるから、$∂f$の線形結合で表されないような独立なサイクルが2つ存在する。
これらを$Γ₁, Γ₂$と書くことにする。
すると、分配関数は$A$によって
\begin{align}
    Z = 2π∫\𝒟{A} &
    ∏_{i=1,2}δ(∮_{Γ_i}A-2πℤ)
    ∏_{f ∈ 𝐹(Λ)}δ((𝑑A)_f-2πℤ) \∅
    &
    ×\exp(-÷{K}{2}∑_{e ∈ 𝐸(Λ)}A_e²)
\end{align}
となる。$2π$の因子は角度の並進$θ_v → θ_v + \const$の自由度を考慮するために入れた。
また
\begin{align}
    δ(x-2πℤ) ≔ ∑_{n ∈ ℤ} δ(x - 2πn)
\end{align}
である。
拘束条件を扱うために、$A$の中の実質的な自由度を取り出すことを考える。
まず
\begin{align}
    m₁ ≔ ÷{1}{2π} ∮_{Γ₁} A,␣ m₂ ≔ ÷1{2π} ∮_{Γ₂} A
\end{align}
とおく。これはトーラスの2つの軸方向に対して$θ$が何回巻き付いたかを表す。
任意の$A$はdivergence-freeな部分$∂ψ$とrotation-freeな部分$𝑑θ$と巻き付きを表す部分によって
\begin{align}
    A = ∂ψ + 𝑑ϕ + 2πm₁a₁ + 2πm₂a₂
\end{align}
と表すことができる。ここで$ψ$は面上、$ϕ$は頂点上で定義された実数変数であり、$a₁,a₂$はそれぞれ$x$方向、$y$方向の辺に対して$1/L$の値をもつ場である。
$a_i$は
\begin{align}
    ∮_{Γ_i}a_j = δ_{ij},␣ 𝑑a_i = 0,␣ ∂a_i = 0
\end{align}
を満たす。
並進の自由度$ψ_f → ψ_f + \const$および$ϕ_v → ϕ_v + \const$を固定すれば、$A$と$(ψ,ϕ,m₁,m₂)$は一対一に対応する。
\footnote{
    自由度を勘定してみると、まず$A$の自由度は$|𝐸|-2g$。ただし$g$は$Λ$の種数で今はトーラスなので$g = 1$。次に$ψ, ϕ$の自由度はそれぞれ$|𝐹|-1, |𝑉|-1$。$m_i$は離散的なので考慮しない。
    これらを等式で結ぶと$|𝐸|-2g = |𝑉| + |𝐹| - 2$。これはEuler標数の定義$2-2g = |𝑉|-|𝐸|+|𝐹|$から確かに成り立っている。
}
ここで、
\begin{align}
    B = 𝑑∂ψ = (𝑑∂+∂𝑑)ψ = -Δψ
\end{align}
より$ψ = -Δ^{-1}B$である。
\footnote{
    $Δ$はゼロ固有値をもつため$Δ^{-1}$は慎重に定義する必要がある。
    今の場合は拘束条件$∑_fB_f = 0$から$B$に$Δ$のゼロモードが含まれないため、逆行列$Δ^{-1}$が定義できる。
}
% ただし、$ψ$は$\ker Δ = \ker ∂$に含まれる成分を持っていないとする。$A$は$∂ψ$を通して$ψ$に依存するため、この仮定は$A$に影響しない。
すると
\begin{align}
    βH
    &
    = ÷{K}{2} ‖∂ψ+𝑑ϕ+2πm₁a₁+2πm₂a₂‖² \∅
    &
    = ÷{K}{2}(‖∂ψ‖² + ‖𝑑ϕ‖² + 4π²m₁²‖a₁‖² + 4π²m₂²‖a₂‖² ) \∅
    &
    = ÷{K}{2}((Δψ,ψ) + ‖𝑑ϕ‖² + 4π²m₁² + 4π²m₂²) \∅
    &
    = ÷{K}{2}((B,Δ^{-1}B) + ‖𝑑ϕ‖² + 4π²m₁² + 4π²m₂²)
\end{align}
となる。
ただし、$∂²ψ = 0,~𝑑²ϕ = 0,~𝑑a_i = 0,~ ∂a_i = 0$を用いた。
変数に対する拘束条件は$B_f = 2πn_f,~ n_f ∈ ℤ$と表される。また$B = 𝑑A$から
\begin{align}
    ∑_{f ∈ 𝐹(Λ)} n_f
    &
    = ÷1{2π}∑_{f ∈ 𝐹(Λ)}(f,𝑑A) 
    = ÷1{2π} ∑_{f ∈ 𝐹(Λ)}(∂f,A) 
    = (0,A) = 0
\end{align}
が成り立つ。ただし$Λ$に境界がないことを用いた。
よって、分配関数は
\begin{align}
    Z = Z_{s.w.}Z_{t}∑_{m₁ ∈ ℤ} ℯ^{-2π²Km₁²}∑_{m₂ ∈ ℤ} ℯ^{-2π²Km₂²}
\end{align}
と書ける。ここで
\begin{align}&
    Z_{s.w.} = 2π∫\𝒟{ϕ} ∏_{v ∈ 𝑉(Λ)}\𝑑{ϕ_v}δ(∑_{v ∈ 𝑉(Λ)} ϕ_v)\exp(- ÷{K}{2}∑_{e ∈ 𝐸(Λ)}(𝑑ϕ)_e²), \\
    &
    Z_{t} = ∑_{\{n_f\}}
    δ(∑_{f ∈ 𝐹(Λ)} n_f)\exp(-2π²K∑_{f,f' ∈ 𝐹(Λ)}n_fΔ^{-1}_{ff'}n_{f'})
\end{align}
である。
$δ(∑_vϕ_v)$は$ϕ_v$の並進に対する自由度を固定するために課した。
$Z_{s.w.}$はスピン波の寄与を表し、$Z_{t}$は渦の寄与を表す。
$Z_{s.w.}$は単なるGaussian理論であり、相転移には寄与しない。
$Z_{t}$は2次元のCoulomb相互作用する電荷の分配関数と等価であり、
低温では正負の電荷が束縛された状態、高温では電荷が自由に動けるプラズマ状態が実現する。
これらの間の相転移がKosterlitz--Thouless転移である。
電荷に対応する$n$は渦度であったから、Kosterlitz--Thouless転移は渦欠陥が束縛された状態と、渦欠陥が自由に動ける状態の間の相転移であると言える。

% \begin{align}
%     ℯ^{2π²KD(𝒓,𝒓',𝒔,𝒔')}
%     &
% \end{align}
% \begin{align}
%     Δ^{-1}(𝒓,𝒓') = Δ^{-1}(𝒓-𝒓')
%     &
%     = - ÷{1}{L²} ∑_{𝒌 ≠ 𝟎} ÷{ℯ^{¡𝒌⋅𝒓}}{4-2\cos k₁ -2 \cos k₂} 
% \end{align}
% $k₁,k₂ ∈ (2π/L)ℤ_L$
% \begin{align}
%     Δ^{-1}(𝒓) ≈ -÷1{2π}\ln ÷{r}{L} - ÷1{4}
% \end{align}
% \begin{align}
%     Δ^{-1}(𝒓) = ÷1{(2π)²} ∫_{|𝒌|>a/L}÷{\𝑑^2{𝒌}}{𝒌²} = 
% \end{align}
% \begin{align}
%     H = 2π²J(÷{\ln L}{2π}-÷1{L}) (∑_𝒓 m(𝒓))² - πJ ∑_{𝒓 ≠ 𝒓'}\ln |𝒓-𝒓'|  m(𝒓)m(𝒓')
% \end{align}
% \begin{align}
%     G(r)
%     &
%     = ÷{1}{(2π)²}∫_{|𝒌|>1/L}\𝑑^2{𝒌} ÷{ℯ^{¡k₁r}}{𝒌²} \∅
%     &
%     = ÷{1}{(2π)²}∫_{|𝒌|>r/L}\𝑑^2{𝒌} ÷{ℯ^{¡k₁}}{𝒌²}
% \end{align}
% \begin{align}
%     ∫_{r/L}^∞ ÷1{k₂²+k₁²} \𝑑{k₂}
%     &
%     = ÷2{|k₁|} ∫_{r/(L|k₁|)}^∞ ÷1{1+x²} \𝑑{x} \∅
%     &
%     = ÷2{|k₁|}(÷{π}{2} -\arctan ÷{r}{L|k₁|}) \∅
%     &
%     = ÷{2}{|k₁|}\arccot÷{L|k₁|}{r}
% \end{align}
% \begin{align}
%     ÷1{π²} ∫_{r/L}^∞ 𝑑{k₁}÷{\cos k₁}{k₁}\arccot÷{L|k₁|}{r}
% \end{align}
% \begin{align}
%     ∫_0^{2π} \𝑑{ϕ} ℯ^{¡kr\cosϕ} = 2πJ₀(kr)
% \end{align}
% \begin{align}
%     ÷1{2π}∫_0^{2π} \𝑑{ϕ} ℯ^{¡kr\cosϕ}
%     ≈ 1 - ÷{k²r²}{4π}
% \end{align}
% \begin{align}
%     G(𝒓)
%     &
%     = ÷1{2π}∫_{1/L}^∞ ÷{J₀(kr)}{k} \𝑑{k} \∅
%     &
%     = ÷{1}{2π}∫_{r/L}^∞ ÷{J₀(k)}{k}\𝑑{k}
% \end{align}
% \begin{align}
%     -÷{J₀(kr)}{2πr} ≈ -÷{1}{2πr}(1-÷{r²}{4L²})
% \end{align}
% \begin{align}
%     G(r) = -÷1{2π}\ln (÷{r}{L})
% \end{align}
% \begin{align}
%     Z = ∫ \𝒟{A} ∏_{f ∈ F(Λ)}∑_{n_f ∈ ℤ}δ(𝑑A(f)-2πn_f)
%      \exp(-÷{J}{2} ∑_{e ∈ E(Λ)} A(e)²)
% \end{align}
% \begin{align}
%     ∑_{n ∈ ℤ}δ(x-2πn) = ÷1{2π} ∑_{m ∈ ℤ}ℯ^{¡mx}
% \end{align}
% \begin{align}
%     Z = ∫ \𝒟{A} ∑_{\{m(f)\}} \exp(¡ ∑_{f ∈ F(Λ)} m(f)𝑑A(f)-÷{J}{2} ∑_{e ∈ E(Λ)} A(e)²)
% \end{align}
% \begin{align}
%     ∑_{e ∈ E(Λ)} A(e)² = ∑_{e ∈ E(Λ)} 𝑑^{-1}B(e)² = ∑_{f ∈ F(Λ)} B(e)Δ^{-1}B(e)
% \end{align}
% \begin{align}
%     Z &= ∫ \𝒟{B} ∑_{\{m(f)\}} \exp( ∑_{f ∈ F(Λ)}[-÷{J}{2} B(f)Δ^{-1}B(f) + ¡m(f)B(f)] ) 
% \end{align}
% さらに
% \begin{align}
%     𝑑^𝑇m(e) ≔ ∑_{f ∈ ∂^𝑇e} m(f)
% \end{align}
% と定義する。ここで$∂^𝑇e$は辺$e$のcoboundaryである。すると部分積分から
% \begin{align}
%     ∑_{f ∈ F(Λ)}m(f)𝑑A(f) = ∑_{e ∈ E(Λ)}𝑑^𝑇m(e)A(e)
% \end{align}
% が成り立つ。 よって
% \begin{align}
%     Z = ∫ \𝒟{A} ∑_{\{m(f)\}} \exp( ∑_{e ∈ E(Λ)}[-÷{J}{2}A(e)² + ¡∂m(e)A(e)])
% \end{align}
% と書ける。
% ここでfluxの自由度$m(f)$のみを残して$A(e)$をintegrate outする。
% \begin{align}&
%     -÷{J}{2} A(e)² + ¡ 𝑑^𝑇m(e)A(e) 
%     = -÷{J}{2} (A(e) - ÷{¡}{J}𝑑^𝑇m(e))² - ÷{1}{2J} 𝑑^𝑇m(e)²
% \end{align}
% より、
% \begin{align}
%     Z
%     &
%     = ∑_{\{m(f)\}}\exp(- ÷{1}{2J} ∑_{e ∈ E(Λ)} 𝑑^𝑇m(e)²) \∅
%     &
%     = ∑_{\{m(f)\}}\exp(- ÷{1}{2J} ∑_{f ∈ F(Λ)} m(f)Δm(f) )
% \end{align}
% ただし
% \begin{align}
%     Δm(f) ≔ 𝑑𝑑^𝑇m(f) = ∑_{f' ∈ ∂^𝑇∂f} m(f')
% \end{align}
% \begin{align}
%     Z = ∫ \𝒟{ϕ} ∑_{\{m(f)\}}\exp(∑_{f}[-÷{1}{2J}ϕ(f)Δϕ(f)+2π¡m(f)ϕ(f)] )
% \end{align}
\end{document}