\documentclass[12pt]{ltjsarticle}
\usepackage{amsmath,ascmac,amssymb,mathtools,siunitx,diffcoeff,inputenc}
\usepackage{mygraphics}
\graphicspath{{../images/}}
\usepackage[backend=biber]{biblatex}
\addbibresource{../ref/ref.bib}
\usepackage[oldfont,exchangeupit]{unicommand}
\usepackage{subfiles}

%% options %%
\renewcommand{\headfont}{\bfseries}
% \numberwithin{equation}{section}
% \renewcommand*\abstractname{}
% \setlength{\parindent}{0pt}
\begin{document}
\title{XY模型とくりこみ群}
\author{政岡凜太郎}
\maketitle

% \section{準備}
% \subsection{微分形式}
% 定義だけ書く。wedge積は
% \begin{align}&
%     𝑑x^μ ∧ 𝑑x^ν = - 𝑑x^ν ∧ 𝑑x^μ,\\
%     &
%     𝑑x^μ ∧ (𝑑x^ν ∧ 𝑑x^λ) = (𝑑x^μ ∧ 𝑑x^ν) ∧ 𝑑x^λ
% \end{align}
% を満たすような形式的な積である。
% $p$-形式を以下のように定義する。
% \begin{itemize}
%     \item 0-形式: $f(x)$
%     \item 1-形式: $f_μ(x)𝑑x^μ$
%     \item 2-形式: $÷1{2}f_{μν}(x)𝑑x^μ ∧ 𝑑x^ν$
%     \item $⋮$
% \end{itemize}
% ここで$f(x), f_μ(x), f_{μν}(x),…$は$ℝ^d$上の関数である。
% $p$-形式 $f = f_{μ₁⋯μ_p} 𝑑x^{μ₁} ∧ ⋯ ∧ 𝑑x^{μ_p}$に対し外微分$𝑑$は
% \begin{align}
%     𝑑f ≔ ÷{∂f_{μ₁⋯μ_p}}{∂x^μ} 𝑑x^μ ∧ 𝑑x^{μ₁} ∧ ⋯ ∧ 𝑑x^{μ_p}
% \end{align}
% と定義される。
% 外微分は冪零性$𝑑² = 0$およびLeibniz則$𝑑(f∧g) = 𝑑f∧g + (-1)^F f∧𝑑g$を満たす。
% またStokesの定理
% \begin{align}
%     ∫_{∂M}f = ∫_M 𝑑f
% \end{align}
% が成り立つ。
% ここで$F$は$p$-形式に対して次数$p$を返す作用素である。

% Hodge star $★$を
% \begin{align}
%     ★(f_{μ₁⋯μ_p}(x) 𝑑x^{μ₁} ∧ ⋯ ∧ 𝑑x^{μ_p}) = ϵ^{μ₁⋯μ_d} f_{μ₁⋯μ_p}  𝑑x_{μ_{p+1}} ∧ ⋯ ∧ 𝑑x_{μ_d}
% \end{align}
% によって定義すると、$p$-形式 $f, g$間の内積が
% \begin{align}
%     (f, g) ≔ ∫_{ℝ^d} f ∧ ★g
% \end{align}
% によって定義される。
% 余微分$𝑑^†$は$(f, 𝑑g) = (𝑑^†, g)$を満たすように定められ、具体的には
% \begin{align}
%     𝑑^† ≔ (-1)^F ★^{-1}𝑑★
% \end{align}
% である。Laplacianは
% \begin{align}
%     Δ ≔ 𝑑𝑑^† + 𝑑^†𝑑
% \end{align}
% と定義される。

\section{離散微分幾何}
まず$d$次元格子は$d$次元ユークリッド空間$ℝ^d$に埋め込まれた点の集合であるが、多くの格子模型においては各点の座標の情報は不要であり、どの点とどの点が隣接しているかのみが重要である。
より一般には点だけでなく辺、面など高次元の隣接関係も重要となるので、以下のように$d$次元格子を再定義しよう:

$d$次元格子とは0次元胞体(点)、1次元胞体、(辺)、2次元胞体(面)、$…$、$d$次元胞体の集合であり、
構造として境界作用素$∂$が入っているようなものである(数学ではCW複体と呼ばれる)。
境界作用素$∂$は$p$次元胞体に対して$(p-1)$次元胞体の整数係数の形式的な線形結合を与える線形作用素である。
境界作用素の詳細な定義は省略するが、低次元の図形の場合には我々の直感から明らかである(そして多くの物理の問題では低次元の図形しか扱わない)。

このノートでは格子$Λ$における$p$次元胞体の集合を$V_p(Λ)$で表す。
$p$次元胞体の形式的な線形結合
\begin{align}
    f = ∑_{i ∈ V_p(Λ)} f(i) i
\end{align}
を$p$-形式と呼ぶ。
これは微分形式の言葉遣いの濫用だが、こうすると格子系と連続極限をとった場の理論を並行に議論できるため便利である。
線形結合の係数は場合によって$ℤ$, $ℝ$, $ℂ$などである。
例えば0-形式は格子点$x$に対して係数$ϕ(x)$を与えるスカラー関数とみなせる。また1-形式は各辺$e$上に割り当てられた係数$A(e)$によって定められる。
これもスカラー関数に思われるかもしれないが、辺$e$が向きをもつため、$A(e)$は辺上の流れの密度として解釈される。
$p$-形式 $f, g$の間の内積を
\begin{align}
    (f, g) ≔ ∑_{i ∈ V_p} f^*(i) g(i)
\end{align}
によって定める。
次に格子上の外微分$𝑑$を
\begin{align}
    𝑑 ≔ ∂^†
\end{align}
によって定義する。
Hermite共役は標準的な基底(点、辺、面、$…$)による成分表示での共役転置として行う。
すると$(p+1)$-形式 $M$と$p$-形式 $f$に対してStokesの定理
\begin{align}
    ∫_{∂M}f ≔ (∂M, f) = (M, 𝑑f) ≕ ∫_M 𝑑f
\end{align}
が成り立つ。
格子余微分は$𝑑^† = ∂$である。
また格子Laplacianを
\begin{align}
    Δ ≔ 𝑑𝑑^† + 𝑑^†𝑑 = ∂^†∂+∂∂^†
\end{align}
によって定義する。
これは2次元格子の場合にはグラフLaplacianと呼ばれる。

最後にHodge star $★$について述べておこう。
ただし離散の世界ではHodge starは連続のときほどは役に立たない。
$d$次元格子$Λ$から、双対格子$\∼Λ$を
\begin{align}
    V_{d-p}(\∼Λ) ≅ V_p(Λ)
\end{align}
によって構成できる。双対格子$\∼Λ$上の$p$-形式に対して、対応する$(d-p)$-形式を与える作用素
\begin{align}
    ★: V_{d-p}(\∼Λ) → V_p(Λ)
\end{align}
をHodge starと呼ぶ。
また双対格子上の境界作用素$\∼∂$を
\begin{align}
    \∼∂ = (-1)^p★^{-1}∂^†★
\end{align}
によって定義する。ここでの$p$は$p'$-形式に対して次数$p'$を返す作用素とする。
連続極限では元の格子と双対格子は区別できなくなるから、チルダ$\∼{\phantom{a}}$は全て外してしまって良い。するとHodge starは$p$-形式と$(d-p)$-形式の同型を表す作用素となる。
また
\begin{align}
    ∂ = (-1)^p ★^{-1}∂^†★,␣ 𝑑^† = (-1)^p ★^{-1}𝑑★
\end{align}
を得る。
% 格子内の$p$次元部分領域$M$を、係数$0, ±1$をもつ$p$-形式として表現し、デルタ関数形式を
% \begin{align}
%     ★^{-1}δ(M) = a^{-p}M
% \end{align}
% によって定める。すると積分を
% \begin{align}
%     ∫_M f = (f, ★^{-1}δ(M)) = ∫ f ∧ δ(M)
% \end{align}
% と表す事ができる。ただし最後の表示は形式的なものである(離散の世界ではwedge積はうまく定義できない)。


\section{XY模型}

XY模型の作用は
\begin{align}
    S[ϕ]
    = β ∑_{e ∈ V₁}\cos(∂e, ϕ) 
    = - β ∑_{⟨ x,y ⟩ ∈ V₁}\cos(ϕ(x)-ϕ(y))
\end{align}
によって与えられる。
ここで$V₁$は辺の集合であり、$⟨ x,y ⟩$は頂点$x, y$をつなぐ辺を表す。
また$ϕ$は$ℝ/2𝜋ℤ$を係数にもつ0-形式とする。
$a$は格子間隔である。
分配関数は
\begin{align}
    Z = ∫ \𝒟{ϕ} \exp(β∑_{e ∈ V₁}\cos(∂e, ϕ)),␣
    ∫\𝒟{ϕ} ≔ ∏_{x ∈ V₀}∫_0^{2𝜋}\𝑑{ϕ(x)}
\end{align}
となる。

\subsection*{高温展開}
XY模型の分配関数を
\begin{align}
    Z &= ∫ \𝒟{ϕ} ∑_{n=0}^∞÷{1}{n!}(÷{β}{2}∑_{e ∈ V₁}(ℯ^{¡(∂e,ϕ)}+ℯ^{-¡(∂e,ϕ)}))^n
\end{align}
と書く。ここで$\𝒟{ϕ} ≔ ∏_{x ∈ V₀}\𝑑{ϕ(x)}/2𝜋$である。
$β ≪ 1$の高温領域において、$β$についての最低次の寄与を取り出すと$Z = 1$である。
次に相関関数
\begin{align}&
    ⟨ \cos(ϕ(x)-ϕ(y)) ⟩ = ⟨ℯ^{¡ϕ(x)-¡ϕ(y)}⟩ = ⟨ℯ^{¡(x-y,ϕ)}⟩ \∅
    &
    = ÷1{Z}∫\𝒟{ϕ} ℯ^{¡(x-y,ϕ)} ∑_{n=0}^∞÷{1}{n!}(÷{β}{2}∑_{e ∈ V₁}(ℯ^{¡(∂e,ϕ)}+ℯ^{-¡(∂e,ϕ)}))^n
\end{align}
を考える。
\begin{align}
    ∫_0^{2𝜋}÷{\𝑑{ϕ(x)}}{2𝜋}ℯ^{¡nϕ(x)} = \begin{cases}
        1 & (n = 0) \\
        0 & (n ≠ 0)
    \end{cases}
\end{align}
から、展開したときに指数関数の肩が相殺するものだけが生き残る。
$n$次の展開係数において指数関数の肩は
\begin{align}
    ¡(x-y ± ∂e₁ ± ∂e₂ ± ⋯ ± ∂e_n, ϕ) ≕ ¡(x-y + ∂Γ, ϕ)
\end{align}
と表されるが、この寄与が非ゼロになる条件は$∂Γ = y-x$である。
よって生き残るのは$x$と$y$をつなぐような曲線である。
$β$の最低次では、$x$と$y$をつなぐ最短経路が主要な寄与を与える。
よって
\begin{align}
    ⟨\cos(ϕ(x) - ϕ(y))⟩
    ∝ (÷{β}{2})^{d(x, y)}
    = \exp(-÷{d(x,y)}{ξ}),␣ ξ = \ln ÷{2}{β}
\end{align}
となる。ここで$d(x,y)$は$x$と$y$の間の距離を表す。
よって高温では相関関数は指数減衰することが分かる。

\subsection*{低温展開}
XY模型の作用は$\cos$が入っていて扱いにくいので、$\cos$をTaylor展開して
\begin{align}
    S[ϕ] ≈ -β∑_{e ∈ V₁}(1-÷1{2}(𝑑ϕ, e)(e, 𝑑ϕ))
    &
    = ÷{β}{2}(𝑑ϕ,𝑑ϕ) + \const \∅
    &
    = ÷{β}{2}(ϕ,Δϕ) + \const
    \label{naive approximation}
\end{align}
とする。ここで$ϕ$が0-形式なことから$𝑑^†𝑑ϕ = Δϕ$となることを用いた。
さらに$ϕ ∈ ℝ/2πℤ$であることを忘れ、$ϕ ∈ ℝ$として扱う。
低温では$ϕ$がほとんど一方向を向くため、$ϕ$の値域のトポロジカルな構造は無視してよい。
すると作用(\ref{naive approximation})は自由ボソン場(Gaussian模型)の作用に一致する。
自由ボソンについて、Wickの定理から
\begin{align}
    ⟨ℯ^{¡n(ϕ(x)-ϕ(y))}⟩
    &
    = ∑_{m=0}^∞ ÷1{(2m)!}(¡n)^{2m}⟨(ϕ(x)-ϕ(y))^{2m}⟩ \∅
    &
    =  ∑_{m=0}^∞ ÷{(2m-1)!!}{(2m)!}(¡n)^{2m}⟨(ϕ(x)-ϕ(y))²⟩^m \∅
    &
    = ∑_{m=0}^∞  ÷1{m!}(-÷{n²}{2}⟨(ϕ(x)-ϕ(y))²⟩)^m \∅
    &
    = \exp(-÷{n²}{2}⟨(ϕ(x)-ϕ(y))²⟩) \∅
    &
    = \exp(÷{n²}{β}(Δ^{-1}(x,y) - Δ^{-1}(0,0)))
\end{align}
が成り立つ。
この結果はWickの定理を使わなくても求めることができる。まず
\begin{align}
    ∂Γ = y-x,␣ ϕ' = ϕ-÷{¡n}{β} Δ^{-1}∂Γ
\end{align}
とおく。$x, y$は$ℝ^d$の元ではなく、$V₀$の元であることに注意。すると、平方完成により
\begin{align}&
    ⟨ℯ^{¡n(ϕ(x)-ϕ(y))}⟩ = ⟨ ℯ^{-¡n(∂Γ,ϕ)} ⟩ \∅
    &
    = ÷1{Z} ∫𝒟ϕ\exp(÷{β}{2}(ϕ,Δϕ) - ¡n(∂Γ,ϕ)) \∅
    &
    = ÷1{Z} ∫𝒟ϕ \exp(÷{β}{2}(ϕ', Δϕ') + ÷{n²}{2β}(∂Γ,Δ^{-1}∂Γ))\∅
    &
    = \exp(÷{n²}{β}(Δ^{-1}(x,y) - Δ^{-1}(0,0))).
\end{align}
$d$次元Euclid空間上LaplacianのGreen関数は$d=1$の場合
\begin{align}
    Δ^{-1}(𝒙,𝒚) = -÷{|𝒙-𝒚|}{2},
\end{align}
$d=2$の場合
\begin{align}
    Δ^{-1}(𝒙,𝒚) = -÷1{2π} \ln |𝒙-𝒚|,
\end{align}
$d > 2$の場合
\begin{align}
    Δ^{-1}(𝒙,𝒚) = -÷{|𝒙-𝒚|^{2-d}}{(d-2)(2π)^{÷{d}{2}}Γ(÷{d}{2})},
\end{align}
で与えられる。離散的なLaplacianの場合でも、長距離の振る舞いは連続極限の場合と同じである。
次に$Δ^{-1}(0, 0)$はFourier変換を用いて
\begin{align}
    Δ^{-1}(0, 0)
    = -∑_{\substack{𝒌 ∈ \mathrm{BZ}\\ 𝒌 ≠ 0}}÷{1}{L^d ∑_{i=1}^d(1-\cos k_i)} ∝ L²
\end{align}
となる。
よって、$d=2$の場合
\begin{align}
    ⟨\cos(ϕ(𝒙)-ϕ(𝒚))⟩ ∼ |𝒙 - 𝒚|^{-1/(2πβ)}
\end{align}
となる。したがって低温では相関関数がべき乗則で減衰することが分かる。
さらにべきの指数が結合定数に依存するという特徴的な振る舞いが見てとれる。


\subsubsection*{Villain模型とCoulombガス}
低温領域での作用の近似(\ref{naive approximation})は実は高温領域では定性的に誤った結果を与える。
問題は作用が$ϕ(x) ↦ ϕ(x)+2𝜋$に対して不変でないことである。
これを解決するために、以下の周期性を保つ近似を用いる。
\begin{align}
    ℯ^{β\cos t} ≈ ℯ^{β} ∑_{l ∈ 2𝜋ℤ} ℯ^{-÷{β}{2}(t+l)²}.
\end{align}
すると、$ℤ$係数の1-形式 $l: V₁ → ℤ$を導入して、作用を以下のように書き直せる。
\begin{align}
    Z = ∫\𝒟{ϕ} \𝒟{l} \exp(-÷{β}{2}(𝑑ϕ+2𝜋l, 𝑑ϕ+2𝜋l) )
\end{align}
ここで
\begin{align}
    ∫\𝒟{ϕ} ≔ ∏_{x ∈ V₀} ∫_0^{2𝜋}\𝑑{ϕ(x)},␣  ∫\𝒟{l} ≔ ∏_{e ∈ V₁} ∑_{l(e) ∈ 2𝜋ℤ}
\end{align}
である。
この模型はVillain模型と呼ばれる。
以下ではVillain模型を用いてXY模型の性質を調べることにしよう。

まずVillain模型を$A ≔ 𝑑ϕ +2𝜋l$によって記述することを考える。
すると自由場の作用
\begin{align}
    S = ÷{β}{2}(A,A)
\end{align}
が得られるが、$A$に課される拘束条件に注意する必要がある。
まず$𝑑A = 2𝜋𝑑l$から、$A$は$2𝜋$単位の離散的なフラックスのみを許す。
また非可縮な閉曲線$Γ_i, i = 1,…,2g$に対し、
\begin{align}
    ∮_{Γ_i} A = (Γ, 𝑑ϕ) + 2𝜋(Γ, l) = 2𝜋(Γ, l) ∈ 2𝜋ℤ
\end{align}
が成り立つ。
逆にこれらの性質が成り立つならば、基準点$x₀$における$ϕ(x₀)$を定め、
そこから$A$を線積分することで$ϕ(x)$が一意に求まる。
したがって分配関数は
\begin{align}
    Z = 2π∫\𝒟{A} &
    ∏_{i=1}^{2g}δ_{2𝜋ℤ}(Γ_i,A)
    ∏_{f ∈ V₂}δ_{2𝜋ℤ}(𝑑A(f)) ℯ^{-÷{β}{2}(A,A)}
\end{align}
となる。$2π$の因子は角度の並進$ϕ(x) ↦ ϕ(x) + \const$の自由度を考慮するために入れた。
また
\begin{align}
    δ_{2𝜋ℤ}(x) ≔ ∑_{n ∈ ℤ} δ(x - 2πn)
\end{align}
である。

次に、Hodge分解によって
\begin{align}
    A = 𝑑ϕ + 𝑑^†ψ + ω
\end{align}
と表す。$ω$は調和形式であり、$𝑑ω = 0,~ 𝑑^†ω = 0$を満たす。
今度は分配関数を$(ϕ, ψ, ω)$で記述しよう。
まず$ϕ, ψ$の不定性を固定するために、$ϕ = 𝑑^†a, ψ = 𝑑b$とおく。
$ϕ = 𝑑^†a$はHodge分解になっているから、$ϕ$は0-調和形式すなわち定数と直交する。
したがって拘束条件
\begin{align}
    ∑_{x ∈ V₀}ϕ(x) = 0
\end{align}
が導かれる。$ϕ$は0-形式で$𝑑χ$のような成分は存在し得ないことに注意する。
また拘束条件を満たす任意の$ϕ$について、
$ϕ ∈ [0, 2𝜋)$と$m ∈ 2𝜋ℤ$ によって$ϕ=ϕ+2𝜋m$と表示すると、$l = 𝑑m$とすれば
$ψ$について、

$ψ = 𝑑b$も同様に調和形式と直交する。$ψ$は2-形式で$𝑑^†χ$のような成分は存在し得ないため、これ以上の拘束条件はない。
$ω$は1-調和形式であり、曲面の種数$g$とすると$2g$個の基底が存在する。
具体的には非可縮な閉曲線$Γ_i$に対して$(Γ_i, ω_j) = δ_{ij}$を満たすように調和形式の基底$ω_i$を構成できる。
また
\begin{align}
    (Γ_i, ω) = (Γ_i, 2𝜋l - 𝑑^†ψ) ∈ ℤ
\end{align}
から、
\begin{align}
    m_i ≔ (ω_i, ω) = (ω_i, A) ∈ 2𝜋ℤ
\end{align}
\begin{align}
    ω = 2𝜋∑_{i=1}^{2g}m_iω_i
\end{align}
と書ける。ここで$m_i ∈ ℤ$である。

作用を$ϕ, ψ, ω$を用いて表示すると、
\begin{align}
    S
    &
    = ÷{β}{2}(𝑑ϕ + 𝑑^†ψ + ω, 𝑑ϕ + 𝑑^†ψ + ω)\∅
    &
    = ÷{β}{2}((𝑑ϕ,𝑑ϕ) + (𝑑^†ψ,𝑑^†ψ) + (ω,ω) + 2(𝑑ϕ,𝑑^†𝑑b) + 2(𝑑ϕ,ω) + 2(𝑑^†ψ,ω) ) \∅
    &
    = ÷{β}{2}((ϕ,Δϕ) + (ψ,Δψ) + (ω,ω)) \∅
    % &
    % = ÷{β}{2}(ϕ,Δϕ) + 2π²β(q,Δ^{-1}q) + ÷{β}{2}(ω,ω)
    \label{action of free-boson and Coulomb gas}
\end{align}
となる。

次に、$ϕ, ψ, ω$から$ϕ, A$を復元できるか考えよう。$ϕ$は$𝑑ϕ$を変えない変換による不定性をもつ。すなわち0-調和形式$ϕ₀$によって$ϕ ↦ ϕ + ϕ₀$としても$ϕ, ψ, ω$は変化しない。
したがって$ϕ₀$も系の自由度に含める必要がある。
さらに
\begin{align}
    Δψ = 𝑑𝑑^†𝑑b = 𝑑A = 2𝜋q
\end{align}
とおいて、$ψ$を渦度$q$で書き直す。
$q$は整数係数の2-形式であり、調和形式と直交する。
以上の議論により、分配関数は、
\begin{align}
    Z = Z_{ϕ₀} Z_ϕ Z_q Z_ω
\end{align}
と表される。ここで
\begin{align}&
    Z_{ϕ₀} = ∫_0^{2𝜋} \𝑑{ϕ₀} = 2𝜋, \\
    &
    Z_ϕ = ∫ 𝒟ϕ δ(÷1{|V₀|} ∑_{x ∈ V₀} ϕ(x)) \exp(÷{β}{2}(ϕ,Δϕ)), \\
    &
    Z_q =  ∑_{\{q\}}δ(∑_{f ∈ V₂} q(f))\exp(2π²β(q,Δ^{-1}q)), \\
    &
    Z_ω = ∏_{i=1}^{2g}(∑_{m_i ∈ ℤ}ℯ^{-2πβm_i²})
\end{align}
である。
$Z_ϕ$は自由ボソンであり、先ほど調べた。
また$Z_ω$は局所的な物理量を含まないため、大域的な演算子の期待値を計算しない限りは無視できる。
この分配関数はCoulomb相互作用する電子系のそれに等しい。
KT転移は$Z_q$によって引き起こされるが、$q$を電荷と考えるとKT転移はプラズマ相転移とみなせる。

平均場近似をしたときの作用の値は
\begin{align}
    S_{eff}
    &
    ∼ 2n \ln n +  2n(πβ\ln ÷{l_n}{l}+\ln÷{L}{l_n}+S_𝑐) - 2n \ln ÷{L²}{l²} \∅
    &
    = 2n ((πβ-2)\ln ÷{l_n}{l} + S_𝑐)
\end{align}
となる。
ここで$l_n = L/√n$である。
$S_{eff}$の最小値が$l/l_n = 0$にあるか、$l/l_n ≠ 0$にあるかによって相転移が起こるので、
転移点は
\begin{align}
    β = ÷{2}{π},␣ T_{KT} = ÷{πφ₀²}{2m}
\end{align}
によって与えられる。

\section*{3.5 Renormalization group}

\subsection*{3.5.1 Relevant and irrelevant perturbations}

くりこみ群の固定点は一般的に共形場理論になることが知られている。
共形場理論の基本的なデータは演算子の共形次元と演算子積展開係数。
演算子の相関関数は
\begin{align}
    ⟨O_i(x)O_j(y)⟩ = ÷{δ_{ij}}{|x-y|^{Δ_i+Δ_j}}
\end{align}
で与えられる。$Δ_i$が演算子の共形次元。
次に演算子積展開は
\begin{align}
    O_i(x)O_j(y) = ∑_{k} C_{ijk}(x-y)O_k(y)
\end{align}
というもの。演算子積展開を用いると$n+1$点関数が$n$点関数から計算できるので、原理的に任意の相関関数が計算可能。

共形場理論の知識を用いて、固定点の近傍についてのくりこみ群フローを考える。
作用が固定点から微小にずれた
\begin{align}&
    S = S₀ - ∑_i ÷{g_i}{l^{d-Δ_i}}∫\𝑑^dx O_i(x)
\end{align}
である場合を考える。
分配関数を摂動展開すると、
\begin{align}
    Z₀( 1 + ∑_{i<j}÷{g_ig_j}{l^{2d-Δ_i-Δ_j}}∫_{|x-y| > l}\𝑑^dx \𝑑^dy ⟨O_i(x)O_j(y)⟩+ ⋯)
\end{align}
% $ℯ^{-S}$を摂動展開すると、
% \begin{align}
%     ℯ^{-S₀}\Bigg( 1
%     &
%     + ∑_i  ÷{g_i}{l^{d-Δ_i}}∫\𝑑^dx O_i(x) \∅
%     &
%     + ∑_{i<j}÷{g_ig_j}{l^{2d-Δ_i-Δ_j}}∫_{|x-y| > l}\𝑑^dx \𝑑^dy O_i(x)O_j(y) + ⋯
%     \Bigg)
% \end{align}
% となる。
座標のスケール変換$x ↦ x' = ℯ^{-δτ}x$を行うと、
\begin{align}&
    ∫\𝑑^d{x'₁}⋯\𝑑^d{x'_n}⟨O_{i₁}(x'₁)⋯O_{i_n}(x'_n)⟩ \∅
    &
    = ℯ^{-∑_{j=1}^n (d-Δ_{i_j})δτ}
    ∫\𝑑^d{x₁}⋯\𝑑^d{x_n}⟨O₁(x₁)⋯O_n(x_n)⟩
\end{align}
となる。この変化を結合定数の変化によって打ち消すためには
\begin{align}
    g_i → ℯ^{(d-Δ_i)δτ}g_i
\end{align}
とすればよい。よって1次のくりこみ群は
\begin{align}
    ÷{𝑑g_i}{𝑑τ} = (d-Δ_i)g_i + 𝒪(g²)
\end{align}
で与えられる。この微分方程式の解の挙動は$d-Δ_i$の符号によって大きく変わる。
$Δ_i > d$の場合、演算子$O_i$はirrelevantであるといい、くりこみ変換によって$g_i$は減少する。
$Δ_i < d$の場合、演算子$O_i$はrelevantであるといい、くりこみ変換に$g_i$は増加する。
$Δ_i = d$の場合、演算子$O_i$はmarginalであるといい、高次の項を参照しないと$g_i$の増減は分からない。
特に高次の項まで厳密に$𝑑g_i/𝑑τ = 0$が成り立つ場合、$O_i$による摂動を加えても理論は固定点にいることになる。
このとき固定点は$O_i$方向に線状に伸びており、$O_i$はexactly marginalであるという。
自明なexactly marginal operatorはfree boson
\begin{align}
    S = ÷κ{2}∫\𝑑^dx ∂_μϕ∂^μϕ
\end{align}
に対する運動項$÷1{2}∂_μϕ∂^μϕ$である。これを摂動として作用に加えても結合定数を$κ ↦ κ + g$とするだけなので、理論は固定点に居続ける。
ただし、$ϕ$のスケール変換によって$κ$の値は自由に変更できるため、意味のある摂動ではない。
この摂動が重要になるのは$ϕ + 2πR = ϕ$によってboson場がコンパクト化されているときである。
このとき$ϕ$を自由にスケール変換することはできないので、運動項$÷1{2}∂_μϕ∂^μϕ$は非自明なexactly marginal operatorになる。

2次のくりこみ群はスケール変換のカットオフ$l$への影響を考えることで得られる。
その際に演算子積展開
% \begin{align}
%     ÷{g_i}{l^{d-Δ_i}}∫\𝑑^dx O_i(x)
%     ↦ ℯ^{(d-Δ_i)δτ}÷{g_i}{l^{d-Δ_i}}∫\𝑑^dx O_i(x)
% \end{align}
% \begin{align}
%     g_i ↦ ℯ^{(d-Δ_i)δτ}g_i
% \end{align}
\begin{align}
    O_i(x)O_j(y)
    = ∑_{k} ÷{C_{ijk}}{|x-y|^{Δ_i+Δ_j-Δ_k}}O_k(y) + ⋯
\end{align}
を用いる。
ここで、スピンをもたないプライマリー演算子以外の寄与は$⋯$の中にまとめている。
スケール変換によって短距離カットオフは$l ↦ ℯ^{-δτ}l$となる。
くりこみ変換の前後でカットオフを合わせるために、$ℯ^{-δτ} < |x-y| < l$の球殻上で相関関数を積分する。
% \begin{align}&
%     ∫_{|x-y| > l}\𝑑^d{x} \𝑑^d{y} ⟨O_i(x)O_j(y)⟩ \∅
%     &
%     = ℯ^{(2d-Δ_i-Δ_j)δτ} ∫_{|x'-y'| > ℯ^{-τ}l}\𝑑^d{x'}\𝑑^d{y'}O_i'(x')O_j'(y') \∅
%     &
%     =  ℯ^{(2d-Δ_i-Δ_j)δτ}(∫_{ℯ^{-δτ}l < |x-y| < l}+∫_{l<|x-y|})\𝑑^d{x} \𝑑^d{y} O_i(x)O_j(y)
% \end{align}
\begin{align}&
    ÷1{2}∑_{i,j}÷{g_ig_j}{l^{2d-Δ_i-Δ_j}} ∫_{ℯ^{-δτ}l < |x-y| < l}\𝑑^d{x} \𝑑^d{y} O_i(x)O_j(y) \∅
    &
    ≈ ÷{1}{2}∑_{i,j}÷{g_ig_j}{l^{d-Δ_i-Δ_j}}S_{d-1}δτ∫\𝑑^d{y} ∑_k ÷{C_{ijk}}{l^{Δ_i+Δ_j-Δ_k}} O_k(y) \∅
    &
    = δτ ∑_k ÷{S_{d-1}}{2l^{d-Δ_k}}∑_{i,j} C_{ijk} g_ig_j ∫\𝑑^d{y} O_k(y)
\end{align}
この寄与を結合定数の変化としてくりこむことで、2次のくりこみ群方程式
\begin{align}
    ÷{𝑑g_k}{𝑑τ} = (d-Δ_k)g_k + ÷1{2} S_{d-1}∑_{i,j}C_{ijk}g_ig_j + 𝒪(g³)
\end{align}
を得る。

\subsection*{3.5.2 The duality between the two-dimensional XY-model and two dimensional clock model}

Coulomb気体の代わりにclock模型を用いる。
分配関数
\begin{align}
    Z = ∏_x ∑_{ρ(x) ∈ ℤ}ℯ^{-S[ρ]},␣
    S = -2π²β (ρ,Δ^{-1}ρ)
\end{align}
に対してPoisson和公式
\begin{align}
    ∑_{x ∈ ℤ} f(x) =  ∑_{k ∈ ℤ} ∫ \𝑑{x} ℯ^{2π¡kx} f(x)
\end{align}
を用いることで、
\begin{align}
    Z = ∫\𝒟{ρ} ∑_{m(x) ∈ ℤ}  ℯ^{-S[ρ] + 2π¡(m,ρ)}
\end{align}
となる。ここで$ρ$をintegrate-outすると、
\begin{align}
    S[m] = ÷1{2β}(m,Δm)
\end{align}
これは固体表面を表すSolid on solid模型であり、プラズマ相転移はroughening相転移に対応する。
さらに$m$が整数であることを忘れて
\begin{align}
    S[ϕ] = ∫\𝑑^2{𝒙}(÷{κ}{2}(∂_𝒙ϕ)² - g\cos ϕ),␣
    κ = ÷{1}{4π²β}
\end{align}
とすることでclock模型が得られる。ただし$m = ϕ/2π$とした。
逆に、clock模型の分配関数は
\begin{align}
    Z
    &
    = ∫\𝒟{ϕ}∑_k ÷1{k!}(g∫\𝑑^2𝒙 ÷{ℯ^{¡ϕ}+ℯ^{-¡ϕ}}{2})^k
    ℯ^{-∫\𝑑^2𝒙 ÷κ{2}(∂_𝒙ϕ)²} \∅
    &
    = Z₀∑_k ÷1{(2k)!}⋅÷{(2k)!}{k!k!}(÷{g}{2})^{2k}\exp(q_iq_j∑_{i<j}^{2k}⟨ϕ(x_i)ϕ(x_j)⟩) \∅
    &
    = Z₀∑_k÷{1}{k!k!}ℯ^{-2kS_c}ℯ^{2πβ ∑_{i<j}^{2k} q_iq_j \ln(r_{ij}/l)}
\end{align}
と書ける。ここで$ℯ^{-S_c} = g/2$である。

% ここで$g = 0$とすればモデルは自由ボソンになることに注意する。
% また周期を$n$倍してより一般的な模型
% \begin{align}
%     S[ϕ] = ∫\𝑑^2{𝒙}(÷{n²κ}{2}(∂_𝒙ϕ)² - g\cos (nϕ))
% \end{align}
% を考えることにする。

\subsection*{3.5.3 Physical properies of clock model}
自由ボソンはくりこみ群の自明な固定点であり、
この固定点における演算子の次元および演算子積展開を求めることで、
固定点の近傍におけるくりこみ群フローを知ることができる。
vertex-operator \footnote{
    正確には正規積によって$V_n(x) ≔ {:ℯ^{¡nϕ}:}$と定義するべき。
}
\begin{align}
    V_n(x) ≔ ℯ^{¡nϕ(x)}
\end{align}
に対し、
\begin{align}
    ⟨ℯ^{¡n(ϕ(x)-ϕ(0))}⟩₀
    = (÷{|𝒙|}{l})^{-n²/2πκ}
\end{align}
である。ただし$⟨⋅⟩₀$は自由ボソンによる期待値を表す。
よって$\cos(nϕ)$の共形次元は$Δ_n = n²/4πκ$である。
$\cos ϕ $がrelevantになる条件は$1/4πκ < 2$であるから、XY模型の相転移点は
\begin{align}
    κ = ÷{1}{8π},␣ β = ÷{2}{π}
\end{align}
となる。先ほどの粗い議論と同じ値が得られた!

次に作用に$V_n(x)$による微小な摂動が加わったとしよう。
$κ < n²/8π$のとき$V_n(x)$はrelevantだから$ℤ_n$対称性が自発的に破れる相が実現する。
逆に$κ > n²/8π$ならば$\U(1)$対称性が低エネルギーで復活する。

\subsection*{3.5.4}
スケール変換と短距離カットオフの変更$l ↦ λ > l$に対して作用がどのように変化するかを見る。
\begin{align}
    ϕ_l = ϕ_λ + δϕ
\end{align}
とする。ただし$δϕ$は短波長のゆらぎを表す。作用は
\begin{align}
    S = & ∫\𝑑^2𝒙(÷{κ_l}{2}(∂_𝒙ϕ_λ)² - g_l\cos(nϕ_λ)+÷{κ_l}{2}(∂_𝒙δϕ)²) \∅
    &
    + ∫ \𝑑^2𝒙 (-ng_l\sin(ϕ_λ)δϕ + ÷{n²}{2}g_l\cos(nϕ_λ)δϕ² )
\end{align}
となる。これを$δϕ$について積分すると
\begin{align}
    S =& ∫\𝑑^2{𝒙}(÷{κ_l}{2}(∂_𝒙ϕ_λ)² - g_l\cos(nϕ_λ) + ÷1{2}g_l \cos(nϕ_λ) K(0)) \∅
    &
    - ∫\𝑑^2𝒙 \𝑑^2𝒚 ÷1{2}(g_l)²\sin(nϕ_λ(𝒙))K(𝒙-𝒚)\sin(nϕ_λ(𝒚))
\end{align}
となる。ここで
\begin{align}
    K(𝒙) &
    = n²⟨δϕ(𝒙)δϕ(0)⟩ \∅
    &
    = ÷{n²}{Z}∫\𝒟{(δϕ)}δϕ(𝒙)δϕ(0)\exp(-∫\𝑑^2{𝒙}÷{κ_l}{2}(∂_𝒙δϕ)²)
\end{align}
である。
また
% \begin{align}
%     \⟨\exp(-∫\𝑑^2𝒙÷{n²}{2}g_l\cos(nϕ_λ)δϕ²)\⟩
%     = \exp(-∫\𝑑^2𝒙÷{1}{2}g_l\cos(nϕ_λ)K(0))
% \end{align}
% とした。
第2項は以下のように書き換えられる。
\begin{align}&
    ∫\𝑑^2𝒙\𝑑^2𝒚 ÷{g_l²}{4}[\sin(nϕ_λ(𝒙))-\sin(nϕ_λ(𝒚))]²K(𝒙-𝒚)
    - ∫\𝑑^2𝒙 ÷{g_l²}{2}\sin²(nϕ_λ)\_K \∅
    &
    = ∫\𝑑^2𝒙\𝑑^2𝒚 ÷{g_l²}{8}\cos²(nϕ_λ(𝒙))(∂_𝒙ϕ_λ(𝒙))²|𝒙-𝒚|²K(𝒙-𝒚) \∅
    &
    - ∫\𝑑^2𝒙 ÷1{2}(g_l)²(\sin(nϕ_λ(𝒙)))² \_K
\end{align}
ここで$\_K = ∫ \𝑑^2 𝒙 K(𝒙)$である。
$\cos(nϕ_λ)$以外の場を無視すると、
\begin{align}
    S = ∫\𝑑^2𝒙 (÷{κ_λ}{2}(∂_𝒙ϕ_λ)² - g_λ\cos(nϕ_λ))
\end{align}
となる。$κ_λ,~g_λ$は以下のように与えられる。
\begin{align}
    g_λ = g_l - ÷1{2}K(0),␣ κ_λ = κ_l + ÷{n²}{8}g_l² K₂
\end{align}
\begin{align}
    K(0) = n²∫_{2π/λ < |𝒌| < 2π/l}÷{\𝑑^2𝒌}{(2π)²}÷{1}{κ_l|𝒌|²} = ÷{n²}{2πκ_l}\ln ÷{λ}{l}
\end{align}
\begin{align}
    K₂ &= ∫ \𝑑^2𝒙 |𝒙|² K(𝒙) \∅
    &
    = n²∫_{2π/λ < |𝒌| < 2π/l} \𝑑^2{𝒙}÷{\𝑑^2𝒌}{(2π)²}÷{𝒙²ℯ^{¡𝒌⋅𝒙-0⁺|𝒙|}}{κ_l 𝒌²} \∅
    &
    = ÷{λ-l}{l}÷{3n²l⁴}{16π⁵κ_l}∫\𝑑{ϕ}÷1{(\cosϕ + ¡0⁺)⁴} = ÷{λ-l}{l}÷{3n²l⁴}{2π⁴κ_l}
\end{align}
くりこみ群方程式は、
\begin{align}&
    ÷{𝑑g}{𝑑τ} = -÷{n²}{4πκ} \\
    &
    ÷{𝑑κ}{𝑑τ} = ÷{3n⁴l⁴}{128π⁵κ} g²
\end{align}
となる。

次にOPEを使ってくりこみ群を求めてみる。
\begin{align}
    S = ∫ \𝑑^2𝒙 (÷κ{2}(∇ϕ)² - ÷{g}{l^{2-Δ_n}}\cos(nϕ))
    % S = ∫ \𝑑^2𝒙 [÷1{2}(∇ϕ)² - g\cos(nϕ)]
\end{align}
\begin{align}
    {:ℯ^{A}:}{:ℯ^{B}:} = ℯ^{⟨AB⟩₀}{:ℯ^{A+B}:}
\end{align}
を用いると、
\begin{align}
    V_n(x) V_m(y)
    &
    = {:ℯ^{¡nϕ(x)}:}{:ℯ^{¡mϕ(y)}:} \∅
    &
    = ℯ^{-nm⟨ϕ(x)ϕ(y)⟩₀}{:ℯ^{¡nϕ(x)+¡mϕ(y)}:} \∅
    &
    = |x-y|^{nm/2πκ}V_{n+m}(y) + ⋯
\end{align}
\begin{align}
    V_n(x)V_{-n}(y)
    &
    = ℯ^{n²⟨ϕ(x)ϕ(y)⟩₀}{:ℯ^{¡n(ϕ(x)-ϕ(y))}:} \∅
    &
    = |x-y|^{-n²/2πκ}(1 - ÷{n²}{4}|x-y|²{:(∇ϕ)²:} + ⋯ )
\end{align}
よって
\begin{align}
    \cos(nϕ)\cos(nϕ) = ÷1{2}|x-y|^{-n²/2πκ} - ÷{n²}{4}|x-y|^{2-n²/2πκ}⋅÷12 {:(∇ϕ)²:} + ⋯
\end{align}
% \begin{align}
%     \exp(¡n((x^μ-y^μ)∂_μϕ + ÷1{2}(x^μ-y^μ)(x^ν-y^ν)∂_μ∂_νϕ + ⋯))
% \end{align}
\begin{align}&
    ÷{𝑑κ}{𝑑τ} = ÷{n²g²}{4} \\
    &
    ÷{𝑑g}{𝑑τ} = (2-÷{n²}{4πκ})g
\end{align}
% \begin{align}
%     ÷{𝑑K}{𝑑τ} = ÷{n²}{4πκ²}÷{𝑑κ}{𝑑τ} = -÷{n²g²}{16πκ²} = -÷{π}{n²}K²g²
% \end{align}
% \begin{align}
%     ÷{𝑑g}{𝑑τ} = (2-K)g
% \end{align}
少し違う結果が出たがいいのだろうか...

\section{3.6 Boson superfluid to Mott insulator transition}

続いて、1+1 D compactified free bosonを考える。
このときはHamilton形式が便利。
\begin{align}
    ϕ(x,t) = ϕ₀(t) + n÷{2𝜋x}{L} + ∑_{k≠0}ϕ_kL^{-1/2}ℯ^{¡kx}
\end{align}
\end{document}