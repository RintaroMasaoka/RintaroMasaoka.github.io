\providecommand{\main}{../main}
\documentclass[\main/main.tex]{subfiles}
\graphicspath{{../images/}}
\begin{document}
\section{エニオンとMPO代数}

\subsection*{MPO代数のデータ}
\begin{definition}[Projector matrix product operators]
    $B ∈ \L(𝒲)⊗\L(𝒱), Δ ∈ \L(𝒲)$によって表される以下のMPO $P_L ∈ \L(𝒱)^{⊗L}$を考える。
    \begin{align}
        P_L = \Tr[Δ\overbrace{B⋯B}^L]
        =∑_{\{i\},\{j\}} \Tr[ΔB^{i₁j₂}⋯B^{i_Lj_L}]|i₁,…,i_L⟩⟨j₁,…,j_L|
    \end{align}
    ただし行列の積は$𝒲$についてとった。
    $P_L$は射影であるとし、$Δ$を挿入する位置に依存しないとする。
    さらに、以下のブロック対角化が成り立つとする。
    \begin{align}&
        𝒲 = ⨁_{a=1}^𝒩 𝒲_a, \\
        &
        B^{ij} = ⨁_{a=1}^𝒩 B^{ij}_a␣B^{ij}_a ∈ \L(𝒲) , \\
        &
        Δ = ⨁_{a=1}^𝒩 w_a 𝟙_a,␣w_a ∈ ℂ.
    \end{align}
    ここで$B_a$はinjectiveなテンソルであるとする。
    すなわち、$\Span\{B^{ij}_a\}_{i,j} = \L(𝒲_a)$である。
    さらに転送行列の最大固有値が$1$になるように規格化されているとする。
    $P_L$が$Δ$の位置に依らないことと、$B_a$のinjectivityから$Δ$に対するブロック行列は$𝟙_a$の定数倍に限られることに注意する。
    このような$P_L$をprojector matrix product operator (PMPO)と呼ぶ。
\end{definition}

\begin{theorem}[Fusion tensor]
    $B$をPMPOを構成するテンソルとし、$\{B_a\}_{a=1}^𝒩$をinjectiveなブロックとする。
    また$O_a^L ≔ \Tr[\overbrace{B_a⋯B_a}^L]$とおき、$\{O_a^L\}_{a=1}^𝒩$がなす行列代数が閉じているとする。
    このとき、$X_{ab,μ}^c: 𝒲_c → 𝒲_a⊗𝒲_b,~μ=1,…,N_{ab}^c$が存在して
    \begin{align}
        {X}_{ab,μ}^{+c}(∑_j B_a^{ij}⊗B_b^{jk})X_{ab,μ}^c
        = B^{ik}_c.
        \label{block diag. of MPO prod.}
    \end{align}
    が成り立つ。
    テンソル$X_{ab,μ}^c$をfusion tensorと呼ぶ。
    % (\ref{block diag. of MPO prod.})の左辺は以下のように図示される。
    % \begin{align}
    %     \mfig[0.4]{Fusion tensor}
    % \end{align}
\end{theorem}
\begin{proof}
    $O_a^LO_b^L$に対する標準形の存在からゲージ変換および非対角ブロックの削除によって
    \begin{align}
        B_a^{ij}⊗B_b^{jk} ↦ ⨁_{c=1}^𝒩 ⨁_{μ=1}^{N_{ab}^c} λ_{ab,μ}^c B^{ik}_c,␣
        λ_{ab,μ}^c ∈ ℂ
    \end{align}
    と表せる。
    ここから
     \begin{align}
        O_a^LO_b^L = ∑_{c,μ} (λ_{ab,μ}^c)^L O_c^L
     \end{align}
     である。よって
     \begin{align}
        P_L = P_L² = ∑_{a,b}w_aw_bO_a^LO_b^L = ∑_{a,b,c,μ} (λ_{ab,μ}^c)^L w_aw_b O_c^L
     \end{align}
     である。$P_L$が任意の$L$において射影であることから$λ_{ab,μ}^c = 1$が分かる。
     よって$𝒲_a ⊗ 𝒲_b$と各ブロック$𝒲_c$の間の変換${X}_{ab,μ}^{c},{X}_{ab,μ}^{+c}$が存在して(\ref{block diag. of MPO prod.})が成り立つ。
\end{proof}
\begin{definition}[Zipper condition]\label{def: zipper condition}
    (\ref{block diag. of MPO prod.})から直ちに以下が成り立つ。
    \begin{align}&
        (∑_j B_a^{ij} ⊗ B_b^{jk})X_{ab,μ}^c = X_{ab,μ}^cB_c^{ik}, \\
        &
        X_{ab,μ}^{+c}(∑_j B_a^{ij} ⊗ B_b^{jk}) = B_c^{ik}X_{ab,μ}^{+c}
    \end{align}
    この式をzipper conditionと呼ぶ。
\end{definition}
\begin{definition}
    行列代数$\{O_a^L\}_{a=1}^𝒩$は
    \begin{align}
        O_a^LO_b^L = ∑_c N_{ab}^c O_c^L,␣ N_{ab}^c ∈ ℤ_{≥0}
    \end{align}
    によって定まる。
    $N$テンソルは$∑_j B_a^{ij}⊗B_b^{jk}$をinjectiveなブロックに分解したときの多重度によって与えられる。
\end{definition}
    % $P_L$が射影であることから、
    % \begin{align}
    %     P_L² = ∑_{a,b = 1}^{𝒩}w_aw_bO_aO_b = ∑_{a=1}^{𝒩}w_aO_a = P_L
    %     \label{idemponent condition for PMPO}
    % \end{align}
    % が成り立つ。

\begin{remark}
    PMPO $P_L = ∑_a w_a O_a^L$と$N_{ab}^c$に対し、以下が成り立つ。
    \begin{align}
        ∑_{a,b=1}^𝒩 N_{ab}^c w_aw_b = w_c
    \end{align}
\end{remark}
\begin{proof}
    $P_L² = P_L$から
    \begin{align}
        P_L² = ∑_{a,b}w_aw_bO_aO_b = ∑_{a,b,c} N_{ab}^c w_aw_b O_c = ∑_c w_c O_c = P_L.
    \end{align}
\end{proof}

\begin{definition}[Duality]
    $P_L$がHermitianであることを課す。
    このときブロック$a$に対して$\_B_a$がinjectiveであることからブロック$a^*$が一意に存在し、
    \begin{align}
        \_w_a = w_{a^*},␣
        O_a^{L†} = O_{a^*}^L
        \label{eq: channel duality}
    \end{align}
    となる。また
    \begin{align}
        N_{ab}^c = N_{b^*a^*}^{c^*}
    \end{align}
    となる。$(a^*)^* = a$であり、$a^*$を$a$に双対なブロックと呼ぶ。
\end{definition}
\begin{definition}[Frobenius Schur indicator]\label{def: Frobenius Schur indicator}
    ブロック$a, a^*$は以下のゲージ変換によって結ばれる。
    \begin{align}
        \_B_a^{ji} = Z_a^{-1}B^{ij}_{a^*}Z_a
    \end{align}
    ここで$Z_a$は
    \begin{align}
        % \begin{cases}
        %     \_Z_{a} = Z_{a^*}^{-1} & (a ≠ a^*) \\
        %     \_Z_a = ±Z_a^{-1} & (a = a^*)
        % \end{cases}
        Z_a\_Z_{a^*} = \_Z_{a^*}Z_a = ϰ_a𝟙,␣
        ϰ_a = \begin{cases}
            1 & (a ≠ a^*) \\
            ±1 & (a=a^*)
        \end{cases}
    \end{align}
    を満たす。$ϰ_a$をFrobenius--Schur indicatorと呼ぶ。
    \end{definition}
\begin{proof}
    $O_a^{L^†} = O_{a^*}^L$の両辺を標準形で表すと、標準形の一意性からゲージ変換$Z_a: 𝒲_a → 𝒲_{a^*}$と$θ ∈ ℝ$が存在して
    \begin{align}
        \_B_a^{ji} = ℯ^{¡θ} Z_a^{-1}B_{a^*}^{ij}Z_a
    \end{align}
    となる。(\ref{eq: channel duality})が$L$に依らずに成り立つことから$θ=0$としてよい。
    これを2回用いると、
    \begin{align}
        B_a^{ji}
        = \_Z_a^{-1}\_B_{a^*}^{ij}\_Z_a
        = \_Z_a^{-1}Z_{a^*}^{-1}B_a^{ji}Z_{a^*}\_Z_a
        % &
        % \_B_{a^*}^{ji}
        % = Z_{a^*}^{-1}B_{a}^{ij}Z_{a^*}
        % = Z_{a^*}^{-1}\_Z_a^{-1}\_B_{a^*}^{ji}\_Z_a Z_{a^*}
    \end{align}
    となる。よって$B_a$のinjectivityから$Z_{a^*}\_Z_a = γ_a𝟙 = \_Z_aZ_{a^*}$と表せる。
    ここで$γ_a$は複素数であり$\_γ_{a^*} = γ_a$を満たす。
    $a ≠ a^*$の場合、$Z_a$を定数倍して再定義することで$γ_a = γ_{a^*} = 1$とできる。
    $a = a^*$の場合、$Z_a\_Z_a = \_Z_aZ_a$から$γ_a$は実数に限られる。
    $Z_a$を定数倍して再定義することで必ず$γ_a = ±1 ≕ ϰ_a$にできる。ただし符号を変えることはできない。
\end{proof}

\begin{lemma}[結合則 (associativity)] \label{lem: associativity}
    $(O_a^LO_b^L)O_c^L = O_a^L(O_b^LO_c^L)$から$N$テンソルは以下の結合則を満たす。
    \begin{align}
        ∑_e N_{ab}^e N_{ec}^d = ∑_f N_{af}^dN_{bc}^f.
    \end{align}
\end{lemma}
\begin{definition}[F行列] \label{def: F matrix}
    F行列 $(F^{abc}_d)^{fλσ}_{eμν}$が存在して以下が成り立つ。
    \begin{align}
        (X_{ab,μ}^e ⊗ 𝟙_c)X_{ec,ν}^d = ∑_{f=1}^{𝒩}∑_{λ=1}^{N_{bc}^f}∑_{σ=1}^{N_{af}^d}(F^{abc}_d)^{fλσ}_{eμν}(𝟙_a ⊗ X_{bc,λ}^f) X_{af,σ}^d.
        \label{eq: F matrix}
    \end{align}
\end{definition}
\begin{proof}
    \hyperref[def: zipper condition]{zipper condition}を2通りの順序で用いることで以下の等式が成り立つ。
    \begin{align}
        ∑_{deμν}\mfig[0.4]{pentagon1}
        = ∑_{dfσλ}\mfig[0.4]{pentagon2}.
    \end{align}
    両辺に右からfusion tensorを掛けると
    \begin{align}
        \mfig[0.4]{pentagon3}
        = ∑_{d'fσλ}\mfig[0.4]{pentagon4}.
    \end{align}
    よって$B_d, B_{d'}$のinjectivityから
    \begin{align}
        \mfig[0.4]{pentagon5}
        =∑_{fσλ}\mfig[0.4]{pentagon6}.
    \end{align}
    よって右辺の2個目の因子を$(F^{abc}_d)^{fλσ}_{eμν} 𝟙_d$と書くことができ、(\ref{eq: F matrix})が示される。
\end{proof}

\begin{definition}[Pentagon equation]
    \begin{align}
        ∑_{h,σλω}(F^{abc}_g)^{fμν}_{hσλ}(F^{ahd}_e)^{gλρ}_{iωκ}(F^{bcd}_i)^{hσω}_{jλδ}
        = ∑_σ (F^{fcd}_e)^{gνρ}_{jγσ}(F^{abj}_e)^{fμσ}_{iδκ}.
    \end{align}
\end{definition}
\begin{proof}
    \begin{align}
        \mfig[0.5]{n13}.
    \end{align}
\end{proof}

\begin{remark}
    \begin{align}
        N_{ab}^c = N_{bc^*}^{a^*} = N_{c^*a}^{b^*}
        = N_{b^*a^*}^{c^*} = N_{cb^*}^a = N_{a^*c}^b.
    \end{align}
\end{remark}
\subsection{Unitarity}
\begin{definition}
    これを以下に図示する。
    \begin{align}
        \mfig[0.4]{zipper} \\ \mfig[0.4]{zipper2}
    \end{align}
    ただし$X_{ab,μ}^c$については多重度の添字$μ$だけ図示した。
\end{definition}
\begin{definition}[Pivotal property]
    \begin{align}
        \mfig[0.4]{pivotal1}
        = ∑_ν (A_{ab}^c)_{μν}\mfig[0.4]{pivotal2}
    \end{align}
    ここで$A_{ab}^c$は$(A_{ab}^c)^† A_{ab}^c = ÷{w_c}{w_b}𝟙$を満たす行列である。
    また$b$の足を曲げることで同様に${A'}_{ab}^c$が定義される。
\end{definition}
\begin{lemma}
    \begin{align}
        \mfig[0.4]{pivotal_conj1}
        = ∑_ν (C_{ab}^c)_{μν}\mfig[0.4]{pivotal_conj2}
    \end{align}
    ここで$C_{ab}^c = A_{a^*b}^c \_{A'}_{a^*c^*}^b A_{b^*c^*}^{a^*}$である。
\end{lemma}
\begin{theorem}[Pulling through equation]
    \begin{align}
        \mfig[0.4]{n18}
    \end{align}
\end{theorem}
\begin{proof}
    \begin{align}
        ∑_{bcμ}w_b\mfig[0.4]{PT_derive1}
        = ∑_{bcν}w_c\mfig[0.4]{PT_derive2}
    \end{align}
\end{proof}
\end{document}