\providecommand{\main}{../main}
\documentclass[\main/main.tex]{subfiles}
\graphicspath{{../images/}}
\begin{document}


\section{Injectivie MPS}
\begin{definition}[TI-MPS]\label{def: MPS}
    $𝒱, ℋ$を有限次元Hilbert空間とする。
    $𝒱$をvirtual space, $ℋ$をphysical spaceまたはlocal Hilbert spaceと呼ぶことにする。
    テンソル$A ∈  \L(𝒱) ⊗ ℋ$を以下のように定義する。
    \begin{align}
        A = ∑_{i,j,s} A^s_{ij} |i⟩⟨j|_{𝒱} |s⟩_{ℋ} = \mfig{A}
    \end{align}
    % このとき以下の状態$|ψ⟩ ∈ ℋ^{⊗L}$を$A$から定まるOBC-MPSと呼ぶ。
    % \begin{align}
    %     |ψ(A)⟩ = ⟨𝑙|_𝒱 A ⋯ A|𝑟⟩_𝒱 = \mfig{OBCMPS}
    % \end{align}
    このとき以下の状態を$A$から構成されるtranslation invariant matrix product state (TI-MPS)と呼ぶ。
    \begin{align}
        |ψ(A)⟩
        =  \Tr_𝒱[A ⋯ A]
        = \mfig{PBCMPS}
    \end{align}
    なおこのノートでは並進対称性のある場合しか扱わない。
\end{definition}

% \begin{definition}[Matrix product operator]
    
% \end{definition}
% \begin{definition}[Projected entangled pair state]
    
% \end{definition}

\begin{definition}[Injectivity]\label{def: injective}
    テンソル$A ∈ \L(𝒱) ⊗ ℋ $が
    \begin{align}
        \Span\{A^s\}_s = { \mfig{alphaA} \Mid α } = \L(𝒱)
    \end{align}
    を満たすとき、$A$がinjectiveであると言う。
\end{definition}

\begin{definition}[一般化逆行列]
    $m × n$行列$X$に対し$n × m$行列$X^𝑔$が$XX^𝑔X = X$を満たすとき、$X^𝑔$は$X$の一般化逆行列(generalized inverse)であるという。
\end{definition}
\begin{definition}[擬逆行列]
    $m × n$行列$X$が特異値分解によって
    \begin{align}
        X = UΣV^†
    \end{align}
    と書かれているとする。ここで$U, V$はユニタリ行列、$Σ$は対角行列である。
    このとき$X$の擬逆行列(pseudoinverse)は以下のように定められる。
    \begin{align}
        X⁺ ≔ VΣ^+U^†
    \end{align}
    ここで
    \begin{align}
         Σ⁺_{ij} = \begin{cases}
            (Σ_{ij})^{-1} & Σ_{ij} ≠ 0 \\
            0 & Σ_{ij} = 0
        \end{cases}
    \end{align}
    である。擬逆行列は一般化逆行列であり、$X$が正則なときは逆行列に一致する。
    $A ∈ \L(𝒱) ⊗ ℋ$に対し、対応する$𝒱 ⊗ 𝒱$から$ℋ$への写像が以下のように定まる。
    \begin{align}
        𝒫(A) = ∑_{i,j,s} A^s_{ij} |s⟩_ℋ ⟨i,j|_{𝒱 ⊗ 𝒱} = \mfig{P(A)}
    \end{align}
    この擬逆行列 $𝒫(A)⁺$を与えるテンソルを$A⁺$とする。すなわち、
    \begin{align}
        𝒫(A)⁺ = ∑_{i,j,s} (A⁺)_s^{ij} |i,j⟩_{𝒱 ⊗ 𝒱} ⟨s|_ℋ = \mfig{P(A)+}
    \end{align}
    とする。
\end{definition}

\begin{remark}
    $𝒫(A)𝒫(A)⁺$および$𝒫(A)⁺𝒫(A)$を図示すると
    \begin{align}
        𝒫(A)𝒫(A)⁺ = \mfig{P(A)P(A)+},␣
        𝒫(A)⁺𝒫(A) = \mfig{P(A)+P(A)}
    \end{align}
    である。これらは擬逆行列の定義から直交射影となる。
    これらの射影の像を考えよう。
    $\im (𝒫(A)𝒫(A)⁺) = \im 𝒫(A) ⊊ ℋ$の場合はlocal Hilbert spaceをとり直すことで常に$\im 𝒫(A) = ℋ$とできる。
    次に$\im (𝒫(A)⁺𝒫(A)) ⊂ 𝒱' ⊗ 𝒱' ⊊ 𝒱 ⊗ 𝒱$の場合、virtual spaceを$𝒱'$に取り直す。
    しかし、それでも$\im (𝒫(A)⁺𝒫(A)) ⊊ 𝒱' ⊗ 𝒱'$となり得る。
    この空間をMPSおよびPEPSの特徴づけとして利用するのがinjectivityの考え方である。
    さらに$\im(𝒫(A)⁺𝒫(A))$は
    \begin{align}&
        \im (𝒫(A)⁺𝒫(A)) = \im(𝒫(A)^† 𝒫(A)^{+†}) = \im 𝒫(A)^† = \Span\{\_A^s\}_s
    \end{align}
    と表せる。多くの場合は$*$演算に対して閉じた行列代数を考えるので、$\Span\{A^s\}_s$を考えれば良い。
\end{remark}


\begin{lemma}
    テンソル$A ∈ \L(𝒱) ⊗ ℋ$について、以下は同値
    \begin{enumerate}
        \item  $\Span\{A^s\}_s  = \L(𝒱) = \Span\{\_A^s\}_s.$
        \item $A$から定まる写像$𝒫(A) :\L(𝒱) → ℋ$が単射(injective)
        \item $𝒫(A)⁺𝒫(A) = 𝟙.$
        これは以下のように図示される。
        \begin{align}
            \mfig{AinvA} ~=~\mfig{idid}
        \end{align}
    \end{enumerate}
\end{lemma}
\begin{proof}
    (1. $⇔$ 2.) 
    $\ker 𝒫(A) = \coker 𝒫(A)^† =\L(𝒱)/\im 𝒫(A)^†$より、
    $\Span\{\_A^s\}_s = \im 𝒫(A)^† = \L(𝒱)$と$\ker 𝒫(A) = 0$は同値。
    (1. $⇔$ 3.)
    $𝒫(A)⁺𝒫(A)$が$\Span\{\_A^s\}_s$への射影であることから直ちに同値性が分かる。
\end{proof}

\begin{lemma}[連結(Concatenation)]
    injectivityはテンソルの連結(concatenation)によって保たれる。 
    つまりinjectiveなテンソル$A^s_{ij},B^s_{ij}$に対し、$∑_j A^{s}_{ij}B^{t}_{jk}$はinjectiveである。
\end{lemma}
\begin{proof}
    $\Span\{A^sB^t\}_{s,t} = \L(𝒱)\L(𝒱) = \L(𝒱).$
\end{proof}

\begin{remark}
    MPS $|ψ(A)⟩$ がinjectiveであるというとき、有限個の$A$の連結$A⋯A$がinjectiveであることを意味することがあるので注意。
    例えばAffleck--Keneddy--Lieb--Tasaki模型の基底状態は
    \begin{align}
        A⁺ = √{÷{3}{2}}σ⁺,␣ A⁰ = -√{÷1{3}}σ^z,␣ A⁻ = -√{÷{3}{2}}σ⁻
    \end{align}
    によるMPSとして与えられるが、local Hilbert spaceの次元は$3$、virtual spaceの次元は$2×2 = 4$であるから$A$は明らかにinjectiveではない。2つの$A$をconcatenateしてはじめてinjectiveになる。
\end{remark}

\begin{theorem}[Intersection property] \label{Intersection property}
    $A, B ∈ \L(𝒱) ⊗ ℋ$をinjectiveなテンソルとする。このとき、
    \begin{align}
        {\mfig{ABM}\Mid M} ∩ {\mfig{NBC}\Mid N} = {\mfig{ABCX}\Mid X}
    \end{align}
\end{theorem}
\begin{proof}
右辺$⊂$左辺は明らかなので左辺$⊂$右辺を示す。
左辺の元に対して、テンソル$X$が存在して
\begin{align}
    \mfig{N} ∝ \mfig{AX}
\end{align}
と書けるならば、右辺にも属することが示される。
このような$X$は具体的に以下のように構成される。
\begin{align}
      \mfig{Intersection1}
    ∝ \mfig{Intersection2}
    = \mfig{Intersection3}
    = \mfig{Intersection4}
\end{align}
\end{proof}

\begin{theorem}[Closure property]\label{Closure property}
    $A, B ∈ \L(𝒱) ⊗ ℋ$をinjectiveなテンソルとする。このとき
    \begin{align}
        {\mfig{AMB}\Mid M} ∩ {\mfig{ANB}\Mid N} = {λ\mfig{AB}\Mid λ ∈ ℂ}
    \end{align}
\end{theorem}
\begin{proof}
    右辺$⊂$左辺は明らか。左辺の元に対して等式
    \begin{align}
        \mfig{Closure1}
    ∝ \mfig{Closure2}
    = \mfig{Closure3}
    = \mfig{Closure4}
    \end{align}
    を代入することで、左辺$⊂$右辺が得られる。
\end{proof}

\begin{definition}[Parent Hamiltonian]
    長さ$L$の周期的な1次元スピン系のHilbert空間を
    \begin{align}
        ℋ_{tot} ≔ ⊗_{i=1}^L ℋ_i,␣ ℋ_i = ℋ,␣ ℋ_{L+1} = ℋ₁ 
    \end{align}
    とする。$ℋ_i ⊗ ℋ_{i+1}$のみに非自明に作用する直交射影$h_i$を、injectiveなテンソル$A ∈ \L(𝒱) ⊗ ℋ$から以下のように定義する。
    \begin{align}&
        \ker h_i = {\mfig{AMA}\Mid M } ⊗ (⨂_{j ≠ i,i+1} ℋ_j)
    \end{align}
    Hamiltonian $H = ∑_{i=1}^L h_i$を(周期境界条件における) $A$に対するparent Hamiltonianと呼ぶ。
\end{definition}

\begin{theorem}
    injectiveなテンソル$A ∈ \L(𝒱) ⊗ ℋ$に対するparent Hamiltonianの基底状態はTI-MPS $|ψ(A)⟩$のみである。
\end{theorem}
\begin{proof}
    $\ker H = ⋂_{i=1}^L \ker h_i$を\hyperref[Intersection property]{intersection property}, \hyperref[Closure property]{closure property}を用いて求める。
    まず、\hyperref[Intersection property]{intersection property}において$A=B=C$とすることで、
    \begin{align}
        \ker h_i ∩ \ker h_{i+1} = {\Tr[XAAA] \mid X ∈ \L(𝒱) } ⊗ (⨂_{j ≠ i,i+1,i+2} ℋ_j)
    \end{align}
    を得る。同様の議論を帰納的に繰り返すことで、
    \begin{align}
        ⋂_{i=1}^{L-1} \ker h_i = {\Tr[X\overbrace{A ⋯ A}^L]\Mid X ∈ \L(𝒱) }
    \end{align}
    となる。また同様の議論から、
    \begin{align}
        ⋂_{i=0}^{L} \ker h_i = {\Tr[AX\overbrace{A ⋯ A}^{L-1}]\Mid X ∈ \L(𝒱) }
    \end{align}
    である。よって\hyperref[Closure property]{closure property}において$B=\overbrace{A ⋯ A}^{L-1}$とすることで
    \begin{align}
        \ker H 
        = (⋂_{i=1}^{L-1} \ker h_i) ∩ (⋂_{i=0}^{L} \ker h_i)
        = {λ\Tr[\overbrace{A ⋯ A}^L]\Mid λ ∈ ℂ }.
    \end{align}
\end{proof}


\begin{theorem}[Energy gap]
    injectiveなテンソルに対するparent Hamiltonianはgappedである。
\end{theorem}
\begin{proof}
    証明は\cite{fannesFinitelyCorrelatedStates1992}にある。
    追っていない。
\end{proof}

\end{document}