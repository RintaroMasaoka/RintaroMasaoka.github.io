\providecommand{\main}{../main}
\documentclass[\main/main.tex]{subfiles}
\graphicspath{{../images/}}
\begin{document}

\section{Introduction}
% トポロジカル秩序 or SPT相を調べる方法として、大きく分けて以下の2つがある。
% \begin{itemize}
%     \item 公理論的アプローチ
%     \item 構成論的アプローチ
% \end{itemize}
% % TQFTの圏論的な特徴づけや、SPT相のトポロジカル不変量による分類は公理論的なアプローチに含まれる。
% Matrix product state (MPS)やprojected entangled pair state (PEPS)は具体的な試行波動関数から始める構成論的アプローチである。

このノートではトポロジカル秩序およびエニオンをPEPSを用いて理解することを目指し、matrix product operator (MPO) algebraによる枠組みを解説する。
内容の多くは\cite{schuchPEPSGroundStates2010,sahinogluCharacterizingTopologicalOrder2021, bultinckAnyonsMatrixProduct2017}に依拠している。
またquantum doubleについて\cite{simonTopologicalQuantum2023}を参照した。
またMPSについての基本的な命題(主に\cite{perez-garciaMatrixProductState2007}の内容)についてappendixで触れている。


\end{document}